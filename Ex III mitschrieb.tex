\documentclass[titlepage,11pt,a4paper,ngerman]{report}
\usepackage[utf8]{inputenc}
\usepackage[T1]{fontenc}
\usepackage[german]{babel}
\usepackage{graphicx}
\usepackage{wrapfig}
\usepackage{amsmath}
\usepackage{amsfonts}
\usepackage{amssymb}
\usepackage{tikz}
\usepackage{tikz-cd}
\usepackage{nicefrac}
\usepackage{mathtools}
\usepackage{enumerate}
\usepackage{cancel}
\usepackage[hidelinks]{hyperref}
\usepackage{cleveref}
\usepackage{tcolorbox}


\usepackage[margin=1in]{geometry}

\usepackage{placeins}
\usepackage{booktabs}
\usepackage{wasysym}

\usepackage{url}

\usetikzlibrary{arrows}


%Environments und Newcommands:

% Allgemein:

% zu zeigen symbol
\newcommand{\zz}{\fontfamily{cmss} \selectfont{Z\kern-.61em\raise-0.7ex\hbox{Z}:}}
% build over
\newcommand{\bov}[2]{\buildrel{#2} \over{#1}}
% better looking := (defined as)
\newcommand*{\defeq}{\mathrel{\vcenter{\baselineskip0.5ex \lineskiplimit0pt \hbox{\scriptsize.}\hbox{\scriptsize.}}}=}

\newcommand{\hfw}{\color{RubineRed}\tx{ $\star$hier fehlt was$\star$ } \color{black}}
\def\checkmark{\tikz\fill[scale=0.4](0,.35) -- (.25,0) -- (1,.7) -- (.25,.15) -- cycle;} 

%Text:
\newcommand{\tx}[1]{\textrm{#1}}
\newcommand{\const}{\tx{const.}}

\newcommand{\ul}[1]{\underline{#1}}
\newcommand{\ol}[1]{\overline{#1}}
\newcommand{\ub}[1]{\underbrace{#1}}
\newcommand{\ob}[1]{\overbrace{#1}}

%Mathe:
\newcommand{\verteq}{\rotatebox{90}{$\,=$}}
\newcommand{\equalto}[2]{\underset{\scriptstyle\overset{\mkern4mu\verteq}{#2}}{#1}}
\newcommand{\equaltoup}[2]{\overset{\scriptstyle\underset{\mkern4mu\verteq}{#2}}{#1}}
\newcommand{\custo}[3]{\underset{\scriptstyle\overset{\mkern4mu\rotatebox{-90}{$\,#1$}}{#3}}{#2}}
\newcommand{\custoup}[3]{\overset{\scriptstyle\overset{\mkern4mu\rotatebox{-90}{$\,#1$}}{#3}}{#2}}
\newcommand{\casess}[4]{\left\{ \begin{array}{ll} {#1} & {#2} \\ {#3} & {#4} \end{array} \right.}


%Spezielles:

%Theo:
\newcommand{\lag}{\mathcal{L}}
\newcommand{\ham}{\mathcal{H}}
\newcommand{\prt}[2]{\frac{\partial #1}{\partial #2}}

%LA:
\newenvironment{bew}[1]{\subsection{Bew: #1}}{\hfill$\square$}
\newcommand{\Bew}[2]{\begin{bew}{#1}#2\end{bew}}

\newcommand{\enph}{F: V \to V \textrm{ Endomorphismus}}
\newcommand{\im}{\tx{im}}
\newcommand{\spa}{\tx{span}}
\newcommand{\adj}{\tx{adj}}
\newcommand{\grad}{\tx{grad}}
\newcommand{\ord}{\tx{ord}}

\newcommand{\basis}[3]{\{#1_{#2}, \dots, #1_{#3}\}}
\newcommand{\ska}[2]{\langle #1 , #2 \rangle}
\newcommand{\dmat}[3]{\begin{pmatrix} #1_{#2}&&\\ &\ddots& \\ && #1_{#3} \end{pmatrix}}

%Ex:
\newcommand{\kq}{\frac{1}{4\pi\epsilon_0}}
\newcommand{\uind}{U_{\tx{ind}}}
\newcommand{\folie}[1]{\color{gray}[Folie: #1]\color{black}}

% anderz
\newcommand{\summ}[2]{\sum_{#1}^{#2}}
\newcommand{\intt}[2]{\int_{#1}^{#2}}
\renewcommand{\vec}[1]{\boldsymbol{#1}}
\newcommand{\lcom}[1]{\color{MidnightBlue}#1\color{black}}
\renewcommand{\epsilon}{\varepsilon}

% Boxen:

\tcbuselibrary{theorems}

% mahlt eine box nur um den text mit tittel
\newtcbox{\fribox}[1]{nobeforeafter,colback=white,colframe=red!75!black,fonttitle=\bfseries,title=#1,sharp corners,tcbox raise base}

% mahlt eine große box um alles mit tittel
\newcommand{\frbox}[2]{\begin{tcolorbox}[colback=white,colframe=red!75!black,fonttitle=\bfseries,title=#1]#2\end{tcolorbox}}

% mahlt eine box nur um den text
\newtcbox{\ribox}{nobeforeafter,colback=white,colframe=red!75!black,sharp corners,tcbox raise base}

% mahlt eine große box um alles was drinnen ist
\newcommand{\rbox}[1]{\begin{tcolorbox}[colback=white,colframe=red!75!black]#1\end{tcolorbox}}

% mahlt eine box um mathe innerhalb mathmode
\newcommand{\rmbox}[1]{\tcboxmath[colback=white,colframe=red!75!black]{#1}}

% super box (looks like regular boxed but wraps around anything)
\newenvironment{supbox}{\begin{tcolorbox}[colback=white,colframe=black,sharp corners,boxrule=.5pt]}{\end{tcolorbox}}

% array type box with title
\newenvironment{zebox}[1]{\begin{array}{|c|}
		\multicolumn{1}{l}{\tx{#1}} \\
		\hline
		\displaystyle
	}{\\ \hline
\end{array}}

% evtl: \renewcommand{\boxed}{\rmbox}


\hbadness=99999

\begin{document}
	
\renewcommand{\thechapter}{\Roman{chapter}}

\title{
	{\Huge Experimentalphysik III}\\[1em]
	{\Large Vorlesung von Prof. Dr. Oliver Waldmann im Wintersemester 2018}}
\author{Markus Österle\\ Andréz Gockel}
\date{16.10.2018}
\maketitle
\tableofcontents


%16.10.18

\chapter{Spezielle Relativitätstheorie}
%\addcontentsline{toc}{section}{section die nur im Inhaltsverzeichniss ist}

\section{Vorgeschichte}

\textbf{1861-1867: Maxwell Gleichungen}\\
Vor Maxwell brauchten alle bekannten Wellen (Wasserwellen, Schall) ein Medium oder Trägerstoff. Daher stammte die Annahme, auch elektro-magnetische Wellen also Licht bräuchte ein Medium: der Äther. Somit wollte man experimentell zu zeigen, wie schnell wir uns durch den Äther bewegen und welches das Inertialsystem, das ausgezeichnete Bezugssystem des Äthers ist.\\
%Man stellte jedoch fest, dass es kein solches Inertialsystem gibt und Geschwindigkeiten nur relativ zueinander existieren und nicht absolut sind.
Mit dem Ziel den Äther nachzuweisen wurde das Michelson-Morley Experiment durchgeführt.

\subsection{Experiment von Michelson-Morley}%I.1.a

Die Idee des Experimentes war es die Bewegung der Erde relativ zum Äther zu messen.\\
\textbf{Anforderungen:}
\begin{itemize}
	\item Erde $ 30 \, \tx{km} \slash \tx{s} $
	\item Licht $ 3 \cdot 10^{5} \, \tx{km} \slash \tx{s}  $
	\item[$ \Rightarrow $] relative Auflösung von circa $ 10^{-4} $ nötig
\end{itemize}
%
%
% insert image von Folie interferrometer
%
%
\textbf{Prinzip}:\\
Lichtlaufzeiten:\\
Arm parallel zum Ätherwind
\begin{equation*}
t_2 = \frac{L_2}{c+v} + \frac{L_2}{c-v} = \frac{2cL_2}{(c^2-v^2)} = \frac{2 L_2}{c} \frac{1}{\sqrt{1 - \frac{v^2}{c^2}}^2}
\end{equation*}
Arm senkrecht zum Ätherwind
\begin{equation*}
t_1 = 2 \frac{L_1'}{c}
\end{equation*}
Nebenrechnung:
\begin{equation*}
L'^2 = L^2 + (vt)^2 \quad \Rightarrow \quad (ct)^2 = L^2 + (vt)^2 \quad \Rightarrow \quad t = \frac{L}{\sqrt{c^2 - v^2}}
\end{equation*}
\begin{equation*}
t_1 = \frac{2 L}{\sqrt{c^2 - v^2}}
\end{equation*}
Somit ist die Zeitdifferenz der beiden Lichtstrahlen
\begin{equation*}
\Delta t = t_2 - t_1 = \frac{2L}{c} \ \frac{1 - \sqrt{1 - \frac{v^2}{c^2}}}{1 - \frac{v^2}{c^2}} \approx \frac{2L}{c} \left(\frac{1}{2} \frac{v^2}{c^2}\right) = \frac{lV^2}{c^2}
\end{equation*}
Hieraus können wir nun den Phasenunterschied der beiden Strahlen berechnen
\begin{equation*}
\Delta \varphi = 2 \pi \frac{c \Delta t}{\lambda}
\end{equation*}
Abschätzung:
\begin{equation*}
L = 2 \times 11 \, \tx{m} \ , \ v = 30 \, \tx{km} \slash \tx{s} \ , \ \lambda = 500 \, \tx{nm} \ \Rightarrow \ \Delta \varphi \slash \pi \approx 0{,}88
\end{equation*}
\textbf{Auflösung}
\begin{equation*}
\frac{\Delta \varphi}{\pi} \approx \frac{1}{4}
\end{equation*}
\textbf{Ergebnis:}\\
Es gibt keine Verschiebung des Interferenzmusters.\\[5pt]
\textbf{Konsequenz:}\\
$ \Rightarrow $ Es gibt keinen Ätherwind.\\
$ \Rightarrow $ Es gibt kein ausgezeichnetes Bezugssystem. Alle Bezugssysteme sind gleichwertig.\\[5pt]
\textbf{Aber:}\\
Kontraktionshypothese von Fitzgerald z Lorenz\\
Wenn der Arm in Richtung der Äthers um Faktor $ \gamma = \frac{1}{\sqrt{1 - \frac{v^2}{c^2}}} $ \\
Arm parallel zum Ä.W.:
\begin{equation*}
t_2 = \frac{2cL_2}{c^2 - v^2} \frac{1}{\gamma} = \frac{2 L_2}{\sqrt{denc^2 - v^2}}
\end{equation*}
Arm senkrecht zum Ä.W.:
\begin{equation*}
t_1 = \frac{2L_1}{\sqrt{c^2 - v^2}}
\end{equation*}
\textbf{Aber:}\\
Äther nicht beobachtbar

\subsection{Lorenz-Invarianz der Maxwell-Gleichungen}%I.1.b

Maxwell-Gleichungen sind Lorenz-invariant und nicht Galilei-invariant.

Beispiel: Relativität der Feder.

$\Rightarrow$ Im Laborsystem: $\vec{F}=q(\vec{v}\times\vec{B})$

$\Rightarrow$ Im Ruhesystem: $v'=0\Rightarrow F_L=0$ \lightning

$\Rightarrow$ ein zusätzliches $E'$-Feld, $\vec{F}_q=q\vec{E}$

$\Rightarrow$ neue „Transformationsgleichungen“

\textbf{Transformation}
\begin{itemize}
	\item $K$: „ruhende“ Bezugssystem
	\item $K'$: „sich in Bezug auf $K$ konstant entlang der $x$-Achse bewegendes“ Bezugssystem
\end{itemize}

$$\vec{E}'=(E_x,\gamma(E_y-vB_z),\gamma(E_z-vB_y))^\top$$
$$\vec{B}'=(B_x,\gamma(B_y+\nicefrac{v}{c^2}E_z),\gamma(B_z-\nicefrac{v}{c^2}E_y))^\top$$

\subsection{Einstein und die Patente}
% andres komments und Folien

Einstein arbeitete um ... beim Patentamt in .... Zu dieser Zeit wurden häufig neue Patente zur Synchronisierung verschiedenen Uhren angemeldet, was zu Einsteins Inspiration und seinen späteren Entdeckungen führte.


% andrez 1


\subsection{Was ist ein Inertialsystem}

Bezugssystem in welchem das Trägheitsgesetz gilt.\\[10pt]
\textbf{Konsequenzen:}\\[5pt]
$\Rightarrow$ Inertialsystem bewegen sich gradlinig - gleichförmig gegeneinander

\subsubsection{Klassisches Relativitätsprinzip}
Grundgesetze der Physik nicht in allen Inertialsystemen gleich (Form invariant)

\subsection{Galilei-Transformation}
\begin{align*}
\vec r' &= \vec{r} - \vec{v} t \\
t' &= t
\end{align*}
$ \vec{v} $: Geschwindigkeit vom $ \vec{O}' $ gegenüber $ \vec{O} $\\[5pt]
Test:
\begin{equation*}
k: \quad \vec{F} = m \vec{a} \quad \Leftrightarrow \quad k': \quad \vec{F}' = m \vec{a}'
\end{equation*}
\begin{equation*}
\vec{a}' = \frac{d^2 \vec{v}'}{dt'^2} = \frac{d^2 \vec{r}}{dt^2} = \vec{a}
\end{equation*}
\textbf{Bemerkungen}
\begin{itemize}
	\item Invarianz unter Drehungen $ \Rightarrow $ Tensoren
\end{itemize}

\subsubsection{Transformation der Geschwindigkeit}
\begin{equation*}
\vec{v}' = \frac{d\vec{v}'}{dt'} = \frac{d\vec{v}'}{dt} \frac{dt}{dt'} = \frac{d\vec{v}'}{dt} = \frac{d\vec{v}}{dt} - \vec{u} = \vec{v} - \vec{u}
\end{equation*}
$ \Rightarrow $ Lichtgeschwindigkeit ist NICHT konstant !%$ \buildrel \mathcal{O} \over{\textbf{.}} $

\subsection{Lorenz-Transformation (LT)} % check if I.2.4

\begin{equation*}
\begin{pmatrix}
x' \\ y' \\ z' \\ t'
\end{pmatrix} = \frac{1}{\sqrt{1 - \frac{v^2}{c^2}}} \begin{pmatrix}
1&0&0&-u\\
0&1&0&0\\
0&0&1&0\\
-\frac{u}{c^2}&0&0&1
\end{pmatrix} \begin{pmatrix}
x\\y\\z\\t
\end{pmatrix} \qquad \tx{LT}
\end{equation*}


% andrez 2 punkt 4 und 5 n telegramm


Transformation der Geschwindigkeiten\\
NR:
\begin{align*}
\frac{dx'}{dt} &= \gamma \left(\frac{dx}{dt} - u\right) = \gamma (v_x - u)\\
\frac{dt'}{dt} &= \gamma \left(1 - \frac{u}{c^2} v_x\right)
\end{align*}
$ \Rightarrow $
\begin{align*}
v_x' &= \frac{dx'}{dt'} = \frac{dx'}{dt} \frac{dt}{dt'} = \frac{v_x - u}{1 - \frac{uv_x}{c^2}} \\
v_y' &= \frac{dy'}{dt'} = \frac{dy'}{dt} \frac{dt}{dt'} = \frac{1}{\gamma} \frac{v_y}{1 - \frac{uv_x}{c^2}}\\
v_z' &= \frac{dz'}{dt'} = \frac{dz'}{dt} \frac{dt}{dt'} = \frac{1}{\gamma} \frac{v_z}{1 - \frac{uv_x}{c^2}}
\end{align*}


% andrez 3



%\bibliographystyle{plain}
%\bibliography{literature}
%\addcontentsline{toc}{section}{Literatur}

\end{document}

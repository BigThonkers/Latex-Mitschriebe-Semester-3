\PassOptionsToPackage{dvipsnames}{xcolor}
\documentclass[titlepage,11pt,a4paper,ngerman]{report}
\usepackage[utf8]{inputenc}
\usepackage[T1]{fontenc}
\usepackage[german]{babel}
\usepackage{graphicx}
\usepackage{wrapfig}
\usepackage{amsmath}
\usepackage{amsfonts}
\usepackage{amssymb}
\usepackage{tikz}
\usepackage{tikz-cd}
\usepackage{nicefrac}
\usepackage{mathtools}
\usepackage{enumerate}
\usepackage{cancel}
\usepackage[hidelinks]{hyperref}
\usepackage{cleveref}
\usepackage{tcolorbox}

\usepackage[margin=1in]{geometry}

\usepackage{placeins}
\usepackage{booktabs}
\usepackage{wasysym}

\usepackage{url}

\usetikzlibrary{calc}
\usetikzlibrary{decorations.pathmorphing,patterns}
\usetikzlibrary{arrows}
\usetikzlibrary{decorations.pathreplacing}


%Environments und Newcommands:

% Allgemein:

% zu zeigen symbol
\newcommand{\zz}{\fontfamily{cmss} \selectfont{Z\kern-.61em\raise-0.7ex\hbox{Z}:}}
% build over
\newcommand{\bov}[2]{\buildrel{#2} \over{#1}}
% better looking := (defined as)
\newcommand*{\defeq}{\mathrel{\vcenter{\baselineskip0.5ex \lineskiplimit0pt \hbox{\scriptsize.}\hbox{\scriptsize.}}}=}
\newcommand*{\eqdef}{=\mathrel{\vcenter{\baselineskip0.5ex \lineskiplimit0pt \hbox{\scriptsize.}\hbox{\scriptsize.}}}}
% integral differential d
\newcommand{\dif}{\mathop{}\!\mathrm{d}}
\newcommand{\difi}[1]{\mathrm{d}#1\mathop{}\!}

\newcommand{\hfw}{\color{RubineRed}\tx{ $\star$hier fehlt was$\star$ } \color{black}}
\def\checkmark{\tikz\fill[scale=0.4](0,.35) -- (.25,0) -- (1,.7) -- (.25,.15) -- cycle;} 

%Text:
\newcommand{\tx}[1]{\textrm{#1}}
\newcommand{\const}{\tx{const.}}

\newcommand{\ul}[1]{\underline{#1}}
\newcommand{\ol}[1]{\overline{#1}}
\newcommand{\ub}[1]{\underbrace{#1}}
\newcommand{\ob}[1]{\overbrace{#1}}

\newcommand{\dd}{\tx{d}}

%Mathe:
\newcommand{\verteq}{\rotatebox{90}{$\,=$}}
\newcommand{\equalto}[2]{\underset{\scriptstyle\overset{\mkern4mu\verteq}{#2}}{#1}}
\newcommand{\equaltoup}[2]{\overset{\scriptstyle\underset{\mkern4mu\verteq}{#2}}{#1}}
\newcommand{\custo}[3]{\underset{\scriptstyle\overset{\mkern4mu\rotatebox{-90}{$\,#1$}}{#3}}{#2}}
\newcommand{\custoup}[3]{\overset{\scriptstyle\underset{\mkern4mu\rotatebox{-90}{$\hspace{-3pt} #1$}}{#3}}{#2}}
\newcommand{\casess}[4]{\left\{ \begin{array}{ll} {#1} & {#2} \\ {#3} & {#4} \end{array} \right.}


%Spezielles:

%Theo:
\newcommand{\lag}{\mathcal{L}}
\newcommand{\ham}{\mathcal{H}}
\newcommand{\gre}{\mathcal{G}}
\newcommand{\prt}[2]{\frac{\partial #1}{\partial #2}}
\newcommand{\prd}[2]{\frac{\tx{d} #1}{\tx{d} #2}}
\newcommand{\eofr}{\vec{E}(\vec{r})}
\newcommand{\pofr}{\Phi(\vec{r})}
\renewcommand{\Phi}{\varPhi}
\newcommand{\delfkt}{\delta(\vec{r} - \vec{r}_0)}
\newcommand{\grr}{\mathcal G(\vec{r},\vec{r}')}

%LA:
\newenvironment{bew}[1]{\subsection{Bew: #1}}{\hfill$\square$}
\newcommand{\Bew}[2]{\begin{bew}{#1}#2\end{bew}}

\newcommand{\enph}{F: V \to V \textrm{ Endomorphismus}}
\newcommand{\im}{\tx{im}}
\newcommand{\spa}{\tx{span}}
\newcommand{\adj}{\tx{adj}}
\newcommand{\grad}{\tx{grad}}
\newcommand{\ord}{\tx{ord}}

\newcommand{\basis}[3]{\{#1_{#2}, \dots, #1_{#3}\}}
\newcommand{\ska}[2]{\langle #1 , #2 \rangle}
\newcommand{\dmat}[3]{\begin{pmatrix} #1_{#2}&&\\ &\ddots& \\ && #1_{#3} \end{pmatrix}}

%Ex:
\newcommand{\kq}{\frac{1}{4\pi\epsilon_0}}
\newcommand{\uind}{U_{\tx{ind}}}
\newcommand{\folie}[1]{\color{gray}[Folie: #1]\color{black}}
\newcommand{\versuch}[1]{\color{red!50!black} Versuch: \color{black} #1\\ }

% ANDREZ
\newcommand{\summ}[2]{\sum_{#1}^{#2}}
\newcommand{\intt}[2]{\int_{#1}^{#2}}
\renewcommand{\vec}[1]{\boldsymbol{#1}}
\newcommand{\lcom}[1]{\color{MidnightBlue}#1\color{black}}
\renewcommand{\epsilon}{\varepsilon}
\newcommand{\vabla}{\boldsymbol{\nabla}}
\newcommand{\bei}{\emph{Beispiel: }}
\newcommand{\bem}{\emph{Bemerkung:}}
\renewcommand{\paragraph}[1]{\subsubsection{#1}}

\newcommand{\mau}{$\buildrel \mathcal{O} \over{\textbf{.}}$}

% Boxen:

\tcbuselibrary{theorems}

% mahlt eine box nur um den text mit tittel
\newtcbox{\fribox}[1]{nobeforeafter,colback=white,colframe=red!75!black,fonttitle=\bfseries,title=#1,sharp corners,tcbox raise base}

% mahlt eine große box um alles mit tittel
\newcommand{\frbox}[2]{\begin{tcolorbox}[colback=white,colframe=red!75!black,fonttitle=\bfseries,title=#1]#2\end{tcolorbox}}

% mahlt eine box nur um den text
\newtcbox{\ribox}{nobeforeafter,colback=white,colframe=red!75!black,sharp corners,tcbox raise base}

% mahlt eine große box um alles was drinnen ist
\newcommand{\rbox}[1]{\begin{tcolorbox}[colback=white,colframe=red!75!black]#1\end{tcolorbox}}

% mahlt eine box um mathe innerhalb mathmode
\newcommand{\rmbox}[1]{\tcboxmath[colback=white,colframe=red!75!black]{#1}}

% super box (looks like regular boxed but wraps around anything)
\newenvironment{supbox}{\begin{tcolorbox}[colback=white,colframe=black,sharp corners,boxrule=.5pt]}{\end{tcolorbox}}

\newcommand{\bbb}[2]{\begin{tcolorbox}[colback=white,colframe=black,fonttitle=\bfseries,title=#1,sharp corners,tcbox raise base]#2\end{tcolorbox}}

% array type box with title
\newenvironment{zebox}[1]{\begin{array}{|c|}
		\multicolumn{1}{l}{\tx{#1}} \\
		\hline
		\displaystyle
	}{\\ \hline
\end{array}}

% evtl: \renewcommand{\boxed}{\rmbox}

\def\centerarc[#1](#2)(#3:#4:#5)% Syntax: [draw options] (center) (initial angle:final angle:radius)
{ \draw[#1] ($(#2)+({#5*cos(#3)},{#5*sin(#3)})$) arc (#3:#4:#5); }

\tikzset{
	annotated cuboid/.pic={
		\tikzset{%
			every edge quotes/.append style={midway, auto},
			/cuboid/.cd,
			#1
		}
		\draw [every edge/.append style={pic actions, densely dashed, opacity=.5}, pic actions]
		(0,0,0) coordinate (o) -- ++(-\cubescale*\cubex,0,0) coordinate (a) -- ++(0,-\cubescale*\cubey,0) coordinate (b) edge coordinate [pos=1] (g) ++(0,0,-\cubescale*\cubez)  -- ++(\cubescale*\cubex,0,0) coordinate (c) -- cycle
		(o) -- ++(0,0,-\cubescale*\cubez) coordinate (d) -- ++(0,-\cubescale*\cubey,0) coordinate (e) edge (g) -- (c) -- cycle
		(o) -- (a) -- ++(0,0,-\cubescale*\cubez) coordinate (f) edge (g) -- (d) -- cycle;
		\path [every edge/.append style={pic actions, |-|}]
		%(b) +(0,-5pt) coordinate (b1) edge ["\cubex \cubeunits"'] (b1 -| c)
		%(b) +(-5pt,0) coordinate (b2) edge ["\cubey \cubeunits"] (b2 |- a)
		%(c) +(3.5pt,-3.5pt) coordinate (c2) edge ["\cubez \cubeunits"'] ([xshift=3.5pt,yshift=-3.5pt]e)
		;
	},
	/cuboid/.search also={/tikz},
	/cuboid/.cd,
	width/.store in=\cubex,
	height/.store in=\cubey,
	depth/.store in=\cubez,
	units/.store in=\cubeunits,
	scale/.store in=\cubescale,
	width=10,
	height=10,
	depth=10,
	units=cm,
	scale=.1,
}

\hbadness=99999

\begin{document}
	

\title{
	{\Huge Experimentalphysik III}\\[1em]
	{\Large Vorlesung von Prof. Dr. Oliver Waldmann im Wintersemester 2018}}
\author{Markus Österle \hspace{5pt} Andréz Gockel}
\date{ \today}%16.10.2018}
\maketitle
\tableofcontents

\setcounter{chapter}{-1}
\chapter{Einführung}

\begin{description}
	\item[Dozent] Prof. Oliver Waldmann, Zi. 202, Physik-Hochhaus
	\item[Zeit] Di, 8-10 Uhr, Mi, 8-10 Uhr
	\item[Ort] Großer Hörsaal Physik
 \end{description}

\section{Termine}
\begin{description}
	\item[Vorlesungsbeginn] Di., 16.10.2018
	\item[Erstes Übungsblatt] Mi., 17.10.2018
	\item[Übungsbeginn] Mo., 29.10.2018 - Fr., 2.11.2018
	\item[Letzte Vorlesung] Mi., 6.2.2019
	\item[Klausur] Sa., 9.2.2019, 9:00 - 12 Uhr
	\item[Wiederholungsklausur]	Sa., 27.4.2019., 10:00-13 Uhr
\end{description}

\section{Programm}
\begin{enumerate}
	\item Spezielle Relativitätstheorie
	\item Fortgeschrittene Optik
	\item Quantenphysik
	\item Einfache atomare Systeme
\end{enumerate}

\section{Literatur}
\begin{itemize}
	\item alle Lehrbücher
	\item Demtröder, Experimentalphysik 3 (Springer)
	\item Tipler/Mosca, Physik (Elsevier)
	\item Bergmann/Schaefer, Lehrbuch der Experimentalphysik, Band 3
	\item Gerthsen, Physik (Springer)
	\item Giancoli, Physik (Pearson)
\end{itemize}

\section{Übungen und Betreuer}
 
Übungsleiter: Krunoslav Prša, Zi. 203, Physik-Hochhaus, Tel. 7631\\
Gruppe 3: Mi 10-12, SR III, Tutor: Fabian Thielemann\\
Gruppe 7: Fr 14-16, SR GMH, Tutor: Rupert Michiels
 
\section{Übungsblätter}
\begin{itemize}
	\item Ausgabe der Übungsblätter jeweils am ENDE der Vorlesung am Mittwoch im großen Hörsaal, oder online auf ILIAS (siehe Ordner "Übungsblätter" unten)
	\item Rückgabe der schriftlichen Lösungen VOR der Vorlesung am Mittwoch, die Lösungen sind im Hörsaal vorne auf den Tisch zu legen
	\item jedes Übungsblatt umfasst typischerweise 4 - 5 Aufgaben mit insgesamt ca. 35 Punkten
	\item jedes Übungsblatt enthält typischerweise eine leichte und eine schwere Aufgabe
\end{itemize}

\section{Übungsgruppen}
\begin{itemize}
	\item bis zu zwei (2) Studierende können ein Lösungsblatt gemeinsam bearbeiten und abgeben (sie müssen dann aber in einer Übungsgruppe sein)
	\item bitte DEUTLICH auf dem Lösungsblatt angeben: Namen der Studierenden, Übungsgruppe (Nr, Name des Übungsleiters, Ort, Zeit)
	\item Gruppenarbeit ist erwünscht, es wird aber erwartet, dass jede(r) Studierende die Aufgaben vorrechnen kann, sonst Punktabzug :-)
	\item Einschreibung in die Übungsgruppen über ILIAS von Di., 17.10, 20:00 Uhr bis Do., 19.10, 20:00 Uhr (siehe unten)
	\item Anwesenheitspflicht \& Vorrechenpflicht!
\end{itemize}

\section{Klausur}
\begin{description}
	\item[Zuslassung zur Klausur] 40\% der Übungspunkte
	\item[Termin] Sa, 9. Februar 2019, 9:00 - 12 Uhr, Großer Hörsaal Physik
	\item[Dauer] 120 Minuten
	\item[Inhalt] Vorlesungsstoff + Übungen, Aufgaben sind ähnlich zu den Übungen gestellt
	\item[Sprache] Die Klausuraufgaben werden in Deutsch zur Verfügung gestellt
	\item[Erlaubt] Din-A4 Blatt mit eigenen Notizen beidseitig beliebigbeschrieben/bedruckt (spezielle Lesehilfen sind nicht erlaubt!), Stifte, Geodreieck/Lineal
	\item[Nicht erlaubt] Es ist alles verboten was nicht erlaubt ist (Taschenrechner, Handy, Bücher, usw.) (leere geheftete Blätter werden verteilt)
	\item[Mitzubringen] Studierendenausweis, Schreibstifte, Geodreieck/Lineal
\end{description}
 
\subsubsection{Anmeldung zur Klausur}
Online Anmeldung, der Ablauf und Termine werden durch die jeweiligen Prüfungsämter bekannt gegeben.
 
\section{Prüfungsleistung}
\begin{itemize}
	\item mindestens 40\% der Übungspunkte werden für die Zulassung zur Klausur benötigt
	\item bestandene schriftliche Klausur oder Nachklausur
	\item Bewertungsgrundlage ist das Ergebnis der Klausur oder Wiederholungsklausur
\end{itemize}

Täuschungsversuche:
Sowohl die Übungen wie die Klausur sind Teil der Prüfungs/Studienleistung! Täuschungsversuche können zum Nicht-Bestehen führen!


%16.10.18
\renewcommand{\thechapter}{\Roman{chapter}}
\chapter{Spezielle Relativitätstheorie}
%\addcontentsline{toc}{section}{section die nur im Inhaltsverzeichniss ist}

\section{Vorgeschichte}

\textbf{1861-1867: Maxwell Gleichungen}\\
Vor Maxwell brauchten alle bekannten Wellen (Wasserwellen, Schall) ein Medium oder Trägerstoff. Daher stammte die Annahme, auch elektro-magnetische Wellen also Licht bräuchte ein Medium: der Äther. Somit wollte man experimentell zu zeigen, wie schnell wir uns durch den Äther bewegen und welches das Inertialsystem, das ausgezeichnete Bezugssystem des Äthers ist.\\
%Man stellte jedoch fest, dass es kein solches Inertialsystem gibt und Geschwindigkeiten nur relativ zueinander existieren und nicht absolut sind.
Mit dem Ziel den Äther nachzuweisen wurde das Michelson-Morley Experiment durchgeführt.

\subsection{Experiment von Michelson-Morley}%I.1.a

Die Idee des Experimentes war es die Bewegung der Erde relativ zum Äther zu messen.\\
\textbf{Anforderungen:}
\begin{itemize}
	\item Erde $ 30 \, \tx{km} \slash \tx{s} $
	\item Licht $ 3 \cdot 10^{5} \, \tx{km} \slash \tx{s}  $
	\item[$ \Rightarrow $] relative Auflösung von circa $ 10^{-4} $ nötig
\end{itemize}
%
%
% insert image von Folie interferrometer
%
%
\textbf{Prinzip}:\\
Lichtlaufzeiten:\\
Arm parallel zum Ätherwind
\begin{equation*}
t_2 = \frac{L_2}{c+v} + \frac{L_2}{c-v} = \frac{2cL_2}{(c^2-v^2)} = \frac{2 L_2}{c} \frac{1}{\sqrt{1 - \frac{v^2}{c^2}}^2}
\end{equation*}
Arm senkrecht zum Ätherwind
\begin{equation*}
t_1 = 2 \frac{L_1'}{c}
\end{equation*}
Nebenrechnung:
\begin{equation*}
L'^2 = L^2 + (vt)^2 \quad \Rightarrow \quad (ct)^2 = L^2 + (vt)^2 \quad \Rightarrow \quad t = \frac{L}{\sqrt{c^2 - v^2}}
\end{equation*}
\begin{equation*}
t_1 = \frac{2 L}{\sqrt{c^2 - v^2}}
\end{equation*}
Somit ist die Zeitdifferenz der beiden Lichtstrahlen
\begin{equation*}
\Delta t = t_2 - t_1 = \frac{2L}{c} \ \frac{1 - \sqrt{1 - \frac{v^2}{c^2}}}{1 - \frac{v^2}{c^2}} \approx \frac{2L}{c} \left(\frac{1}{2} \frac{v^2}{c^2}\right) = \frac{lV^2}{c^2}
\end{equation*}
Hieraus können wir nun den Phasenunterschied der beiden Strahlen berechnen
\begin{equation*}
\Delta \varphi = 2 \pi \frac{c \Delta t}{\lambda}
\end{equation*}
Abschätzung:
\begin{equation*}
L = 2 \times 11 \, \tx{m} \ , \ v = 30 \, \tx{km} \slash \tx{s} \ , \ \lambda = 500 \, \tx{nm} \ \Rightarrow \ \Delta \varphi \slash \pi \approx 0{,}88
\end{equation*}
\textbf{Auflösung}
\begin{equation*}
\frac{\Delta \varphi}{\pi} \approx \frac{1}{4}
\end{equation*}
\textbf{Ergebnis:}\\
Es gibt keine Verschiebung des Interferenzmusters.\\[5pt]
\textbf{Konsequenz:}\\
$ \Rightarrow $ Es gibt keinen Ätherwind.\\
$ \Rightarrow $ Es gibt kein ausgezeichnetes Bezugssystem. Alle Bezugssysteme sind gleichwertig.\\[5pt]
\textbf{Aber:}\\
Kontraktionshypothese von Fitzgerald z Lorenz\\
Wenn der Arm in Richtung der Äthers um Faktor $ \gamma = \frac{1}{\sqrt{1 - \frac{v^2}{c^2}}} $ \\
Arm parallel zum Ä.W.:
\begin{equation*}
t_2 = \frac{2cL_2}{c^2 - v^2} \frac{1}{\gamma} = \frac{2 L_2}{\sqrt{denc^2 - v^2}}
\end{equation*}
Arm senkrecht zum Ä.W.:
\begin{equation*}
t_1 = \frac{2L_1}{\sqrt{c^2 - v^2}}
\end{equation*}
\textbf{Aber:}\\
Äther nicht beobachtbar

\subsection{Lorenz-Invarianz der Maxwell-Gleichungen}

Maxwell-Gleichungen sind Lorenz-invariant und nicht Galilei-invariant.

Beispiel: Relativität der Feder. 




% tikz felder Bilder abzeichnen Jan



$\Rightarrow$ Im Laborsystem: $\vec{F}=q(\vec{v}\times\vec{B})$

$\Rightarrow$ Im Ruhesystem: $v'=0\Rightarrow F_L=0$ \begin{LARGE}\lightning\end{LARGE}

$\Rightarrow$ ein zusätzliches $E'$-Feld, $\vec{F}_q=q\vec{E}$

$\Rightarrow$ neue „Transformationsgleichungen“

\textbf{Transformation}
\begin{itemize}
	\item $K$: „ruhende“ Bezugssystem
	\item $K'$: „sich in Bezug auf $K$ konstant entlang der $x$-Achse bewegendes“ Bezugssystem
\end{itemize}

$$\vec{E}'=(E_x,\gamma(E_y-vB_z),\gamma(E_z-vB_y))^\top$$
$$\vec{B}'=(B_x,\gamma(B_y+\nicefrac{v}{c^2}E_z),\gamma(B_z-\nicefrac{v}{c^2}E_y))^\top$$

\subsection{Einstein und die Patente}



% andres komments und Folien



Einstein arbeitete um ... beim Patentamt in .... Zu dieser Zeit wurden häufig neue Patente zur Synchronisierung verschiedenen Uhren angemeldet, was zu Einsteins Inspiration und seinen späteren Entdeckungen führte.

\section{Bezugsysteme und Inertialsysteme}
Zur Diskussion der Relativitätstheorie ist ein bestimmtes Vokabular mit klaren Definitionen Notwendig. Hierzu soll dieser Abschnitt dienen.

\subsection{Was ist ein Bezugssystem (BZS)}

Ein Bezugssystem ist ein Koordinatensystem bezüglich dessen man die Bewegung von Objekten beschreibt.\\
\textbf{Konsequenzen:}
\begin{itemize}
	\item es gibt einen Koordinatenursprung $ \vec{O} $
	\item Koordinatenursprünge verschiedener BZS können sich gegeneinander bewegen
\end{itemize}
$ \Rightarrow $ Transformationsgesetze\\[5pt]
\underline{ACHTUNG}\\
Unterscheide sorgfältig zwischen Basisvektoren, Vektoren und Koordinaten
\begin{equation*}
\vec{v} = \vec{v}' \quad \begin{array}{c}
\vec{v} = \sum_i x_i \vec{e}_i \\ \vec{v}' = \sum_i x_i' \vec{e}_i'
\end{array} \ \ \rightarrow \ \ \vec{x} = \begin{pmatrix}
x_1 \\ x_2 \\ x_3
\end{pmatrix} \neq \vec{x}'
\end{equation*}

\subsection{Was ist ein Inertialsystem (IS)}
Bezugssystem in welchem das Trägheitsgesetz gilt.\\[10pt]
\textbf{Konsequenzen:}\\[5pt]
$\Rightarrow$ Inertialsystem bewegen sich gradlinig - gleichförmig gegeneinander

\subsubsection{Klassisches Relativitätsprinzip}
Grundgesetze der Physik nicht in allen Inertialsystemen gleich (Form invariant)

\subsection{Galilei-Transformation}
\frbox{Galilei-Trafo}{\begin{align*}
\vec r' &= \vec{r} - \vec{v} t \\
t' &= t
\end{align*}}
$ \vec{v} $: Geschwindigkeit vom $ \vec{O}' $ gegenüber $ \vec{O} $\\[5pt]
\textbf{Test:}
\begin{equation*}
k: \quad \vec{F} = m \vec{a} \quad \Leftrightarrow \quad k': \quad \vec{F}' = m \vec{a}'
\end{equation*}
\begin{equation*}
\vec{a}' = \frac{d^2 \vec{v}'}{dt'^2} = \frac{d^2 \vec{r}}{dt^2} = \vec{a}
\end{equation*}
\emph{Bemerkung}\\
Invarianz unter Drehungen $ \Rightarrow $ Tensoren

\subsubsection{Transformation der Geschwindigkeit}
\begin{equation*}
\vec{v}' = \frac{d\vec{v}'}{dt'} = \frac{d\vec{v}'}{dt} \frac{dt}{dt'} = \frac{d\vec{v}'}{dt} = \frac{d\vec{v}}{dt} - \vec{u} = \vec{v} - \vec{u}
\end{equation*}
$ \Rightarrow $ Lichtgeschwindigkeit ist NICHT konstant ! %$ \buildrel \mathcal{O} \over{\textbf{.}} $

\subsection{Lorenz-Transformation (LT)}
\frbox{Lorenz-Trafo}{\begin{equation*}
\begin{pmatrix}
x' \\ y' \\ z' \\ t'
\end{pmatrix} = \frac{1}{\sqrt{1 - \frac{v^2}{c^2}}} \begin{pmatrix}
1&0&0&-u\\
0&1&0&0\\
0&0&1&0\\
-\frac{u}{c^2}&0&0&1
\end{pmatrix} \begin{pmatrix}
x\\y\\z\\t
\end{pmatrix} \qquad \tx{LT}
\end{equation*}}
Ziel der speziellen Relativitätstheorie:\\
Die Gesetze der Mechanik sollen Lorenz-invariant sein !\\[5pt]
\emph{Bemerkung}\\
Die Galilei-Transformation ergibt sich aus der Lorenz-Transformation für $ \frac{u}{v} \rightarrow 0 $ somit wird dann $ \gamma \rightarrow 1 $ gehen.

\section{Einsteins Axiome der speziellen Relativitätstheorie (SRT)}
\begin{enumerate}[1.]
	\item Relativitätsprinzip:\\
	Alle Naturgesetze nehmen in allen IS die gleiche Form an.
	\item Die Lichtgeschwindigkeit im Vakuum ist Konstant.
\end{enumerate}
\emph{Bemerkungen}
\begin{itemize}
	\item Punkt 1 legt die Lorenz-Trafo fest
	\item Das Umgekehrte gilt nicht
\end{itemize}

\section{Konsequanzen aus der Konstanz der Lichtgeschwindigkeit}

Da durch, dass die Lichtgeschwindigkeit eine Konstante sein soll ergeben sich schon bei einfachen Beispielen Schwierigkeiten. Ein solches Beispiel wäre die Addition von Geschwindigkeiten. Fährt man nun in einem Zug der die Geschwindigkeit $ 0{,}8 c $ hat und schießt ein Geschoss mit $ 0{,}3 c $ in Fahrtrichtung ab so sollte dieses mit $ 1{,}1 c $ schneller als die Lichtgeschwindigkeit sein. Warum dies nicht der Fall ist und wie man damit umgeht und welche weiteren Konsequenzen aus der Konstanz der Lichtgeschwindigkeit folgen wird im folgenden Abschnitt erläutert.

\subsection{Lorenz-Trafo und Lorenz-Invarianz}

\begin{enumerate}[1)]
	\item Lichtstrahlen:\\
	Im IS$ \phantom' $ $ \ \ K: \quad x = ct $\\
	Im IS' $ \ \ K: \quad x = ct' $
	\item Alle IS sind gleichberechtigt\\
	Trafo $ K\phantom' \rightarrow K': \quad x' = \gamma (x\phantom' - ut) $\\
	Trafo $ K' \rightarrow K\phantom': \quad x\phantom' = \gamma (x' - ut) $
	\item $ \Rightarrow xx' = \gamma^2 xx' \left(1 - \frac{v^2}{c^2}\right) $\\
	$ \Rightarrow \rmbox{\gamma = \frac{1}{\sqrt{1 - \frac{v^2}{c^2}}}} $
	
	% xi ?
	
	\item
	$t' = \gamma (t - \xi x)$\\
	$t' = \frac{x'}{c} \gamma \frac{1}{c} (x - ut) = \gamma (t - \frac{ux}{c^2})$
	\item Insgesamt:
	\frbox{Lorenz-Trafo}{\begin{equation*}
	x' = \gamma (x - ut) \qquad \quad y' = y \qquad \quad z' = z \qquad \quad t' = \gamma \left(t - \frac{ux}{c^2}\right)
	\end{equation*}}
\end{enumerate}
Transformation der Geschwindigkeiten\\
NR:
\begin{align*}
\frac{dx'}{dt} &= \gamma \left(\frac{dx}{dt} - u\right) = \gamma (v_x - u)\\
\frac{dt'}{dt} &= \gamma \left(1 - \frac{u}{c^2} v_x\right)
\end{align*}
Daraus folgt dann für die Geschwindigkeitsadditionen in verschiedene räumliche Komponenten:
\begin{align*}
v_x' &= \frac{dx'}{dt'} = \frac{dx'}{dt} \frac{dt}{dt'} = \frac{v_x - u}{1 - \frac{uv_x}{c^2}} \\
v_y' &= \frac{dy'}{dt'} = \frac{dy'}{dt} \frac{dt}{dt'} = \frac{1}{\gamma} \frac{v_y}{1 - \frac{uv_x}{c^2}}\\
v_z' &= \frac{dz'}{dt'} = \frac{dz'}{dt} \frac{dt}{dt'} = \frac{1}{\gamma} \frac{v_z}{1 - \frac{uv_x}{c^2}}
\end{align*}
\textbf{Test:}\\
\begin{itemize}
	\item $ u \ll c \Rightarrow $ Galilei-Trafo
	\item Umkehrung
\end{itemize}

\subsubsection{Lorenz-Invarianz}
Feststellung $ x^2 - c^2t^2 = \const = x'^2 - c^2 t'^2 $\\[5pt]
Raum-Zeit-Abstand: $ L = \sqrt{x^2 - c^2 t^2} $\\
$ \rightarrow $ RZ-Abstand bleibt unter Lorenz-Trafo invariant.

\subsubsection{Minkiwski Raum}
,,Drehung`` im M-Raum $ \Leftrightarrow $ Lorenz-Trafo
\begin{enumerate}[A)]
	\item \begin{align*}
	\vec{x} &= (x,y,z,ict) \\
	\vec{x} \vec{x} &= x^2 + y^2 + z^2 - c^2 t^2 = l^2
	\end{align*}
	\item \begin{equation*}
	\vec{x} = (x,y,z,ct) \quad  \tx{Metrik:} \quad  \overline{g} = \begin{scriptsize}\begin{pmatrix}
	1 & & & 0 \\ & 1\\ & & 1 \\ 0 & & & -1
	\end{pmatrix}\end{scriptsize} \end{equation*}
	\begin{equation*}
	\Rightarrow l^2 = \vec{x} \overline{g} \vec{x} = x^2 + y^2 + z^2 - c^2 t^2
	\end{equation*}
\end{enumerate}
$ \Rightarrow $ Kovarianten und Kontravarianten Vierervektoren

\subsubsection{Elementare Diskussion des M-Raums}
\emph{Frage:} Wie sieht ein anderes sich bewegendes IS (K') im eigenen IS (K) aus ?\\[5pt]
K: Die Achsen $ ct, x $ stehen rechtwinklig aufeinander.\\
K': wie liegen $ ct', x' $ in K ?\\[5pt]
\textbf{Für die Lage von $ ct' $:}
Betrachte $ x' = 0 \rightarrow \tx{ LT } \Rightarrow x = \frac{u}{c} ct $\\
\textbf{Für die Lage von $ x' $:}
Betrachte $ ct' = 0 \rightarrow \tx{ LT } \Rightarrow ct = \frac{u}{c}x $

\subsection{Relativität der Gleichzeitigkeit}
Lichtlaufzeit und Ereignisse:\\
Das beobachtete Licht (z.B. von Galaxien) entspricht einem Blick in die Vergangenheit.\\
Was soll gleichzeitig bedeuten?

\subsubsection{Definition der GLeichzeitigkeit}
In einem IS sind die Ereignisse A und C gleichzeitig, wenn die von den beiden Ergebnissen ausgehenden Lichtpulse einen Beobachter B in der Mitte der beiden Punkte zur selben Zeit erreichen.

\subsubsection{Synchronisation von Uhren}
\begin{itemize}
	\item man nehme zwei Uhren un einem IS, ruhend
	\item man sende zwei Photonen von der Mitte der Verbindungsstrecke aus
	\item[$ \Rightarrow $] die beiden Photonen kommen gleichzeitig an den beiden Orten an
\end{itemize}

\subsection{Zeitdilatation}
\underline{Uhren} $ \Rightarrow $ Lichtuhr\\
\textbf{Zeitdilatation: bewegte Uhren laufen langsamer}\\
\underline{Grund:} Licht muss größere Wege zurücklegen.\\[5pt]
im Eigensystem: $ \Delta t' = 2 \frac{L}{c} $\\
in ,,unserem`` System: $ \Delta t = 2 \frac{L}{c} $
\begin{equation*}
\overline{L}^2 = L^2 + \left(\frac{u \Delta t}{2}\right)^2
\end{equation*}
\begin{equation*}
\Delta t^2 = \Delta t'^2 + \frac{u^2}{c^2} \Delta t^2
\end{equation*}
\begin{equation*}
\Rightarrow \Delta t = \frac{1}{\sqrt{1 - \frac{u^2}{c^2}}} \Delta t' = \gamma \Delta t' \qquad ; \Delta t > \Delta t'
\end{equation*}

\subsubsection{Betrachtung im Minkowski Diagramm}
Betrachtung mit Lorenz-Trafo
\begin{align*}
x_0' &= 0 \qquad  \qquad \qquad  \Rightarrow  \qquad \qquad  x = ut \\
t'\phantom{_0} &= \gamma \left(t - \frac{u}{c^2}  u t\right) = \gamma \left(1 - \frac{u^2}{c^2}\right) t = \frac{1}{\gamma} t
\end{align*}
\frbox{Zeitdilatation}{\begin{equation*}
	t' = \frac{1}{\gamma} t
	\end{equation*}}
\textbf{Eigenzeit}\\
$ \tau $: Tickdauer im Ruhesystem der Uhr
\emph{Beispiele:}
\begin{itemize}
	\item Myonenzerfall - Lebensdauer $ \tau = 2 \cdot 10^{-6} \, \tx{s} $\\
	Entstehung in Erdatmosphäre:\\
	gemessene Lebensdauer $ \tau' = 30 \cdot 10^{-6} \, \tx{s} $
	\item Nebelkammer
\end{itemize}

\textbf{Zwillingsparadoxon}
Lösung: \lcom{Erdzwilling hat recht.}

\textbf{Modernes Experiment}



% glaube hier fehlt was



\subsection{Längenkontraktion}
Die Längenkontraktion besagt, dass bewegte Gegenstände in Bewegungsrichtung kürzer erscheinen.
\begin{equation*}
l = \sqrt{1 - \frac{u^2}{c^2}} l' = \frac{1}{\gamma} l'
\end{equation*}
\frbox{Längenkontraktion}{\begin{equation*}
	l' = \gamma l
	\end{equation*}}
Minkowski-Diagramm:\\[5pt]
Benutze Lorenz-Invarianz\\
Lorenz-Trafo:\\
$ t = 0 $
\begin{align*}
\tx{LT für Ort:} \quad x' &= \gamma(x\phantom{_2} - ut) = \gamma x\\
\Rightarrow l' = x_2' - x_1' &= \gamma (x_2 - x_1) = \gamma l
\end{align*}



%Stimmt das? :



\textbf{Ruder-Filme}\\
Lichtlaufzeit berücksichtigen

\subsection{Doppler Effekt}
\begin{equation*}
f_B = f_S \frac{c + v_B}{c - v_S}
\end{equation*}
\begin{itemize}
	\item bewegter Empfänger: $ v_S = 0 \Rightarrow f_B = \left(1 + \frac{u_B}{c}\right) \cdot f_S $
	\item bewegter Sender: $ v_B = 0 \Rightarrow f_B = f_S \frac{1}{1 - \frac{v_S}{c}} \approx f_S \left(1 + \frac{v_S}{c} + \frac{v_S^2}{c^2} + \dots \right) $
\end{itemize}

\subsubsection{Relativistischer Dopplereffekt}
\begin{equation*}
f_B = f_S \frac{\sqrt{1 - \frac{v^2}{c^2}}}{1 - \cos \alpha \frac{u}{c}}
\end{equation*}
\begin{itemize}
	\item $ \cos \alpha = \pm 1 $: longitudinaler DE
	\item $ \cos \alpha = 0 $: transversaler DE
\end{itemize}

\subsection{Abberation des Lichts}

\folie{Folien zur Abberation des Lichts}
% folien

\section{Relativistische Dynamik}
Newton: \begin{equation*}
\vec{F} = m \vec{a}
\end{equation*}
Erhaltungssätze (Energie, Impuls, Drehimpuls)\\[5pt]
Energieerhaltung: Homogenität der Zeit\\
Impulserhaltung: Homogenität des Raums\\[5pt]
$ \Rightarrow $ $ E $ und $ p $ Erhaltung auch in SRT

\subsection{Impulserhaltung und relativistischer Impuls}
\textbf{Gedankenexperiment}\\[5pt]
\begin{tikzpicture}
\node at (.15,.7) (a) {};
\node[anchor=east,xshift=-5pt] at (a) {\tx{Vorher: } $ \quad $};
\node at (0,0) (b) {};
\node at ($(b) + (.5,0)$) (b2) {};
\node[anchor=east] at (b) {\tx{Nacher: } $ \quad $};
\node[circle,fill=gray,inner sep=2pt, minimum size=2pt] at (a) {};
\node[circle,fill=gray,inner sep=2pt, minimum size=2pt] at (.35,.7) {};
\node[circle,fill=gray,inner sep=2pt, minimum size = 2pt] at (b) {};
\node[circle,fill=gray,inner sep=2pt, minimum size = 2pt] at (.5,0){};
\draw[->] (b) -- (-.5,0) node[above,xshift=5pt] {$ v_1 $};
\draw[->] (b2) -- ($(b) + (1,0)$) node[above,xshift=-5pt] {$ v_2 $};
\node[below,yshift=-3pt] at (b) {\footnotesize{$ A $}};
\end{tikzpicture}\\
Im Laborsystem gilt $ \vec{p}_{\tx{vorher}} = \vec{p}_{\tx{nacher}} $:
\begin{equation*}
\Rightarrow 0 = m_1 v_1 + m_2 v_2 = m_0 (v-v) = 0
\end{equation*}
Im Ruhesystem von Körper $ A $ gilt dann:
\begin{equation*}
u = - v_1 \quad , \quad v_1' = 0
\end{equation*}
\begin{equation*}
\Rightarrow \quad (m_1 + m_2) v_1 = m_2 v_2'
\end{equation*}
Klassisch gilt somit:
\begin{equation*}
m_1 = m_2 = m_0
\end{equation*}
\begin{equation*}
v_2' = v_2 + v = 2v
\end{equation*}
In der \textbf{SRT} gelten aber folgende Additionstheoreme für Geschwindigkeiten:
\begin{equation*}
v_2' = \frac{v_2 - u}{1 - v_2 \frac{u}{c^2}} = \frac{2 v}{1 + \frac{v^2}{c^2}}
\end{equation*}
Dies liefert jedoch einen Widerspruch mit unserem zuvorigen Ergebnis.
\begin{equation*}
\Rightarrow v_2' < 2v \quad \tx{\lightning}
\end{equation*}

\subsubsection{Relativistischer Impuls}
\begin{equation*}
\vec{p} = m(v) \cdot \vec{v} = \rmbox{\gamma(v) m_0 \vec{v}}
\end{equation*}
\begin{equation*}
\gamma(v) = \frac{1}{\sqrt{1 - \frac{v^2}{c^2}}}
\end{equation*}
$ \vec{\omega} $ ist die Geschwindigkeit der Systeme zueinander.\\
$ \vec{v} $ Die Geschwindigkeit des Körpers.

\subsection{Energieerhaltung und relativistische Energie}
\begin{minipage}{.5\linewidth}
	\textbf{Gedankenexperiment:}\\[5pt]
	Beschleunigung eines Körpers in einem konstanten Kraftfeld.
\end{minipage}
\begin{minipage}{.5\linewidth}
	\hspace{50pt}
	\begin{tikzpicture}
	\node at (0,0) (a) {};
	\node[circle, fill = gray ,inner sep = 3pt, minimum size = 2pt] at (a) {};
	\draw[thick,->] (a) -- ($(a) + (1,0)$) node[above,xshift=-15pt] {$ \vec{F} $};
	\draw[thick,->] ($(a) + (0,-.5)$) -- ($(a) + (1,-.5)$) node[below,xshift=-15pt] {$ \vec{v} $};
	\end{tikzpicture}
\end{minipage}
$ \Rightarrow $ Körper gewinnt kinetische Energie.
\begin{equation*}
E_{\tx{kin}} = \int \vec{F} \cdot d\vec{s}
\end{equation*}
\begin{align*}
dE_{\tx{kin}} &= F ds \buildrel ? \over = \frac{dp}{dt} ds\\
&= m_0 d(\gamma v) \cdot v = m_0 v (v d \gamma + \gamma dv)
\end{align*}
Trick:
$ 1 = \gamma^2 \left( 1 - \frac{v^2}{c^2} \right) $\\
$ \Rightarrow 0 = c \gamma d \gamma \left(1 - \frac{v^2}{c^2}\right) + \gamma ^2 \left(- \frac{2v}{c^2} dv \right) $\\
$ \Rightarrow c^2 d \gamma = v^2 d\gamma + \gamma v dv $\\
$ \Rightarrow d E_{\tx{kin}} = m_0 c^2 d \gamma $
\begin{align*}
E_{\tx{kin}} &= m_0 c^2 \left[\gamma(v) - \gamma(0)\right]\\
&= \gamma m_0 c^2 - m_0 c^2
\end{align*}
Einstein: $ \qquad E = \gamma m_0 c^2 \ \widehat{ = }\ \tx{,,Gesamtenergie``} $
\begin{equation*}
\rmbox{E = E_{\tx{kin}} + m_0 c^2}
\end{equation*}

\subsection{Relativistische Energie-Impuls-Beziehung und Viererimpuls}
Dispersionsrelation: $ E(\vec{p}) $\\
Klassisch: $ E(\vec{p}) = \frac{p^2}{2m} $\\
SRT: $ E = \gamma m_0 c^2 $\\
$ \Rightarrow E^2 = \gamma ^2 (m_0 c^2)^2 $ mit $ \gamma ^2 = 1 + \gamma^2 \left(\frac{v}{c}\right)^2 $\\
$ \Rightarrow m_0 c^2 \gamma ^2 = m_0^2 c^2 + \gamma^2 m_0^2 v^2 = m_0^2 c^2 + \vec{p}^2 $
\begin{equation*}
\rmbox{\Rightarrow E^2 = (m_0 c^2)^2 + c^2 \vec{p}^2}
\end{equation*}
\begin{equation*}
\rmbox{E = \sqrt{(m_0 c^2)^2 + c^2 \vec{p^2}}}
\end{equation*}
\begin{equation*}
\Rightarrow \left(\frac{E}{c}\right)^2 - \vec{p}^2 = (m_0 c)^2
\end{equation*}

\subsubsection{Viererimpuls}
\begin{equation*}
\hat{\vec{p}} = \left(\frac{E}{c} , p_x, p_y, p_z\right)
\end{equation*}
\begin{equation*}
\Rightarrow \left(\frac{E}{c}\right)^2 - \vec{p}^2 \qquad \tx{ist Lorenz-invariant !}
\end{equation*}
$ \Rightarrow $ Die Ruhemasse $ m_0 $ ist Lorenz-invariant.

\subsubsection{Was ist Masse?}
Masse charakterisiert die Energie-Impuls-Beziehung relativistischer Teilchen $ \widehat{=} $ masseloser Teilchen $ \widehat{=}\ v = c \Leftrightarrow m_0 = 0 $.
\begin{equation*}
\Rightarrow R(\vec{p}) = c |\vec{p}|
\end{equation*}
Massebehaftete Teilchen, $ m_0 > 0 \ , \ v < c$
\begin{equation*}
E(\vec{p}) = \sqrt{(m_0 c^2)^2 + c^2 \vec{p}^2} \buildrel v \to 0 \over \longrightarrow \frac{\vec{p}^2}{2 m_0}
\end{equation*}


% 30.10.18 

\subsection{Kraft und Energie-Impuls Erhaltung}
klassisch gilt:
\begin{equation*}
\vec{F} = \frac{d\vec{p}}{dt}
\end{equation*}
und Impulserhaltung:
\begin{equation*}
\vec{F} = 0 \quad \Leftrightarrow \quad \vec{p} = \const \quad , \quad E = \const
\end{equation*}
In der SRT gilt:
\begin{equation*}
\vec{\hat F} = \gamma (m_0 c \frac{d\gamma}{dt}, F_x, F_y, F_z)
\end{equation*}
\begin{equation*}
\rmbox{\vec{\hat F} = \frac{d}{dt} \vec{\hat{p}}}
\end{equation*}
\begin{equation*}
d \tau = \frac{1}{\gamma} dt
\end{equation*}
\underline{Relativistische Energie-Impuls Erhaltung}
\begin{equation*}
\vec{\hat{F}} = 0 \quad \Leftrightarrow \quad \rmbox{\vec{\hat{p}} = \const}
\end{equation*}
\begin{itemize}
	\item Zeitanteil $ \widehat{=} $ Energieerhaltung: $ E_{\tx{ges}} = E(\vec{p}) + W_{\tx{pot}} \dots = \const $
	\item Raumanteil $ \widehat{=} $ Impulserhaltung: $ \vec{p}_{\tx{nacher}} = \vec{p}_{\tx{vorher}} $
\end{itemize}

\subsection{Anwendungsbeispiele}
\subsubsection{Äquivalenz von Masse und Energie}
\lcom{$m_0 c^2$ ist eine Energie, also muss sie, rein physikalisch gesehen, frei umwandelbar sein.}\\

\lcom{Bindungsenergie: ein mehr Teilchensystem hat mehr Energie als die einzelnen Teilchen, diese zusätzliche Energie ist die Bindungsenergie und sie äußert sich als Massenänderung.}

\begin{equation*}
E = m_0 c^2 
\end{equation*}
\begin{equation*}
1 kg \ \widehat{=} \ m c^2 = 10^{17} \, \tx{J}
\end{equation*}
Weltenergieverbrauch ca.: $ 4 \cdot 10^{20} \, \tx{J} \ \widehat{=} \ 4 \, \tx{T} / \, \tx{Jahr} $
\subsubsection{Bindungsenergie $ \vec{\widehat{=}} $ Massenänderung}
\folie{Fusion/Kernspaltung}
\begin{equation*}
p: \qquad m_p = 938{,}27 \, \frac{\tx{MeV}}{c^2}
\end{equation*}
\begin{equation*}
n: \qquad m_n = 939{,}54 \, \frac{\tx{MeV}}{c^2}
\end{equation*}
\begin{equation*}
^4He: \qquad m_{He} = 3727 \, \frac{\tx{MeV}}{c^2} \ \ \ \;
\end{equation*}
\begin{equation*}
2n + 2p = 3754 \, \frac{\tx{MeV}}{c^2}
\end{equation*}
\begin{equation*}
\Rightarrow \tx{ Bindungsenergie } \quad 24 \, \frac{\tx{MeV}}{c^2}
\end{equation*}
\folie{Bindungsenergie von Atomen} Kernspaltung/Kernfusion\\[5pt]
\emph{Beispiel:} Compton Effekt\\
Streuung eines Photons an einem (quasi-) freien, ruhenden Elektron
\begin{equation*}
\Delta R = \lambda_c (1 - \cos \theta)
\end{equation*}
\begin{equation*}
\lambda_c = \frac{k}{mc} = 2{,}13 \cdot 10^{-24} \, \tx{m}
\end{equation*}
\lcom{Allgemeine Relativitäts Theorie: die Grundlage ist Masse in Gravitation.}
\section{Von der SRT zur ART}
ART $ \widehat{=} $ allgemeine Relativitätstheorie

\subsection{Äquivalenzprinzip (\emph{die heilige Kuh der Physik})}
\lcom{Träge und schwere Masse sind zwei unterschiedliche Grössen aus den zwei Formeln. Fahrstuhl hat mein kein Bezug auf die Umwelt, man spürt nur eine Kraft.}\\

Beobachtung: Träge Masse $ \widehat{=} $ schwere Masse $ \quad m_t = m_s $
\begin{equation*}
\vec{F} = m_t \vec{a} \qquad \tx{träge Masse}
\end{equation*}
\begin{equation*}
\vec{F}_{12} = G \frac{m_{1S} m_{2S}}{r_{12}^2} \frac{\vec{r}_{12}}{|\vec{r}_{12}|} \quad \rightarrow \quad \vec{F} = m_s \vec{g} \qquad \tx{schwere Masse}
\end{equation*}
\folie{Fahrstuhl-Äquivalenzprinzip}\\
$ \Rightarrow $ In einem geschlossenen Raum gilt: Die Beschleunigung aufgrund einer Kraft ist nicht von der Gravitation zu unterscheiden \mau\\
\noindent
\textbf{Experimente:}
\begin{itemize}
	\item Pendel
	\item Satellite: lokales Inertialsystem das Kräftefrei ist
	\item Freie-Fall-Experimente (Parabelflug bei Flugzeugen oder Freifall Experiment Turm) \\
	\folie{Freifall Experiment Turm in Bremen}
\end{itemize}
\lcom{Äquivalenzprinzip war anfangs auf 1\,g bestätigt aber mittlerweile sehr genau, also wollen wir es aufrecht erhalten.}

\folie{Wie genau ist das Äquivalenzprinzip bestätigt}\\
\folie{Verletzungen des Äquivalenzprinzips}

\subsubsection{Äquivalenz Prinzip, drei Formulierungen}
\lcom{Inertialsysteme gelten nur in kleinen Raumbereichen da die Gravitationskraft abhängig von $r^2$ ist.}

\folie{Verletzung des Äquivalenzprinzips}

\lcom{Die gravitation ist bis heute noch ein ungelöstes problem. Ähnlich wie Newton und Maxwell, an irgend einer stelle müssen wir was korrigieren da die Theorien der Wechselwirkungen nicht vereinbar sind. Vielleicht existiert eine fünfte fundamentale Kraft.} 

\begin{enumerate}[1)]
	\item Ein kleines Labor, welches in einem Schwerefeld frei fällt ist äquivalent zu einem lokalen Inertialsystem, in welchem die Gesetze der SRT gelten.
	\item Beobachte im Fahrstuhl, Raumschiff (ohne Antrieb):\\
	$ \Rightarrow $ Es ist nicht entscheidbar ob man sich gradlinig im freien Raum oder beschleunigt im Gravitationsfeld bewegt.
	\item Der freie Fall ist Äquivalent zur Schwerelosigkeit
\end{enumerate}

\subsection{Gravitation und Raumkrümmung}
Die Abweichung von einer geraden Bahn kann man auf zwei Arten definieren oder festlegen:
\begin{enumerate}[(i)]
	\item Kraft $ \qquad $ oder
	\item Krümmung des Raums
\end{enumerate}
\folie{zur Raumkrümmung}\\
\textbf{Idee:} Gravitation nicht mehr als Kraft, sondern als Krümmung des Raums betrachten.\\
\lcom{Gilt nur für die Gravitationskraft!}\\
\textbf{Mathematisch:} Gravitation beeinflusst die Metrik des Raums\\[5pt]
\lcom{Ein weg ist $ict^2$ zu verwenden damit ein minus bei dem Skalarprodukt auftaucht, eine andere option ist das Skalarprodukt anders zu definieren zum Beispiel mit einem Vektor $(1,1,1,-1)$. Die Feldgleichungen der ART sind tatsächlich sehr kompliziert und werden meistens nicht behandelt.}\\
\textbf{Konsequenzen:}
\begin{itemize}
	\item Raum und Zeit sind nicht statisch, sondern dynamisch
	\item Inertialsystem nur noch lokal, im homogenen Schwerefeld
\end{itemize}

\subsection{Periheldrehung des Merkurs}
Kepler'sche Bahngesetze \folie{zur Periheldrehung des Merkur}\\
$ \Rightarrow $ geschlossene Ellipsen mit der Sonne im Brennpunkt\\[5pt]
Beobachtung beim Merkur:\\
Periheldrehung $ 5{,}74" \, \tx{pro Jahr} $\\[5pt]
\folie{Merkur Umlaufbahn}
\lcom{Merkur ist nah an der Sonne da treten diese Periheldrehungen deutlicher auf. Was kann diese Bewegung laut Newton beeinflussen? Andere Planeten.}\\
Rechnungen nach Newton mit Einbeziehung der Potentiale der anderen Planeten: $ 0{,}4311" \, \tx{pro Jahr} $ mehr als die gemessene\\[5pt]
ART: Überschuss von $ 0{,}4303" \, \tx{pro Jahr} $

\subsection{Ablenkung von Licht durch Gravitation}
\folie{zur gravitativen Ablenkung von Licht} \folie{Bilder zur Ablenkung von Licht um die Sonne bei einer Sonnenfinsternis}\\
\lcom{Durch die Sonnenfinsternis konnte ein Stern beobachtet werden dessen Position am Himmel sich durch der Lichtbeugung verändert wärend die Sonne sich in der nähe bewegt. Dies hat dieses Phänomen bestätigt.}\\
$ \Rightarrow $ Sonnenfinsternis 1919\\
\folie{Mehr Bilder zu Gravitationslinsen} \folie{Video: zu Gravitationslinsen}

\subsection{Gravitationswellen}
Indirekte Beobachtung umkreisender Neutronensterne PSR1913+16\\
Abstrahlung von Gravitationswellen $ \Rightarrow $ Energieverlust $ \rightarrow $ Sterne kommen sich immer näher. Nobel Preis 1993\\[5pt]
Direkte Beobachtung: Detektoren durch Relativbewegung von Massen
\begin{enumerate}[(i)]
	\item Resonanzdetektor
	\item interferrometrische Detektoren (Micheson-Morley)\\
	\folie{zu Michelson Interferrometern}\\
	11. Februar 2016: LIGO\\
	Nobel Preis 2017
\end{enumerate}

\subsection{Global Positioning System (GPS)}
gegeben: Satelliten, Atomuhren, $ 3{,}87 \, \frac{\tx{km}}{\tx{s}} $ in Höhe von ca. $ 20000 \, \tx{km} $\\[5pt]
SRT: $ \Rightarrow $ Zeitdilatation um $ \approx 7 \, \mu \tx{s} / \tx{Tag} $\\
ART: $ \Rightarrow $ Zeitdilatation um $ \approx 45 \, \mu \tx{s/\tx{Tag}} $\\[5pt]
$ \Rightarrow $ Ortsmessung um ca $ 11{,}4 \, \tx{km} $ verschoben




% chek if correkt hier fehlt was glaube ich

% was?

 
% 31.10.18
 
\chapter{Geometrische Optik}

\section{Lichtstrahlen}
Licht hat eine \textbf{Ausbreitungsgeschwindigkeit} $ s = ct $. Diese Geschwindigkeit ist nur im Vakuum konstant und hängt von der Materie die der Lichtstrahl durchläuft ab. Diese Geschwindigkeit im Medium ist gegeben durch den \textbf{Brechungsindex}.
\begin{equation*}
\rmbox{n = \frac{c_{\tx{Vakuum}}}{c_{\tx{Medium}}}} \qquad \qquad \begin{array}{ll}
\tx{Luft:} & n = 1{,}00027 \\
\tx{Wasser:} & n = 1{,}333 \\
\tx{Diamant:} & n = 2{,}417
\end{array}
\end{equation*}
\lcom{Warum ist das denn so? Wie interagieren die elektomagnetischen Felder des Lichstrahls also mit der Materie?}\\
Antwort:\\
\lcom{Die Ladungsträger in der Materie erfahren aufgrund der elektrischen Felder eine Kraft. Die Ströme der bewegten Ladungen Wechselwirkungen mit den Magnetischen Feldern, jedoch ist dieser Effekt viel kleiner als der der $ \vec{E} $-Felder.\\
Die Elektronen kann man sich mit einer Feder an der Atomkern gebunden Vorstellen. Mit dem $ \vec{E} $-Feld des Lichtstrahls haben wir das Modell eines getriebenen gedämpften harmonischen Oszillators: Das Lorenz-Lorenz-Oszillator Modell}\\
Genaugenommen kann man sich das auch so Vorstellen, dass die Photonen immer wieder absorbiert und nach einer Zeitverzögerung abgestrahlt werden, jedoch ist dies schwer zu berechnen.\\[10pt]
\emph{Bemerkung:}\\
zwei Medien $ n_1 > n_2 $\\
$ n_1 $: optisch dichter\\
$ n_2 $: optisch dünner

\section{Das Fermat'sche Prinzip}

\subsection{Fermat'sches Prinzip}

\begin{minipage}{.6\linewidth}
	Die Ausbreitung des Lichts zwischen zwei Punkten erfolgt auf dem Weg, für den die benötigte Zeit extremal ist.
\end{minipage}
\begin{minipage}{.4\linewidth}
	%t1:
	\centering
	\begin{tikzpicture}[scale=1.5]
	\node[circle,fill=black,inner sep=1pt, minimum size=1pt] at (0,0) (A) {};
	\node[circle,fill=black,inner sep=1pt, minimum size=1pt] at (2,-.3) (B) {};
	\node[left] at (A) {A};
	\node[right] at (B) {B};
	\draw (A) to[out=30,in=190] (1,.5) to[out=10,in=135] (B);
	\draw (A) to[out=-45,in=190] (1,-.7) to[out=10,in=200] (B);
	\draw (A) -- (B);
	\end{tikzpicture}
\end{minipage}

\noindent
$ \Rightarrow $ Konsequenz $\rmbox{\tx{Der Stahlengang ist umkehrbar \mau}}$

\subsubsection{Mathematische Handwerkzeuge}
\begin{equation*}
t = \frac{1}{c_{\tx{Vakuum}}} \int_{P_1}^{P_2} n(s) ds
\end{equation*}

Parametrisierung des Wegs $ \tau $
\begin{equation*}
\vec{x} = \vec{x} (\tau)
\end{equation*}
\begin{equation*}
t(\vec{x}) = \frac{1}{c_{\tx{Vakuum}}} \int_{P_1}^{P_2} n(\vec{x(\tau)}) \sqrt{\left(\frac{d\vec{x}}{d\tau}\right)^2} d\tau
\end{equation*}

\subsection{Weg zwischen zwei Punkten A und B}

%t2:
\begin{tikzpicture}
	\node[circle,fill=black,inner sep=1pt, minimum size=1pt] at (0,0) (A) {};
	\node[circle,fill=black,inner sep=1pt, minimum size=1pt] at (2,0) (B) {};
	\node[left] at (A) {A};
	\node[right] at (B) {B};
	\draw (A) -- (B);
\end{tikzpicture}

\subsection{Reflexionsgesetz}

\begin{minipage}{.6\linewidth}
	\folie{zu Reflexion an Oberflächen}\\
	Berechnung der Wegstrecke
	\begin{equation*}
	s(x) = \sqrt{a^2 + x^2} + \sqrt{d^2 + (b-x)^2}
	\end{equation*}
\end{minipage}
\begin{minipage}{.4\linewidth}
	%t3:  % folie:
	\centering
	\begin{tikzpicture}[scale=1.75]
	\node at (0,0) (m) {};
	\node at (-1,0) (l) {};
	\node at (1,0) (r) {};
	\node at (0,1) (mu) {};
	\node at (-1,1) (lu) {};
	\node at (1,1) (ru) {};
	\draw[thick,->] (-1,1) -- (0,0);
	\draw[thick,->] (0,0) -- (1,1);
	\draw (-1,1) -- (-1,0) -- (0,0) -- (1,0) -- (1,1);
	\draw[dashed] (0,0) -- (0,1.2);
	\draw (.5,.5) arc (45:90:20pt);
	\draw (-.5,.5) arc (135:90:20pt);
	\node at (.16,.4) () {$ \beta $};
	\node at (-.16,.4) () {$ \alpha $};
	\node[below] at (-.5,0) {$ x $};
	\node[below] at (.5,0) {$ b $};
	\node[right] at (1,.5) {$ d $};
	\node[left] at (-1,.5) {$ a $};
	\end{tikzpicture}
\end{minipage}

Extremum:\\
\lcom{Da wir und in einem homogenen Medium befinden und die Lichtgeschwindigkeit konstant ist, ist die Strecke $ s $ proportional zur Zeit $ t $ und wir können das Extremum der Strecke berechnen.}
\begin{equation*}
\frac{ds}{dx} = 0
\end{equation*}
\begin{equation*}
\Rightarrow \ub{\frac{x}{\sqrt{a^2 + x^2}}}_{\sin \alpha} = \ub{\frac{b-x}{\sqrt{d^2 + (b-x)^2}}}_{\sin \beta}
\end{equation*}
\begin{minipage}{.2\linewidth}
	$ \phantom{M} $
\end{minipage}
\begin{minipage}{.6\linewidth}
	\frbox{Reflexionsgesetz}{\begin{equation*}
		\sin \alpha = \sin \beta
		\end{equation*}}
\end{minipage}\\[5pt]
\lcom{Einfallswinkel gleich Ausfallswinkel.}

\subsection{Brechungsgesetz von Snellius}

\begin{equation*}
n_1 \neq n_2 \qquad \Leftrightarrow \qquad n(s) \neq \const
\end{equation*}

\begin{minipage}{\linewidth}
	%t4:
	\centering
	\begin{tikzpicture}[scale=1.5]
	\draw[draw=none,fill=gray!10] (-1.5,0) rectangle (1.5,-1.1);
	\node[circle,fill=black,inner sep=1pt, minimum size=1pt] at (-.75,.4) (A) {};
	\node[circle,fill=black,inner sep=1pt, minimum size=1pt] at (1,-.75) (B) {};
	\node[anchor=south east] at (A) {A};
	\node[anchor=north west] at (B) {B};
	\node[circle,fill=black,inner sep=1pt, minimum size=1pt] at (-.5,0) (a) {};
	\node[circle,fill=black,inner sep=1pt, minimum size=1pt] at (.5,0) (b) {};
	\node[circle,fill=black,inner sep=1pt, minimum size=1pt] at (-.9,0) (c) {};
	\draw[thick] (A) -- (b) -- (B) -- (a) -- (A);
	\draw[thick] (A) -- (c) -- (B);
	\draw (-1.5,0) -- (1.5,0);
	\filldraw[draw=none,pattern=north east lines] (-1.5,0) rectangle (1.5,-.1);
	\node at (-2,.5) {$ c_1 n_1 $};
	\node at (-2,-.5) {$ c_2 n_2 $};
	\draw[dashed] (0,-1) -- (0,1);
	\end{tikzpicture} \hspace{80pt}
	\begin{tikzpicture}[scale=1.5]
	\draw[draw=none,fill=gray!10] (-1.5,0) rectangle (1.5,-1.1);
	\filldraw[draw=none,pattern=north east lines] (-1.5,0) rectangle (1.5,-.1);
	\node[circle,fill=black,inner sep=1pt, minimum size=1pt] at (-1,.75) (A) {};
	\node[circle,fill=black,inner sep=1pt, minimum size=1pt] at (1,-.75) (B) {};
	\node[anchor=south east] at (A) {A};
	\node[anchor=north west] at (B) {B};
	\node[circle,fill=black,inner sep=1pt, minimum size=1pt] at (-.5,0) (a) {};
	\node[circle,fill=black,inner sep=1pt, minimum size=1pt] at (.5,0) (b) {};
	\draw (A) -- (b) -- (B) -- (a) -- (A);
	\draw (-1.5,0) -- (1.5,0);
	\draw[thick,dashed] (A) -- (-1,0);
	\draw[dashed] (0,-1) -- (0,1);
	\draw[thick,->] (-1,0) -- (b) node[anchor=south west] {$ x $};
	\end{tikzpicture} \hspace{30pt}
\end{minipage}\\[10pt]
\noindent
Berechnung des Laufzeit des Lichts
\begin{equation*}
t(x) = \frac{\sqrt{x^2 + a^2}}{c_1} + \frac{\sqrt{(b-x)^2 + d^2}}{c_2} = \frac{1}{c} \left(n_1 \sqrt{x^2 + a^2} + n_2 \sqrt{(b-x)^2 + d^2}\right)
\end{equation*}
\folie{zu Brechung}
$ \frac{dt}{dx} = 0 $

\begin{minipage}{.55\linewidth}
	\frbox{Brechungsgesetz}{\begin{equation*}
		\Rightarrow n_1 \sin \alpha = n_2 \sin \beta
		\end{equation*}}
\end{minipage}
\begin{minipage}{.45\linewidth}
	%t5:
	\centering
	\begin{tikzpicture}[scale=1.5]
	\draw[draw=none,fill=gray!10] (-1.5,0) rectangle (2,-.8);
	\node[circle,fill=black,inner sep=1pt, minimum size=1pt] at (-1,.75) (A) {};
	\node[circle,fill=black,inner sep=1pt, minimum size=1pt] at (1.5,-.5) (B) {};
	\node[anchor=south east] at (A) {A};
	\node[anchor=north west] at (B) {B};
	\node[circle,fill=black,inner sep=1pt, minimum size=1pt] at (-1,0) (a) {};
	\node[circle,fill=black,inner sep=1pt, minimum size=1pt] at (1.5,0) (b) {};
	\node[circle,fill=black,inner sep=1pt, minimum size=1pt] at (0,0) (o) {};
	\draw[thick] (A) -- (o) -- (B);
	\draw (A) -- (a) -- (o) -- (b) -- (B);
	\draw (-1.5,0) -- (2,0);
	\draw[thick,dashed] (A) -- (-1,0);
	\draw[dashed] (0,-1) -- (0,1);
	\draw (0,-.7) arc (-90:-25:23pt);
	\draw (0,.7) arc (90:143.5:20pt);
	\node at (-.2,.35) {$ \alpha $};
	\node at (.3,-.4) {$ \beta $};
	\node[left] at (-1,.375) {$ a $};
	\node[below] at (-.5,0) {$ x $};
	\node[above] at (.75,0) {$ b $};
	\node[right] at (1.5,-.25) {$ d $};
	\node at (-1.7,-.25) {$ n_2 $};
	\node at (-1.7,.25) {$ n_1 $};
\end{tikzpicture}
\end{minipage}

\section{Einfache Anwendungen}

\subsection{Totalreflexion}

$ \rightarrow $ siehe Experimentalphysik II\\
\folie{zur Totalreflexion z.B. unter Wasser}
\begin{equation*}
\sin\alpha_{\tx{TV}} = \frac{n_2}{n_1}
\end{equation*}

\subsection{Optisches Prisma}
\lcom{Bei Drehung des Prismas kann der Winkel $ \delta $ verändert werden. Hier gibt es es ein Minimum, jedoch kann er nie null werden, da das Licht nie gerade durch das Prisma hindurch kommen kann.}

\subsubsection{Beobachtung I}

\begin{minipage}{.55\linewidth}
	Die Ablenkung ist minimal für den symmetrischen  Strahlengang
	\begin{equation*}
	\sin \frac{\delta_{\tx{min}} + \gamma}{2} = n \sin \frac{\gamma}{2}
	\end{equation*}
	\folie{Demtröder: Prisma}
	\vspace{40pt}
\end{minipage}
\begin{minipage}{.45\linewidth}
	%t6:
	\centering
	\begin{tikzpicture}
		\draw[fill=gray!10] (2,0) -- ++(120:4cm) -- ++(240:4cm) -- ++(360:4cm);
		\coordinate (d) at ($(60:2cm) + (120:2cm)$);
		\coordinate (a) at (120:2cm); %($ (-2,0)!.4!(d) $);
		\coordinate (c) at (60:2cm);
		\coordinate (b) at ($ (2,0)!.6!(d) $) {};
		\draw[dashed] ($(a) + (150:1cm)$) -- ++(-30:2cm);
		\draw[dashed] ($(b) + (30:1cm)$) -- ++(210:2cm);
		\draw[thick] (3,1.5) -- (b) -- (a) -- ++(220:2cm);
		\draw[dashed] (a) -- ++(40:4cm);
		\centerarc[](a)(150:220:.8cm);
		\node[left,xshift=-5pt,yshift=-2pt] at (a) {$ \alpha_1 $};
		\centerarc[](a)(-30:10:.8cm);
		\node[anchor=north west,yshift=-4pt] at (a) {$ \beta_1 $};
		\centerarc[](b)(190:210:.8cm);
		\node[anchor=north east,yshift=-5pt] at (b) {$ \beta_2 $};
		\centerarc[](b)(30:-14:.9cm);
		\node[right,xshift=8pt,yshift=1pt] at (b) {$ \alpha_2 $};
		\coordinate (o) at (-.26,2.34) {};
		\centerarc[](o)(40:-14:2.5cm);
		\node at ($ (b) + (1.7,.8) $) {$ \delta $};
		\centerarc[](d)(-120:-60:.6cm);
		\node[below,yshift=-3pt] at (d) {$ \gamma $};
		\node[above] at (-1,0) {$ n = n_2 $};
		\node[below] at (-1.5,-.3) {$ n = n_1 $};
	\end{tikzpicture}
\end{minipage}
\textbf{Berechnung:}
\begin{enumerate}[(i)]
	\item $ \gamma = \beta_1 + \beta_2 $
	\item $ \delta = \alpha_1 - \beta_1 + \alpha_2 - \beta_2 $
	\item $ \sin \alpha_1 = n \sin \beta_1 \quad , \quad \sin \alpha_2 = n \sin \beta_2 $
\end{enumerate}
gesucht ist $ \frac{d\delta}{d \alpha_1} = 0 $\\
$ \Rightarrow $
\begin{enumerate}[(i)]
	\item $ \tx{d}\gamma = \tx{d}\beta_1 + \tx{d}\beta_2 = 0 $
	\item $ \tx{d}\alpha_1 = \tx{d}\alpha_2 \qquad (\delta = \alpha + \alpha - \gamma) \qquad (\mathrm{d} \delta = 0) $
	\item $ \cos \alpha_1 \tx{d} \alpha_1 = n \cos \beta_1 \tx{d} \beta_1 $\\
	$ \cos \alpha_1 \tx{d} \alpha_2 = n \cos \beta_2 \tx{d} \beta_2 $
\end{enumerate}

$$\Rightarrow\frac{\cos\alpha_1\cancel{\tx{d}\alpha_1}}{\cos\alpha_2\cancel{\tx{d}\alpha_2}}=\frac{\cos\beta_1\cancel{\tx{d}\beta_1}}{\cos\beta_1\cancel{\tx{d}\beta_2}}$$
$$\Rightarrow\frac{1-\sin^2\alpha_1}{1-\sin^2\alpha_2}=\frac{1-\sin^2\beta  a_1}{1-\sin^2\beta_2}=\frac{n^2-\sin^2\alpha_1}{n^2-\sin^2\alpha_2}$$\\[-10pt]
$$\left.\begin{array}{c}
\alpha_1=\alpha_2 \\
\beta_1=\beta_2\end{array} \right\} \textrm{ symmetrischer Strahlengang}$$
$$\Rightarrow\delta_\tx{min}=\alpha_1+\alpha_2-\gamma=2\alpha-\gamma$$
$$\sin\alpha=n\sin\beta\Rightarrow\rmbox{\sin\frac{\delta_\tx{min}+\gamma}{2}  =n\sin\frac{\gamma}{2}}$$

\noindent
$ \Rightarrow \quad \delta_{\tx{min}} = \alpha_1 + \alpha_2 - \gamma = 2 \alpha - \gamma \qquad \gamma = 2 \beta $
\begin{equation*}
\Rightarrow \quad \sin \alpha = n \sin \beta \quad \Rightarrow \quad \rmbox{\sin \frac{\delta_{\tx{min}} + \gamma}{2} = n \sin \frac{\gamma}{2}}
\end{equation*}
\begin{flushright}
	$ \square $
\end{flushright}
\emph{Bemerkung:}\\
Dieser Zusammenhang wurde früher zur Bestimmung von $ n $ genutzt \mau\\
\lcom{*Professor redet darüber, dass wir immer kompliziertere Experimente verstehen können, und diese aufgrund des Aufwands bei der Ausführung nicht mehr immer in der Vorlesung gezeigt werden können. Wir sollen lernen uns Experimente vorzustellen ohne sie direkt zu sehen und die Formeln als Handlungsanweisungen zu sehen die wir nutzen können. Abgesehen davon sollen wir uns auch im klaren darüber sein wie genau verschiedene Messmethoden sind.*}

\subsubsection{Beobachtung II}
optische Dispersion:
\begin{equation*}
\frac{dn(\lambda)}{d\lambda} < 0 \qquad (\tx{normale Dispersion})
\end{equation*}
(Bei der anormalen Dispersion ist $ \frac{d n(\lambda)}{d\lambda} > 0 $)\\
\folie{zur Dispersion: Beispiel Regenbogen}\\[5pt]
\lcom{Wieso gibt es Dispersion? Vorstellung: Die Atome mit ihren Elektronen haben Resonanzfrequenzen. Somit ist die erzwungene Schwingung die die Licht-Welle in der Materie anregt, von der Frequenz des Lichts abhängig.\\
Die meisten Resonanzfrequenzen liegen im Ultravioletten-Bereich. Deshalb sehen wir meist normale Dispersion. Im Bereich über Ultraviolettem Licht ist auch anormale Dispersion Beobachtbar.}\\[5pt]
$ \Rightarrow $ Experimentalphysik II
%\lcom{*Das geht wirklich nicht! Was soll man denn da machen? Das ist echt lässtig!*}

\section{Sphärische, dünne Linse}
\lcom{Zunächst einmal betrachten wir sphärische Linsen speziell dünne Linsen: Wir betrachten die dünne Linse als eine unendlich-dünne Linse, da wir hierbei einige vereinfachungen machen können (Taylor-entwicklungs-Terme fallen weg).}

\subsection{Brennpunkt und Brennebene}
\folie{zur Sammel- und Streulinsen}\\
Wir betrachten zunächst den Strahlengang einer Linse:\\

\rbox{\begin{minipage}{\linewidth}
		\lcom{Lieblingsprüfungsfrage: Parallel einfallendes Licht, das nicht parallel zur optischen Achse verläuft, in eine Sammellinse. Strahlen werden immer noch fokussiert aber Brennpunkt verschoben. Konstruktion mit Zentralstrahlengang \mau}
	\end{minipage}}
\noindent
\folie{zur der Lieblingsfrage}

\subsubsection{3 Regeln zur Konstruktion von Strahlengängen}
Nur für dünne Linsen:
\begin{enumerate}[1)]
	\item Parallel einfallende Strahlen fallen durch den Brennpunkt aus
	\item Durch Brennpunkt einfallender Strahlen fallen parallel aus
	\item Der Zentralstrahl wird nicht abgelenkt
\end{enumerate}

\subsection{Berechnung des Brennpunktes bzw. der Brennweite}
\lcom{Bei der Streulinse genau dasselbe nur ein negatives Vorzeichen an der Brennweite.}\\
\folie{zu Strahlengängen durch Sammellinsen}\\[5pt]
Ein Linsenstück im Abstand $ h $ von der optischen Achse betrachten wir als Prisma.\\[5pt]
Näherungen der ,,dünnen`` Linse\\[5pt]
alle Winkel sind klein $ \rightarrow $ trigonometrische Näherungen\\
$ \sin(x) \approx x \qquad \cos(x) \approx 1 \qquad \tan(x) \approx x $\\[5pt]
\textbf{Berechnung:}\\
Ablenkung des Strahls

\begin{minipage}{.5\linewidth}
	\begin{enumerate}
		\item[(i)] $ \delta = \alpha_1 - \beta_1 + \alpha_2 - \beta_2 $
		\item[(ii)] $ \beta_1 + \beta_2 = \gamma $
		\item[(iii)] $ n_1 \alpha_1 \approx n_2 \beta_1 \qquad n_2 \beta_2 \approx n_1 \alpha_2 $
		\begin{equation*}
		\Rightarrow \quad \delta = \gamma\left(\frac{n_2}{n_1} - 1\right)
		\end{equation*}
	\end{enumerate}
\end{minipage}
\begin{minipage}{.5\linewidth}
	%t8:
	\centering
	\begin{tikzpicture}
		\draw[dashed] (0,4) arc (150:170:400pt);
		\draw[dashed] (0,4) arc (30:10:400pt);
		\centerarc[](0,4)(-118:-62:.8cm);
		\node[below,yshift=-7pt] at (0,4) {$ \gamma $};
		%\node[circle,fill=black,inner sep=1pt,minimum size=1pt] at (1.55,0) {};
		%\node[circle,fill=black,inner sep=1pt,minimum size=1pt] at (.95,2) {};
		\coordinate (x) at (1.55,0);
		\coordinate (x2) at (-1.55,0);
		\coordinate (y) at (.95,2);
		\coordinate (y2) at (-.95,2);
		\draw[fill=gray!10] (x) -- (y) -- (y2) -- (x2) -- (x);
		\coordinate (b) at ($ (x)!.6!(y) $);
		\coordinate (a) at ($ (x2)!.4!(y2) $);
		%\draw(a) -- (b);
		% winkel der seiten ist 107 grad bzw 73 grad
		\draw[dashed] ($(a) + (163:1cm)$) -- ++(-17:2cm);
		\draw[dashed] ($(b) + (197:1cm)$) -- ++(17:2.75cm);
		\draw[thick] (3,.5) -- (b) -- (a) -- ++(220:2.5cm);
		\draw[dashed] (a) -- ++(40:5cm);
		\centerarc[](a)(163:220:.8cm);
		\node[left,xshift=-5pt,yshift=-2pt] at (a) {$ \alpha_1 $};
		\centerarc[](a)(-17:10:.8cm);
		\node[anchor=north west,yshift=-4pt] at (a) {$ \beta_1 $};
		\centerarc[](b)(190:197:.8cm);
		\node[anchor=north east,yshift=-5pt] at (b) {$ \beta_2 $};
		\centerarc[](b)(17:-20:.9cm);
		\node[right,xshift=8pt,yshift=-1pt] at (b) {$ \alpha_2 $};
		\centerarc[](b)(17:-20:1.5cm);
		\node[right,xshift=25pt,yshift=2pt] at (b) {$ \delta_2 $};
		\centerarc[](a)(40:12:4.02cm);
		\node[right,xshift=87pt,yshift=45pt] at (a) {$ \delta_1 $};
		\coordinate (o) at (-.22,1.71) {};
		%\node[circle,fill=black,inner sep=1pt,minimum size=1pt] at (-.22,1.71) {};
		%\draw (b) -- ++(160:3cm);
		\centerarc[](o)(40:-20:3.4cm);
		\node at ($ (b) + (2.2,1) $) {$ \delta $};
		\node[above] at (1,0) {$ n_2 $};
		\node[below] at (1.2,-.3) {$ n_1 $};
	\end{tikzpicture}
\end{minipage}

\noindent
Winkel $ \gamma $:\\[10pt]
\begin{minipage}{.035\linewidth}
	$ \phantom{.} $
\end{minipage}
\begin{minipage}{.475\linewidth}
	\begin{enumerate}
		\item[(iv)] $ \gamma = \gamma_1 + \gamma_2 $
		\item[(v)] $ \gamma_1 \approx \frac{h}{R_1} \quad ;\quad \gamma_2 \approx \frac{h}{R_2} $
		\begin{equation*}
		\Rightarrow \qquad \gamma = h \left(\frac{1}{R_1} - \frac{1}{R_2}\right)
		\end{equation*}
	\end{enumerate}
\end{minipage}
\begin{minipage}{.5\linewidth}
	%t9:
	\centering
	\begin{tikzpicture}[decoration=brace,scale=.8]
		\coordinate (a) at ( .25,0) {};
		\coordinate (b) at (-.25,0) {};
		\draw[fill=gray!10,draw=none] ($(b)+(135:2cm) + (-0.02,0)$) rectangle ($(a)+(-28:3cm)$);
		\node[circle,fill=black,inner sep=1pt,minimum size=1pt] at (a) {};
		\node[circle,fill=black,inner sep=1pt,minimum size=1pt] at (b) {};
		\centerarc[thick,fill=gray!10](a)(-28:28:3cm);
		\centerarc[thick,fill=gray!10](b)(135:225:2cm);
		\draw[dashed] (-3,0) -- (4,0);
		\draw (a) -- ++(19:3cm);
		\draw (b) -- ++(150:2cm);
		\draw[color=gray!70] ($(b)+(135:2cm)$) -- ($(a)+(28:3cm)$);
		\draw[color=gray!70] ($(b)+(225:2cm)$) -- ($(a)+(-28:3cm)$);
		\draw[red] (-2.5,0) -- (-2.5,1);
		\draw[red] (-2.6,1) -- (-2.4,1);
		\draw[red,dashed] (-2.5,1) -- (-2,1);
		\draw[decorate, xshift=-4pt,red]  (-2.5,0) -- node[left=0.4pt] {$h$}  (-2.5,1);
		\node[below] at (-1.25,0) {$ R_1 $};
		\node[below] at (1.8,0) {$ R_2 $};
		\centerarc[](b)(180:150:1.2cm);
		\centerarc[](a)(0:19:1.5cm);
		\node[right,yshift=4pt,xshift=17pt] at (a) {$ \gamma_2 $};
		\node[left,yshift=4pt,xshift=-10pt] at (b) {$ \gamma_1 $};
	\end{tikzpicture}
\end{minipage}

\noindent
Winkel $ \delta $:\\[10pt]
\begin{minipage}{.035\linewidth}
	$ \phantom{.} $
\end{minipage}
\begin{minipage}{.475\linewidth}
	\begin{enumerate}
		\item[(vi)] $ \delta \approx \frac{h}{f} $
	\end{enumerate}
\end{minipage}
\begin{minipage}{.5\linewidth}
	%t10:
	\centering
	\begin{tikzpicture}
		\draw[fill=gray!10] (0,0) ellipse (.2cm and 1.7cm);
		\draw[->] (-2,0) -- (2,0) node[anchor=north west] {$ x $};
		\draw[->] (0,-2) -- (0,2) node[anchor=south east] {$ y $};
		\draw[red] (-2,1.2) -- (0,1.2) -- (2,-1.2);
		\draw[red,thick,<->] (0,0) -- (1,0) node[below,xshift=-14pt] {$ f $};
		\draw[red,thick,<->] (-1,0) -- (-1,1.2) node[left,yshift=-17pt] {$ h $};
		\node[circle,fill=black,inner sep=1pt,minimum size=1pt] at (1,0) {};
		\draw[blue,dashed] (0,1.2) -- (2,1.2);
		\centerarc[blue](0,1.2)(0:-50:1cm);
		\node[blue] at (.6,.9) {$ \delta $};
	\end{tikzpicture}
\end{minipage}




% 6.11.18 Vorlesung 7

\frbox{Linsenschleiferformel}{\begin{equation*}
\Rightarrow \quad \frac{1}{f} = \frac{n_2 - n_1}{n_1} \left(\frac{1}{R_1} - \frac{1}{R_2}\right)
\end{equation*}}
$ \Rightarrow $ Die Brennweite $ f $ ist unabhängig von $ h $. \lcom{Hätte wir oben keine Näherungen für die Winkelfunktionen benutzt (bei allen Rundungssymbolen), dann wäre die Brennweite abhängig von der Höhe $ h $. Hier sehen wir, dass die Gleichung für eine dünne Linse nur eine annäherung an echte Linsen ist.}\\[10pt]
\textbf{Def: Brechkraft}
\begin{equation*}
D = \frac{n}{f}
\end{equation*}
\emph{Bemerkung:}\\
$ D = D_{\tx{Vorne}} + D_{\tx{Hinten}} $

\subsection{Abbildung (am Beispiel einer Sammellinse)}

\begin{minipage}{.5\linewidth}
	Berechnung:
	\begin{enumerate}[i)]
		\item $ \delta = \alpha + \beta $
		\item $ \alpha \approx \frac{h}{a} \quad \beta \approx \frac{h}{b} $
		\item $ \delta \approx \frac{h}{f} $
	\end{enumerate}
\end{minipage}
\begin{minipage}{.5\linewidth}
	%t1:
	\centering
	\begin{tikzpicture}
		\draw[fill=gray!10] (0,0) ellipse (.2cm and 1.7cm);
		\draw[dashed]  (0,-2) -- (0,2);
		\draw (-3,0) -- (3,0);
		\coordinate (A) at (-2.5,0);
		\coordinate (B) at (2.5,0);
		\node[below] at (A) {$ A $};
		\node[below] at (B) {$ B $};
		\draw[thick] ($ (A) + (0,-.1) $) -- ++(0,.2);
		\draw[thick] ($ (B) + (0,-.1) $) -- ++(0,.2);
		\coordinate (M) at (0,1);
		\draw (A) -- (M) -- (B);
		\node[below] at ($ (0,0)!.5!(A) $) {$ a $};
		\node[below] at ($ (0,0)!.5!(B) $) {$ b $};
		\draw[dashed,thick] (M) -- ++($ (B) + (M) $);
		\centerarc[](M)(22:-22:1cm);
		\node[right,xshift=10pt] at (M) {$ \delta $};
		\centerarc[](A)(22:0:1.5cm);
		\node[right,xshift=20pt,yshift=5pt] at (A) {$ \alpha $};
		\centerarc[](B)(158:180:1.5cm);
		\node[left,xshift=-25pt,yshift=6pt] at (B) {$ \beta $};
		\draw (A) -- ++(-158:1cm);
		\draw (B) -- ++(-22:1cm);
	\end{tikzpicture}
\end{minipage}

\frbox{Abbildungsgesetz}{\begin{equation*}
\Rightarrow \frac{1}{f} = \frac{1}{a} + \frac{1}{b}
\end{equation*}}
\textbf{Abbildung eines Gegenstands $ G $:}\\
Graphisch:
\begin{enumerate}[(1)]
	\item parallel ein $ \rightarrow $ durch Brennpunkt aus
	\item durch Brennpunkt ein $ \rightarrow $ parallel aus
	\item Zentralstrahl läuft gerade
\end{enumerate}

\begin{minipage}{.5\linewidth}
	\textbf{Rechnerisch}
	\begin{enumerate}[(i)]
		\item $ \frac{1}{f} = \frac{1}{a} + \frac{1}{b} $
		\item $ \frac{A}{a} = \frac{B}{b} $
	\end{enumerate}
\end{minipage}
\begin{minipage}{.5\linewidth}
	%t2:
	\centering
	\begin{tikzpicture}
		\draw[fill=gray!10] (0,0) ellipse (.2cm and 1.7cm);
		\draw[dashed]  (0,-2) -- (0,2);
		\draw (-3,0) -- (3,0);
		\coordinate (A) at (-2.5,1);
		\coordinate (B) at (2.5,-1);
		\node[above] at (A) {$ A $};
		\node[below] at (B) {$ B $};
		\draw[thick] ($ (-2.5,0) + (0,-.1) $) -- ++(0,.2);
		\draw[thick] ($ (2.5,0) + (0,-.1) $) -- ++(0,.2);
		\coordinate (M) at (0,1);
		\coordinate (m) at (0,-1);
		\draw (A) -- (M) -- (B);
		\draw (A) -- (B);
		\draw (A) -- (m) -- (B);
		\node[below] at ($ (0,0)!.5!(-2.5,0) $) {$ f $};
		\node[above] at ($ (0,0)!.5!(2.5,0) $) {$ f $};
		\draw[thick] ($ (1.25,0) + (45:.1) $) -- ++(-135:.2);
		\draw[thick] ($ (1.25,0) + (-45:.1) $) -- ++(135:.2);
		\draw[thick] ($ (-1.25,0) + (45:.1) $) -- ++(-135:.2);
		\draw[thick] ($ (-1.25,0) + (-45:.1) $) -- ++(135:.2);
		\node[below] at (-2,0) {$ a $};
		\node[below] at (2,0) {$ b $};
		\draw[very thick,->] (-2.5,0) -- (A);
		\draw[very thick,->] (2.5,0) -- (B);
		\node[circle,draw=black,inner sep=1pt] at (-1,1.5) {1};
		\node[circle,draw=black,inner sep=1pt] at (-.6,.7) {2};
		\node[circle,draw=black,inner sep=1pt] at (-.7,-1) {3};
	\end{tikzpicture}
\end{minipage}




\section{Abbildungsfehler}

$ \Rightarrow $ Beschränkung durch Beschreibung mit geometrischer Optik\\
$ \Rightarrow $ Näherungen (dünne Linse)

\subsection{Dicke Linsen}

\folie{Dicke Linsen}\\
Brechung der Lichtstrahlen an zwei Hauptebenen


% Bilder von Volie Dicke Linsen einfügen



\subsection{Sphärische Abberation}

\folie{Sphärische Abberation}

\subsection{Chromatische Abbreration}

\folie{Chromatische Abberation}\\
optische Dispersion

\subsection{Astigmatismus}

\folie{Astigmatismus}

\subsection{Absorption}

Bei verunreinigten Stoffen wird Licht absorbiert.

\subsection{Beugung}

\lcom{Werden wir später behandeln.}

\subsection{Optisches Auflösungsvermögen}

\rbox{Mit freilaufenden Wellen lassen sich keine Strukturen auflösen, die deutlich kleiner sind als die Wellenlänge $ \lambda $ sind.}

\section{Optische Instrumente}

\subsection{Vergrößerung}

\textbf{Sehwinkel:}
%t3:
\hspace{40pt}
\begin{tikzpicture}
	\draw ($ (-.5,0) + (25:.7cm) $) -- ++(-155:.7cm) -- ++(-25:.7cm);
	\centerarc[](-.5,0)(-25:25:.5cm);
	\draw (3,0) -- (0,0) -- (3,1);
	\draw[thick,->] (3,0) -- (3,1) node[right,yshift=-15pt] {$ G $};
	\centerarc[](0,0)(0:18.5:1.5cm);
	\node[yshift=5pt,xshift=30pt] at (0,0) {$ \alpha $};
\end{tikzpicture}

\begin{equation*}
\tan\alpha = \frac{\tx{Größe des Objekts}}{\tx{Entfernung des Objekts}}
\end{equation*}
\textbf{Winkelvergrößerung}
\begin{equation*}
V = \frac{\tx{Sehwiinkel mit Instrument}}{\tx{Sehwinkel ohne Instrument}}
\end{equation*}
\textbf{Lateralvergrößerung oder Abbildungsmaßstab}
\begin{equation*}
L = \frac{\tx{Bildgröße}}{\tx{Gegenstandsgröße}}
\end{equation*}

\subsection{Das menschliche Auge}

\folie{menschliches Auge}\\
Hornhaut \& Linse. Die Hornhaut hat den größten Beitrag zur Brechung\\[5pt]
\textbf{Definition: Brennweite der deutlichen Sehweite}
\begin{equation*}
25 \, \tx{cm}
\end{equation*}
\begin{equation*}
\Rightarrow \quad \tan \alpha_{\tx{Auge}} = \frac{G}{25 \, \tx{cm}}
\end{equation*}
Vergrößerung
\begin{equation*}
V = \frac{\tx{Sehwinkel mit Instrument}}{\alpha_{\tx{Auge}}} = (\tx{Sehwinkel mit Instrument}) \frac{25 \, \tx{cm}}{G}
\end{equation*}

\subsection{Optisches Instrument mit einer Linse: Lupe}

%t4:
\begin{center}
\begin{tikzpicture}
	\draw[fill=gray!10] (0,0) ellipse (.2cm and 1.7cm);
	\draw[dashed] (0,-2) -- (0,2);
	\draw (-4,0) -- (3,0);
	\node[below] at ($ (0,0)!.5!(-3.6,0) $) {$ f $};
	\node[above] at ($ (0,0)!.5!(3.6,0) $) {$ f $};
	\draw[thick] ($ (1.8,0) + (45:.1) $) -- ++(-135:.2);
	\draw[thick] ($ (1.8,0) + (-45:.1) $) -- ++(135:.2);
	\draw[thick] ($ (-1.8,0) + (45:.1) $) -- ++(-135:.2);
	\draw[thick] ($ (-1.8,0) + (-45:.1) $) -- ++(135:.2);
	\coordinate (B) at (-3.6,1.2);
	\coordinate (G) at (-1.2,.4);
	\coordinate (f) at (.9,0);
	\coordinate (f2) at (-.9,0);
	\draw[very thick,->] (-1.2,0) -- (G) node[yshift=-18pt] {$ G $};
	\draw[dashed] (B) -- ($ (B) + (3.6,-.8) $);
	\draw[dashed] (B) -- ($ (B) + (3.6,0) $);
	\draw[dashed] (B) -- (0,0);
	\draw[dashed] (-1.8,0) -- (0,1.2);
	\draw[very thick,->] (-3.6,0) -- (B) node[left,yshift=-20] {$ B $};
	\draw (G) -- (0,1.2) -- ++(2.5,0) node[right] {$ (3) $};
	\draw (G) -- (0,.4) -- ++($(0,0)!.8!(3.6,-.8)$) node[right] {$ (2) $};
	\draw (G) -- (0,0) -- ++($ (0,0)!.7!(3.6,-1.2) $) node[right] {$ (1) $};
	\draw[->] (0,-.5) -- ++(-1.2,0) node[below,xshift=15pt] {$ g $};
	\draw[->] (0,-1.2) -- ++(-3.6,0) node[below,xshift=50pt] {$ b $};
\end{tikzpicture}
\end{center}
Gegenstand nahezu in Brennebenen, $ g < f $\\
Graphisch: siehe oben\\
Rechnerisch: Abbildungsgesetz der Lupe $ \frac{1}{g} - \frac{1}{b} = \frac{1}{f} $\\
Abbildungsmaßstab
\begin{equation*}
\frac{B}{G} = \frac{b}{g} = \frac{f}{f-g} = \frac{f+b}{f}
\end{equation*}
Vergrößerung:\\
Sehwinkel des Instruments
\begin{equation*}
\alpha \approx \tan \alpha = \frac{G}{f} \qquad (g \approx f)
\end{equation*}
\begin{equation*}
\Rightarrow V = \frac{\alpha}{\alpha_{\tx{Auge}}} = \frac{G}{f} \frac{25 \, \tx{cm}}{g} = \frac{25 \, \tx{cm}}{f}
\end{equation*}

\subsection{Optische Instrumente mit zwei Linsen}

zwei Linsen $ = $ drei Fälle:\\[5pt]
Doppellinse, Fernrohr, Mikroskop\\[5pt]
\begin{enumerate}[1)]
	\begin{minipage}{.7\linewidth}
		\item \textbf{Doppellinse: Abstand der Linse $ \ll $ Brennweiten $ f_1,f_2 $}\\
		\folie{zu Doppellinsen}
		\begin{equation*}
		\frac{1}{f} = \frac{1}{f_1} + \frac{1}{f_2}
		\end{equation*}
	\end{minipage}
	\begin{minipage}{.3\linewidth}
		%t5:
		\centering
		\begin{tikzpicture}
			\coordinate (1) at (-1,0);
			\coordinate (2) at (-.5,0);
			\coordinate (3) at (.5,0);
			\node at (0,0) {$ \widehat{=} $};
			\draw[fill=gray!10] (1) ellipse (.1cm and .5cm);
			\draw[fill=gray!10] (2) ellipse (.1cm and .5cm);
			\draw[fill=gray!10] (3) ellipse (.1cm and .5cm);
			\node[yshift=-25pt] at (1) {$ f_1 $};
			\node[yshift=-25pt] at (2) {$ f_2 $};
			\node[yshift=-25pt] at (3) {$\qquad \ \  \qquad \frac{1}{f} = \frac{1}{f_1} + \frac{1}{f_2} $};
		\end{tikzpicture}
	\end{minipage}
	Berechnung:
	\begin{equation*}
	b_1 \ \tx{der 1ten Linsen} = \tx{virtuelle Gegenstandsweite} \ a_2 = b_1 - D \ \tx{der zweiten Linse}
	\end{equation*}
	\begin{itemize}
		\item[1. Linse] $ \frac{1}{a} + \frac{1}{b_1} = \frac{1}{f_1} $
		\item[2. Linse] $ -\frac{1}{b_1 - D} + \frac{1}{b} = \frac{1}{f_2} $
	\end{itemize}
	\begin{equation*}
	\Rightarrow \qquad \rmbox{\frac{1}{f} = \frac{1}{f_1} + \frac{1}{f_2}} \qquad \rmbox{D_{\tx{ges}} = D_1 + D_2}
	\end{equation*}
	\item \textbf{Fernrohr (Kepler): Abstand der Linsen $ = $ Summe der Brennweiten $ f_1 + f_2 $}\\
	
	\begin{minipage}{.5\linewidth}
		$ \Rightarrow $ Zwischenbild in der Brennebene\\[10pt]
		Sehwinkel vor Objektiv:
		\begin{equation*}
		\alpha \approx \frac{B}{f_1}
		\end{equation*}
		Sehwinkel nach Okular:
		\begin{equation*}
		\beta \approx \frac{B}{f_2}
		\end{equation*}
	\end{minipage}
	\begin{minipage}{.5\linewidth}
		%t6:
		\centering
		\begin{tikzpicture}
		\coordinate (1) at (-2,0);
		\coordinate (2) at (1,0);
		\draw[fill=gray!10] (1) ellipse (.15cm and 1.3cm);
		\draw[fill=gray!10] (2) ellipse (.15cm and 1.3cm);
		\draw (-4,0) -- (2.5,0);
		\draw[dashed] ($ (1) + (0,-1.5) $) -- ++(0,3);
		\draw[dashed] ($ (2) + (0,-1.5) $) -- ++(0,3);
		\coordinate (B) at (0,-1);
		\draw[thick,->] (0,0) -- (B) node[left,yshift=15pt] {$ B $};
		\coordinate (1u) at ($ (1) + (0,.5) $);
		\coordinate (1d) at ($ (1) + (0,-1) $);
		\coordinate (2u) at ($ (2) + (0,.5) $);
		\coordinate (2d) at ($ (2) + (0,-1) $);
		\draw (1u) -- (B) -- (2u);
		\draw (1) -- (B) -- (2);
		\draw (1d) -- (B) -- (2d);
		\draw (1u) -- ++(157.5:1.5cm);
		\draw (1) -- ++(157.5:1.5cm);
		\draw (1d) -- ++(157.5:1.5cm);
		\draw (2u) -- ++(45:1.5cm);
		\draw (2) -- ++(45:1.5cm);
		\draw (2d) -- ++(45:1.5cm);
		\node[above] at (.5,0) {$ f_1 $};
		\node[above] at (-1,0) {$ f_2 $};
		\node[circle,fill=black,inner sep=1pt,minimum size=1pt] at (0,0) {};
		\node[circle,fill=black,inner sep=1pt,minimum size=1pt] at (2,0) {};
		\centerarc[thick,<->](-2,0)(180:157.5:1.25cm);
		\node[left,yshift=4pt,xshift=-15pt] at (-2,0) {$ \alpha $};
		\centerarc[thick,<->](1,0)(0:45.5:1.25cm);
		\node[right,yshift=7pt,xshift=15pt] at (1,0) {$ \beta $};
		\end{tikzpicture}
	\end{minipage}
	$ \Rightarrow $ Vergrößerung $ V = \frac{\beta}{\alpha} = \frac{B}{f_2} \frac{f_1}{B} = \frac{f_1}{f_2} $
	\emph{Bemerkung:}
	\begin{itemize}
		\item Kepler Fernrohr: Bild wird invertiert
		\item Galilei Fernrohr: Sammellinse und Zerstreuungslinse\\
		keine Invertierung und kürzere Bauweise
		\item Fernglas:\\
		\folie{Galilei Fernrohr und Fernrohr mit drei Linsen}
	\end{itemize}
	\item \textbf{Mikroskop: Abstand der Linsen $ \gg $ Summe der Brennweiten $ f_2 + f_1 $}\\
	Gegenstand nah am Objektiv
	
	% t7:
	\begin{center}
	\begin{tikzpicture}
		\coordinate (1) at (-3,0);
		\coordinate (2) at (1,0);
		\draw[fill=gray!10] (1) ellipse (.15cm and 1.3cm);
		\draw[fill=gray!10] (2) ellipse (.2cm and 1.8cm);
		\draw (-5,0) -- (3,0);
		\draw[dashed] ($ (1) + (0,-1.5) $) -- ++(0,3);
		\draw[dashed] ($ (2) + (0,-2) $) -- ++(0,4);
		\coordinate (ZB) at (0,-1);
		\coordinate (B) at ($ (2) + (1,1) $);
		\coordinate (G) at ($ (1) + (-1.5,.5) $);
		\draw[thick,->] (0,0) -- (ZB) node[left,yshift=15pt] {$ ZB $};
		\draw[thick,->] ($ (1) + (-1.5,0) $) -- (G) node[left,yshift=-8pt] {$ G $};
		\draw[thick,->] ($ (2) + (1,0) $) -- (B) node[right,yshift=-15pt] {$ B $};
		\coordinate (1u) at ($ (1) + (0,.5) $);
		\coordinate (1d) at ($ (1) + (0,-1) $);
		\coordinate (2u) at ($ (2) + (0,1) $);
		\coordinate (2d) at ($ (2) + (0,-1) $);
		\draw (G) -- (ZB) -- (B);
		\draw (G) -- (1u) -- (ZB) -- (2u) -- (B);
		\draw (G) -- (1d) -- (ZB) -- (2d) -- (B);
		\coordinate (f1) at ($ (1) + (1,0) $);
		\coordinate (f2) at ($ (1) + (-1,0) $);
		\coordinate (f3) at ($ (2) + (-.5,0) $);
		\coordinate (f4) at ($ (2) + (.5,0) $);
		\draw[thick] ($ (f1) + (45:.1) $) -- ++(-135:.2);
		\draw[thick] ($ (f1) + (-45:.1) $) -- ++(135:.2);
		\draw[thick] ($ (f2) + (45:.1) $) -- ++(-135:.2);
		\draw[thick] ($ (f2) + (-45:.1) $) -- ++(135:.2);
		\draw[thick] ($ (f3) + (45:.1) $) -- ++(-135:.2);
		\draw[thick] ($ (f3) + (-45:.1) $) -- ++(135:.2);
		\draw[thick] ($ (f4) + (45:.1) $) -- ++(-135:.2);
		\draw[thick] ($ (f4) + (-45:.1) $) -- ++(135:.2);
		\node[below,xshift=-5pt] at (f2) {$ f_1 $};
		\node[below,xshift=5pt] at (f4) {$ f_2 $};
		\node[yshift=-55pt] at (1) {$ \substack{\tx{1. Linse}\\\tx{Objektiv}} $};
		\node[yshift=-67.5pt] at (2) {$ \substack{\tx{2. Linse (Lupe)}\\\tx{Okular}} $};
	\end{tikzpicture}
	\end{center}
	\folie{anderes Bildschema Mikroskop} Veränderung durch Änderung der Tubuslänge\\
	Verstärkung durch Objektiv $ V_1 = \frac{B}{G} = \frac{b}{f_1} $\\
	Verstärkung durch Okular $ V_2 = \frac{25 \, \tx{cm}}{f_2} $\\
	$ \Rightarrow $ Vergrößerung: $$ V = V_1 V_2 = \frac{t \cdot 25 \, \tx{cm}}{f_1 f_2} $$
	$ t \approx 28 \, \tx{cm} $ wurde so definiert um den Vergleich von Geräten zu vereinfachen.
\end{enumerate}

% Vorlesung 7.11.18 (8)

\section{Elektronenoptik}

\folie{Elektronen Strahl}\\
\lcom{Die Idee hier ist das man einen Elektronen der von einem $E$-Feld abgelenkt wird als wie eine optische Beugung betrachten...}\\
Elektronenstrahl Ablenkung durch elektrische und magnetische Felder

$$\vec{F} = q \vec{E}$$
$$\vec{F} = q (\vec{v} \times \vec{B})$$
\subsection{Beispiel: Brechungsgesetz für Elektronen}

$$\sin \alpha = \frac{v_{1,x}}{v_1} \qquad \sin \beta = \frac{v_{2,x}}{v_2}$$
el. Feld entlang $x$
$$v_{1,x} = v_{2,x}$$
$$\Rightarrow v_1 \sin \alpha = v_2 \sin \beta$$
\textbf{Definition}
$$\frac{n_2}{n_1} = \frac{v_2}{v_1}$$
Energieerhaltung
$$\frac{1}{2} mv^2_2 = \frac{1}{2} mv_1^2 + eU$$
$$v_1 = \sqrt{\frac{2eU_0}{m}}, \quad v_2 = \sqrt{\frac{2e(U+U_0)}{m}}$$
\frbox{$\Rightarrow$ Brechindex}{$$\frac{n_2}{n_1} = \sqrt{1 + \frac{U}{U_0}}$$}

\subsection{Elektrische Rohrlinse}
\subsection{Magnetische Linsen}

Lange Spule = homogenes Magnetfeld\\
Lorentz Kraft $\Rightarrow$ Spiralbahnen der Elektronen $\Rightarrow$ Bild wird gedreht\\
$\Rightarrow$ kleine Spule = inhomogene Magnetfelder $\Rightarrow$ ?? %% was
\folie{Elektronen in Magnetfeldern}
\subsection{Elektronenmikroskope}
\folie{Elektronmikroskop}
\lcom{Man kann ein Material untersuchen in dem man die Streuelektronen analysiert die bei dem Elektronenmikroskop entstehen.}
\lcom{Bei der SRT muss man nach Gedankenfehler suchen, bei Optik müssen die Näherungen anschauen als Fehlerquelle, Zahnmedizinerinnen würden alle Fehlerpunkte auswendig gelernt haben, während ein Physiker überlegen ''ok, wir haben die geometrische Optik angewandt also alles was da schief gehen kann und so weiter.''}

\chapter{Wellenoptik}
\section{Wiederholung (Schwingungen und Wellen)}
\subsection{Schwingungen}
\folie{Harmonische Schwingungen}
\lcom{Zwei grössen: Amplitude und Phase, und zwei komponenten: inphase, und out of phase}
\subsection{Wellen}
\folie{Harmonische ebene Welle, Polarisation, Kugelwellen}\\
Harmonische Ebene Welle in 1D
\begin{equation*}
A(x,t) = A_0 \cos (\omega t - kx + \varphi)= \Re\left[\hat{A} e^{-i(\omega t - kx)}\right]
\end{equation*}
\lcom{Das Minus in der Komplexen Schreibweise Stammt aus der Wellengleichung. Es kommt auch aus der Lorenzinvarianten. }\\
\lcom{Da EM-Wellen Transversalwellen sind und Ladungserhaltung/Eichinvarianz gilt, kann es nur zwei Polarisationsrichtungen geben. }\\
\folie{Wellengleichung}

\subsection{Fourier-Transformation}
\lcom{Ein Thema das nie richtig behandelt wird, aber wichtig ist, vielleicht leiste ich heute ein Beitrag dazu.}\\
Harmonische Schwingungen:
$$A(t) = A_0 \cos (\omega t + \varphi) = \Re [\hat A_0 e^{-i\omega t}]$$
Periodische Schwingungen: Fourier-Reihe
$$A(t) = A(t + T) = \summ{n = 0}{\infty} A_n \cos(m \omega_0 t + \varphi_n) = \Re \left[\summ{n = 0}{\infty} \hat A_n e^{-i n \omega_0 t}\right]$$
$$\omega_m \leq n \omega_0$$
Beliebige Zeitabhängigkeit: Fourier Transformation
$$A(t) = \intt{-\infty}{+\infty} A(\omega) \cos (\omega t + \varphi(\omega)) d \omega = \Re [\int \hat A (\omega) e^{-i\omega t} d \omega]$$
Bemerkung: \lcom{Wir können zwei Fourier Transformationen machen, nach Ort und nach Zeit.}
\begin{itemize}
	\item Umkehr Transformation: Faktor $2 \pi$ taucht auf
	\item Analog für $e^{i \vec{k} \vec{x}}$
	\item Die Fourier Transformation impliziert eine ''Unschärfe Relation'' \lcom{Eine Eigenschaft der FT}
		$$\Delta \omega t \approx \mathcal O(2 \pi) \qquad \Delta \vec{k} \Delta \vec{x} \approx \mathcal O(2\pi)$$
\end{itemize}
Beispiel: Gedämpfte Schwingung, Zerfall, Relaxtion\\


% relaxtion ???


\noindent
Exponentielles Abklingverhalten
$$A(t) \propto e^{-\frac{t}{\tau}} \quad \Leftrightarrow \quad \textrm{ = Lorentz-Funktion}$$
$\tau$: Zeitkonstante
\folie{Zeitraum/Frequenzraum}\\
\lcom{Ein exponentieller Zerfall ist immer mit der Lorentz-Funktion verbunden. Eine ungedämpfte harmonische Schwingung: $\tau = \infty$. }\\
\lcom{Eine unendlich langer, unendlich langsamer Schwingung ist im Frequenzraum unendlich scharf. Eine ungedämpfte harmonische Schwingung ist also im Frequenzraum unendlich scharf und wird damit zu einer Dirac-Delta-Funktion mit dem Peak bei der Frequenz der Schwingung. Eine solche Welle ist im Ortsraum überall verteilt.}\\
Beispiel: Wellenpaket
ebene Welle: $e(\vec{x},t) = E_0 e^{-i(\omega t - \vec{k} \vec{x})}$
$$\Rightarrow |E|^2 = E_0^2 = \const$$
\folie{Wellenpakete}\\


%Worum geht es hier bei sie???

\noindent
\lcom{Im Wellenvektorraum ist sie jedoch wieder mit einer Delta-Funktion beschrieben. Ein Wellenpaket ist im Ortsraum lokalisiert aber im Wellenvektorraum sehr unscharf verteilt.}\\
\lcom{Die Fouriertransformation einer Gauß-Verteilung (wird oft für Wellenpakete verwendet) ist wieder eine Gauß-Verteilung. Die Fouriertransformation einer Exponentiell Abfallenden Welle ist eine Lorenz-Funktion.}
\subsection{Kohärenz}
\begin{description}
	\item[Definition] Zwei oder mehr Wellen sind Kohärent, wenn die Zeitabhängigkeit der Auslenkungen bis auf eine Phasendifferenz der gleiche ist\mau
\end{description}
Viele gründe warum Wellen nicht kohärent sind:
\begin{itemize}
	\item spontan emittiertes Licht vielr unabhängiger Lichtemmiter (heiße Atome)
	\item zwei Lichtquellen schwanken leicht in Frequenz Amplitude, Phase (nicht stabil)
	\item zu ausgedehnte Lichtquelle
\end{itemize}
Gegenbeispiele:
\begin{itemize}
	\item stimulierte Emission: einfallendes Licht zwingt angeregte Atome synchron mit dem Lichtfeld zu Schwingen und abzustrahlen.
\end{itemize}
Kohärenzlänge:\\
\lcom{Kohärenzlänge: Keine Lichtquelle kann unendlich kohärentes Licht erstellen. Die minimal unterschiedlichen Frequenzen sorgen dafür, dass das Licht nach einer Weile nicht mehr kohärent ist. Die Flugweite der Lichtstrahlen bis die Kohärenz nicht mehr gegeben ist, nennt man Kohärenzlänge.}\\
Kohärenzzeit:\\
\lcom{Die dauer in der Kohärentes Licht nicht nicht mehr kohärent wird nennt man Kohärenszzeit. }
Kohärenzzeit = Kohärenzlänge/$c$\\
Experimente: 
\begin{enumerate}[(i)]
	\item Licht mit genügend großer Kohärenzlänge
	\item Licht aus \textbf{einer} Quelle aufteilen, wenn mehrere Kohärente Strahlen benötigt werden.
\end{enumerate}


% 13.11.18

\section{Inteferenz und Beugung I}
\subsection{Inteferenz am Doppelspalt}

Berechnung der Maxima und Minima
\begin{description}
	\item[Gangunterschied] $x = d \sin \alpha$
\end{description}
Maxima wenn $x$ ein Vielfaches von $\lambda$ ist (konstruktive inteferenz)
\begin{itemize}
	\item[$\rightarrow$] $x = n \lambda$
	\item[$\Rightarrow$] Maxima: $\sin\alpha = n \frac{\lambda}{d}$
	\item[$\Rightarrow$] Minima: $\sin\alpha = \left(n + \frac{1}{2}\right)\frac{\lambda}{d}$
\end{itemize}

Triviale aber Wichtige Feststellung: nicht nur $d$, nicht nur $\lambda$, sondern $\frac{\lambda}{d}$\\
\emph{Bemerkung: } $$k= \frac{2\pi}{\lambda}$$
$$\Rightarrow \quad kx = 2 \pi n$$
$$\Delta f = 2\pi n = kx = \frac{2 \pi}{\lambda}x$$
Feststellung: Ortsraum $k$-Raum reziprok 


% f = varphi a ? = alpha

\subsection{Beugung am Einzelspalt}

Brechung der Intensitätsverteilungen $ I(\alpha) $, in 6 Schritten

\begin{enumerate}[(1)]
	\item Strahl über Spalt wird in $ N $ Teilstrahlen zerlegt im Abstand: $ \Delta b = \frac{b}{N} $
	\item Amplitude eines Teilstrahls = $ a = \frac{\sqrt{I_0}}{N} $
	\item Gangunterschied benachbarter Strahlen: $ x = \Delta b \sin \alpha $
	\item Phasenunterschied benachbarter Strahlen: $ \Delta \varphi = k x = k \Delta b \sin \alpha  $
	\item Gesamtamplitude $ \widehat{=} $ Summe der $ N $ Teilstrahlen
	\begin{equation*}
	\Rightarrow \quad A = a \sum_{n=0}^{N-1} e^{-i(\omega t + f_n)} = a e_{-i \omega t} \sum_{n=0}^{N-1} e^{i n \Delta \varphi}
	\end{equation*}
	$ f_n $ ist die Phase des $ n $-ten Strahls
	\begin{equation*}
	A = e^{-i\omega t} \frac{e^{i N \Delta \varphi}}{e^{i \Delta \varphi}} = a e^{- i \omega t} e^{i \frac{N \Delta \varphi}{2}} \frac{\sin\left(\frac{N}{2} \Delta \varphi\right)}{\sin\left(\frac{1}{2} \Delta \varphi\right)}
	\end{equation*}
	\begin{equation*}
	\Rightarrow \quad I(\alpha) = A^2 = a^2 \frac{\sin^2\left(\frac{N}{2} \Delta \varphi \right)}{\sin^2 \left(\frac{1}{2} \Delta \varphi \right)} = a^2 \frac{\sin^2 (x)}{\sin^2 \left(\frac{x}{N}\right)} \qquad x \defeq \pi \frac{b}{\lambda} \sin \alpha
	\end{equation*}
	\item Grenzübergang $ \Delta b \rightarrow 0 $ oder $ N \rightarrow \infty $:
	\begin{equation*}
	\Rightarrow \quad \sin^2(\frac{x}{N}) = \frac{x^2}{N^2}
	\end{equation*}
	\begin{equation*}
	\Rightarrow \quad I(\alpha)  = N^2 a^2 \frac{\sin^2(x)}{x^2}
	\end{equation*}
	\textbf{Lösung:}
	\begin{equation*}
	\rmbox{I(x) = I_0 \frac{\sin^2(x)}{x^2}} \qquad x \defeq \pi \frac{b}{\lambda} \sin \alpha
	\end{equation*}
\end{enumerate}
\textbf{Diskussion:}\\[5pt]
Hauptmaxima: $ x = 0 \quad \rightarrow \quad \alpha = 0 $\\
Minima: $ \sin x = 0 \quad \rightarrow \quad x = n \pi \quad \Rightarrow \quad \sin \alpha = n \frac{\lambda}{b} $\\
Nebenmaxima: $ |\sin \alpha| = 1 \quad \rightarrow \quad x = \left(n + \frac{1}{2}\right) \pi \quad \Rightarrow \quad  \sin \alpha = \left(n + \frac{1}{2}\right) \frac{\lambda}{b} $\\[5pt]
\lcom{Die Gleichungen sind unabhängig vom Abstand des Schirms. Die stammt aus der Annahme es handelt sich um Ebene Wellen, es nur die nur im Fernbereich gibt. Daher gelten unsere Gleichungen und Annahmen auch nur im Fernbereich. Ist der Abstand zum Schirm nicht Wesentlich größer als die Spaltbreite oder der Abstand der Spalten, gelten sie nicht mehr. Mehr dazu im Abschnitt Interferenz und Beugung II.}\\[5pt]
ACHTUNG:\\
Implizit Annahme ebener Wellen $ \Rightarrow $ Rechnung nur richtig im Fernbereich \mau

\subsection{Beugung am Gitter}

\textbf{Definition:}\\
Gitter: Regelmäßige Anordnung von ,,Strichen``
\folie{Licht am optische Gitter} (wieder Annahme ebener Wellen)\\
Brechungs
\begin{enumerate}[(1)]
	\item Überlagerung von $ N $ Strahlen im Abstand $ d $
	\item Amplitude eines Strahls $ a = \sqrt{I_0} $
	\item Gangunterschied von benachbarten Strahlen $ x = d \sin \alpha $
\end{enumerate}	
Weiter wie beim Einzelspalt:
\begin{enumerate}[(4)]
	\item[(4)] Phasenunterschied $ \Delta \varphi = k x = k d \sin \alpha $
	\item[(5)] Gesamtamplitude $ \widehat{=} $ Summe der Teilstrahlen
	\begin{equation*}
	\Rightarrow \quad A = a e^{i \omega t} \sum_{n=0}^{N-1} e^{i n \Delta \varphi}
	\end{equation*}
	\begin{equation*}
	\Rightarrow \quad A(\alpha) = A_0 \frac{\sin \left(\pi \frac{N d}{\lambda} \sin \alpha \right)}{\sin \left( \pi \frac{d}{\lambda} \sin \alpha \right)} \qquad A_0: \tx{ geeignet definiert}
	\end{equation*}
	\begin{equation*}
	I(\alpha) = A^2
	\end{equation*}
	\folie{Interferenzmuster in Abhängigkeit der Spaltenzahl}
\end{enumerate}
\textbf{Diskussion:}\\[5pt]
Hauptmaxima: Nenner klein $ \pi \frac{d}{\lambda} \sin \alpha = m \pi $
\begin{equation*}
\Rightarrow \qquad \sin \alpha = m \frac{\lambda}{d}
\end{equation*}
Nebenmaxima: Zähler klein $ \Rightarrow \pi \frac{N d}{\lambda} \sin \alpha = m \pi $
\begin{equation*}
\Rightarrow \qquad \sin \alpha = m \frac{\lambda}{N d}
\end{equation*}
Breite der Hauptmaxima:\\
Verhält sich wie der Abstand benachbarter Nebenmaxima
\begin{equation*}
\Rightarrow \qquad B = \frac{2 \lambda}{N d}
\end{equation*}
Dispersion der Hauptmaxima ($ \lambda $ bzw. $ \omega $ Abhängigkeit)\\
Lage: $ \sin \alpha = m \frac{\lambda}{b} $
\begin{equation*}
\Rightarrow \quad \frac{\dd \alpha}{\dd \lambda} = \frac{m}{d} \arccos \alpha \quad \Rightarrow \quad \frac{\dd \alpha}{\dd \lambda} > 0
\end{equation*}

\lcom{Wenn ihr eine scharfe Linie wollt sollte euch sofort Gitter einfallen. }

\subsection{Beugung am Raum- oder Oberflächengitter}

\textbf{Definition:}\\
Raumgitter: Kristall $ \widehat{=} $ regelmäßige Anordnung von Atomen im 3D Raum \folie{Diamant Kristallgitter}
\begin{equation*}
\Rightarrow \quad 2 d \sin \alpha = m \lambda \qquad \tx{Bragg-Bedingung}
\end{equation*}
$ d $ ist der Abstand von zwei Kristall-Ebenen oder Netzebenen 
\versuch{Doppelspalt}
% unten angstöm ändern evtl

\noindent
Typischer Zahlenwert:
\begin{equation*}
d \approx 1 \, \overset{\circ}{\tx{A}} \quad \Rightarrow \quad \lambda \approx 1 \, \overset{\circ}{\tx{A}} \quad \approx \quad \tx{Röntgenstrahlung oder X-Ray}
\end{equation*}
\versuch{Gitter}
\versuch{Kristall Struktur Messung it Röntgenstrahlung}\\
Anwendung von Kristallen: Röntgenstrukturanalyse mittels Röntgendefraktometer\\
X-Ray: \lcom{Die Wechselwirkung von Röntgenstrahlung mit Materie ist die mit den Elektronen also der Lorentz-Lorenz-Oszillator $ \vec{F} = q \vec{E} $. Die Röntgenstrahlung Wechselwirkt also mit den Ladungen. Was wir beobachten ist also die Lage der Elektronen also nur indirekt die Lage der Atome.}


% Vorlesung 10

% 14.11.18

% andrez comment 1
\noindent
\subsubsection{\emph{Beispiel:} \textbf{Srukturanalyse von Kristallen}}
\begin{itemize}
	\item X-ray: $ \vec{F} = q \vec{E} $
	\begin{itemize}
		\item Streuung an der Elektronenhülle
		\item Bestimmung der Elektronenladungshülle
	\end{itemize}
	\item Neutronen $n$: starke WW
	\begin{itemize}
		\item Streuung am Kern
		\item Bestimmung der Kernpositionen
	\end{itemize}
	\item Spin: magnetische Dipol-Dipol-WW
	\begin{itemize}
		\item Streuung an Elektronenspins
		\item $ \Rightarrow $ Mangetismus
	\end{itemize}
\end{itemize}
\subsubsection{\emph{Beispiel:} \textbf{Untersuchung von Oberflächen LEED} (Low Energy Electron Diffraction)}
\begin{itemize}
	\item $ e^- $:
	\begin{itemize}
		\item Streuung an der Elektronenhülle
		\item Eindringtiefe $ \approx 100 \, \overset{\circ}{\tx{A}} \Rightarrow $ Oberflächenanalyse
	\end{itemize}
\end{itemize}
\lcom{Die \textbf{Energie von Raumtemperatur} ist ungefähr \underline{$ \frac{1}{40} \, e \tx{V} $}. Sollte man Wissen.}\\
\folie{zu LEED}

\subsection{Interferenz an planparallelen Glasplatten}
\folie{zu Reflexion an einer Glasplatte}\\
\subsubsection{Reflexion}
\begin{enumerate}[(1)]
	\item Länge der Wege $ \ol{AB} $ und $ \ol{BC} $
	\begin{equation*}
	x_{AB} = \frac{d}{\cos \beta} \quad x_{BC} = \frac{d}{\cos \beta}
	\end{equation*}
	\item Weg $ \ol{AD} $
	\begin{equation*}
	x_{AD} = \ol{AC} \sin \alpha = 2 d \tan \beta \sin \alpha
	\end{equation*}
	\item Gangunterschied
	\begin{equation*}
	x = n \left(x_{AB} + x _{BC}\right) - x_{AB}
	\end{equation*}
	$ n $: Brechungsindex der Glasplatte\\
	mit $ \sin \alpha = n \sin \beta $ gilt:
	\begin{equation*}
	x = 2 d n \cos \beta = 2 d \sqrt{n^2 - \sin^2 \alpha}
	\end{equation*}
	\item Phasenunterschied\\
	Phasenunterschied aufgrund des Gangunterschieds $ \Delta \varpi = kx = \frac{2\pi}{\lambda} x $\\
	$ \approx $ senkrechter Einfall:\\
	Phasensprung um $ \pi $ bei der Reflexion am optisch dichteren Medium \color{red!75!black} \mau \color{black}
	\begin{equation*}
	\Rightarrow \Delta \varphi = k x  + \pi
	\end{equation*}
	\item Für konstruktive Interferenz (Maximua): $ \Delta \varphi = m 2 \pi $
	\begin{equation*}
	\Rightarrow \qquad 2 d \sqrt{n^2 - \sin^2 \alpha} = \left(m + \frac{1}{2}\right) \lambda
	\end{equation*}
\end{enumerate}
\subsubsection{Transmission}
\begin{enumerate}[(1)]
	\item Gangunterschied wie zuvor
	\begin{equation*}
	x = n (x_{AB} + x_{BC}) - x_{AD} = 2 d \sqrt{n^2 \sin^2 \alpha}
	\end{equation*}
	\item Phasenunterschied\\
	Gangunterschied $ \Rightarrow \Delta \varphi = kx $\\
	Phasensprünge $ \Rightarrow 0 $
	\begin{equation*}
	\Rightarrow \qquad \Delta \varphi = kx
	\end{equation*}
	\item Maxima:
	\begin{equation*}
	\Rightarrow \qquad 2d\sqrt{n^2 \sin^2 \alpha} = m \lambda
	\end{equation*}
\end{enumerate}
Test:\\
Annahme: keine Absorption in der Glasplatte: \lcom{Die Intensitäten der Reflexion und die der Streuung sollten sich zu Gesamtintensität aufsummieren. }
\begin{equation*}
\Rightarrow \tx{ zu erwarten: } \quad I = I_{\tx{reflektiert}} + I_{\tx{transmittiert}}
\end{equation*}
Wie erwartet da wo die Transmission Maximal ist ist die Reflexion minimal und umgekehrt. Immer um $ \frac{\lambda}{2} $ verschoben.
\subsubsection{Anwendung}
\begin{itemize}
	\item \folie{Fres'nelscher Doppelspiegel}
	\item \folie{Newton's Ringe}
\end{itemize}

\subsection{Vielstrahlinterferenz, Fabry-Parot Interferometer}

\folie{Vielstrahlinterferenz an Planparallelen Platten}\\
\lcom{Die Interferenz ähnelt der bei einem Gitter. Der große Unterschied liegt darin, dass die Intensität der Teilstrahlen mit zunehmenden Reflexionen abnimmt. }\\
Reflexion (Transmission aus $ I = I_{\tx{ref}} + I_{\tx{trans}} $)
\begin{enumerate}[(1)]
	\item Intensitäten der jeweiligen Teilstrahlen berücksichtigen\\
	$ \Rightarrow $ Reflexionskoeffizient $ R $
	
	
	% make align:
	
	
	
	\begin{equation*}
	|A_1| = \sqrt{R} |A_0| \quad |B_1| = \sqrt{1 - R} |A_0| \quad |D_1| = \sqrt{1-R} |A_0| = (1-R) |A_0|
	\end{equation*}
	\begin{equation*}
	|C_1| = \sqrt{R} |B_1| = \sqrt{R}\sqrt{1 - R} |A_0| \quad |A_2| = \sqrt{1 - R} |C_1| \quad |B_1| = \sqrt{R} |C_1| = R \sqrt{1 - R} |A_0|
	\end{equation*}
	\begin{equation*}
	\Rightarrow \qquad |A_{m+1}| = \sqrt{1-R} |C_m| = \sqrt{1-R} \sqrt{R} |B_m| = \sqrt{1-R} \sqrt{R} \sqrt{R} |C_{m-1}| = R |A_m|
	\end{equation*}
	\begin{equation*}
	\Rightarrow \qquad |D_{m+1}| = R |D_m|
	\end{equation*}
	\item Gangunterschied benachbarter Strahlen (siehe Platte)
	\begin{equation*}
	x = 2d \sqrt{n^2 \sin^2 \alpha} 
	\end{equation*}
	\item Phasenunterschied
	\begin{equation*}
	\Delta \varphi = \custo{\leftarrow}{kx}{\mathclap{\tx{Gangunterschied}}} + \tx{Phasensprünge}
	\end{equation*}
	Bei den Phasensprüngen folgt nach längerer Arbeit, dass nur der Phasensprung $ A_1 $ übrig bleibt.
	\item Gesamtamplitude
	\begin{equation*}
	A = \color{red} \pm \color{black} \sum_{m=1}^{N} A_m e^{i(m-1) \Delta \varphi}
	\end{equation*}
	$ N $: Anzahl der betrachteten Teilstrahlen\\
	$ \color{red} \pm \color{black} $: Berücksichtigt die Phasensprünge
	\item Grenzübergang $ N \to \infty $ (unendlichlange Glasplatte):
	\begin{equation*}
	\Rightarrow \qquad A = \pm A_0 \sqrt{R} \frac{1 - e^{i \Delta \varphi}}{1 - R e^{i \Delta \varphi}}
	\end{equation*}
	\begin{equation*}
	\Rightarrow \rmbox{I_R = |A^2| = 2 I_0 R \frac{1 - \cos \Delta \varphi}{1 - 2 R \cos \Delta \varphi + R^2}}
	\end{equation*}
\end{enumerate}

\subsubsection{Ergebnis}

Fabry Airy ??? Formeln

%big box 


\begin{equation*}
I_R = I_0 \frac{F \sin^2 \left(\frac{\Delta \varphi}{2}\right)}{1 + F \sin^2\left(\frac{\Delta \varphi}{2}\right)}
\end{equation*}

\begin{equation*}
Hier fehlt eine Gleichung: I_T = I_0 XXX\frac{XXX}{1 + F \sin^2 \left(\frac{\Delta \varphi}{2}\right)}
\end{equation*}
% end box

\begin{equation*}
F = \frac{4 R}{(1-R)^2} \qquad \frac{\Delta \varphi}{2} = 2 \pi \frac{d}{2} \sqrt{n^2 - \sin^2 \alpha}
\end{equation*}
\emph{Beispiel:} senkrechter Einfall
\begin{equation*}
\Rightarrow \alpha = 0 \qquad I_T = I_0 \frac{1}{1 + F \sin^2 (n d k)}
\end{equation*}
wobei $ k = \frac{2 \pi}{\lambda} $
Maxima: $$ 2 n d = m \lambda $$
FWHM (full width at half maximum)
$$ \epsilon = 4 \arcsin\sqrt{\frac{1}{F}} \approx \frac{4}{\sqrt{F}} $$
\folie{Maxima beim FWHM}\\
\textbf{Anwendung: Fabry Parot Interferometer}\\[5pt]
\folie{zu Fabry-Parot Interferometern}\\[5pt]

Versuche

\versuch{Natrium-Dampf-Lampe Interferenz}
\folie{Interferenz mit einer Natrium-Dampf-Lampe}
\versuch{Abpumpen der Luft beim Interferenzmuster der Natrium-Dampf-Lampe}
\lcom{Dar Muster verändert sich erheblich obwohl der Unterschied der Brechungsindizes so gering war. ALso hat das Farbinterferometer eine sehr hohe Auflösung. }



Versuche :

Seifenblase

interferenz an planparalleser Folie

interferenz mit doppelspiegel Fres'nelscher Doppelspiegel






%\bibliographystyle{plain}
%\bibliography{literature}
%\addcontentsline{toc}{section}{Literatur}

\end{document}

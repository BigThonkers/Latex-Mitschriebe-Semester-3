\setcounter{chapter}{-1}
\chapter{Einführung}

\begin{description}
	\item[Dozent] Prof. Oliver Waldmann, Zi. 202, Physik-Hochhaus
	\item[Zeit] Di, 8-10 Uhr, Mi, 8-10 Uhr
	\item[Ort] Großer Hörsaal der Physik
 \end{description}

\section{Termine}

\begin{description}
	\item[Vorlesungsbeginn] Di., 16.10.2018
	\item[Erstes Übungsblatt] Mi., 17.10.2018
	\item[Übungsbeginn] Mo., 29.10.2018 - Fr., 2.11.2018
	\item[Letzte Vorlesung] Mi., 6.2.2019
	\item[Klausur] Sa., 9.2.2019, 9:00 - 12 Uhr
	\item[Wiederholungsklausur]	Sa., 27.4.2019., 10:00-13 Uhr
\end{description}

\section{Programm}

\begin{enumerate}
	\item Spezielle Relativitätstheorie
	\item Fortgeschrittene Optik
	\item Quantenphysik
	\item Einfache atomare Systeme
\end{enumerate}

\section{Literatur}

\begin{itemize}
	\item alle Lehrbücher
	\item Demtröder, Experimentalphysik 3 (Springer)
	\item Tipler/Mosca, Physik (Elsevier)
	\item Bergmann/Schaefer, Lehrbuch der Experimentalphysik, Band 3
	\item Gerthsen, Physik (Springer)
	\item Giancoli, Physik (Pearson)
\end{itemize}

\section{Übungen und Betreuer}
 
Übungsleiter: Krunoslav Prša, Zi. 203, Physik-Hochhaus, Tel. 7631\\
Gruppe 3: Mi 10-12, SR III, Tutor: Fabian Thielemann\\
Gruppe 7: Fr 14-16, SR GMH, Tutor: Rupert Michiels
 
\section{Übungsblätter}

\begin{itemize}
	\item Ausgabe der Übungsblätter jeweils am ENDE der Vorlesung am Mittwoch im großen Hörsaal, oder online auf ILIAS (siehe Ordner "Übungsblätter" unten)
	\item Rückgabe der schriftlichen Lösungen VOR der Vorlesung am Mittwoch, die Lösungen sind im Hörsaal vorne auf den Tisch zu legen
	\item jedes Übungsblatt umfasst typischerweise 4 - 5 Aufgaben mit insgesamt ca. 35 Punkten
	\item jedes Übungsblatt enthält typischerweise eine leichte und eine schwere Aufgabe
\end{itemize}

\section{Übungsgruppen}

\begin{itemize}
	\item bis zu zwei (2) Studierende können ein Lösungsblatt gemeinsam bearbeiten und abgeben (sie müssen dann aber in einer Übungsgruppe sein)
	\item bitte DEUTLICH auf dem Lösungsblatt angeben: Namen der Studierenden, Übungsgruppe (Nr, Name des Übungsleiters, Ort, Zeit)
	\item Gruppenarbeit ist erwünscht, es wird aber erwartet, dass jede(r) Studierende die Aufgaben vorrechnen kann, sonst Punktabzug :-)
	\item Einschreibung in die Übungsgruppen über ILIAS von Di., 17.10, 20:00 Uhr bis Do., 19.10, 20:00 Uhr (siehe unten)
	\item Anwesenheitspflicht \& Vorrechenpflicht!
\end{itemize}

\section{Klausur}

\begin{description}
	\item[Zuslassung zur Klausur] 40\% der Übungspunkte
	\item[Termin] Sa, 9. Februar 2019, 9:00 - 12 Uhr, Großer Hörsaal Physik
	\item[Dauer] 120 Minuten
	\item[Inhalt] Vorlesungsstoff + Übungen, Aufgaben sind ähnlich zu den Übungen gestellt
	\item[Sprache] Die Klausuraufgaben werden in Deutsch zur Verfügung gestellt
	\item[Erlaubt] Din-A4 Blatt mit eigenen Notizen beidseitig beliebigbeschrieben/bedruckt (spezielle Lesehilfen sind nicht erlaubt!), Stifte, Geodreieck/Lineal
	\item[Nicht erlaubt] Es ist alles verboten was nicht erlaubt ist (Taschenrechner, Handy, Bücher, usw.) (leere geheftete Blätter werden verteilt)
	\item[Mitzubringen] Studierendenausweis, Schreibstifte, Geodreieck/Lineal
\end{description}
 
\subsubsection{Anmeldung zur Klausur}

Online Anmeldung, der Ablauf und Termine werden durch die jeweiligen Prüfungsämter bekannt gegeben.
 
\section{Prüfungsleistung}

\begin{itemize}
	\item mindestens 40\% der Übungspunkte werden für die Zulassung zur Klausur benötigt
	\item bestandene schriftliche Klausur oder Nachklausur
	\item Bewertungsgrundlage ist das Ergebnis der Klausur oder Wiederholungsklausur
\end{itemize}
Täuschungsversuche:\\
Sowohl die Übungen wie die Klausur sind Teil der Prüfungs/Studienleistung! Täuschungsversuche können zum Nicht-Bestehen führen!
\PassOptionsToPackage{dvipsnames}{xcolor}
\documentclass[titlepage,11pt,a4paper,ngerman]{report}
\usepackage[utf8]{inputenc}
\usepackage[T1]{fontenc}
\usepackage[german]{babel}
\usepackage{graphicx}
\usepackage{wrapfig}
\usepackage{amsmath}
\usepackage{amsfonts}
\usepackage{amssymb}
\usepackage{tikz}
\usepackage{tikz-cd}
\usepackage{nicefrac}
\usepackage{mathtools}
\usepackage{enumerate}
\usepackage{cancel}
\usepackage[hidelinks]{hyperref}
\usepackage{cleveref}
\usepackage{tcolorbox}

\usepackage[margin=1in]{geometry}

\usepackage{placeins}
\usepackage{booktabs}
\usepackage{wasysym}

\usepackage{url}

\usetikzlibrary{calc}
\usetikzlibrary{decorations.pathmorphing,patterns}
\usetikzlibrary{arrows}
\usetikzlibrary{decorations.pathreplacing}


%Environments und Newcommands:

% Allgemein:

% zu zeigen symbol
\newcommand{\zz}{\fontfamily{cmss} \selectfont{Z\kern-.61em\raise-0.7ex\hbox{Z}:}}
% build over
\newcommand{\bov}[2]{\buildrel{#2} \over{#1}}
% better looking := (defined as)
\newcommand*{\defeq}{\mathrel{\vcenter{\baselineskip0.5ex \lineskiplimit0pt \hbox{\scriptsize.}\hbox{\scriptsize.}}}=}
\newcommand*{\eqdef}{=\mathrel{\vcenter{\baselineskip0.5ex \lineskiplimit0pt \hbox{\scriptsize.}\hbox{\scriptsize.}}}}
% integral differential d
\newcommand{\dif}{\mathop{}\!\mathrm{d}}
\newcommand{\difi}[1]{\mathrm{d}#1\mathop{}\!}

\newcommand{\hfw}{\color{RubineRed}\tx{ $\star$hier fehlt was$\star$ } \color{black}}
\def\checkmark{\tikz\fill[scale=0.4](0,.35) -- (.25,0) -- (1,.7) -- (.25,.15) -- cycle;} 

%Text:
\newcommand{\tx}[1]{\textrm{#1}}
\newcommand{\const}{\tx{const.}}

\newcommand{\ul}[1]{\underline{#1}}
\newcommand{\ol}[1]{\overline{#1}}
\newcommand{\ub}[1]{\underbrace{#1}}
\newcommand{\ob}[1]{\overbrace{#1}}

\newcommand{\dd}{\tx{d}}

%Mathe:
\newcommand{\verteq}{\rotatebox{90}{$\,=$}}
\newcommand{\equalto}[2]{\underset{\scriptstyle\overset{\mkern4mu\verteq}{#2}}{#1}}
\newcommand{\equaltoup}[2]{\overset{\scriptstyle\underset{\mkern4mu\verteq}{#2}}{#1}}
\newcommand{\custo}[3]{\underset{\scriptstyle\overset{\mkern4mu\rotatebox{-90}{$\,#1$}}{#3}}{#2}}
\newcommand{\custoup}[3]{\overset{\scriptstyle\underset{\mkern4mu\rotatebox{-90}{$\hspace{-3pt} #1$}}{#3}}{#2}}
\newcommand{\casess}[4]{\left\{ \begin{array}{ll} {#1} & {#2} \\ {#3} & {#4} \end{array} \right.}


%Spezielles:

%Theo:
\newcommand{\lag}{\mathcal{L}}
\newcommand{\ham}{\mathcal{H}}
\newcommand{\gre}{\mathcal{G}}
\newcommand{\prt}[2]{\frac{\partial #1}{\partial #2}}
\newcommand{\prd}[2]{\frac{\tx{d} #1}{\tx{d} #2}}
\newcommand{\eofr}{\vec{E}(\vec{r})}
\newcommand{\pofr}{\Phi(\vec{r})}
\renewcommand{\Phi}{\varPhi}
\newcommand{\delfkt}{\delta(\vec{r} - \vec{r}_0)}
\newcommand{\grr}{\mathcal G(\vec{r},\vec{r}')}

%LA:
\newenvironment{bew}[1]{\subsection{Bew: #1}}{\hfill$\square$}
\newcommand{\Bew}[2]{\begin{bew}{#1}#2\end{bew}}

\newcommand{\enph}{F: V \to V \textrm{ Endomorphismus}}
\newcommand{\im}{\tx{im}}
\newcommand{\spa}{\tx{span}}
\newcommand{\adj}{\tx{adj}}
\newcommand{\grad}{\tx{grad}}
\newcommand{\ord}{\tx{ord}}

\newcommand{\basis}[3]{\{#1_{#2}, \dots, #1_{#3}\}}
\newcommand{\ska}[2]{\langle #1 , #2 \rangle}
\newcommand{\dmat}[3]{\begin{pmatrix} #1_{#2}&&\\ &\ddots& \\ && #1_{#3} \end{pmatrix}}

%Ex:
\newcommand{\kq}{\frac{1}{4\pi\epsilon_0}}
\newcommand{\uind}{U_{\tx{ind}}}
\newcommand{\folie}[1]{\color{gray}[Folie: #1]\color{black}}
\newcommand{\versuch}[1]{\color{red!50!black} Versuch: \color{black} #1\\ }

% ANDREZ
\newcommand{\summ}[2]{\sum_{#1}^{#2}}
\newcommand{\intt}[2]{\int_{#1}^{#2}}
\renewcommand{\vec}[1]{\boldsymbol{#1}}
\newcommand{\lcom}[1]{\color{MidnightBlue}#1\color{black}}
\renewcommand{\epsilon}{\varepsilon}
\newcommand{\vabla}{\boldsymbol{\nabla}}
\newcommand{\bei}{\emph{Beispiel: }}
\newcommand{\bem}{\emph{Bemerkung:}}
\renewcommand{\paragraph}[1]{\subsubsection{#1}}

\newcommand{\mau}{$\buildrel \mathcal{O} \over{\textbf{.}}$}

% Boxen:

\tcbuselibrary{theorems}

% mahlt eine box nur um den text mit tittel
\newtcbox{\fribox}[1]{nobeforeafter,colback=white,colframe=red!75!black,fonttitle=\bfseries,title=#1,sharp corners,tcbox raise base}

% mahlt eine große box um alles mit tittel
\newcommand{\frbox}[2]{\begin{tcolorbox}[colback=white,colframe=red!75!black,fonttitle=\bfseries,title=#1]#2\end{tcolorbox}}

% mahlt eine box nur um den text
\newtcbox{\ribox}{nobeforeafter,colback=white,colframe=red!75!black,sharp corners,tcbox raise base}

% mahlt eine große box um alles was drinnen ist
\newcommand{\rbox}[1]{\begin{tcolorbox}[colback=white,colframe=red!75!black]#1\end{tcolorbox}}

% mahlt eine box um mathe innerhalb mathmode
\newcommand{\rmbox}[1]{\tcboxmath[colback=white,colframe=red!75!black]{#1}}

% super box (looks like regular boxed but wraps around anything)
\newenvironment{supbox}{\begin{tcolorbox}[colback=white,colframe=black,sharp corners,boxrule=.5pt]}{\end{tcolorbox}}

\newcommand{\bbb}[2]{\begin{tcolorbox}[colback=white,colframe=black,fonttitle=\bfseries,title=#1,sharp corners,tcbox raise base]#2\end{tcolorbox}}

% array type box with title
\newenvironment{zebox}[1]{\begin{array}{|c|}
		\multicolumn{1}{l}{\tx{#1}} \\
		\hline
		\displaystyle
	}{\\ \hline
\end{array}}

% evtl: \renewcommand{\boxed}{\rmbox}

\def\centerarc[#1](#2)(#3:#4:#5)% Syntax: [draw options] (center) (initial angle:final angle:radius)
{ \draw[#1] ($(#2)+({#5*cos(#3)},{#5*sin(#3)})$) arc (#3:#4:#5); }

\tikzset{
	annotated cuboid/.pic={
		\tikzset{%
			every edge quotes/.append style={midway, auto},
			/cuboid/.cd,
			#1
		}
		\draw [every edge/.append style={pic actions, densely dashed, opacity=.5}, pic actions]
		(0,0,0) coordinate (o) -- ++(-\cubescale*\cubex,0,0) coordinate (a) -- ++(0,-\cubescale*\cubey,0) coordinate (b) edge coordinate [pos=1] (g) ++(0,0,-\cubescale*\cubez)  -- ++(\cubescale*\cubex,0,0) coordinate (c) -- cycle
		(o) -- ++(0,0,-\cubescale*\cubez) coordinate (d) -- ++(0,-\cubescale*\cubey,0) coordinate (e) edge (g) -- (c) -- cycle
		(o) -- (a) -- ++(0,0,-\cubescale*\cubez) coordinate (f) edge (g) -- (d) -- cycle;
		\path [every edge/.append style={pic actions, |-|}]
		%(b) +(0,-5pt) coordinate (b1) edge ["\cubex \cubeunits"'] (b1 -| c)
		%(b) +(-5pt,0) coordinate (b2) edge ["\cubey \cubeunits"] (b2 |- a)
		%(c) +(3.5pt,-3.5pt) coordinate (c2) edge ["\cubez \cubeunits"'] ([xshift=3.5pt,yshift=-3.5pt]e)
		;
	},
	/cuboid/.search also={/tikz},
	/cuboid/.cd,
	width/.store in=\cubex,
	height/.store in=\cubey,
	depth/.store in=\cubez,
	units/.store in=\cubeunits,
	scale/.store in=\cubescale,
	width=10,
	height=10,
	depth=10,
	units=cm,
	scale=.1,
}

\hbadness=99999

\begin{document}

%\renewcommand{\thechapter}{\Roman{chapter}}

\title{
	{\Huge Theoretische Physik II\\[3pt]Elektrodynamik}\\[1em]
	{\Large Vorlesung von Prof. Dr. Michael Thoss im Wintersemester 2018}}
\author{Markus Österle \hspace{5pt} Andréz Gockel}
\date{ \today}%15.10.2018}
\maketitle
\tableofcontents

\setcounter{chapter}{-1}
\chapter{Einführung}

%\addcontentsline{toc}{section}{section die nur im Inhaltsverzeichniss ist}
%\bibliographystyle{plain}
%\bibliography{literature}
%\addcontentsline{toc}{section}{Literatur}

\section{Zur Vorlesung}

\begin{description}
	\item[Dozent] Michael Thoss
	\item[Übungen] Donnerstag/Freitag (ILIAS) beginnt 18./19.10.18
	\item[Übungsleiter] Jakob Bätge
	\item[Abgabe der Hausaufgaben] bus Dienstag 12:00 - Briefkasten GuMi
	\item[Klausur] 13.02.19, 10-12 Uhr, Hörsaal Anatomie (Nachklausur: 26.19, 10-12 Uhr)
	\item[Ankündigungen] ILIAS Pass: theophy2.thoss18
	\item[Angaben] Vorlesung: 4 SWS, Übung: 2 SWS, ECTS: 7 
	\item[Vorkenntnisse] Mathematik: Analysis für Physiker (Vektor Rechnung), Theoretische Physik I, Experimental Physik II. 
\end{description}

\begin{tcolorbox}[colback=white,colframe=black,fonttitle=\bfseries,title=Hinweis zu den Übungen,sharp corners,tcbox raise base]
	\begin{itemize}
		\item[-] Keine Anwesenheitspflicht.
		\item[-] Keine Punktzahl nötig für Klausurzulassung.
		\item[-] Kann auch wehrend Übungen abgegeben werden.
	\end{itemize}
\end{tcolorbox}
\noindent
Lehrbücher: 
\begin{itemize}
	\item W. Nolting, \textit{Grundkurs Theoretische Physik 3: Elektrodynamik} (Springer)
	\item D.J. Griffiths, \textit{Elektrodynamik: Eine Einführung} (Pearson) 
	\item T. Fließbach, \textit{Elektrodynamik} (Spektrum Akademischer Verlag)
	\item J.D. Jackson, \textit{Klassische Elektrodynamik} (Walter de Gruyter) \lcom{geht dieser Vorlesung hinaus}
\end{itemize}

\section{Einführung und Überblick}

Die vier fundamentalen Wechselwirkungen (WW):
\begin{itemize}
	\item Starke WW
	\item \textbf{Elektromagnetische WW}\  \lcom{Wird in dieser Vorlesung betrachtet}
	\item Schwache WW
	\item Gravitation
\end{itemize}

\subsection{Rückblick}
Theoretische Physik 1: 
\begin{itemize}
	\item Mechanik 
	\item Punktmechanik: Bahnkurven von Körpern
	\item Bewegungsgleichung: $m \vec{\ddot r} = \vec{F}$
\end{itemize}
\subsection{Elektrodynamik}
\begin{itemize}
	\item Grundlegende Größen
	\item Felder
	\item 
	\begin{equation*}
	\begin{array}{cc}
	\vec{E}(\vec{r},t)\ \ \  & \vec{B}(\vec{r},t) \\[5pt]
	\tx{elektrisches Feld \ \ \ } &  \tx{Magnetfeld}
	\end{array}
	\end{equation*}
	\item[$\boldsymbol{\rightarrow}$] Feldtheorie \lcom{sehr wichtiges Konzept}
\end{itemize}
\lcom{Wie sind Elektrische Felder definiert?}\\
Experimentelle Definition als Messgröße: Kraft auf Ladung
$$\vec{F} = q(\vec{E}(\vec{r},t) + \vec{v} \times \vec{B}(\vec{r},t))$$
Theoretische Definition ist Mathematisch: Feldgleichungen-Maxwellgleichungen
$$ \ \qquad \vec\nabla \cdot\vec{E} = \frac{1}{\epsilon_0}\rho \qquad \qquad \ \qquad  \vec\nabla\cdot\vec{B} = 0$$
$$\vec\nabla\times\vec{E} + \prt{\vec{B}}{t} = 0 \qquad \vec\nabla\times\vec{B} - \mu_0 \epsilon_0 \prt{\vec{E}}{t} = \mu_0 \vec{j}$$
Hierbei steht $\rho$ für die Ladungsdichte und $\vec{j}$ für die Stromdichte.

\section{Aufbau der Vorlesung}
\textbf{1./2.} Statische Phänomene: $ \ \prt{\vec{E}}{t} = 0 = \prt{\vec{B}}{t}$
$$\Rightarrow \vec\nabla\cdot\vec{E} = \frac{1}{\epsilon_0}\rho \qquad\ \vec\nabla \cdot \vec{B} = 0$$
$$\ \ \, \quad\underbrace{\vec\nabla\times\vec{E} = 0}_{\textrm{1. Elektrostatik}} \qquad \underbrace{\vec\nabla\times\vec{B} = 0}_{\textrm{2. Magnetostatik}}$$
\textbf{3.} Zeitabhängige magnetische/elektrische Felder\\
\textbf{4.} Relativistische Formulierung der Elektrodynamik

\chapter{Elektrostatik}

Wir beschäftigen uns in diesem Kapitel mit \textbf{ruhenden Ladungen} und \textbf{zeitunabhängigen Feldern}. Das Grundproblem besteht darin, dass wir eine Ladungsverteilung haben und das Elektrische Feld und dessen Potential bestimmen wollen.\\[5pt]
\FloatBarrier
\begin{wrapfigure}{r}{3cm}
	\vspace{-1.2cm}
	\fbox{\begin{tikzpicture}
		\draw node[circle,fill,inner sep=1pt,label=left:$q_1$](a){} -- (1,0);
		\draw[xshift=0.5cm,yshift=0.7cm] node[circle,fill,inner sep=1pt,label=right:$q_2$](a){} -- (1,0);
		\draw[xshift=0.5cm,yshift=-0.2cm] node[circle,fill,inner sep=1pt,label=right:$q_3$](a){} -- (1,0);
		\end{tikzpicture}}
\end{wrapfigure}
\FloatBarrier
$\rightarrow$ Feld $\vec{E}(\vec{r})$, el. Potential $\Phi(\vec{r})$
\section{Elektrische und Coulombsches Gesetz}
Ladung: Beobachtungstatsachen:
\begin{enumerate}[i)]
	\item Zwei Arten ,,+``, ,,-``
	\item Abgeschlossenes System: Ladung erhalten: $q = \sum_i q_i = \const$
	\item Ladung ist quantisiert in Einheiten der Elementarladung: $$q = ne,\ n \in \mathbb{Z},\ e = 1,602\cdot 10^{-19}\,\textrm{C}$$ 
	$n = -1$: für ein Elektron wäre ein Beispiel einer Punktladung
\end{enumerate}
\begin{minipage}{.6\linewidth}
	Kontinuierliche Ladungsverteilung Ladungsdichte\\
	$\rho(\vec{r}) = \frac{\textrm{Ladung}}{\textrm{Volumen}} = \frac{\Delta q}{\Delta V}$
	Gesamtladung in $V$: $$Q = \int_V d^3 r\, \rho(\vec{r})$$
\end{minipage}
\begin{minipage}{.4\linewidth}
	\hspace{30pt}
	
	%t2 done:
	
	\begin{tikzpicture}
	\node at (1.5,1.5) (a) {};
	\draw  plot [smooth cycle, tension=1] coordinates { (0,0) (.5,1.5) (.5,2.5) (2,2) (3,2) (2.5,1) (1.5,0) };
	\pic at (a) {annotated cuboid={width=3,height=3,depth=3}};
	\draw[] (a) node[right,xshift=10pt] {$ \Delta V $};
	\draw[] (a) node[right,xshift=-10pt,yshift=10pt] {$ \Delta q $};
	\draw[->] (3,.5) to[out=180,in=-10] (1.5,.7);
	\node[right] at (3,.5) (b) {$ V $};
	\draw[thick,->] (-.5,-.5) -- (-.5,2) node[anchor=south east] {y};
	\draw[thick,->] (-.5,-.5) -- (2,-.5) node[anchor=north west] {x};
	\draw[thick,->] (-.5,-.5) -- (1.24,1.24) node[below,xshift=-10,yshift=-20pt] {$ \vec{r} $};
	\end{tikzpicture}
\end{minipage}

\subsection{Coulombsches Gesetz}

\begin{minipage}{.585\linewidth}
	Die Kraft, welche eine am Ort $\vec{r}_2$ lokalisierte Punktladung auf eine Punktladung am Ort $\vec{r}_1$ ausübt, ist gegeben durch:
	$$\vec{F}_{12} = k \frac{q_1 q_2}{|\vec{r}_1 - \vec{r}_2|^2} \underbrace{\frac{\vec{r}_1 - \vec{r}_2}{|\vec{r}_1 - \vec{r}_2|}}_{\vec{e}_{r_{12}}}$$
\end{minipage}
\begin{minipage}{.4\linewidth}
	%t3 done:
	\hspace{30pt}
	\begin{tikzpicture}
	\node at (1,2) (a) {};
	\node at (2,1) (b) {};
	\draw[thick,->] (0,0) -- (0,2.5);
	\draw[thick,->] (0,0) -- (2.5,0);
	\draw[->] (0,0) -- (a) node[above] {$ q_1 $};
	\draw[->] (0,0) -- (b) node[right] {$ q_2 $};
	\node at (.5,1) (c) {};
	\node at (1,.5) (d) {};
	\draw[] (c) node[above,yshift=5,xshift=-3] {$ \vec{r}_1 $};
	\draw[] (d) node[below,yshift=2,xshift=5] {$ \vec{r}_2 $};
	\end{tikzpicture}
\end{minipage}

\begin{enumerate}
	\item $\vec{F}_{12} \sim q_1 q_2$
	\item $\vec{F}_{12} \sim \frac{1}{|\vec{r}_1 - \vec{r}_2|^2}$
	\item $\vec{F}_{12} \sim q_1 q_2\, \vec{e}_{r_{12}}$
	\item $\vec{F}_{12} = -\vec{F}_{21}$
\end{enumerate}
Es gilt das Superpositionsprinzip: Das heißt, durch vektorielle Addition der Kräfte kann die Gesamtkraft ermittelt werden.

$$\vec{F}_1 = k \summ{j = 2}{N} \frac{q_1 q_j}{r_{1j}^2}\vec{e}_{r_{1j}}$$

\paragraph{Zur Konstanten $k$:}
Die Konstante ist abhängig von dem verwendeten Maßsystemen.
\begin{enumerate}[i)]
	\item Gauß-System (cgs): $k \equiv 1$, dyn = $\frac{\textrm{g}\cdot \textrm{cm}}{\textrm{s}^2} = 10^{-5}\,\textrm{N}$
	1 dyn = $\frac{(1\textrm{ESE})^2}{\textrm{cm}^2} \quad 1\textrm{ESE} = \frac{\sqrt{\textrm{g}\cdot \textrm{cm}^3}}{\textrm{s}}$
	\item SI (MKSA-System): Definition von A = Amp\`ere 
	
	
	%t4 done: 
	
	\vspace{-15pt}
	$$\frac{\Delta F}{\Delta l}= 2 \cdot 10^{-7}\,\frac{\textrm{N}}{\textrm{m}} \qquad \qquad \begin{tikzpicture}
	\draw (0,0) -- (3,0);
	\draw (0,1) -- (3,1);
	\draw[thick,<->] (.75,0) -- (.75,1) node[anchor=north east,yshift=-6pt] {1 m};
	\draw[thick,->] (2,0) -- (2,.3);
	\draw[thick,->] (2,1) -- (2,.7);
	\draw[->] (2,0) -- (2.5,0) node[above] {$ I $};
	\draw[->] (2,1) -- (2.5,1) node[above] {$ I $};
	\end{tikzpicture}$$
	Strom = $\frac{\textrm{Ladung}}{\textrm{Zeit}}$ $\Rightarrow$  \ 
	$
	1 \tx{A} = \frac{1 \tx{C}}{1 \tx{s}} \quad \rightarrow \quad e = 1{,}602 \cdot 10^{-19} \, \tx{C} \qquad c \approx 3 \cdot 10^{8} \, \frac{\tx{m}}{\tx{s}}
	$\\[5pt]
	$
	\frac{\Delta F}{\Delta l} = k \frac{2 I^2}{c^2 d} \qquad \rightarrow k = 2 \cdot 10^{-7} \ \frac{\tx{N}}{\tx{m}} \frac{c^2 1 \tx{m}}{2 (1 \tx{A})^2} = 10^{-7} c^2 \, \frac{\tx{N}}{\tx{A}^2}
	$
	\begin{equation*}
	k = \frac{1}{4 \pi \epsilon_0}
	\end{equation*}
	Damit erhalten wir für die Dielektrizitätskonstante des Vakuums:
	\begin{equation*}
	\epsilon_0 = 8{,}85 \cdot 10^{-12} \frac{\tx{C}^2}{\tx{N} \tx{m}^2}
	\end{equation*}
\end{enumerate}


\section{Elektrisches Feld}
\subsection{Feld eines Systems von Punktladungen}

\begin{minipage}{.5\linewidth}
	$N$-Ladungen $q_1, \dots, q_N$ ruhen an den Orten $\vec{r}_1, \dots, \vec{r}_N$. Nun bringen wir eine Testladung $q$ am Ort $\vec{r}$ mit ein. 
\end{minipage}
\begin{minipage}{.5\linewidth}
	%t5 done:
	\hspace{50pt}
	\begin{tikzpicture}
	\node at (3,1) (a) {};
	\node at (2.5,2) (b) {};
	\node at (1,2) (c) {};
	\draw[thick,->] (0,0) -- (0,2.5) node[anchor=south east] {y};
	\draw[thick,->] (0,0) -- (2.5,0) node[anchor=north west] {x};
	\draw[->] (0,0) -- (a) node[above] {$ q_1 $};
	\draw[->] (0,0) -- (b) node[right] {$ q_2 $};
	\draw[->] (0,0) -- (c) node[above] {$ q $};
	\node at (1.5,.5) (c) {};
	\node at (1.25,1) (d) {};
	\node at (.5,1) (e) {};
	\draw[] (c) node[below,xshift=10pt] {$ \vec{r}_1 $};
	\draw[] (d) node[below,xshift=5pt] {$ \vec{r}_2 $};
	\draw[] (e) node[left] {$ \vec{r} $};
	\end{tikzpicture}
\end{minipage}
Kraft von $q_1,\ q_2$ auf $q$
$$\vec{F} = \frac{1}{4\pi \epsilon_0} q \summ{j = 1}{N} q_n \frac{\vec{r} - \vec{r}_j}{|\vec{r} - \vec{r}_j|^3} = q \vec{E}(\vec{r})$$
Somit ist das elektrisches Feld:
$$\rmbox{\vec{E}(\vec{r}) = \frac{1}{4 \pi \epsilon_0} \summ{j = 1}{N} q_j \frac{\vec{r} - \vec{r}_j}{|\vec{r} - \vec{r}_j|^3}}$$
\textbf{\emph{Bemerkung}}
\begin{enumerate}[i)]
	\item Testladung klein (formal: $ \lim_{q\to0}\frac{\vec{F}}{q} $)
	\item math. $ \vec{E}(\vec{r}) $ Vektorpfeil\\
	\begin{equation*}
	\tx{kartesisch:} \quad \vec{E}(\vec{r}) = \begin{pmatrix}
	E_x(\vec{r}) \\ E_y(\vec{r}) \\ E_z(\vec{r})
	\end{pmatrix}
	\end{equation*}
	\item Wechselwirkungsprozess: 2 Teile
	\begin{equation*}
	q_j \rightarrow \vec{E}(\vec{r}) \rightarrow \vec{F} = q \vec{E} (\vec{r})
	\end{equation*}
	\item Superpositionsprinzip gilt
\end{enumerate}

\subsection[Feld einer kontinuierlichen Ladungsverteilung]{Feld einer kontinuierlichen Ladungsverteilung $\rho(\vec{r})$}

\begin{minipage}{.5\linewidth}
	$$\vec{E}(\vec{r}) =\frac{1}{4 \pi \epsilon_0} \ub{\int\limits_V}_{\mathclap{\substack{\tx{schließt alle} \\ \tx{Ladungen ein}}}} d^3 r'\, \rho(\vec{r}')\frac{\vec{r} - \vec{r}_j}{|\vec{r} - \vec{r}_j|^3}$$
	$\rho(\vec{r}_j) = \frac{\Delta q_j}{\Delta V_j}$
\end{minipage}
\begin{minipage}{.4\linewidth}
	%t6 done:
	\hspace{50pt}
	\begin{tikzpicture}
	\node at (1.5,1.5) (a) {};
	\draw  plot [smooth cycle, tension=1] coordinates { (0,0) (.5,1) (.5,2.5) (2,2.2) (3,2) (2.5,.5) (1.5,-.2) };
	\pic at (a) {annotated cuboid={width=3,height=3,depth=3}};
	\draw[] (a) node[right,xshift=10pt] {$ \Delta V $};
	\draw[] (a) node[right,xshift=-10pt,yshift=10pt] {$ \Delta q_j $};
	\draw[->] (3,.5) to[out=180,in=-40] (2.2,.8);
	\node[right] at (3,.5) (b) {$ V $};
	\draw[thick,->] (-.5,-.5) -- (-.5,2) node[anchor=south east] {y};
	\draw[thick,->] (-.5,-.5) -- (2,-.5) node[anchor=north west] {x};
	\draw[thick,->] (-.5,-.5) -- (.25,.25) node[anchor=south east] {z};
	\draw[] (1,0) node[above] {$ \rho(\vec{r}) $};
	\end{tikzpicture}
\end{minipage}

\begin{align*}
\vec{E}(\vec{r}) &= k\sum_j\Delta q_j\frac{\vec{r}-\vec{r}_j}{|\vec{r}-\vec{r}_j|^3}\\
&= k\sum_j\Delta V_j\rho(\vec{r}_j)\frac{\vec{r}-\vec{r}_j}{|\vec{r}-\vec  {r}_j|^3}\\
\scriptstyle{\tx{mit } \Delta V_j \to 0}\ \ &\rightarrow k\int_V d^3r'\rho(\vec{r}')\frac{\vec{r}-\vec{r}'}{|\vec{r}  -\vec{r}'|^3}
\end{align*}

%%%%%% Vorlesung 2 %18.10.18

\subsection{Ladungsdichte einer Punktladung}

\begin{minipage}{.6\linewidth}
	\paragraph{Deltafunktion}
	
	$$\rho (\vec{r}) = q \delta(\vec{r}-\vec{r}_0)$$
	Punktladung in $\vec{r}_0 \Rightarrow \rho(\vec{r}) = 0 \quad \vec{r} \neq \vec{r}_0$\\
	Ladungsdichte divergiert in $\vec{r}_0$
	$$\rho(\vec{r}_0) = \infty$$
\end{minipage}
\begin{minipage}{.4\linewidth}
	%t1 done:
	\centering
	\begin{tikzpicture}
	\draw[->] (0,0) -- (0,2);
	\draw[->] (0,0) -- (2,0);
	\draw[->] (0,0) -- (1.5,1.5) node [right] {$q$};
	\draw (1,1) node [above,left] {$\vec{r}_0$};
	\end{tikzpicture}
\end{minipage}

Modell für Punktladung:\\
Ladung $q$ in Kugel mit Radius $\epsilon$  um $\vec{r}_0,\ \epsilon \rightarrow 0$
$$\rho_2 (\vec{r}) = \left\{ \begin{array}{cc}
\frac{q}{v_k} & |\vec{r}| \leq \epsilon\\
0 & \textrm{sonst}	
\end{array} \right\} = \frac{q}{\frac{4}{3}\pi \epsilon^3}\underbrace{\Theta (\epsilon - |\vec{r}|)}_{\mathclap{\textrm{Stufenfunktion}}}$$


%% T?


$$\rho(\vec{r}) = \lim_{\epsilon \rightarrow 0}\ \rho_\epsilon (\vec{r}) = \left\{ \begin{array}{cc}
\infty & \vec{r} = 0\\
0 & \vec{r} \neq 0
\end{array}\right.$$
Divergenz muss so sein, dass $$ \int\limits_{V \atop \vec{r}_0 \in V} d^3r\ \rho(\vec{r}) = q$$

\paragraph{Definition Delta-Funktion (Diracsche Deltafunktion)}

\begin{enumerate}
	\item 
	$$\delta (\vec{r} - \vec{r}_0) = \left\{ \begin{array}{cc}
		0 & \vec{r} \neq \vec{r}_0 \\
		\infty & \vec{r} = \vec{r}_0
	\end{array}\right.$$
	\item 
	$$\int_V d^3 r\ f(\vec{r}) \delfkt = \left\{ \begin{array}{cc}
		f(\vec{r}_0) & \vec{r}_0 \in V\\
		0 & \vec{r}_0 \notin V
	\end{array}\right.$$ 
\end{enumerate}

\paragraph{Mathematik}
Distribution - Funktional

Funktional: Abb. Funktionen $\mapsto \mathbb R, \mathbb C$
$$\delta_{\vec{r}_0}: f \mapsto f(\vec{r}_0)$$

\paragraph{Physik}

$$\int d^3 r\ f(\vec{r}) \delta (\vec{r}-\vec{r}_0) = f(\vec{r})$$
$\delta$-Fkt. als Grenzwert einer Folge von Funktionen im Integral
$$\int d^3 r\ f(\vec{r}) \delta (\vec{r} - \vec{r}_0) = \lim_{\epsilon \to 0} \quad \int d^3\ f(\vec{r} g_\epsilon(\vec{r}-\vec{r}_0) $$
mit
$$\lim_{\epsilon \to 0} g_\epsilon (\vec{r}-\vec{r}_0) =  \left\{ \begin{array}{cc}
0 & \vec{r} \neq \vec{r}_0 \\
\infty & \vec{r} = \vec{r}_0
\end{array}\right.$$
$$\int_V d^3 r\ g_\epsilon (\vec{r}-\vec{r}_0) = 1$$

Beispiel: $g_\epsilon (\vec{r}-\vec{r}_0) = \frac{\Theta(\epsilon - |\vec{r}|)}{\frac{4}{3}\pi \epsilon^3}$

Mehrere Punktladungen $q_j$ in $\vec{r}_j$
$$\rho(\vec{r}) = \sum_j q_j \delta(\vec{r}-\vec{r}_j)$$

\begin{align*}
 	\Rightarrow \vec{E}(\vec{r}) &= \kq \int_V d^3 r'\ \rho(\vec{r}') \frac{\vec{r} - \vec{r}'}{|\vec{r} - \vec{r}'|^3}\\
	&= \kq \int_V d^3 r'\ \sum_j q_j \delta(\vec{r} - \vec{r}_j) \frac{\vec{r} - \vec{r}'}{|\vec{r} - \vec{r}'|^3}\\
	&= \kq \sum_j q_j \int_V d^3 r'\ \delta(\vec{r} - \vec{r}_j) \frac{\vec{r} - \vec{r}'}{|\vec{r} - \vec{r}'|^3}\\
	&= \kq \sum_j q_j \frac{\vec{r} - \vec{r}_j}{|\vec{r} - \vec{r}_j|^3} \qquad \checkmark\\
\end{align*}

\subsection{Flächenladungsdichte}

\begin{minipage}{.5\linewidth}
	$\sigma(\vec{r}) = \frac{\textrm{Ladung}}{\textrm{Fläche}} = \frac{\Delta q}{\Delta A}$
\end{minipage}
\begin{minipage}{.5\linewidth}
	%t2 eig %t1:
	\centering
	\begin{tikzpicture}
		\draw[->] (0,0) -- (0,1);
		\draw[->] (0,0) -- (1,0);
		\draw[->] (0,0) -- (-.5,-.5);
		\draw (.5,.2) -- ++(30:1.3) -- ++(90:1) -- ++(-150:1.3) -- (.5,.2);
		\draw (.9,.7) -- ++(30:.3) -- ++(90:.3) -- ++ (-150:.3) -- (.9,.7);
		\draw[thick,->] (0,0) -- (.9,.7) node[yshift=-7pt,xshift=-18pt] {$ \vec{r} $};
		\draw ($ (.9,.7) + (30:.3) + (90:.3) + (-120:.15) $) to[out=65,in=150] (2,2) node[right] {$ \Delta A $};
		\draw ($ (.5,.2) + (33:1.1) $) to[out=-50,in=-150] (2,.5) node[right] {$ A $};
		\draw ($ (.9,.7) + (30:.3) + (90:.3) + (-100:.2) $) to[out=20,in=180] (2.2,1.35) node[right] {$ \Delta q $};
	\end{tikzpicture}
\end{minipage}
erzeugtes elektrisches Feld:
$$\vec{E}(\vec{r}) = \kq \int_A \underbrace{df'}_{\mathclap{\textrm{Flächenelement}}}\ \sigma (\vec{r}) \frac{\vec{r} - \vec{r}'}{|\vec{r} - \vec{r}'|^3}$$
\begin{minipage}{.65\linewidth}
	\bei Elektrisches Feld einer homogenen Flächenladung
\end{minipage}
\begin{minipage}{.35\linewidth}
	%t3 eig %t2: -- $\sigma = \const$ in $x$-$y$-Ebene
	\centering
	\begin{tikzpicture}
		\draw[->] (0,0) -- (0,1) node[anchor=south east] {$ z $};
		\draw[->] (0,0) -- (2,0) node[anchor=north west] {$ y $};
		\draw[->] (0,0) -- (-1,-1) node[anchor=north east] {$ x $};
		\node at (2,1) {$ \substack{\sigma = \const \tx{in} \\ x\tx{-}y \tx{-Ebene} } $};
		\draw (-.3,-.3) -- ++(1.8,0);
		\draw (-.6,-.6) -- ++(1.8,0);
		\draw (-.9,-.9) -- ++(1.8,0);
		\draw (1.8,0) -- ++(-.9,-.9);
		\draw (1.2,0) -- ++(-.9,-.9);
		\draw (.6,0) -- ++(-.9,-.9);
	\end{tikzpicture}
\end{minipage}

$$\vec{E}(\vec{r}) = \kq \intt{-\infty}{+\infty}dx'\intt{-\infty}{+\infty}dy'\ \sigma \frac{\vec{r} - \vec{r}'}{|\vec{r} - \vec{r}'|^3} \qquad \vec{r}' = (x', y', 0)$$
Symmetrie: $\vec{E}$ unabhängig von $x,y \qquad \vec{r} = (0,0,z)$
$$\vec{r} - \vec{r}' = (-x',-y',z),\ |\vec{r} - \vec{r}'|^3 = (x'^2+y'^2+z^2)^{\nicefrac{3}{2}}$$
$$E_x \sim \sigma \intt{-\infty}{+\infty}dx'\intt{-\infty}{+\infty}dy'\ \frac{(-x')}{(x'^2 + y'^2 + z^2)^{\nicefrac{3}{2}}} = 0 = E_y$$

$\vec{E} = (0,0,E_z)$

\begin{align*}
 	E_z &= \kq \sigma_z \intt{-\infty}{+\infty}dx'\underbrace{\intt{-\infty}{+\infty}dy'\ \frac{(x')}{(x'^2 + y'^2 + z'^2)^{\nicefrac{3}{2}}}}_{\mathclap{\frac{1}{x'^2 +z^2}\left.\frac{y'}{(x'^2 + y'^2 + z^2)^{\nicefrac{3}{2}}} \right\rvert_{-\infty}^{+\infty} = \frac{1}{x'^2 +z^2}\left.\frac{\textrm{sgn}(y')}{\sqrt{1 + \frac{x'^2 + z^2}{y'^2}}} \right\rvert_{-\infty}^{+\infty} = \frac{2}{x'^2 +z^2}}}\\
 	&= \frac{1}{2 \pi \epsilon_0} \sigma_z \underbrace{\intt{-\infty}{+\infty} dx'\ \frac{1}{x'^2 + z^2}}_{\left. \frac{1}{z} \arctan \left(\frac{x'}{2}\right)\right\rvert_{-\infty}^{+\infty} = \frac{1}{z} \textrm{sgn}(z)\pi}\\
 	E_z &= \frac{\sigma}{2 \epsilon_0} \textrm{sgn}(z)
\end{align*}
\begin{minipage}{.6\linewidth}
	Grenzfläche: $z \rightarrow 0$
	$$\vec{E} \underset{z \to 0} \longrightarrow \left\{ \begin{array}{cc}
	\frac{\sigma}{2 \epsilon_0} \vec{e}_z & z > 0\\
	-\frac{\sigma}{2 \epsilon_0} \vec{e}_z & z < 0
	\end{array}\right.$$
	$$\vec{E}_{\perp_+} - \vec{E}_{\perp_-} = \frac{\sigma}{\epsilon_0}, \qquad \vec{E}_\parallel = 0$$
\end{minipage}
\begin{minipage}{.4\linewidth}
	%t3:
	\centering
	\begin{tikzpicture}
		\draw[->] (0,-1.2) -- (0,1.2) node[above] {$ E_z $};
		\draw[->] (-2,0) -- (2,0) node[anchor=north west] {$ z $};
		\draw[thick] (2,.8) -- (0,.8) ;
		\draw (0,.8) -- (-.15,.8) node[left] {$ \frac{\sigma}{2 \epsilon_0} $};
		\draw[thick] (-2,-.8) -- (0,-.8);
		\draw (0,-.8) -- (.15,-.8) node[right] {$\frac{\sigma}{2 \epsilon_0}$};
	\end{tikzpicture}
\end{minipage}
%t4: plattenkondensator
%t5:
\begin{center}
\begin{tikzpicture}
	\coordinate (o) at (-8,0);
	\draw[thick,->] ($ (o) + (-.5,0)$) -- ++(2.5,0) node[anchor=north west] {$ z $};
	\draw[very thick] ($ (o) + (0,-1) $) -- ++(0,2) node[above] {$ \sigma $};
	\draw[very thick] ($ (o) + (1.5,-1) $) -- ++(0,2) node[above] {$ -\sigma $};
	
	\draw[->] (-2,0) -- (2,0) node[anchor=north west] {$ z $};
	\draw[->] (0,-1) -- (0,1.5) node[anchor=south east] {$ E_z $};
	\draw[very thick] (-2,0) -- (-.75,0);
	\draw[very thick] (.75,0) -- (2,0);
	\draw[very thick] (-.75,1) -- (.75,1) node[anchor=south west] {$ \frac{\sigma}{\epsilon_0} $};
	\draw[ultra thick,draw=orange,dashed] (-2,.5) -- (.75,.5);
	\draw[ultra thick,draw=orange,dashed] (.75,-.5) -- (2,-.5) node[right] {$ - \sigma $};
	\draw[very thick,draw=black!25!green,dashed] (-.75,-.5) -- (-2,-.5) node[left] {$ \sigma $};
	\draw[very thick,draw=black!25!green,dashed] (2,.5) -- (-.75,.5);
	\draw (.75,-.1) -- (.75,.1);
	\draw (-.75,-.1) -- (-.75,.1);
	\draw (-.1,.5) -- (.1,.5) node[right,xshift=55pt] {$ \frac{\sigma}{2 \epsilon_0} $};
	\draw (-.1,-.5) -- (.1,-.5) node[right,xshift=-2pt] {$ \frac{- \sigma}{2 \epsilon_0} $};
\end{tikzpicture}
\end{center}

\subsection{Linenladungsdichte}

\begin{minipage}{.5\linewidth}
	$$\lambda(\vec{r}) = \frac{\textrm{Ladung}}{\textrm{Länge}} = \frac{\Delta q}{\Delta s}$$
	$$\vec{E}(\vec{r}) = \kq \underbrace{\int_\gamma}_{\mathclap{\textrm{Linienintegral}}} ds'\ \lambda(\vec{r}') \frac{\vec{r} - \vec{r}'}{|\vec{r} - \vec{r}'|^3}$$
\end{minipage}
\begin{minipage}{.5\linewidth}
	%t6:
	\centering
	\begin{tikzpicture}
		\draw[->] (0,0) -- (0,1);
		\draw[->] (0,0) -- (1,0);
		\coordinate (m) at (2,3);
		\draw[thick,->] (0,0) -- (m) node[xshift=-30pt,yshift=-35pt] {$ \vec{r} $};
		\draw ($ (m) + (45:0.2) $) -- ++(180:.3) node[anchor=south east] {$ \Delta s $} -- ++(-90:.3) -- ++(0:.3) -- ++(90:.3) node[anchor=north west] {$ \Delta q $};
		\draw[thick] (-.5,1.5) to[out=80,in=-170] (1,2.7) to[out=10,in=-160] (m) to[out=20,in=-100] (3,4) node[right] {$ \gamma $};
	\end{tikzpicture}\vspace{15pt}
\end{minipage}\\
\emph{Beispiel:} Elektrisches Feld einer homogenen Linienladung $ \lambda = \const $\\
\begin{minipage}{.2\linewidth}
	%t7 $\lambda = \const$
	\centering
	\begin{tikzpicture}
		\draw[->] (0,0) -- (1,0) node[anchor=north west] {$ x $};
		\draw[->] (0,0) -- (.75,.75) node[anchor=south west] {$ y $};
		\draw[->] (0,0) -- (0,3) node[anchor=south west] {$ z $};
		\draw[very thick,draw=red] (0,0) -- (0,2.5) node[left,yshift=-30pt] {$ \lambda $};
	\end{tikzpicture}
\end{minipage}
\begin{minipage}{.8\linewidth}
	\begin{align*}
	\vec{E}(\vec{r}) &= \kq \int_\gamma ds'\ \lambda \frac{\vec{r} - \vec{r}'}{|\vec{r} - \vec{r}'|^3} \qquad \gamma: z' \mapsto \vec{r}'(z') = \begin{pmatrix} 0 \\ 0\\ z \end{pmatrix}\\
	&= \frac{\lambda}{4 \pi \epsilon_0} \intt{-\infty}{\infty}dz'\ \frac{\vec{r} - \begin{pmatrix} 0 \\ 0\\ z \end{pmatrix}}{(x^2 + y^2 + (z-z')^2)^{\nicefrac{3}{2}}}
	\end{align*}
\end{minipage}

$\tilde{z} = z' - z$:
$$E_x = \frac{\lambda x}{4 \pi \epsilon_0}\intt{-\infty}{\infty}dz'\ \frac{1}{(x^2 + y^2 + (z-z')^2)^{\nicefrac{3}{2}}} = \frac{\lambda x}{4 \pi \epsilon_0}\underbrace{\intt{-\infty}{\infty}d\tilde{z}\ \frac{1}{(x^2 + y^2 + \tilde{z}^2)^{\nicefrac{3}{2}}}}_{\frac{2}{x^2 + y^2}} = \frac{\lambda}{2 \pi \epsilon_0} \frac{x}{x^2 + y^2}$$
$$E_y = \frac{\lambda}{2 \pi \epsilon_0}\frac{x}{x^2 + y^2}$$
$$E_z = \frac{\lambda}{4 \pi \epsilon_0} \intt{-\infty}{\infty}dz'\ \frac{z-z'}{(x^2 + y^2 + (z-z')^2)^{\nicefrac{3}{2}}} = 0$$
$$\vec{E}(\vec{r}) = \frac{\lambda}{2 \pi \epsilon_0} \frac{1}{x^2 + y^2} \begin{pmatrix}x\\ y\\ 0\end{pmatrix}$$
$$\rho = \sqrt{x^2 + y^2} = \frac{\lambda}{2 \pi \epsilon_0} \frac{1}{\rho}\vec{e}_\rho, \qquad \vec{e}_\rho = \begin{pmatrix}\cos \varphi\\ \sin \varphi\\ 0\end{pmatrix}$$

\section{Feldgleichungen und elektrostatische Potential}
$$\vec{E}(\vec{r}) = \kq \int d^3 r'\ \rho(\vec{r}') \frac{\vec{r} - \vec{r}'}{|\vec{r} - \vec{r}'|^3}$$
\subsection{Elektrostatisches Potential}
elektrische Feld ist ein Potentialfeld $\vec{E}(\vec{r}) = - \nabla\phi(\vec{r}) = - \bigg( \vec{e}_x \prt{\phi}{x} + \vec{e}_y \prt{\phi}{y} + \vec{e}_z \prt{\phi}{z}\bigg)$
$$\frac{\vec{r} - \vec{r}'}{|\vec{r} - \vec{r}'|^3} = -\nabla\frac{1}{|\vec{r}-\vec{r}'|}$$
$$-\prt{}{x} \frac{1}{[(x-x')^2 + (y-y')^2 + (z-z')^2]^{\nicefrac{1}{2}}} =  \frac{- \left(-\frac{1}{2}\right) \quad 2(x-x')}{[(x-x')^2 + (y-y')^2 + (z-z')^2]^{\nicefrac{3}{2}}} = \frac{(x-x')}{|\vec{r} - \vec{r}'|^3}$$
$$\Rightarrow \vec{E}(\vec{r}) = \kq \int d^3 r'\ \rho(\vec{r}') \bigg(-\nabla_{\vec{r}} \frac{1}{|\vec{r} - \vec{r}'|} \bigg) = \nabla_F \kq \int d^3 r'\ \frac{\rho(\vec{r}')}{|\vec{r} - \vec{r}'|}$$
$\rightarrow$ elektrostatisches Potential
$$\phi(\vec{r}) = \kq \int d^3 r'\ \frac{\rho(\vec{r}')}{|\vec{r} - \vec{r}'|} + c$$
übliche Konvention: $c = 0\ (\phi(\vec{r})\buildrel{\rightarrow}\over{|\vec{r}| \to \infty} 0)$

Potential einer Punktladung in $\vec{r}_0$:
$$\rho(\vec{r}) = q \delta(\vec{r}-\vec{r}_0)$$
$$\phi(\vec{r}) = \int_{\mathbb R^3} d^3 r'\ \frac{q\delta(\vec{r}' - \vec{r}_0)}{|\vec{r}-\vec{r}'|} = \kq \frac{q}{|\vec{r} - \vec{r}_0|}$$

$$\eofr = - \vabla \phi = \kq q \vabla \frac{1}{|\vec{r} - \vec{r}_0|} = \kq q \frac{\vec{r} - \vec{r}_0}{|\vec{r} - \vec{r}_0|^3}$$

%%%% Vorlesung 3 22.10.18

\lcom{(Funktional-Analysis Siegfried Großmann Springer)\\
	(Landau-Lipschitz Buch geht weit der Vorlesung hinaus)}

\subsection{Feldgleichugn (differentielle Form)}

\paragraph{Rotation (Wirbel)}
$$\textrm{rot}\, \vec{E} \nabla \times \vec{E} = \vec{e}_x \bigg(\prt{E_z}{y} - \prt{E_x}{z}\bigg) + \vec{e}_y \bigg(\prt{E_x}{z} - \prt{E_z}{x}\bigg) + \vec{e}_z \bigg(\prt{E_y}{x} - \prt{E_x}{y}\bigg) \Rightarrow \nabla \times$$
$$\Rightarrow \nabla \times \vec{E} = - \nabla \times (\nabla\phi) = 0$$

Mathe: Es sind äquivalent 
\begin{enumerate}[i]
	\item $\vec{E} = - \nabla\phi$
	\item $\nabla \times \vec{E} = 0$ (auf einfach zusammenhängendem Gebiet)
	\item Kurvenintegral $\int_\gamma d\vec{r} \cdot \vec{E}$ ist Wegunabhängig 
	
	
	
	%T1
	
	
	
	
	$$\intt{\vec{r}_1}{\vec{r}_2} d\vec{r} \cdot \vec{E} = - \intt{\vec{r}_1}{\vec{r}_2} dt \underbrace{\frac{d\vec{r}}{dt}\times \nabla\phi(\vec{r}(t))}_{\frac{d\phi}{dt}} = \underbrace{(\phi(\vec{r}_2) - \phi(\vec{r}_1))}_{\textrm{Potentialdifferenz}} $$
\end{enumerate}

\subsection{Divergenz (Quellen)}

$$\div \vec{E} = \nabla \cdot \vec{E} = \prt{E_x}{x} + \prt{E_y}{y} + \prt{E_z}{z}$$

\begin{align*}
\nabla \cdot \vec{E}(\vec{r}) &= \nabla_{\vec{r}} \cdot \kq \int d^3r'\ \rho(\vec{r}') \frac{1}{|\vec{r} - \vec{r}'|^3}\\
&= \kq \int_V d^3 r' \rho (\vec{r}') \nabla_{\vec{r}} \cdot \frac{1}{|\vec{r} - \vec{r}'|^3}
\end{align*}

$x$-Anteil:
\begin{align*}
\frac{\partial}{\partial x}\frac{x-x'}{[(x-x')^2 + (y-y')^2 + (z-z')^2]^{\nicefrac{3}{2}}} &= \frac{1 \cdot [\dots]^{\nicefrac{3}{2}}(x - x')(x - x')^{\nicefrac{3}{2}}\cdot 2[\dots]^{\nicefrac{1}{2}}}{[\dots]^3}\\
&= \frac{[\dots]^{\nicefrac{1}{2}}((x - x')^2 + (y - y')^2 + (z - z')^2 - 3(x - x')^2)}{[\dots]^{\nicefrac{3}{2}}}\\
&= \frac{(y - y')^2 + (z - z')^2 - 2(x - x')^2}{[\dots]^{\nicefrac{3}{2}}}
\end{align*}

$$\frac{\partial}{\partial y} \frac{y-y'}{[(x-x')^2 + (y-y')^2 + (z-z')^2]^{\nicefrac{3}{2}}} = \frac{(x - x')^2 + (z - z')^2 - 2(y - y')^2}{[\dots]^{\nicefrac{3}{2}}}$$ 

$$\frac{\partial}{\partial z} \frac{z-z'}{[(x-x')^2 + (y-y')^2 + (z-z')^2]^{\nicefrac{3}{2}}} = \frac{(x - x')^2 + (y - y')^2 - 2(z - z')^2}{[\dots]^{\nicefrac{3}{2}}}$$ 

$$\nabla \cdot \frac{\vec{r} - \vec{r}'}{|\vec{r} - \vec{r}'|^3} = 0 \quad \textrm{falls} \quad \vec{r} \neq \vec{r}'$$

$$\Rightarrow \textrm{falls} \quad \vec{r} \notin V, \textrm{ d.h. } \vec{r} \textrm{ in Gebiet ohne Ladungsdichte } \rho(\vec{r}) = 0$$

$$\Rightarrow \nabla \cdot \vec{E}(\vec{r}) = 0$$ 



%T?





$\vec{r} \in V$: Grenzwertbetrachtung (Regularisierung des Integranden)\\
statt
$$\frac{\vec{r} - \vec{r}'}{|\vec{r} - \vec{r}'|^3} = \frac{\vec{r}-\vec{r}'}{[(x-x')^2 + (y-y')^2 + (z-z')^2]^{\nicefrac{3}{2}}}$$
betrachten wir:
$$\vec{f}_a (\vec{r}-\vec{r}') = \frac{\vec{r}-\vec{r}'}{[(x-x')^2 + (y-y')^2 + (z-z')^2]^{\nicefrac{3}{2}}} = \frac{\vec{r}-\vec{r}'}{[(\vec{r}-\vec{r}')^2 + a^2]^{\nicefrac{3}{2}}} \quad a \in \mathbb R,\ a>0$$

am Ende Grenzwert $\displaystyle{\lim_{a \to 0}}$
$$\nabla \cdot \vec{E} (\vec{r}) = \kq \lim_{a \to 0} \int_V d^3r'\ \rho(\vec{r}')\ \rmbox{\hspace{-10pt}\nabla_{\vec{r}}\cdot f_a(\vec{r}-\vec{r}')}$$

\begin{align*}
\prt{}{x} \frac{x-x'}{[(\vec{r}-\vec{r}')^2 + a^2]^{\nicefrac{3}{2}}} &= \frac{[\dots +a^2]^{\nicefrac{3}{2}} - (x - x') \frac{3}{2} \cdot 2(x-x') [\dots + a^2]^{\nicefrac{3}{2}}}{[\dots + a^2]^3}\\
&= \frac{(y - y')^2 + (z - z')^2 + a^2 - 2(x - x')^2}{[\dots + a^2]^{\nicefrac{3}{2}}}
\end{align*}

$$\nabla_{\vec{r}} \cdot f_a(\vec{r}-\vec{r}') = \frac{3a^2}{[(\vec{r}-\vec{r}')^2 + a^2]^{\nicefrac{5}{2}}}$$
$$\lim_{a \to 0} f_a(\vec{r}-\vec{r}') = \left\{ \begin{array}{cc}
0 & \vec{r} \neq \vec{r}'\\
\infty & \vec{r} = \vec{r}'	
\end{array} \right.$$
$$\Rightarrow \textrm{zum Integral } \int_V d^3r' \dots \textrm{ trägt (in Limes }a \to 0) \textrm{ nur der Bereich } \vec{r}' \approx \vec{r} \textrm{ bei}$$
$$K_R (\vec{r}) = \{\vec{r}' \in \mathbb R^3: |\vec{r} - \vec{r}'| \leq R\}$$


%T?




\begin{align*}
 	\hspace{100pt} &\lim_{a \to 0} \int_V d^3 r'\ \rho(\vec{r}') \nabla_{\vec{r}} \cdot f_a(\vec{r}- \vec{r}')\\
	&= \lim_{a \to 0} \int_{K_R (\vec{r})} d^3 r' \ \rho(\vec{r}') \frac{3a^2}{[(\vec{r}-\vec{r}')^2 + a^2]^{\nicefrac{5}{2}}}\\
 	&+ \underbrace{\lim_{a \to 0} \int_{V / K_R (\vec{r})} d^3 r' \rho(\vec{r}') \frac{3a^2}{[(\vec{r}-\vec{r}')^2 + a^2]^{\nicefrac{5}{2}}}}_{= 0}
\end{align*}

Wähle $R$ klein genug, dass man innerhalb $K_R (\vec{r})$\\
$\rho(\vec{r}')$ in Taylorreihe um $\vec{r}$ entwickeln kann.

$$\vec{\tilde{r}} = \vec{r}' - \vec{r},\ d^3 r' = d^3 \tilde{r}$$
$$\int_{K_R (\vec{r})} d^3 r' \ \rho(\vec{r}') \frac{3a^2}{[(\vec{r}-\vec{r}')^2 + a^2]^{\nicefrac{5}{2}}} = \int_{K_R (0)} d^3 \tilde{r} \ \rho(\vec{r} + \vec{\tilde{r}}) \frac{3a^2}{[\vec{\tilde{r}}^2 + a^2]^{\nicefrac{5}{2}}}$$
Taylorentwicklung von $\rho(\vec{r} + \vec{\tilde{r}})$ zum $\vec{\tilde{r}} = 0$
$$\rho(\vec{r} + \vec{\tilde{r}}) = \rho(\vec{r}) + \vec{\tilde{r}} \cdot \vabla \rho(\vec{r}) + \dots$$
$$= \int_{K_R (0)} d^3 \tilde{r} \ (\rho(\vec{r}) + \vec{\tilde{r}} \cdot \vabla \rho(\vec{r}) + \dots) \frac{3a^2}{[\vec{\tilde{r}}^2 + a^2]^{\nicefrac{5}{2}}}$$

\begin{enumerate}
	\item Integral:
	\begin{align*}
	\int_{K_R (0)} d^3 \tilde r\ \rho(\vec{r}) \frac{3a^2}{(\tilde{r}^2 + a^2)^{\nicefrac{5}{2}}} &= \rho(\vec{r}) \underbrace{\intt{0}{R} d\tilde r\ \frac{3a^2}{(\tilde{r}^2 + a^2)^{\nicefrac{5}{2}}}}_{\left[\frac{\tilde r^3}{(\vec{\tilde{r}}^2 + a^2)^{\nicefrac{3}{2}}}\right]_0^R} \underbrace{\int d \equaltoup{\Omega}{\mathclap{\sin \theta d \theta d \varphi}}}_{= 4\pi}\\
	&= 4 \pi \rho(\vec{r}) \frac{R^3}{(R^2 + a^2)^{\nicefrac{3}{2}}} \underset{a \to 0}{\longrightarrow} 4 \pi \rho(\vec{r})
	\end{align*} 
	\item Integral:
	$$\int_{K_R (0)} d^3 \tilde{r}\ \equalto{\vec{\tilde{r}}}{\tilde{r}\vec{e}_{\vec{\tilde{r}}}} \cdot \nabla_{\vec{r}} \rho(\vec{r}) \frac{3 a^2}{(\tilde{r}^2 + a^2)^{\nicefrac{5}{2}}} = \underbrace{\intt{0}{R} d \tilde r\ \frac{3a^2 \tilde r^3}{(\tilde r^2 + a^2)^{\nicefrac{3}{2}}}}_{\frac{2}{3}a -3a^2 \left(\frac{R^2 +\frac{2}{3}a^2}{(R^2 + a^2)^{\nicefrac{3}{2}}}\right)} \underbrace{\int d \Omega\ \vec{e}_{\tilde{r}} \cdot \nabla \rho(\vec{r})}_{\textrm{unabh. von } a} \underset{a \to 0}{\longrightarrow} 0$$
	gilt auch für alle höheren Terme
\end{enumerate}

$$\lim_{a \to 0} \int_V d^3 r' \rho(\vec{r}) \nabla_{\vec{r}}\cdot \frac{(\vec{r}-\vec{r}'}{[(\vec{r}-\vec{r}')^2 + a^2]^{\nicefrac{3}{2}}} = 4 \pi \rho (\vec{r})$$
$$\Rightarrow \nabla \cdot \vec{E}(\vec{r}) = \kq \lim_{a \to 0} '' = \frac{1}{\epsilon_0} \rho(\vec{r})$$

\rbox{$$\nabla \vec{E}(\vec{r}) = \frac{1}{\epsilon_0} \rho(\vec{r}) \quad \vec{r} \in \mathbb R^3$$}

\subsection{Zusammenfassung:}

\paragraph{Feldgleichungen der Elektrostatik}
Mathe: partielle DGL
\rbox{$$\nabla \vec{E}(\vec{r}) = \frac{1}{\epsilon_0} \rho(\vec{r}) \textrm{ inhomogene DGL}$$ 
	$$\nabla \times \vec{E}(\vec{r}) = 0\textrm{ homogene DGL}$$}
DGL für Potential $\phi$: 
$\vec{E} = - \nabla \phi$

\begin{align*}\nabla \cdot \vec{E} = \nabla \cdot (- \nabla \phi) &= - \nabla \cdot \begin{pmatrix} 
\partial_x \phi \\
\partial_y \phi \\ 
\partial_z \phi 
\end{pmatrix} \\
&= - \underbrace{\left( \frac{\partial^2 \phi}{\partial x^2} + \frac{\partial^2 \phi}{\partial y^2} + \frac{\partial^2 \phi}{\partial  z^2} \right) }_{=: \Delta \phi}\end{align*}

Partielle DGL 2. Ordnung:
\frbox{Poissongleichung}{$$\Delta \Phi(\vec{r}) = - \frac{1}{\epsilon_0} \rho(\vec{r})$$}

für Gebiete mit $\rho (\vec{r}) = 0$:
$$\Delta \phi (\vec{r}) = 0 \qquad \textrm{Laplacegleichung}$$

\paragraph{Darstellung der Deltafunktion:}

$$\lim_{a \to 0} \int_{\mathbb{R}^3} d^3r'\ \rho(\vec{r}')\underbrace{\nabla_{\vec{r}} \cdot \frac{(\vec{r} - \vec{r}')}{[(\vec{r} - \vec{r}')^2 + a^2]^{\nicefrac{3}{2}}}}_{\frac{3 a^2}{[(\vec{r} - \vec{r}')^2 + a^2]^{\nicefrac{5}{2}}} =: g_a(\vec{r}' - \vec{r})} = 4 \pi \rho(\vec{r})$$

$$\frac{1}{4 \pi} g_a \textrm{ liefert Grenzwertdarstellung der $\delta$-funktion.}$$

$$\lim_{a\to 0} \int_{\mathbb{R}^3} d^3r'\ \rho(\vec{r}') \frac{1}{4\pi}g_a(\vec{r}'-\vec{r})=\rho(\vec{r})$$

$$\lim_{a \to 0} g_a(\vec{r}' - \vec{r}) = \left\{ \begin{array}{cc}
0 		& \vec{r} \neq  \vec{r}'\\
\infty 	& \vec{r} = 	\vec{r}'
\end{array} \right.$$

$$\delta(\vec{r}) = \lim_{a \to 0} \frac{1}{4 \pi} \nabla_{\vec{r}} \cdot \frac{r^2}{(r^2+a^2)^{\nicefrac{3}{2}}}$$

$$\overset{\textrm{formal}}{=} \frac{1}{4 \pi} \nabla \cdot \underbrace{\frac{\vec{r}}{r^3}}_{\mathclap{= - \nabla \frac{1}{r}}} = - \frac{1}{4 \pi} \nabla \cdot \bigg(\nabla \frac{1}{r}\bigg) = \frac{-1}{4 \pi} \Delta \frac{1}{r} \Rightarrow \Delta \frac{1}{r} = -4 \pi \delta(\vec{r})$$

z.B. Potential einer Punktladung $\rho$ q in $\vec{r}_0$:
$$\phi(\vec{r}) = \kq \frac{q}{|\vec{r} - \vec{r}_0|}$$
$$\Delta \phi(\vec{r}) = \kq q \underbrace{\Delta_{\vec{r}} \frac{1}{|\vec{r} - \vec{r}_0|} }_{= - 4 \pi \delta(\vec{r} - \vec{r}_0)} = - \frac{1}{\epsilon_0} \underbrace{q \delta(\vec{r} - \vec{r}_0)}_{= \rho(\vec{r})} = \frac{1}{\epsilon_0} \rho(\vec{r})$$

%%%%%%%%% Vorlesung 4 25.10.18

\bbb{Wiederholung}{\begin{align*}
	\vec{\nabla} \cdot \vec{E}(\vec{r}) &= \frac{1}{\epsilon_0} \rho(\vec{r})\\
	\vec{\nabla} \times \vec{E}(\vec{r}) &= 0 \\[10pt]
	\Rightarrow \quad \vec{E} &= - \vec{\nabla} \Phi \\
	\Rightarrow \quad \Delta \Phi (\vec{r}) &= - \frac{1}{\epsilon_0} \rho(\vec{r})
	\end{align*}
}

\subsection{Integralsätze der Vektoranalysis}

\paragraph{1) Gaußscher Satz:} 
Sei $\vec{A}(\vec{r})$ ein Vektorfeld im Volumen $V \subset \mathbb R^3$, so gilt:
$$\int_V d^3r\ \nabla \cdot \vec{A}(\vec{r}) = \int_{\partial V} d\vec{f}\ \cdot \vec{A}(\vec{r})$$
$$\partial V \tx{ Rand von } V$$
$$d \vec{f} = \custo{\rightarrow}{\vec{n}}{\mathclap{\textrm{nach aussen orientierter Normaleneinheutsvektor}}}\ df$$




%T1

\noindent
\emph{Bemerkung:}\\
\begin{enumerate}[i)]
	\item Analogie 1D: Fundamentalsatz der Integralrechnung:
	\begin{equation*}
	\int_{a}^{b} dx \frac{df}{dx} = f(b) - f(a)
	\end{equation*}
	\item Geometrische / physikalische Integration:\\
	Fluss des Vektorfeldes $ \vec{A} $ durch $ \partial V $
	\begin{equation*}
	\int_{\partial V} d\vec{f} \cdot \vec{A}
	\end{equation*}
	Integral über die Quellen von $ \vec{A} $
	\begin{equation*}
	\int_V d^3r \vec{\nabla} \cdot \vec{A}
	\end{equation*}
	\begin{equation*}
	\vec{A} = \const \rightarrow \vec{\nabla} \cdot \vec{A} = 0
	\end{equation*}
	\emph{Beispiel:} Geschwindigkeit einer Flüssigkeit: $ \vec{A}(\vec{r}) = \vec{v}(\vec{r}) $\\
	
	%T1
	
	$$ \vec{v} = \const \quad \vec{\nabla} \cdot \vec{v} = 0 \quad \int_{\partial V} d\vec{f} \cdot \vec{v} = 0 $$
	$ \Rightarrow $ Es gibt keine Quellen von $ \vec{v} $
	
	%T2
	
	$$ \vec{\nabla} \cdot \vec{r} \neq 0 \quad \int_{\partial V} d\vec{f} \cdot \vec{v} \neq 0 $$
	\item
	
	%T3  5 T4 fehlt
	
	\begin{align*}
	\int_V d^3r \vec{\nabla} \cdot \vec{A}(\vec{r})  =  \int_{0}^{\Delta x} dx \int_{0}^{\Delta y} dy \int_{0}^{\Delta z} \left(\frac{\partial A_x}{\partial x} + \frac{\partial A_y}{\partial y} + \frac{\partial A_z}{\partial z}\right)  \\
	\end{align*}
	\begin{align*}
	&\int_{0}^{\Delta z} dz \int_{0}^{\Delta y} dy \ub{\int_{0}^{\Delta x} dx \frac{\partial A_x}{\partial x}}_{A_x(\Delta x , y , z) - A_x(0,y,z)}\\
	& = \int_{0}^{\Delta z} dz \int_{0}^{\Delta y} A_x(\Delta x,y,z) - \int_{0}^{\Delta z} dz \int_{0}^{\Delta y} dy A_x(0,y,z)\\
	& = \int_{F^+_A} d\vec{f} \cdot \vec{A} + \int_{F^-_A} d\vec{f} \cdot \vec{A}
	\end{align*}
	\begin{equation*}
	F_x^+: \quad d\vec{f} = \vec{e}_x dy dz \qquad F_x^-: \quad d\vec{f} = - \vec{e}_x dy dz
	\end{equation*}
	ebenso gilt dann für die anderen Koordinaten:
	\begin{align*}
	\int_{0}^{\Delta x} dx \int_{0}^{\Delta z} dz \int_{0}^{\Delta y} dy \frac{\partial A_y}{\partial y} &= \int_{F^+_y} d\vec{f} \cdot \vec{A} + \int_{F^-_y} d\vec{f} \cdot \vec{A} \\
	\int_{0}^{\Delta x} dx \int_{0}^{\Delta y} dy \int_{0}^{\Delta z} dz \frac{\partial A_z}{\partial z} &= \int_{F^+_z} d\vec{f} \cdot \vec{A} + \int_{F^-_z} d\vec{f} \cdot \vec{A} \\
	\end{align*}
\end{enumerate}

$$\Rightarrow \int_V d^3 r \nabla \cdot \vec{A} = \int_{\partial V} d \vec{f} \cdot \vec{A}$$

\paragraph{2) Stokescher Satz}
Sei $\vec{A}(\vec{r})$ ein Vektorfeld, $F$ eine Fläche mit Randkurve $\partial F$, so gilt:
$$\int\limits_{\mathclap{\textrm{Linienintegral}\rightarrow\,\partial F \phantom{\textrm{Linienintegral}\rightarrow\,}}} d\vec{r} \vec{A}(\vec{r}) = \int\limits_{\mathclap{\phantom{\, \leftarrow \textrm{Oberflächenint.}} F\, \leftarrow \textrm{Oberflächenint.}}} d \vec{f} \cdot (\nabla \times \vec{A}(\vec{r}))$$


%T5


$$d\vec{f} = \vec{n} df$$
Richtung von $d\vec{f}$ und Umlauf sinn von $\partial F$: rechte Hand Regel.

\emph{Beispiel:}
\begin{equation*}
\vec{A}(\vec{r}) = \begin{pmatrix}
-y \\ x \\ 0
\end{pmatrix}
\end{equation*}

%T6





\begin{equation*}
\vec{\nabla} \times \vec{A} = \begin{pmatrix}
\frac{\partial A_z}{\partial y} - \frac{\partial A_y}{\partial z} \\
\frac{\partial A_x}{\partial z} - \frac{\partial A_z}{\partial x} \\
\frac{\partial A_y}{\partial x} - \frac{\partial A_x}{\partial y}
\end{pmatrix} = \begin{pmatrix}
0 \\ 0 \\ 1 + 1
\end{pmatrix} = 2 \vec{e}_z
\end{equation*}

%T7





\begin{equation*}
\vec{r}(\varphi) = R \begin{pmatrix}
\cos \varphi \\ \sin \varphi \\ 0
\end{pmatrix} \qquad \varphi \in [0,2\pi]
\end{equation*}
\begin{equation*}
\frac{\partial \vec{r}}{\partial \varphi} = R \begin{pmatrix}
- \sin \varphi \\ \cos \varphi \\ 0
\end{pmatrix}
\end{equation*}
\begin{align*}
\int_{\partial F} d \vec{r} \cdot \vec{A}(\vec{r}) &= \int_{0}^{2 \pi} d \varphi \frac{\partial \vec{r}}{\partial \varphi} \cdot \vec{A}( \vec{r} ( \varphi)) \\
&= \int_{0}^{2 \pi} d \varphi R ( + \sin^2 \varphi + \cos^2 \varphi) = 2 \pi R^2\\
\int_F d\vec{f} \ob{\left(\vec{\nabla} \times  \vec{A}\right)}^{2 \vec{e}_z} &= 2 \pi R^2
\end{align*}

Vektorfeld ohne Wirbel z.B. $\vec{A} = \const$\\
%T8
$\nabla \times \vec{A} = 0$\\
%T9
\bem  %T10 , %T11

\subsection{Integrale Form der Feldgleichung}
\subsection{Gaußsches Gesetz}
\begin{equation*}
\vec{\nabla} \cdot \vec{E} = \frac{1}{\epsilon_0}
\end{equation*}

%T tikz mit unförmigem volumen und einem V drinnen

\begin{align*}
\int_{V} d^3 r \vec{\nabla} \cdot \vec{E}(\vec{r}) &= \frac{1}{\epsilon_0} \int_{V} d^3 r \rho (\vec{r}) = \frac{1}{\epsilon_0} Q_V\\
&= \int_V d\vec{f} \cdot \vec{E}(\vec{r})
\end{align*}

\begin{equation*}
\rmbox{\int_{\partial V} d \vec{f} \cdot \vec{E}(\vec{r}) = \frac{1}{\epsilon} Q_V}
\end{equation*}

\subsubsection{Berechnung elektrischer Felder für hochsymmetrische Ladungsverteilungen}
\emph{Beispiel:}\\
Homogen geladene Kugel mit Radius $ R $ und Gesamtladung $ Q $. Damit ist die Ladungsdichte innerhalb der Kugel:
\begin{equation*}
\rho = \frac{Q}{V} = \frac{Q}{\frac{4}{3} \pi R^3}
\end{equation*}

%T12

\begin{equation*}
\vec{E}(\vec{r}) = E_r (r) \vec{e}_r
\end{equation*}
\begin{equation*}
\vec{r} = r \begin{pmatrix}
\sin \theta \cos \varphi \\ \sin \theta \sin \varphi \\ \cos \theta
\end{pmatrix}
\end{equation*}
$ \vec{e}_r = \frac{\vec{r}}{r} $

Fluss von $\vec{E}$ durch Oberfläche einer Kugel mit Radius $r$


%T13


$$d \vec{f} = \vec{e}_r r^2 \sin \theta d \theta d \varphi \Rightarrow d \vec{f} \cdot \vec{E} = E_r(r)r^2\sin \theta d \theta d \varphi $$

\begin{align*}
\int_{\partial K_r (0)} d\vec{f}\ \vec{E} &= \int_0^T d \theta \ \intt{0}{2 \pi} d \varphi E_r (r) r^2 \sin \theta\\
&= E_r (r) r^2 4 \pi\\
&= \frac{1}{\epsilon_0} Q_{K_r (0)} = \frac{1}{\epsilon_0} \int_{K_r (0)} d^3 r\ \rho(\vec{r}) = \frac{1}{\epsilon_0} \left\{ \begin{array}{cc}
Q & r > R\\
Q \frac{r^3}{R^3} & r \leq R 		
\end{array}\right.
\end{align*}

$$\Rightarrow E_r(r) = \frac{Q}{4 \pi \epsilon_0}\left\{ \begin{array}{cc}
\frac{1}{r^2} & r > R\\
\frac{r}{R^3} & r \leq R 		
\end{array}\right.$$




%T14 


\subsection{Satz von Stokes}
\begin{equation*}
\vec{\nabla} \times \vec{E} = 0
\end{equation*}
\textbf{Definition:} $ \gamma = \partial F $\\
$ \int_\gamma $ ist dann ein Linienintegral über eine geschlossene Kurve
\begin{equation*}
\rmbox{\int_\gamma d \vec{r} \cdot \vec{E} = \int_F d\vec{f} \cdot (\vec{\nabla} \times \vec{E}) = 0}
\end{equation*}

\subsection{Zusammenfassung: Feldgleichungen der Elektrostatik}

\textbf{differentielle Darstellung:}
\begin{equation*}
\vec{\nabla}\cdot \vec{E} = \frac{1}{\epsilon_0} \rho \quad \vec{\nabla} \times \vec{E} = 0 \quad \rightarrow \quad \vec{E} = - \vec{\nabla} \Phi \quad \rightarrow \quad \Delta \Phi = - \frac{1}{\epsilon_0} \rho
\end{equation*}
\textbf{Integral Darstellung:}
\begin{equation*}
\int_{\partial V} d\vec{f} \cdot \vec{E} = \frac{1}{\epsilon_0} Q_V \qquad , \qquad \oint_\gamma d\vec{r} \cdot \vec{E} = 0
\end{equation*}

\section{Elektrostatische Energie}
\textbf{potentielle Energie einer Punktladung im äußeren elektrischen Feld}\\
Kraft auf Ladung $ q $:
\begin{equation*}
\vec{F} = q \vec{E}
\end{equation*}
Die Arbeit bei Verschiebung der Ladung von $ \vec{a} $ nach $ \vec{b} $
\begin{align*}
W &= - \int_{\vec{a}}^{\vec{b}} d\vec{r} \cdot \vec{F} = - q \int_{\vec{a}}^{\vec{b}} d\vec{r} \cdot \vec{E}(\vec{r})\\
&= q \int_{\vec{a}}^{\vec{b}} d\vec{r} \cdot \vec{\nabla} \Phi = q \ub{( \Phi(\vec{b}) - \Phi(\vec{a})}_{\tx{Potentialdifferenz}}
\end{align*}
Die Arbeit um $ q $ aus dem unendlichen $ \infty $ nach $ \vec{r} $ zu bringen ist dann:

% tikz T16
\begin{equation*}
W = q (\Phi(\vec{r}) - \Phi(\infty))
\end{equation*}
Zur Referenz: $ \Phi(\infty) = 0 $\\
Damit ist die Energie der Ladung $ q $ im äußeren Feld:
\begin{equation*}
\Rightarrow W = q (\Phi(\vec{r}))
\end{equation*}
\begin{equation*}
\vec{E} = - \vec{\nabla} \Phi
\end{equation*}

%29.10.18

%T -1 evtl = T16 checken

\begin{tikzpicture}
\draw[thick,->] (0,0) -- (1,1) node[below,xshift=-10pt,yshift=-10pt] {$  \vec{r} $};
\draw[<-] (1.15,1.15) to[out=60,in=120] (3,1) node[right] {$ q $};
\draw[fill=black] (1.1,1.1)circle(2pt);
\draw[->] (-.5,.5) to[out=20,in=-90] (.5,1.5);
\draw[->] (.5,-.5) to[out=70,in=180] (1.5,.5) node[below,yshift=-10pt] {$ \vec{E} $};
\end{tikzpicture}

%andrez 1:
\subsection{Elektrostatische Potentielle Energie}

\paragraph{Energie einer Verteilung von Punktladungen}

$N$ Ladungen $q$: an Orten $\vec{r}_i$\\
Zunächst: $\underbrace{i - 1}_{\mathclap{\textrm{erzeugen am Ort } \vec{r}_i}}$ Ladungen $q_j$ bei $\vec{r}_j$\\
Das Potential
$$\Phi(\vec{r}_i) = \kq \summ{j = 1}{i - 1} \frac{q_i}{|\vec{r}_j - \vec{r}_i|}$$ 

%t1:
\begin{tikzpicture}
\draw[thick,->] (0,0) -- (1,1) node[above] {$ q_1 $};
\draw[thick,->] (0,0) -- (2,1) node[right] {$ q_2 $};
\draw[thick,->] (0,0) -- (1.8,0) node[right] {$ q_3 $};
\node[above,xshift=-5pt] at (.5,.5) {$ \vec{r}_1 $};
\node[right,yshift=-5pt] at (1,.5) {$ \vec{r}_2 $};
\node[below] at (.9,0) {$ \vec{r}_3 $};
\end{tikzpicture}

Arbeit um $ i $-te Ladung aus dem unendlichen nach $ \vec{r} $ zu bringen:
\begin{equation*}
W_i = q_i \Phi(\vec{r}_i) = \frac{1}{4 \pi \epsilon_0} \sum_{i=1}^{j-1} \frac{q_i q_j}{r_{ij}}
\end{equation*}
Somit ergibt sich die gesamte Arbeit für $ N $ Ladungen als:
\begin{align*}
W = \sum_{i=2}^{N} W_i &= \frac{1}{4 \pi \epsilon_0} \sum_{i=2}^{N} \sum_{j=1}^{i-1} \frac{q_i q_j}{r_{ij}}\\
&= \frac{1}{2} \sum_{i=1}^{N} \sum_{i=1}^{N} \frac{q_i q_j}{r_{ij}}
\end{align*}

\begin{equation*}
\Rightarrow W = \frac{1}{8 \pi \epsilon_0} \sum_{\substack{i,j\\i\neq j}} \frac{q_i q_j}{r_{ij}}
\end{equation*}


% andrez 2:
\begin{align*}
W &= \frac{1}{2} \summ{i = 1}{N} q_i \underbrace{\bigg( \sum_{\substack{j\\ j \neq i}} \kq \frac{q_i}{r_{ij}}\bigg)}_{\Phi_{/ \hspace{-3.5pt}i}(\vec{r}_i)}\\
&= \frac{1}{2} \summ{i = 1}{N} q_i \Phi_{/ \hspace{-4pt}i}(\vec{r}_i)
\end{align*}

\paragraph{Energie einer kontinuierlichen lokalisierten Ladungsverteilung}

\begin{minipage}{.3\linewidth}
	%t2:
	\centering
	\begin{tikzpicture}
	\draw  plot [smooth cycle, tension=.7] coordinates { (0,0) (.25,.5) (.25,1) (.75,.75) (1,0) (.5,-.25)};
	\node at (.5,.35) {$ \rho $};
	\end{tikzpicture}
\end{minipage}
\begin{minipage}{.7\linewidth}
	\begin{align*}
	W &= \frac{1}{8 \pi \epsilon_0} \int d^3r\ \int d^3r'\ \frac{\rho(\vec{r}) \rho(\vec{r}')}{|\vec{r}_j - \vec{r}_i|}\\
	&= \frac{1}{2} \int d^3r\ \rho(\vec{r}) \kq \underbrace{\int_{\mathbb R^3} d^3r'\ \frac{\rho(\vec{r}')}{|\vec{r}_j - \vec{r}_i|}}_{\Phi(\vec{r})}
	\end{align*}
\end{minipage}\\
\begin{minipage}{.3\linewidth}
	%t3:
	\centering
	\begin{tikzpicture}
	\draw  plot [smooth cycle, tension=.7] coordinates { (0,0) (.25,.5) (.25,1) (.75,.75) (1,0) (.5,-.25)};
	\node at (.5,.35) {$ \rho $};
	\draw[thick] (-.5,-1) to[out=20,in=160] (1.5,-1);
	\draw[thick] (-1,-.5) to[out=70,in=-70] (-1,1.5);
	\draw[thick,->] (-.5,-1) to[out=20,in=180] (.5,-.8);
	\draw[thick,->] (-1,-.5) to[out=70,in=-90] (-.8,.5);
	\node at (-1.4,.5) {$ \vec{E}_{\tx{ext}} $};
	\end{tikzpicture}
\end{minipage}
\begin{minipage}{.7\linewidth}
	$$W_{\textrm{ext}} = \int d^3 r\ \rho(\vec{r}) \Phi_{\textrm{ext}}(\vec{r})$$
\end{minipage}

\subsubsection{Energie $ W $ durch $ \vec{E} $ ausdrücken:}

\begin{align*}
\vec\Delta \Phi &= - \frac{1}{\epsilon_0} \rho \quad \Rightarrow \quad W = -\frac{1}{2} \int d^3r \epsilon_0 \ub{\vec\Delta \Phi(\vec{r}) \Phi(\vec{r})}_{\vec{\nabla} \cdot (\Phi \vec{\nabla} \Phi) - \equalto{(\vec{\nabla} \Phi)^2}{\vec{E}} }\\
&= -\frac{\epsilon_0}{2} \ub{\int_{\mathbb{R}^3} d^3r \vec{\nabla} \cdot (\Phi \vec{\nabla} \Phi)} + \frac{\epsilon_0}{2} \int d^3r \vec{E}(\vec{r})\\[5pt]
& \qquad \qquad \lim\limits_{R\to \infty} \int_{K_R(0)} d^3r \vec{\nabla} \cdot (\Phi \vec{\nabla} \Phi) = \lim\limits_{R \to \infty} \int_{\partial K_R(0)} \ub{d\vec{f} \cdot \ub{(\Phi \vec{\nabla}\Phi)}_{\buildrel {R\to \infty} \over \sim \frac{1}{R^3}}}_{\sim \frac{1}{R}} = 0\\
&= \frac{\epsilon_0}{2} \int d^3r \vec{E}(\vec{r})
\end{align*}
Zur Umformung oben wurde benutzt:
\begin{equation*}
\Phi \buildrel R\to \infty \over \sim \frac{1}{R} \qquad \vec{\nabla} \Phi \sim \frac{1}{R^2}\qquad d\vec{f} = \vec{n} \ub{df}_{\sim R^2}
\end{equation*}
Damit ergibt sich für die Energie einer Verteilung von Punktladungen 

% andrez 3:
$$\Rightarrow \quad \rmbox{W = \frac{\epsilon_0}{2} \int d^3 r\ \vec{E}^2(\vec{r})} \qquad \textrm{nicht für Punkladungen}$$
\begin{minipage}{.6\linewidth}
	Energiedichte des elektrostatischen Feldes
	$$\rmbox{w(\vec{r}) = \frac{\epsilon_0}{2} \vec{E}^2(\vec{r})}$$
	\textbf{\emph{Beispiel: }Plattenkondensator}\\
	Fläche $F$, Ladung $\rightarrow\ r = \frac{q}{F}\ \rightarrow\ \vec{E} = \frac{r}{\epsilon_0} \vec{e}_x$\\[5pt]
	$ \rightarrow $ Die Energiedichte ist: $ w = \frac{\epsilon_0}{2} \vec{E}^2 = \frac{\sigma^2}{2 \epsilon_0} $ (nicht für Punktladungen)\\[5pt]
	$ \rightarrow $ Die Energie beträgt: $ W = \int d^3 r w(\vec{r}) = l \cdot F \cdot \frac{\sigma^2}{2 \epsilon_0} $
\end{minipage}
\begin{minipage}{0.1\linewidth}
	$ \phantom{M} $
\end{minipage}
\begin{minipage}{.3\linewidth}
	%t4: %t5:
	\begin{tikzpicture}
	\draw[thick,->] (-1,0) -- (3,0);
	\draw (-.05,0) rectangle (.05,1.5) node[above] {$ \sigma $};
	\draw (2-.05,0) rectangle (2+.05,1.5) node[above] {$ -\sigma $};
	\node[below] at (0,0) {0};
	\node[below] at (2,0) {$ l $};
	\draw[thick,->] (.5,.4) -- (1.5,.4);
	\draw[thick,->] (.5,.8) -- (1.5,.8);
	\draw[thick,->] (.5,1.2) -- (1.5,1.2);
	\node at (1,1.8) {$ \vec{E} $};
	\end{tikzpicture}
	\begin{tikzpicture}
	\draw[->] (-1,0) -- (3,0);
	\node[above] at (0,2)  {$ \phantom{M} $};
	\draw[<-] (0,2) -- (0,-.3) node[below] {0};
	\draw (2,.2) -- (2,-.2) node[below] {$ l $};
	\draw[thick] (-1,0.01) -- (0,0.01);
	\draw[thick] (2,0.01) -- (3,0.01);
	\draw (0,1.2) -- (-.3,1.2) node[left] {$ \frac{\sigma}{\epsilon_0} $};
	\draw[thick] (2,1.2) --(0,1.2);
	\draw[thick] (2,1.21) -- (0,1.21);
	\end{tikzpicture}
\end{minipage}

\subsubsection{Potentialdifferenz - Spannung}
\begin{equation*}
\Phi(\vec{r}) - \Phi(0) = - \int_{0}^{\vec{r}} d\vec{r}' \cdot \vec{E} (\vec{r}') = - \int_{0}^{x} dx' \frac{\sigma}{\epsilon_0} = - \frac{\sigma}{\epsilon} x
\end{equation*}

%t6:
%t7:
$$
\begin{tikzpicture}
\draw[->] (0,0) -- (0,1) node[anchor=south east] {$ z $};
\draw[->] (0,0) -- (.7,.7) node[anchor=south] {$ y $};
\draw[->] (0,0) -- (1,0) node[anchor=north west] {$ x $};
\draw[dashed] (.7,.7) -- (.7+1,.7);
\draw[dashed] (1,0) -- (1+.7,.7);
\draw[thick,->] (0,0) -- (1.8,1.2) node[right,yshift=-5pt] {$ \vec{r} $};
\end{tikzpicture}
\qquad \qquad 
\begin{tikzpicture}
\draw[->] (-.5,0) -- (1.5,0) node[anchor=north west] {$ x $};
\draw[->] (0,-1) -- (0,1) node[anchor=south east] {$ \Phi $};
\draw[thick] (0,0) -- (1,-1);
\draw[dashed] (1,-1) -- (1,.2) node[anchor=north west,yshift=-5pt] {$ l $};
\end{tikzpicture}
$$

\noindent
Die Spannung zwischen zwei Kondensatorplatten ist dann:
\begin{equation*}
U = \Phi(0) - \Phi(l) = \frac{\sigma}{\epsilon_0} l = \frac{q}{\epsilon_0 F} l
\end{equation*}
Die Kapazität ist also:
\begin{equation*}
C = \frac{q}{U} = \frac{\epsilon_0 F}{l}
\end{equation*}

% andrez 4:
\noindent
\lcom{Was ist die Energie bei einer Verteilung von Punktladungen und bei einer kontinuierlichen Ladungsverteilung. Bei einer kontinuierlichen Ladungsverteilung haben wir herausgefunden:}
$$W = \frac{1}{8 \pi \epsilon_0} \sum_{\substack{i,j\\ i \neq j}} \frac{q_i q_j}{r_{ij}} \qquad \textrm{ für Punktladungen}$$
\lcom{Die Energie der Punktladung selbst steckt hier nicht drinnen. Man muss dabei aufpassen, welche Gleichung man für welches Modell benutzt.}
$$\vec{E} = \kq q \frac{\vec{r}}{r^3} \qquad \int d^3 r\ \vec{E}^2 = \int d^3 r\ \frac{1}{r^4} = \infty$$

\section{Verhalten des el. Feldes an Grenzflächen mit Flächenladung}
$ \rightarrow $ Diskontinuitäten von $ \vec{E} $\\
\emph{Beispiel:} Wir betrachten eine homogene Flächenladung.\\[10pt]
\begin{minipage}{.4\linewidth}
	%%% dat tafel dude
	%t8: - %t11: auf blatt:
	\begin{tikzpicture}
	\draw (-1,-1) to[out=45,in=180] (1.2,0);
	\draw (-1.2,-.4) to[out=45,in=180] (1,.6);
	\draw (-.8,-1.6) to[out=45,in=180] (1.4,-.6);
	\draw (0,1) to[out=-30,in=90] (1,-1);
	\draw (-.7,.7) to[out=-30,in=90] (.3,-1.3);
	\draw (-1.4,.2) to[out=-30,in=90] (-.4,-1.6);
	\draw[thick,<-] (.5,.3) to[out=60,in=180] (1.5,1) node[right] {$ \sigma(\vec{r}) = \frac{\tx{Ladung}}{\tx{Fläche}} $};
	\end{tikzpicture}
\end{minipage}
\begin{minipage}{.3\linewidth}
	\begin{tikzpicture}
	\draw[thick,->] (0,0) -- (0,.75) node[anchor=south east] {$ z $};
	\draw[thick,->] (0,0) -- (1,0) node[anchor=north west] {$ y $};
	\draw[thick,->] (0,0) -- (-.75,-.75) node[anchor=south east] {$ x $};
	\draw (-.2,-.2) -- (-.2+1,-.2);
	\draw (-.4,-.4) -- (-.4+1,-.4);
	\draw (-.6,-.6) -- (-.6+1,-.6);
	\node at (.5,-.75) {$ \sigma $};
	\end{tikzpicture}
\end{minipage}\nolinebreak
\begin{minipage}{.3\linewidth}
	\begin{tikzpicture}
	\draw[thick,->] (0,-1.2) -- (0,1.2) node[above] {$ E_z $};
	\draw[thick,->] (-2,0) -- (2,0) node[anchor=north west] {$ z $};
	\draw[thick] (2,.8) -- (0,.8) ;
	\draw (0,.8) -- (-.15,.8) node[left] {$ \frac{\sigma}{2 \epsilon_0} $};
	\draw[thick] (-2,-.8) -- (0,-.8);
	\draw (0,-.8) -- (.15,-.8) node[right] {$\frac{\sigma}{2 \epsilon_0}$};
	\end{tikzpicture}
\end{minipage}
\begin{equation*}
\Rightarrow \vec{E} = \frac{\sigma}{2 \epsilon_0} \tx{sgn}(z) \vec{e}_z
\end{equation*}
\begin{align*}
\vec{E}_\perp &= \pm \frac{\sigma}{2 \epsilon_0} \vec{e}_z\\
\vec{E}_\parallel &= 0
\end{align*}
\begin{minipage}{.6\linewidth}
	Das elektrische Feld $ \vec{E}_\parallel $ ist gleich der Ableitung des elektrischen Potentials:\\
	Das elektrische Potential ist also stetig.
\end{minipage}
\begin{minipage}{.4\linewidth}
	\hspace{30pt}
	\begin{tikzpicture}
	\draw[thick,->] (0,-.7) -- (0,.7) node[anchor=north west] {$ z $};
	\draw[thick,->] (-1,0) -- (1.2,0) node[anchor=south east] {$ \Phi(z) $};
	\draw (-1,-.7) -- (0,0) -- (1,-.7);
	\end{tikzpicture}
\end{minipage}\\
\begin{minipage}{.4\linewidth}
	%andrez 5:
	\paragraph{Normalkomponente $\vec{E}_\perp$}
	
	Gaußscher Satz für $V$:
	\vspace{50pt}
\end{minipage}
\begin{minipage}{.6\linewidth}
	%t12:
	\centering
	\begin{tikzpicture}
	\coordinate (o) at (0,0);
	\coordinate (O) at (-4,-2);
	\draw[fill=gray!7.5] (-3,-1.4) -- (1.5,-1.4) -- (3,1.4) -- (-1.5,1.4) -- (-3,-1.4);
	\draw[->] (O) -- ++(.4,0);
	\draw[->] (O) -- ++(0,.4);
	\draw[->] (O) -- ++(.3,.3);
	\draw[red,fill=red!5] (o) ellipse (1cm and .5cm);
	\draw[red,fill=red!5] ($ (o) + (0,2) $) ellipse (1cm and .5cm);
	\draw[red,dashed,fill=red!5] ($ (o) + (0,-2) $) ellipse (1cm and .5cm);
	\draw[red] (-1,2) -- (-1,0);
	\draw[red] (1,2) -- (1,0);
	\draw[red,dashed] (-1,-2) -- (-1,0);
	\draw[red,dashed] (1,-2) -- (1,0);
	\draw[red,thick,->] (o) -- ++(0,3) node[anchor=west] {$ \vec{n}_+ $};
	\draw[blue,thick,->] (o) -- ++(0,-3) node[anchor=east] {$ \vec{n}_- $};
	\draw[thick,->] (O) -- (o) node[xshift=-60pt,yshift=-20pt] {$ \vec{r} $};
	\node[right] at (o) {$ F $};
	\node[anchor=north west] at ($ (o) + (0,1.2) $) {$ \partial V_+ $};
	\node[anchor=north west] at ($ (o) + (0,-.8) $) {$ \partial V_- $};
	\draw[decorate, decoration={brace,amplitude=10pt,mirror,raise=2pt}, xshift=-4pt,red]  (-1,2) -- node[left=12pt] {$\Delta z$}  (-1,-2);
	\node at (2.2,1) {$ \sigma(\vec{r}) $};
	\draw ($ (o) + (.5,.3) $) to[out=30,in=150] (2.7,0) node[right] {$ V = \Delta z \cdot F $};
	\node[circle,fill=blue,inner sep=1pt,minimum size=1pt] at (0,-2) {};
	\node[circle,fill=red,inner sep=1pt,minimum size=1pt] at (0,2) {};
	\end{tikzpicture}
\end{minipage}

\begin{align*}
\int_V d^3 r'\ \vabla \cdot \vec{E}(\vec{r}) &= \int_{\partial V} d\vec{f}'\ \vec{E}(\vec{r})\\
&=\custo{\rotatebox{90}{$\scriptscriptstyle{\Delta z \to 0}$}\atop \longrightarrow}{\int_{\textrm{Mantel}} d\vec{f}'\ \vec{E}}{\hspace{-23.5pt}0} + \custo{\rotatebox{90}{$\scriptscriptstyle{\Delta z \to 0}$}\atop \longrightarrow}{\int_{\partial V_+} d\vec{f}'\ \vec{E}(\vec{r})}{\int_F df'\ \vec{n} \cdot \vec{E}_+} + \custo{\rotatebox{90}{$\scriptscriptstyle{\Delta z \to 0}$}\atop \longrightarrow}{\int_{\partial V_-} d\vec{f}'\ \vec{E}}{- \int_F df'\ \vec{n} \cdot \vec{E}_-}
\end{align*}
$ \vec{E}_{\pm} $ ist das Feld auf beiden Seiten der Grenzfläche

% eaualto zwischen Zeile 1 und 2 der equations ...
\begin{equation*}
\int_{\partial V} d\vec{f}' \vec{E} \buildrel \Delta z \to 0 \over \longrightarrow \int_{F} df \vec{n} \cdot (\vec{E}_+ - \vec{E}_-) \buildrel F \to 0 \over \longrightarrow F\  \vec{n} \cdot \big(\vec{E}_+(\vec{r}) - \equalto{\vec{E}_-(\vec{r})}{}\big)
\end{equation*}
\begin{equation*}
\int_V d^2r' \equalto{\vec{\nabla} \cdot \vec{E}(\vec{r}')}{\frac{1}{\epsilon_0} \rho(\vec{r}')} = \frac{1}{\epsilon_0} \int_V d^3r' \rho(\vec{r}) = \frac{1}{\epsilon_0} \int_F df'  \sigma(\vec{r}') \buildrel F \to 0 \over \longrightarrow \frac{1}{\epsilon_0} F \sigma(\vec{r})
\end{equation*}
\begin{equation*}
\Rightarrow \vec{n} \cdot \left(\vec{E}_+(\vec{r}) - \vec{E}_-(\vec{r}) \right) = \frac{1}{\epsilon_0} \sigma(\vec{r})
\end{equation*}
\begin{equation*}
E_{\perp_\pm} = \vec{n} \cdot \vec{E}_{\pm} \qquad E_{\perp_+}(\vec{r}) - E_{\perp_-} (\vec{r}) = \frac{1}{\epsilon_0} \sigma(\vec{r})
\end{equation*}

%t13: und Tafel vor M4:
\noindent
\begin{minipage}{.4\linewidth}
	\paragraph{Tangentialkomponente $ E\parallel $}
	
	Satz von Stokes:
	\vspace{50pt}
\end{minipage}
\begin{minipage}{.6\linewidth}
	\centering
	\begin{tikzpicture}[scale=1.25]
		\draw (-2,1) to[out=-10,in=135] (0,0) to[out=-45,in=100] (1,-2);
		%\draw (-.3,1.25) -- (1.25,.3) -- (.3,-1.25) -- (-1.25,-.3) -- (-.3,1.25);
		\draw[->] ($ (0,0) + (-.25,1.25) + (-45:2cm) + (-135:1.5cm) $) -- ($ (0,0) + (-.25,1.25) + (-45:2cm) + (-135:.6cm) $);
		\draw[->] ($ (0,0) + (-.25,1.25) + (-45:2cm) $) -- ($ (0,0) + (-.25,1.25) + (-45:1cm) $);
		\draw[->] ($ (0,0) + (-.25,1.25) $) -- ($ (0,0) + (-.25,1.25) + (-135:.75cm) $);
		\draw[->] ($ (0,0) + (-.25,1.25) + (-135:1.5cm) $) -- ($ (0,0) + (-.25,1.25) + (-135:1.5cm) + (-45:1cm) $);
		\draw[draw=none,fill=gray,opacity=.1] ($ (0,0) + (-.25,1.25) $) -- ++(-135:1.5cm) -- ++(-45:2cm) -- ++(45:1.5cm) -- ++(135:2cm);
		\draw ($ (0,0) + (-.25,1.25) $) -- ++(-135:1.5cm) -- ++(-45:2cm) -- ++(45:1.5cm) -- ++(135:2cm);
		\draw[decorate, decoration={brace,amplitude=10pt,raise=2pt},red] ($ (0,0) + (-.25,1.25) $) -- node[above=10pt,xshift=13pt] {$ L $} ($ (0,0) + (-.25,1.25) + (-45:2cm) $);
		\draw[decorate, decoration={brace,amplitude=10pt,mirror,raise=2pt},red] ($ (0,0) + (-.25,1.25) $) -- node[above=10pt,xshift=-12pt] {$ \Delta z $} ($ (0,0) + (-.25,1.25) + (-135:1.5cm) $);
		\node at (1,-1) {$ \gamma $};
		\coordinate (O) at (-1.5,-1.5);
		\draw[->] (O) -- ++(.4,0);
		\draw[->] (O) -- ++(0,.4);
		\draw[->] (O) -- ++(.25,.25);
		\node[anchor=north east] at (O) {$ O $};
		\draw[thick,->] (O) -- (0,0) node[anchor=north east,yshift=-15pt,xshift=-25pt] {$ \vec{r} $};
		\draw[thick,->] (0,0) -- ++(135:.75cm) node[anchor=south west,yshift=-10pt,xshift=5pt] {$ \vec{E} $};
		\node at (-2,.3) {$ \sigma $};
	\end{tikzpicture}
\end{minipage}
\begin{equation*}
0 = \oint_r d\vec{r} \vec{E}(\vec{r}') = \ub{
	\int\limits_{\begin{tikzpicture}[scale=.7]
		\draw[dashed] (0,0) -- (1,0) -- (1,.7) -- (0,.7) -- (0,0);
		\draw[thick,->] (1,0) -- (1,.7);
		\end{tikzpicture}} \dots + 
	\int\limits_{\begin{tikzpicture}[scale=.7]
		\draw[dashed] (0,0) -- (1,0) -- (1,.7) -- (0,.7) -- (0,0);
		\draw[thick,->] (0,0) -- (0,.7);
		\end{tikzpicture}} \dots}_{= 0 \tx{ für } \Delta z \to 0} + \ub{
	\int\limits_{\begin{tikzpicture}[scale=.7]
		\draw[dashed] (0,0) -- (1,0) -- (1,.7) -- (0,.7) -- (0,0);
		\draw[thick,->] (1,.7) -- (0,.7);
		\end{tikzpicture}} d\vec{r}' \vec{E} + 
	\int\limits_{\begin{tikzpicture}[scale=.7]
		\draw[dashed] (0,0) -- (1,0) -- (1,.7) -- (0,.7) -- (0,0);
		\draw[thick,->] (0,0) -- (1,0);
		\end{tikzpicture}} d\vec{r}' \vec{E} }_{
	\int\limits_{\begin{tikzpicture}[scale=.4]
		\draw[dashed] (0,0) -- (1,0) -- (1,.7) -- (0,.7) -- (0,0);
		\draw[thick,->] (1,0) -- (0,0);
		\end{tikzpicture}} d\vec{r}' (\vec{E}_+ -  \vec{E}_-)}
\end{equation*}
\begin{equation*}
0 = \oint_\gamma d\vec{r}' \cdot \vec{E} \buildrel \Delta z \to 0 \over \longrightarrow \int_{-\frac{L}{2}}^{-\frac{L}{2}} ds \vec{t} \cdot \left(\vec{E}_+ - \vec{E}_-\right) \buildrel L \to 0 \over \longrightarrow L \vec{t} \cdot \left(\vec{E}_+(\vec{r}) - \vec{E}_-(\vec{r})\right) = 0
\end{equation*}
\begin{equation*}
\rightarrow \vec{t} \cdot (\vec{E}_+ (\vec{r}) - \vec{E}_- (\vec{r}) = 0
\end{equation*}
$\rightarrow $ Die Tangentialkomponente ist stetig
\begin{equation*}
E_{\parallel_{+}} = E_{\parallel_{-}}
\end{equation*}
Insgesamt ergibt sich damit:
\begin{equation*}
\vec{E}_+(\vec{r}) - \vec{E}_-(\vec{r}) = \frac{\sigma}{\epsilon_0} \vec{n}
\end{equation*}
\begin{minipage}{.6\linewidth}
	Das elektrische Potential $ \Phi $ ist damit stetig.
	\begin{equation*}
	\ub{\Phi(\vec{r}_b) - \Phi(\vec{r}_a)}_{\Phi_+(\vec{r}) - \Phi_-(\vec{r})} = \int_{\vec{r}_a}^{\vec{r}_b} d\vec{r}' \cdot \vec{E} \quad \buildrel \Delta z \to 0 \over \longrightarrow 0
	\end{equation*}
\end{minipage}
\begin{minipage}{.4\linewidth}
	%t14:
	\centering
	\begin{tikzpicture}
		\draw[fill=gray!10] (-1.2,-.7) -- (.8,-.7) -- (1.2,.7) -- (-.8,.7) -- (-1.2,-.7);
		\draw[thick,<-] (0,0) -- (1.5,-1) node[anchor=west] {$ \vec{r} $};
		\node[anchor=south west] at (1,.7) {$ \sigma $};
		\draw[thick,->] (0,0) -- (0,1) node[anchor=south west] {$ \vec{r}_n $};
		\draw[thick,dashed] (0,0) -- (0,-.7);
		\draw[thick,->] (0,-.7) -- (0,-1) node[anchor=north west] {$ \vec{r}_a $};
		%\draw[decorate, decoration={brace,amplitude=10pt,raise=2pt},red] ($ (0,0) + (-.25,1.25) $) -- node[above=10pt,xshift=13pt] {$ L $} ($ (0,0) + (-.25,1.25) + (-45:2cm) $);
		\draw[decorate, decoration={brace,amplitude=10pt,mirror,raise=2pt+10pt}] (0,1) -- node[left,xshift=-28pt] {$ \Delta z $} (0,-1);
	\end{tikzpicture}
\end{minipage}

% andrez 6 oder 5:
\subsection{Randbedingungen an el. Leitern}

\begin{minipage}{.6\linewidth}
	Leiter: Material mit freibeweglichen Ladungsträgern (Metall)\\
	Eigenschaften von $\vec{E}$ im Leiter:
\end{minipage}
\begin{minipage}{.4\linewidth}
	%% T15:
	\centering
	\begin{tikzpicture}
		\draw[thick] (0,0) arc (90:0:2.5cm);
		\draw[draw=none,pattern=north east lines] (0,0) arc (90:0:2.5cm) -- ++(-2.5,0) -- (0,0);
		\node[right] at (1.5,0) {Vakuum};
		\node[fill=white,opacity=.8,rounded corners] at (1,-2) {Leiter};
		\node[fill=white,opacity=.8,rounded corners] at (1,-1) {$ \vec{E} = 0 $};
	\end{tikzpicture}
\end{minipage}\\[10pt]
\begin{minipage}{.6\linewidth}
	\begin{enumerate}[i)]
		\item $\vec{E} = 0$
		\item $0 = \vabla \cdot \vec{E} = \frac{1}{\epsilon_0} \rho, \qquad \rho(\vec{r}) = 0$
		\item Nettoladung befinden sich an Oberfläche %t16
		\item Potential $\Phi(\vec{r}_b) - \Phi(\vec{r}_a) = 0 \quad \rightarrow \quad \Phi(\vec{r}) = \const$ % t17
	\end{enumerate}
\end{minipage}
\begin{minipage}{.4\linewidth}
		%% t16:
		\centering
		\begin{tikzpicture}[scale=.8]
		\draw[pattern=north east lines] (0,0) circle (1cm);
		\draw[fill=white,opacity=.6,draw=none] (0,0) circle (.995cm);
		\foreach \x/\xtext in {0/$+$,45/$+$,90/$+$,135/$+$,180/$+$,225/$+$,270/$+$,315/$+$}
		\node at (\x:.8cm) {\xtext};
		\draw[<-] (-45:1.2cm) -- ++(.5,-.2) node[right] {$ \substack{\tx{Flächenladung} \\ \sigma} $};
		\end{tikzpicture}\\
		%% t17:
		\begin{tikzpicture}[scale=.8]
		\draw[pattern=north east lines] (0,0) circle (1cm);
		\draw[fill=white,opacity=.7,draw=none] (0,0) circle (.995cm);
		\node[circle,fill=black,inner sep=1pt,minimum size=1pt] (b) at (-.7,.25) {};
		\node[circle,fill=black,inner sep=1pt,minimum size=1pt] (a) at (.7,.25) {};
		\draw (a) --(b);
		\node[below] at (a) {$ \vec{r_a} $};
		\node[below] at (b) {$ \vec{r_b} $};
		\end{tikzpicture}\hspace{66pt}
\end{minipage}

\paragraph{Randbedingungen}
%% T18
$$\vec{E}_+ - \vec{E}_- = \frac{\sigma^-}{\epsilon_0} \vec{n}$$
$$\vec{E}_- = 0$$
$$\rightarrow \vec{E}_+(\vec{r}) = \frac{\sigma(\vec{r})}{\epsilon_0} \vec{n} {(\vec{r})}$$
\folie{Ladung an Oberfläche eines Leiters}


% 5.11.18

\section{Randwertprobleme (RWP) der Elektrostatik und \\ Lösungsmethoden}

\subsection{Formulierung des Randwertproblems}

Das elektrische Potential: $ \Phi(\vec{r}): \quad \vec{E}(\vec{r}) = - \vec{\nabla} \Phi(\vec{r}) $\\
$$ \vec{\Delta} \Phi (\vec{r}) = -\frac{1}{\epsilon_0} \rho(\vec{r})  \qquad \tx{Poisson-Gleichung}$$
Für eine gegebene lokale Ladungsverteilung $ \rho $ gilt:
\begin{equation*}
\Phi(\vec{r}) = \frac{1}{4 \pi \epsilon_0} \int d^3 r' \frac{\rho(\vec{r}')}{|\vec{r} - \vec{r}'|}
\end{equation*}
\begin{equation*}
\rightarrow \Phi(\vec{r}) \buildrel |\vec{r}| \rightarrow 0 \over \longrightarrow 0
\end{equation*}
Typische Problemstellung:\\
Ladungsverteilung $\rho$ + Werte des Potentials auf Randfläche\\[5pt]
\bei\\
\begin{minipage}{.6\linewidth}
	Randwertproblem: Gegeben: $\rho(\vec{r}')$ im Raumbereich $V$\\
	$\Phi(\vec{r})$ oder $\vec{E}(\vec{r})$ auf Randfläche $\partial V$\\
	Gesucht: $\Phi(\vec{r}),\ \vec{E}(\vec{r})$ überall in $V$
\end{minipage}
\begin{minipage}{.4\linewidth}
	%t1:
	\centering
	\begin{tikzpicture}
		\draw[pattern=north east lines] (0,0) circle (.7cm);
		\draw[thick] (-.7,0) -- ++(-.5,0) -- ++(0,-1);
		\coordinate (e) at (-1.2,-1);
		\draw[thick] ($ (e) + (-.25,0) $) -- ++(.5,0);
		\draw[thick] ($ (e) + (-.125,0) + (0,-.1) $) -- ++(.25,0);
		\draw[thick] ($ (e) + (-.0625,0) + (0,-.2) $) -- ++(.125,0);
		\draw (.25,.15) to[out=55,in=190] (1.2,.8) node[right] {$ \Phi = 0 $};
		\node[fill=black,circle,inner sep=1pt,minimum size=1pt] (q) at (1.5,0) {};
		\node[right] at (q) {$ q $};
	\end{tikzpicture}
\end{minipage}
Zwei Fälle:
\begin{enumerate}[i)]
	\item $ \Phi(\vec{r}) $ ist auf der Randfläche gegeben\\
	$ \rightarrow $ \textbf{Dirichlet-Randbedingung}
	\item $ \vec{E}(\vec{r}) $ ist auf der Randfläche gegeben\\
	$ \rightarrow $ \textbf{Neumannsche Randbedingung}
\end{enumerate}
\noindent
\begin{minipage}{.6\linewidth}
	Gegeben sei: $ \vec{n} \cdot \vec{E} $ dies ist gleich der \textbf{Normalenableitung}:
	\begin{equation*}
	\vec{n} \cdot \vec{E} = - \vec{n} \vabla \Phi = - \prt{\Phi}{n}
	\end{equation*}
\end{minipage}
\begin{minipage}{.4\linewidth}
	%t2:
	\centering
	\begin{tikzpicture}
		\draw[draw=none,pattern=north east lines] (0,0) -- (120:1.5cm) arc (120:-20:1.5cm) -- (0,0);
		\draw ($ (0,0) + (120:1.5cm)$) arc (120:-20:1.5cm);
		\draw[thick,->] ($ (50:1.5cm) $) -- ++(50:1cm) node[xshift=-15pt,yshift=-5pt] {$ \vec{n} $};
	\end{tikzpicture}
	\vspace{5pt}
\end{minipage}
\lcom{Wir beschränken uns vorwiegend auf den ersten Fall. }
\lcom{Zur Lösung dieser Probleme gibt es einige Methoden. Zum Einstieg und zur Wiederholung betrachten wir zunächst die Methode der Spiegelladung. }

\subsection{Methode der Bildladung (Spiegelladung)}

\subsubsection{Punktladung vor leitender, geerdeter Metallplatte}

\begin{minipage}{.6\linewidth}
	\begin{equation*}
	\vec{\Delta} \Phi(\vec{r}) = - \frac{1}{\epsilon_0} \rho (\vec{r}) = - \frac{q}{\epsilon_0} \delta(\vec{r} - \vec{r}_0)
	\end{equation*}
	$ \vec{r} \in V  \qquad \vec{r}_0 = (d,0,0) \qquad V = \{\vec{r} \in \mathbb{R}^3 , x > 0\}$\\[5pt]
	Randbedingungen:
	\begin{equation*}
	\pofr = 0 \quad \tx{für} \quad \vec{r} \in \partial V ,\quad \quad \tx{d.h.} \ \vec{r} = (0,y,z)
	\end{equation*}
\end{minipage}
\begin{minipage}{.4\linewidth}
	%t3:
	\centering
	\begin{tikzpicture}
		\draw[thick,->] (-.7,0) -- (2,0) node[anchor=north west] {$ x $};
		\draw ($ (-.1,0) + (0,-1.5) $) -- ++(0,3);
		\draw ($ (.1,0) + (0,-1.5) $) -- ++(0,3);
		\draw[draw=none,pattern=north east lines] ($ (-.1,0) + (0,-1.5)  $) rectangle ($ (.1,0) + (0,1.5) $);
		\draw[thick] (0,-.1) -- (0,.1) node[anchor=north east] {$ 0 $};
		\node[fill=black,circle,inner sep=1pt,minimum size=1pt] (q) at (1.5,0) {};
		\node[above] at (q) {$ q $};
		\draw[decorate, decoration={brace,amplitude=5pt,mirror,raise=2pt}, yshift=-4pt,black] (0,0) -- node[below=10pt] {$ d $} (1.5,0);
		\coordinate (e) at (-1,-1.5);
		\draw[thick] (e) -- ++(0,.7) -- ++(.9,0);
		\draw[thick] ($ (e) + (-.25,0) $) -- ++(.5,0);
		\draw[thick] ($ (e) + (-.125,0) + (0,-.1) $) -- ++(.25,0);
		\draw[thick] ($ (e) + (-.0625,0) + (0,-.2) $) -- ++(.125,0);
		\draw (0,.5) to[out=120,in=10] (-1.2,.8) node[left] {$ \Phi = 0 $};
		\node at (1.8,1.2) {$ V \ \ \Phi(\vec{r}) = ? $};
	\end{tikzpicture}
	\vspace{10pt}
\end{minipage}\\
\textbf{Idee:} Ersetze ursprüngliche Problem durch ''Fiktives'' Problem mit zusätzlichen Ladungen außerhalb von $V$, welche die Randbedingungen simulieren.\\[5pt]
Potential der Punkladungen in $\vec{r}_0$:
$$\Phi_q (\vec{r}) = \kq \frac{q}{|\vec{r} - \vec{r}_0|}$$
\noindent
\begin{minipage}{.5\linewidth}
	addiere Ladung $-q$ in $\vec{r}'_0 = (-d, 0, 0) = -\vec{r}_0$\\[5pt]
	\begin{equation*}
	\pofr = \kq \left( \frac{q}{|\vec{r} - \vec{r}_0|} - \frac{q}{|\vec{r}+\vec{r}_0|} \right)
	\end{equation*}
\end{minipage}
\begin{minipage}{.5\linewidth}
	%t4:
	\centering
	\begin{tikzpicture}
		\draw[->] (-2,0) -- (2,0) node[anchor=north west] {$ x $};
		\draw[->] (0,-1) -- (0,1) node[anchor=south east] {$ y $};
		\coordinate (q) at (1.5,0);
		\coordinate (-q) at (-1.5,0);
		\node[circle,fill=black,inner sep=1pt,minimum size=1pt] at (-q) {};
		\node[circle,fill=black,inner sep=1pt,minimum size=1pt] at (q) {};
		\node[above] at (-q) {$ -q $};
		\node[above] at (q) {$ q $};
		\draw[decorate,decoration={brace,amplitude=5pt,raise=2pt},yshift=-4pt,black] (0,0) -- node[below=10pt] {$ d $} (-1.5,0);
		\draw[decorate,decoration={brace,amplitude=5pt,mirror,raise=2pt},yshift=-4pt,black] (0,0) -- node[below=10pt] {$ d $} (1.5,0);
	\end{tikzpicture}
\end{minipage}\\
Schauen wir nun nach ob dies die Poisson-GLeichung erfüllt:
\begin{align*}
\vec{\Delta} \Phi &= \frac{q}{4 \pi \epsilon_0} \Bigg( \ub{\Delta \frac{1}{|\vec{r} - \vec{r}_0|}}_{= - 4 \pi \delta(\vec{r} - \vec{r}_0)} - \ub{- \Delta \frac{1}{|\vec{r} + \vec{r}_0|}}_{= - 4 \pi \delta(\vec{r} + \vec{r}_0)}\Bigg)\\
&= \underset{\checkmark}{- \frac{q}{\epsilon_0} \delta(\vec{r} - \vec{r}_0)} + \frac{q}{\epsilon_0} \ub{\delta(\vec{r} + \vec{r}_0)}_{\mathclap{\qquad = 0 \ \tx{für} \ \vec{r} \neq - \vec{r}_0 \ \checkmark \quad \forall \vec{r} \in V}}
\end{align*}

\subsubsection{Diskussion der Lösung}
\begin{enumerate}[i)]
	\item \textbf{Struktur}
	$$\pofr = \underbrace{\frac{q}{4 \pi \epsilon_0} \frac{1}{|\vec{r}-\vec{r}_0|}}_{\eqdef\ \Phi_{\textrm{s}}(\vec{r})} + \underbrace{\frac{(-q)}{4 \pi \epsilon_0} \frac{1}{|\vec{r}+\vec{r}_0|}}_{\eqdef\ \Phi_{\textrm{hom}}(\vec{r})}$$ 
	$ \vec{r} \in V $
	\begin{equation*}
	\Delta \Phi_\tx{s}(\vec{r}) = - \frac{1}{\epsilon_0} \rho(\vec{r}) \quad \tx{Poisson-Gleichung}
	\end{equation*}
	\begin{equation*}
	\Delta \Phi_{\tx{hom}} (\vec{r}) = 0 \qquad \tx{Laplace-Gleichung}
	\end{equation*}
	Mathematisch: Lösung inhomogener DGL
	\begin{equation*}
	\pofr = \Phi_\tx{s}(\vec{r}) + \Phi_{\tx{hom}}(\vec{r})
	\end{equation*}
	$ \Phi_{\tx{hom}} $ wird so gewählt, dass die Randbedingungen erfüllt werden:
	\begin{equation*}
	\vec{r} \in \partial V: \quad \Phi_\tx{o} (\vec{r}) = \Phi_\tx{s}(\vec{r}) + \Phi_{\tx{hom}}(\vec{r})
	\end{equation*}	
	\item \textbf{Elektrisches Feld}
	\begin{equation*}
	\vec{E} = - \vec{\nabla} \Phi = \frac{q}{4 \pi \epsilon_0} \left(\frac{(x-d,y,z)}{|\vec{r} - \vec{r}_0|^3} - \frac{(x+d,y,z)}{|\vec{r} + \vec{r}_0|^3}\right)
	\end{equation*}
	An der Oberfläche $ x \to 0 $, $ x \ge 0 $\\
	$ |\vec{r} \pm \vec{r}_0|^3 \rightarrow (d^2 + y^2 + z^2) $\\
	\begin{minipage}{.5\linewidth}
		\begin{equation*}
		\eofr \bigg|_{\vec{r} \in \partial V} = - \frac{qd}{2 \pi \epsilon_0} \frac{1}{(d^2 + y^2 + z^2) ^{3/2}} \vec{e}_x
		\end{equation*}
		\lcom{Durch das externe elektrische Feld verschieben sich die Ladungsträger im Metall und es entsteht eine Influenzladung an der Oberfläche.}
	\end{minipage}
	\begin{minipage}{.5\linewidth}
		\centering
		%t5: \draw  plot [smooth cycle, tension=1] coordinates { (0,0) (.5,1.5) (.5,2.5) (2,2) (3,2) (2.5,1) (1.5,0) };
		\begin{tikzpicture}
			\draw[->] (-.5,0) -- (2.5,0) node[anchor=north west] {$ x $};
			\coordinate (q) at (1.5,0);
			\node[circle,fill=black,inner sep=1pt,minimum size=2pt] at (q) {};
			\node[yshift=15pt,xshift=30pt] at (q) {$ q > 0 $};
			\draw[->] (0,-2) -- (0,2) node[anchor=south west] {$ z $};
			\draw[draw=none,pattern=north west lines] (-.3,1.5) node[above,xshift=-5pt] {$ \sigma(z) $} rectangle (0,-1.5);
			\draw[thick,->] (q) to[out=180,in=0] (0,0);
			\draw[thick,->] (q) to[out=155,in=0] (0,.20);
			\draw[thick,->] (q) to[out=90,in=0] (0,.50);
			\draw[thick,->,looseness=1.3] (q) to[out=65,in=0] (0,.85);
			\draw[thick,->,looseness=1.6] (q) to[out=45,in=0] (0,1.3);
			\draw[thick,->,looseness=1.3] (q) to[out=25,in=-45] ($ (q) + (.4,1) $);
			\draw[thick,->] (q) to[out=-155,in=0] (0,-.20);
			\draw[thick,->] (q) to[out=-90,in=0] (0,-.50);
			\draw[thick,->,looseness=1.3] (q) to[out=-65,in=0] (0,-.85);
			\draw[thick,->,looseness=1.6] (q) to[out=-45,in=0] (0,-1.3);
			\draw[thick,->,looseness=1.3] (q) to[out=-25,in=45] ($ (q) + (.4,-1) $);
			\coordinate (n) at (-.5,0);
			\node[rounded corners,align=center,left] at (n) {$ \vec{E}_{\parallel} = 0 $ \\ \\ $ \vec{E}_{\perp} \neq 0 $};
		\end{tikzpicture}
	\end{minipage}
	\item \textbf{Influenzladung auf Metalloberfläche}
	$$\vec{E}_+ - \equalto{\vec{E}_-}{0} = \frac{\sigma}{\epsilon_0} \vec{n} \qquad \vec{n} = \vec{e}_x$$
	$\vec{r} \in \partial V$:
	$$\sigma (\vec{r}) = \epsilon_0 \vec{E}_+(\vec{r}) = - \frac{qd}{2 \pi(d^2 + y^2 + z^2)^{\nicefrac{3}{2}}}$$
	gesamte influenzierte Ladung
	$$q_i = \int_{\partial V} df\ \sigma(\vec{r}) = \dots = -q$$
	\item \textbf{Kraft zwischen Punktladungen und Metallplatte}
	$$\vec{F} = q \vec{\tilde{E}}(\vec{r}_0) = \frac{-q^2}{4 \pi \epsilon_0 (2d)^2}\vec{e}_x$$
\end{enumerate}

% andrez 5:
\subsubsection{Eindeutigkeit der Lösung des Randwertproblems}
Dirichlet-Randwertproblem:\\
\begin{minipage}{.5\linewidth}
	\begin{align*}
	\Delta \pofr = -\frac{1}{\epsilon_0} \rho(\vec{r}) &\qquad \vec{r} \in V\\
	\pofr = \Phi_0 (\vec{r}) \quad \; &\qquad \vec{r} \in \partial V \quad \;
	\end{align*}
\end{minipage}
\begin{minipage}{.5\linewidth}
	\centering
	\begin{tikzpicture}
		\draw (0,1) -- (0,-1);
		\draw[draw = none,pattern = north east lines] (-.3,1) rectangle (0,-1);
		\node[fill=black,circle,inner sep=1pt,minimum size=1pt] (p) at (-.1,0) {};
		\node[fill=black,circle,inner sep=1pt,minimum size=1pt] (q) at (1,0) {};
		\node[right] at (q) {$ q $};
		\node at (1,.5) {$ V $};
		\node at (0,1.3) {$ \partial V $};
		\draw[thick] (p) to[out=135,in=-10] (-.5,.3) node[left] {$ \Phi_0 $};
	\end{tikzpicture}
\end{minipage}
Annahme: $\Phi_1,\ \Phi_2$ lösen RWP
\begin{align*}
\textrm{d.h. }\qquad \Delta\Phi_1(\vec{r}) = \frac{1}{\epsilon_0} \rho(\vec{r}) = \Delta \Phi_2 (\vec{r}) &\qquad \vec{r} \in V\\
\Phi_1 (\vec{r}) = \Phi_0(\vec{r}) = \Phi_2(\vec{r}) &\qquad \vec{r} \in \partial V
\end{align*}
Setze :
\begin{equation*}
\psi(\vec{r}) \defeq \Phi_1(\vec{r}) - \Phi_2(\vec{r})
\end{equation*}
\begin{equation*}
\Delta \Phi(\vec{r}) = 0 \quad \vec{r} \in V
\end{equation*}
\begin{equation*}
\vec{r} \in \partial V \quad \psi(\vec{r}) = \Phi_1(\vec{r}) - \Phi_2(\vec{r}) = 0
\end{equation*}

\subsubsection{Greensche Identität:}
$ g,h $ Funktionen an V:
\begin{align*}
&\ \ \ \, \int_V d^3 r \left[\left(\vec{\nabla} (\vec{r})\right) \cdot \left(\vec{\nabla} h(\vec{r}))\right) + g(\vec{r}) \Delta h(\vec{r})\right]\\
&= \int_{\partial V} d\vec{f} \cdot \left(g(\vec{r}) \vec{\nabla} h(\vec{r}\right)\\
&= \int_{\partial V} d\vec{f} g(\vec{r}) \ub{\vec{n} \cdot \vec{\nabla} h(\vec{r})}_{= \frac{\partial h}{\partial n} (\vec{r})}
\end{align*}

% andrez 6 % lost board von jan holen:
$h = g = \psi$
$$\Rightarrow \quad \int_V d^3 r\ ((\vabla \psi)^2 + \psi(\vec{r}) \underbrace{\Delta \psi(\vec{r})}_{=0}) = \int_{\partial V} df\ \underbrace{\psi(\vec{r})}_{=0} \frac{\partial \psi(\vec{r})}{\partial n} $$
\begin{equation*}
\Rightarrow \int_V d^3 r \left(\vec{\nabla} \psi(\vec{r}) \right)^2 = 0 \Rightarrow \vec{\nabla} \psi(\vec{r}) = 0 \qquad \vec{r} \in V
\end{equation*}
\begin{equation*}
\psi(\vec{r}) = \const \qquad \psi(\vec{r}) = 0 \ \tx{in} \ V \ \ \Rightarrow \ \ \Phi_1(\vec{r}) = \Phi_2(\vec{r})
\end{equation*}


\subsection{Formale Lösungen des elektrostatischen Randwertproblems mit \\ Greenschen Funktionen}
GF: generelle Methode um inhomogene DGL zu lösen
\begin{equation*}
\Delta \pofr = - \frac{1}{\epsilon_0} \rho(\vec{r})
\end{equation*}
Greensche Funktionen der Poisson-Gleichung: $ \gre(\vec{r},\vec{r}') $ mit
\frbox{Greensche Funktionen der Poisson-Gleichung}{\begin{equation*}
\Delta_{\vec{r}} \gre(\vec{r},\vec{r}') = - \frac{1}{\epsilon_0} \delta(\vec{r} - \vec{r}')
\end{equation*}}
\lcom{Diese Gleichung geht vor einer Punktladung mit $ q=1 $ aus, ist hier aber zunächst einmal eine Definition.}

% andrez 7:
$\mathcal G$ bekannt 
$$\rightarrow \pofr = \int d^3 r'\ \mathcal G(\vec{r}, \vec{r}') \rho(\vec{r}')$$
$$\Delta_{\vec{r}} \pofr = \int d^3 r'\ \underbrace{(\Delta_{\vec{r}} \mathcal G(\vec{r}, \vec{r}'))}_{= \frac{1}{\epsilon_0 \delta(\vec{r}-\vec{r}')}} \rho (\vec{r}) = - \frac{1}{\epsilon_0} \rho (\vec{r})\ \checkmark$$

\begin{equation*}
\Delta_{\vec{r}} \frac{1}{|\vec{r} - \vec{r}'|} = 4 \pi \delta(\vec{r} - \vec{r}')
\end{equation*}
\begin{equation*}
\gre(\vec{r},\vec{r}') = \frac{1}{4 \pi \epsilon_0} \frac{1}{|\vec{r} - \vec{r}'|}
\end{equation*}
\begin{equation*}
\rightarrow \Delta_{\vec{r}} \gre(\vec{r}, \vec{r}') = - \frac{1}{\epsilon_0} \delta(\vec{r} - \vec{r}')
\end{equation*}
\begin{equation*}
\pofr = \frac{1}{4 \pi \epsilon_0} \int d^3 r \frac{\rho(\vec{r}')}{|\vec{r} - \vec{r}'|}
\end{equation*}
\begin{equation*}
\pofr \buildrel |\vec{r}| \to \infty \over \longrightarrow 0
\end{equation*}


% andrez 8:
$$\rightarrow\ \Delta_{\vec{r}} \mathcal G(\vec{r}, \vec{r}') = - \frac{1}{\epsilon_0} \delta(\vec{r} - \vec{r}')$$
$$\mathcal G(\vec{r}, \vec{r}') \underset{|\vec{r}| \to \infty} \longrightarrow 0 $$

\subsubsection{Dirichlet-Randwertproblem}

$$\Delta \pofr = -\frac{1}{\epsilon_0} \rho(\vec{r}) \quad \vec{r} \in V$$
$$\pofr = \Phi_0(\vec{r}) \quad \vec{r} \in \partial V$$
GF: $$\Delta_{\vec{r}} \mathcal G(\vec{r}, \vec{r}') = \ \frac{1}{\epsilon_0} \delta(\vec{r} - \vec{r}') \quad \vec{r},\vec{r}' \in V$$
$$\mathcal G(\vec{r}, \vec{r}') = 0 \textrm{ für } \substack{\vec{r} \in \partial V \\ \vec{r}' \in V}$$
\lcom{Hiermit haben wir das Grenzwertproblem auf eine Integration zurückgeführt. Dies werden wir nun Beweisen:}\\[5pt]
Die 2. Greensche Identität lautet: 
\begin{align*}
&\ \ \ \, \int_V d^3 r' \left( g(\vec{r}') \Delta_{\vec{r}'} h(\vec{r}')  - h(\vec{r}') \Delta_{\vec{r}'} g(\vec{r}') \right)\\
&= \int_{\partial V} d\vec{f}' \cdot \left(g(\vec{r}') \vec{\nabla}_{\vec{r}'} h(\vec{r}') - h(\vec{r}') \vec{\nabla} _{\vec{r}'} g(\vec{r}') \right)
\end{align*}
\begin{equation*}
g(\vec{r}') \defeq \Phi(\vec{r}') \qquad h(\vec{r}') \defeq \gre(\vec{r}',\vec{r})
\end{equation*}
\begin{align*}
\Rightarrow & \ \ \ \, \int_V d^3 r' \Bigg[ \Phi(\vec{r}') \ub{\Delta_{\vec{r}'} \gre(\vec{r}',\vec{r})}_{= -\frac{1}{\epsilon_0} \delta(\vec{r}' - \vec{r})} - \gre(\vec{r}',\vec{r}) \ub{\Delta_{\vec{r}'} \Phi(\vec{r}')}_{= - \frac{1}{\epsilon_0} \rho(\vec{r}')} \Bigg]\\
&= \int_{\partial V} d\vec{f}' \Bigg[ \ub{\Phi(\vec{r}')}_{= \Phi_0(\vec{r}')} \vec{\nabla}_{\vec{r}'} \gre(\vec{r}',\vec{r}) - \ub{\gre(\vec{r}',\vec{r})}_{= 0} \vec{\nabla}_{\vec{r}'} \Phi(\vec{r}') \Bigg]\\ % check if last phi(r') is right % andrez 9:
\Rightarrow &= - \frac{1}{\epsilon_0} \pofr + \frac{1}{\epsilon_0} \int_V d^3 r'\ \mathcal G(\vec{r}, \vec{r}') \rho(\vec{r})\\
&= \int_{\partial V} d \vec{f}'\ \Phi_0 (\vec{r}') \vabla_{\vec{r}'} \mathcal G(\vec{r}, \vec{r}')\\
&= \int_{\partial V} d \vec{f}'\ \Phi_0 (\vec{r}') \prt{\mathcal G}{n'}(\vec{r}, \vec{r}')
\end{align*}
\begin{equation*}
\Rightarrow \pofr = \int_V d^3r'\ \mathcal G(\vec{r}, \vec{r}') \rho(\vec{r}') - \epsilon_0 \int_{\partial V} df'\ \Phi_0 (\vec{r}') \prt{\mathcal G}{n'}(\vec{r}, \vec{r}')
\end{equation*}

Es gilt (HA):
\begin{equation*}
\gre(\vec{r},\vec{r}') = \gre(\vec{r}',\vec{r}) \qquad \tx{Reziprozität}
\end{equation*}
\begin{equation*}
\rightarrow \vec{\nabla}_{\vec{r}'} \gre(\vec{r},\vec{r}') = \vec{\nabla}_{\vec{r}'} \gre(\vec{r}',\vec{r})
\end{equation*}
\begin{equation*}
\Delta_{\vec{r}} \gre(\vec{r},\vec{r}') = \Delta_{\vec{r}'} \gre(\vec{r},\vec{r}')
\end{equation*}

\rbox{\begin{equation*}
\Phi(\vec{r}) = \int_V \dd^3 r' \gre(\vec{r},\vec{r}') \rho(\vec{r}') - \epsilon_0 \int_{\partial V} \dd f' \Phi_0(\vec{r}') \frac{\partial \gre}{\partial n'} (\vec{r},\vec{r}')
\end{equation*}}


% 8.11.18

% Wiederholung nachholen

\emph{Bemerkungen:}
\begin{enumerate}[i)]
	\item Spezialfälle:
	\begin{enumerate}[1)]
		\begin{minipage}{.5\linewidth}
			\item $ V $ Ladungsfrei ($ \rho(\vec{r}) = 0 $ in $ V $)
			\begin{equation*}
			\rightarrow \Phi(\vec{r}) = - \epsilon \int_{\partial V} \dd f' \Phi_0(\vec{r}) \frac{\partial \mathcal{G}}{\partial n'} (\vec{r},\vec{r}')
			\end{equation*}
		\end{minipage}
		\begin{minipage}{.5\linewidth}
			%t1:
			\centering
			\begin{tikzpicture}[scale=1.5]
				\draw[draw=none,pattern=north east lines,rounded corners] ($(-.7,.5) + (-.1,.1)$) rectangle ($(.7,-.5) + (.1,-.1)$);
				\draw[fill=gray!10] (-.7,.5) rectangle (.7,-.5);
				\draw (-.7,-.5) -- (.7,-.5) -- (.7,.5) -- (-.7,.5) -- (-.7,-.5);
				\node[left] at (-.7,0) {$ \Phi_1 $};
				\node[right] at (.7,0) {$ \Phi_1 $};
				\node[above] at (0,.5) {$ \Phi_2 $};
				\node[below] at (0,-.5) {$ \Phi_2 $};
				\node at (0,0) {$ V $};
			\end{tikzpicture}
		\end{minipage}
		%T2
		\begin{align*}
		\Rightarrow \Phi(\vec{r}) &= - \epsilon \Phi_0 \ub{\int_{\partial V} \dd f' \frac{\partial \gre}{\partial n'} (\vec{r},\vec{r}') }_{\int \dd f' \vec{n} \cdot \vec{\nabla}_{\vec{r}'} \gre}\\
		&= - \epsilon \Phi_0 \int \dd \vec{f}' \cdot \vec{\nabla}_{\vec{r}'} \gre \\
		&\buildrel \mathclap{\tx{S.v.G.}} \over = \int_V \dd^3 \vec{r}' \ub{\vec{\nabla}_{\vec{r}'} \cdot (\vec{\nabla}_{\vec{r}'} \mathcal{G})}_{\Delta_{\vec{r}'} \cdot \mathcal{G} = - \frac{1}{\epsilon_0} \delta(\vec{r} - \vec{r}')}\\
		&= - \frac{1}{\epsilon_0}\\
		\Rightarrow \Phi(\vec{r}) &= \Phi_0
		\end{align*}
		\item $ V = \mathbb{R}^3 $, lokalisierte Ladungsverteilung $ \rho $
		\begin{equation*}
		\Phi(\vec{r}) = \frac{1}{4 \pi \epsilon_0} \int \dd^3 \vec{r}' \frac{\rho(\vec{r})}{|\vec{r} - \vec{r}'|}
		\end{equation*}
		\begin{equation*}
		\gre(\vec{r},\vec{r}') = \kq \frac{1}{|\vec{r} - \vec{r}'|} \qquad \int_{\partial V} \dots \rightarrow 0
		\end{equation*}
		eine spezielle Lösung für $ \gre $
	\end{enumerate}
	\item $ \gre $ ist auch die Lösung einer inhomogenen partiellen DGL
	\begin{equation*}
	\gre(\vec{r},\vec{r}') = \ub{\gre_s(\vec{r},\vec{r}')}_{\substack{\tx{spezielle} \\ \tx{Lösung der} \\ \tx{inhomogenen} \\ \tx{DGL}}} + \ub{F(\vec{r},\vec{r}')}_{\substack{\tx{Lösung} \\ \tx{zugehörigen} \\ \tx{homogenen} \\ \tx{DGL}}}
	\end{equation*}
	\begin{align*}
	\Delta_{\vec{r}'} \gre_s(\vec{r},\vec{r}') &= - \frac{1}{\epsilon_0} \delta(\vec{r} - \vec{r}')\\
	\Delta_{\vec{r}'} F(\vec{r},\vec{r}') \, &= 0
	\end{align*}
	\begin{equation*}
	\gre_j(\vec{r},\vec{r}') = \kq \frac{1}{|\vec{r} - \vec{r}'|} \qquad \tx{Laplace anwenden !}
	\end{equation*}
	\begin{equation*}
	\gre(\vec{r},\vec{r}') = \ub{\kq \frac{1}{|\vec{r} - \vec{r}'|}}_{\tx{immer zur Lösung}} \quad + \ub{F(\vec{r},\vec{r}')}_{\substack{\tx{so wählen, dass} \\ \tx{Randbedingungen erfüllt}}}
	\end{equation*}
	$ F(\vec{r},\vec{r}') $ so wählen, dass die Randbedingungen erfüllt sind: $ \gre(\vec{r},\vec{r}') = 0 \quad \vec{r} \in \partial V $.
\end{enumerate}

\subsection{Greensche Funktion des Dirichlet Randwertproblems einer Ebene}

\begin{minipage}{.55\linewidth}
	\begin{align*}
	\Delta_{\vec{r}'} \gre(\vec{r},\vec{r}') &= - \frac{1}{\epsilon_0} \delta(\vec{r} - \vec{r}') & \vec{r},\vec{r}' \in V \\
	\gre(\vec{r},\vec{r}') &= 0 & \vec{r} \in \partial V \, \tx{ \scriptsize{(z=0)}} , \quad \vec{r} \in V
	\end{align*}
	\begin{equation*}
	V=\{\vec{r} \in \mathbb{R}^3 | z < 0\}
	\end{equation*}
\end{minipage}
\begin{minipage}{.45\linewidth}
	%t3:
	\centering
	\begin{tikzpicture}
		\draw[draw=none,pattern=north east lines] (-.7,0) -- (2.7,0) -- ++(-1.2,-1.2) -- ++(-3.4,0);
		\draw[draw=none,fill=white,opacity=.6] (-.7,0) -- (2.7,0) -- ++(-1.2,-1.2) -- ++(-3.4,0);
		\draw[->] (0,0) -- (0,1.5) node[anchor=south east] {$ z $};
		\draw[->] (0,0) -- (-1.5,-1.5) node[anchor=north east] {$ x $};
		\draw[->] (-1,0) -- (3,0) node[anchor=north west] {$ y $};
		\node at (2,-.4) {$ \partial V $};
		\draw[thick,->] (0,0) -- (.7,.5) node[left,yshift=3pt] {$ \vec{r}' $};
		\draw[thick,->] (0,0) -- (.7,-1);
		\node[fill=white,opacity=.8,rounded corners] at (0,-.7) {$ \tilde{\vec{r}}' $};
		\node[fill=black,circle,inner sep=1pt,minimum size=1pt] at (.7,-.3) {};
		\node[fill=black,circle,inner sep=1pt,minimum size=1.5pt] (q1) at (.7,.5) {};
		\node[fill=black,circle,inner sep=1pt,minimum size=1.5pt] (q2) at (.7,-1) {};
		\node[anchor=south west] at (q1) {$ q = 1 $};
		\node[anchor=north west,yshift=-3pt] at (q2) {$ \substack{ \tx{\normalsize{$ - q $}} \\ \mathclap{\tx{\scriptsize{Spiegelladung}}}} $};
		\draw[thick,dashed] (.7,.5) -- (.7,-1);
		\draw[thick,->] (0,0) -- (-.7,.9) node[left] {$ \vec{r} $};
	\end{tikzpicture}
\end{minipage}
Analog: Punktladung ,,$ q = 1 $`` in $ \vec{r}' $ vor leitender Ebene mit Potential 0
\begin{equation*}
\Phi(\vec{r}) = \frac{q}{4 \pi \epsilon_0} \left(\frac{1}{|\vec{r} - \vec{r}'|} - \frac{1}{|\vec{r} - \tilde{\vec{r}}'|}\right) \qquad \tilde{\vec{r}}' = (x',y',-z')
\end{equation*}
\rbox{\begin{equation*}
\gre(\vec{r},\vec{r}') = \frac{1}{q} \Phi(\vec{r}) = \kq \left(\frac{1}{|\vec{r} - \vec{r}'|} - \frac{1}{|\vec{r} - \tilde{\vec{r}}'|}\right)
\end{equation*}}

\emph{Beweis:}
\begin{equation*}
\Delta_{\vec{r}} = \kq \bigg(\equalto{\Delta_{\vec{r}} \frac{1}{|\vec{r} - \vec{r}'|}}{- 4 \pi \delta(\vec{r} - \vec{r}')} - \equalto{\Delta_{\vec{r}} \frac{1}{|\vec{r} - \tilde{\vec{r}}'|}}{- 4 \pi \delta(\vec{r} - \tilde{\vec{r}}') = 0 }\bigg) = - \frac{1}{\epsilon_0} \delta(\vec{r} - \vec{r}')
\end{equation*}
1. Teil: $ \vec{r} \in \partial V \ : \quad z = 0 \ ,\qquad$ 2. Teil = 0: $ \tilde{\vec{r}}' \notin V $.
\begin{align*}
\frac{1}{|\vec{r} - \vec{r}'|} &= \frac{1}{\sqrt{(x-x')^2 + (y - y')^2 + (z')^2}} = \frac{1}{\sqrt{(x-x')^2 + (y-y')^2 + (-z)^2}}\\
&= \frac{1}{|\vec{r} - \tilde{\vec{r}}'|}
\end{align*}
$ \gre(\vec{r},\vec{r}') = 0 \quad \vec{r} \in \partial V $\\[5pt]
\emph{Bemerkung:}
\begin{enumerate}[i)]
	\item $ \hspace{1.5pt} \gre(\vec{r},\vec{r}') = \phantom{-} \kq \frac{1}{|\vec{r} - \vec{r}'|} + F(\vec{r},\vec{r}') $\\
	$ F(\vec{r},\vec{r}') = - \kq \frac{1}{|\vec{r} - \tilde{\vec{r}}'|} $\\
	$ \Delta_{\vec{r}} F(\vec{r},\vec{r}') = 0 $
	\item Symmetrie der Greenschen Funktion (Reziprozitätsrelation):
	\begin{equation*}
	\gre(\vec{r},\vec{r}') = \gre(\vec{r}',\vec{r})
	\end{equation*}
	$ \rightarrow $ formale Lösung des Randwertproblems für eine beliebige Ladungsverteilung und Randwerte $ \Phi_0(\vec{r}) $ in der Ebene:\\
	\begin{minipage}{.55\linewidth}
			\begin{equation*}
			\Phi(\vec{r}) = \int_V \dd^3 r' \gre(\vec{r},\vec{r}') \rho(\vec{r}') - \epsilon_0 \int_{\partial V} \dd f' \Phi(\vec{r}') \prt{\gre}{n'}
			\end{equation*}
			\begin{equation*}
			\rho \equiv 0 \ \ \ \Rightarrow \ \ \ \Phi(\vec{r}) = \epsilon_0 \hspace{-15pt} \int\limits_{\sqrt{x^2 + y^2} \le R} \hspace{-15pt} \dd y' \dd x' \Phi_0(x',y',0) \prt{\gre}{n'}
			\end{equation*}
	\end{minipage}
	\begin{minipage}{.45\linewidth}
		%t4:
		\centering
		\begin{tikzpicture}
			\draw[draw=none,pattern=north east lines] ($ (-1.3,0) + (.6,.6) $) -- ++(3.1,0) -- ++(-1.4,-1.4) -- ++(-3.1,0);
			\draw[draw=none,fill=white,opacity=.7]  ($ (-1.3,0) + (.6,.6) $) -- ++(3.1,0) -- ++(-1.4,-1.4) -- ++(-3.1,0);
			\draw[fill=white] (0,0)ellipse (1cm and .5cm);
			\draw[->] (0,0) -- (0,1) node[anchor=south east] {$ z $};
			\draw[->] (.7,.7) -- (-1,-1) node[anchor=north east] {$ x $};
			\draw[->] (-1.5,0) -- (2,0) node[anchor=north west] {$ y $};
			\draw[pattern=north west lines] (0,0) ellipse (1cm and .5cm);
			\draw[thick] (0,0) -- (.45,-.45) node[fill=black,circle,inner sep=1pt,minimum size=1pt] {};
			\node[below,fill=white,opacity=.8,inner sep=2pt,rounded corners] at (.3,-.5) {$ R $};
			\node[fill=white,opacity=.8,inner sep=2pt,rounded corners] at (2,.5) {$ \Phi \equiv 0 $};
			\draw (-.5,.3) to[out=100,in=-10] (-1,1) node[left] {$ \Phi_0(x,y,0) $};
		\end{tikzpicture}
	\end{minipage}
\end{enumerate}

\subsection{Separation der Variablen und Entwicklung nach orthogonalen Funktionen}

\textbf{Eine allgemeine Methode zur Lösung partieller DGL.}\\[5pt]
Zur Vereinfachung: \textbf{Laplace.Gl $ \boldsymbol{ \Delta \Phi = 0 \ +} $ Randbedingung}\\[5pt]
Verbindung zur Poisson-Gl: $ \Delta \Phi(\vec{r}) = - \frac{1}{\epsilon_0} \rho(\vec{r}) $
\begin{equation*}
\Phi(\vec{r}) = \Phi_s(\vec{r}) + \Phi_{\tx{hom}} \qquad \Phi(\vec{r}) = \kq \int \dd^3 r' \frac{\rho(\vec{r}')}{|\vec{r} - \vec{r}'|} + \Phi_{\tx{hom}}
\end{equation*}

\subsubsection{Motivation: 1-Dim Randwertproblem}

\begin{minipage}{.5\linewidth}
	%t5:
	\centering
	\begin{tikzpicture}
		\coordinate (a) at (-1,0);
		\coordinate (b) at (1,0);
		\draw[->] (-2,0) -- (2,0) node[anchor=north west] {$ x $};
		\draw ($ (a) + (0,-1) $) -- ++(0,2) node[anchor=south east] {$ \Phi_1 $};
		\draw ($ (b) + (0,-1) $) -- ++(0,2) node[anchor=south west] {$ \Phi_2 $};
		\node[anchor=north west] at (a) {$ 0 $};
		\node[anchor=north east] at (b) {$ l_x $};
		\draw[draw=none,pattern=north east lines] ($ (a) + (0,-1) $) rectangle ++ (-.3,2);
		\draw[draw=none,pattern=north east lines] ($ (b) + (0,-1) $) rectangle ++ (.3,2);
		\draw[thick] ($ (a) + (45:.1) $) -- ++(-135:.2);
		\draw[thick] ($ (a) + (-45:.1) $) -- ++(135:.2);
		\draw[thick] ($ (b) + (45:.1) $) -- ++(-135:.2);
		\draw[thick] ($ (b) + (-45:.1) $) -- ++(135:.2);
	\end{tikzpicture}
	\vspace{5pt}
\end{minipage}
\begin{minipage}{.5\linewidth}
	$$ \Phi(x) = ? \qquad \rho = 0 $$
	\begin{equation*}
	\Delta \Phi(x) = \frac{\dd^2 \Phi}{\dd x^2} = 0
	\end{equation*}
	\begin{equation*}
	\Rightarrow \Phi(x) = c_1 + c_2 x
	\end{equation*}
\end{minipage}
\textbf{Randbedingungen:}\\
$$ \Phi(0) = c_1 = \Phi_1  \qquad \Phi(l_x) = \Phi_1 + c_2 l_x = \Phi_2 $$
\begin{equation*}
\rightarrow c_2 = \frac{\Phi_2 - \Phi_1}{l_x} \quad \rightarrow \quad \Phi(x) = \Phi_1 + \frac{\Phi_2 - \Phi_1}{l_x} x
\end{equation*}
\begin{equation*}
\Rightarrow \vec{E} = - \vec{\nabla} \Phi = - \frac{\Phi_2 - \Phi_1}{l_x} \vec{e}_x
\end{equation*}

\subsubsection{2-Dim Randwertproblem}

\begin{minipage}{.5\linewidth}
	%t6:
	\centering
	\begin{tikzpicture}
		\coordinate (x) at (3.5,0);
		\coordinate (y) at (0,2.5);
		\draw[draw=none,pattern=north east lines,rounded corners] ($ (0,0) + (-.3,-.3) $) rectangle ($ (x) + (y) + (.3,.3)$);
		\draw[draw=none,fill=white,opacity=.5,rounded corners] ($ (0,0) + (-.3,-.3) $) rectangle ($ (x) + (y) + (.3,.3)$);
		\draw[draw=none,fill=white] (0,0) rectangle ($ (x) + (y) $);
		\node[xshift=-10pt,left] at ($ (0,0)!.5!(y) $) {$ \Phi = 0 $};
		\node[xshift=10pt,right] at ($ (x) + (y) - (0,0)!.5!(y) $) {$ \Phi = 0 $};
		\node[yshift=-10pt,below] at ($ (0,0)!.5!(x) $) {$ \Phi = 0 $};
		\node[yshift=10pt,above] at ($ (y) + (0,0)!.5!(x) $) {$ \Phi_R(x) $};
		\draw[->] (-.5,0) -- ($ (x) + (.6,0) $) node[anchor=north west] {$ x $};
		\draw[->] (0,-.5) -- ($ (y) + (0,.6) $) node[anchor=south east] {$ y $};
		\draw ($ (x) + (0,.1)$) -- ++(0,-.3) node[below,yshift=-5pt] {$ l_x $};
		\draw ($ (y) + (.1,0) $) -- ++(-.3,0) node[left,xshift=-5pt] {$ l_y $};
		\draw (y) -- ($ (x) + (y) $);
		\draw (x) -- ($ (x) + (y) $);
		\node at ($ (0,0)!.5!(x) + (0,0)!.5!(y) $) {$ V $};
	\end{tikzpicture}
\end{minipage}
\begin{minipage}{.5\linewidth}
	Wir suchen: $ \Phi = \Phi(x,y) \quad $ mit $ \rho = 0 $
	\begin{equation*}
	0 = \Delta \Phi = \frac{\partial ^2 \Phi}{\partial x^2} + \frac{\partial^2 \Phi}{\partial y^2}
	\end{equation*}
\end{minipage}
\textbf{Randbedingungen:}\\
\begin{enumerate}[i)]
	\item $ \Phi(\vec{r}) = 0 \phantom{\Phi_R(x)} \qquad y = 0 $
	\item $ \Phi(\vec{r}) = 0 \phantom{\Phi_R(x)} \qquad x = 0 $
	\item $ \Phi(\vec{r}) = 0 \phantom{\Phi_R(x)} \qquad x = l_x $
	\item $ \Phi(\vec{r}) = \Phi_R(x) \phantom{0} \qquad y = l_y $
\end{enumerate}
\textbf{Separationsansatz:}
$ \Phi(x,y) = f(x) g(y) $
\begin{align*}
0 = \Delta \Phi &= \left(\prt{^2}{x^2} + \prt{^2}{y^2}\right) f(x) g(y)\\
&= \prt{^2 f}{x^2} g(y) + f(x) \prt{^2 g}{y^2}\\
&= \Delta \Phi = \prd{^2 f}{x^2} g(y) + f(x) \prd{^2 g}{y^2}
\end{align*}
\begin{equation*}
\rmbox{0 = \Delta \Phi = \prd{^2 f}{x^2} g(y) + f(x) \prd{^2 g}{y^2} } \qquad \left\rvert \cdot \, \frac{1}{f\,g} \right.
\end{equation*}
umformen:
\begin{equation*}
\Rightarrow \ub{\frac{1}{f(x)} \prd{^2 f}{x^2}}_{\tx{Fkt. von} x} = - \ub{\frac{1}{g(y)} \prd{^2 g}{y^2}}_{\tx{Fkt. von} y} = \const = - \alpha ^2
\end{equation*}
\begin{equation*}
\prd{^2 f}{x^2} = - \alpha^2 f(x) \quad \tx{mit } e^{i\alpha x} \qquad \prd{^2 g}{y^2} = \alpha^2 g(y) \quad \tx{mit } e^{\alpha y}
\end{equation*}
\begin{equation*}
e^{i \alpha x} \Rightarrow f(x) = a \sin(\alpha x) + b cos(\alpha x) \qquad e^{\alpha y} \Rightarrow g(x) = c \sinh(\alpha y) + d \cosh(\alpha y)
\end{equation*}
$ \Phi(x,y) = f(x) \cdot g(y) $\\[5pt]
\textbf{Randbedingungen:}
\begin{enumerate}[i)]
	\item $ 0 = \Phi(x,0) = f(x) \cdot d \quad \Rightarrow d = 0 $
	\item $ 0 = \Phi(0,y) = b \cdot g(y) \quad \Rightarrow b = 0 $
	\begin{equation*}
	\Rightarrow \Phi(x,y) = a \sin(\alpha x) c \sinh(\alpha y) = \equalto{A}{a\cdot c} \sin(\alpha x) \sinh(\alpha y)
	\end{equation*}
	\item $ 0 = \Phi(l_x,y) = A \sin(\alpha l_x) \sinh(\alpha y) $
	$ \rightarrow \sin(\alpha l_x) = 0 \quad \Rightarrow \quad \alpha = \frac{n \pi}{l_x} \qquad n \in \mathbb{Z} (\tx{oder } n \in \mathbb{N}) $
	\begin{equation*}
	\rightarrow \Phi_n(x,y) = A_n \sin\left(\frac{n \pi x}{l_x}\right) \sinh\left(\frac{n \pi y}{l_y}\right)
	\end{equation*}
	\item $ \Phi(x,l_y) = \Phi_R(x) $
	\begin{equation*}
	\Rightarrow \Phi_R(x) \custo{\leftarrow}{=}{} A_n \sin\left(\frac{n \pi x}{l_x}\right) \sinh\left(\frac{n \pi l_y}{l_x}\right) \qquad \forall x \in [0,l_y]
	\end{equation*}
	im allgemeinen ist dies nicht möglich, aber da es sich um eine lineare DGL ($ \Delta \Phi = 0 $) handelt:\\
	$ \rightarrow $ Linearkombinationen von Lösungen sind auch Lösungen
\end{enumerate}
\textbf{Ansatz für allgemeine Lösung:}
\begin{equation*}
\rmbox{\Phi(x,y) = \sum_{n=1}^{\infty} A_n \sin \left(\frac{n \pi x}{l_x}\right) \sinh \left(\frac{n \pi y}{l_x}\right)}
\end{equation*}
Der Ansatz erfüllt $ \Delta \Phi = 0 $ und erfüllt die Randbedingungen i), ii), iii). Um iv) zu erfüllen fordern wir:
\begin{equation*}
\rmbox{\Phi_R(x) \buildrel ! \over = \ub{\sum_{n=1}^{\infty} A_n \sin\left(\frac{n \pi x}{l_x}\right)}_{\tx{Entwicklung}} \ub{\sinh\left(\frac{n \pi l_y}{l_x}\right)}_{\const}}
\end{equation*}
Der erste Teil des Ausdrucks entspricht der Entwicklung von $ \Phi_R(x) $ nach Funktionen $ \sin \left(\frac{n \pi x}{l_x}\right) $ also einer Fourier-Reihe.\\[5pt]
\textbf{Bestimmung von $ A_n $:}
Multipliziere mit $ \sin\left(\frac{n \pi x}{l_x}\right) \quad m \in \mathbb{N} $ und danach Integration:
\begin{align*}
\int_{0}^{l_x} \dd x \sin\left(\frac{m \pi x}{l_x}\right) \Phi_R (x) &= \sum_{n=1}^{\infty} A_n \sinh\left(\frac{m \pi l_y}{l_x}\right) \int_{0}^{l_x} \dd x \ub{\custoup{\leftarrow}{\sin\left(\frac{m \pi x}{l_x}\right)}{\qquad \qquad \quad \mathclap{\tx{zueinander orthogonale Vektoren}}} \custoup{\leftarrow}{\sin\left(\frac{n \pi x}{l_x}\right)}{}}_{= \frac{l_x}{2} \delta_{nm}}\\
&= A_m \frac{l_x}{2} \sinh\left(\frac{n \pi l_y}{l_x}\right)
\end{align*}
\begin{equation*}
A_m = \frac{2}{l_x \sinh\left(\frac{n \pi l_y}{l_x}\right)} \int_{0}^{l_x} \dd x \sin \left(\frac{n \pi x}{l_x}\right) \Phi_R(x)
\end{equation*}
in $ \Phi(x,y) $ einsetzen\\[5pt]

% 12.11.18

\bbb{Wiederholung}{$$\Delta \Phi(\vec{r}) = 0 + \textrm{Randbedingungen}$$
$$\Phi = \Phi(x,y) = f(x)g(y)$$
$$\Phi_n(x,y) = A_n \sin\left(\frac{n \pi x}{l_x}\right) \sinh\left(\frac{n \pi y}{l_x}\right)$$
$$n \in \mathbb N$$
$$\Phi(x,y) = \sum_n A_n \sin\left(\frac{n \pi x}{l_x}\right) \sinh\left(\frac{n \pi y}{l_x}\right)$$
$$A_n = \frac{2}{l_x \sinh\left(\frac{n \pi l_y}{l_x}\right)} \intt{0}{l_x}\textrm{d}x\ \sin\left(\frac{n \pi x}{l_x}\right) \Phi_R (x)$$} % t1 = t6 letzten vorlesung

\subsection{Vollständige Orthonormale Funktionensysteme (VONS)}
Betrachte Funktionen $g(x),h(x)$ auf $I = [a,b] \subset \mathbb R$
$$h,g: \quad I \to \mathbb R\ (\mathbb C)$$
Skalarprodukt: $(g,h) = \int_a^b \textrm{d}x\ g^* (x) h(x)$\\
$(g,h) = 0$: $g$ und $h$ orthogonal, $(g,g) = 1:\ g$ normiert\\
Norm: $||g|| = \sqrt{(g,g)}$

\noindent
Ein abzählbarer Satz von Funktionen $\{f_n\} = \{f_1, f_2, \dots \}$\\
Heißt orthonormiert falls: $(f_m,f_n) = \delta_{nm}\ \rightarrow$ \underline{Orthonormalsystem}\\
\underline{Vollständigkeit:} Ein Satz von Funktionen heißt vollständig (VONS) falls \underline{jede} quadratintegrable\footnote{$\textrm{Falls }\int \textrm{d}x\ |g(x)|^2 \textrm{ existiert}$} Funktion $g: I \to \mathbb R (\mathbb C)$ in der Form $g(x) = \summ{n = 1}{\infty} a_n f_n(x)$ dargestellt werden kann.\\
Genauer: $\lim\limits_{n \to \infty} \intt{a}{b} \textrm{d}x\ |\ g(x) - \summ{n = 1}{\infty} a_n f_n(x)| = 0$\\[5pt]
Bestimmung der Koeffizient $ a_n $:
$$g(x) = \sum_n a_n f_n(x)\ \qquad \left| \int \textrm{d}x\ f_m^*(x) \right.$$
$$\intt{a}{b}\textrm{d}x\ f_m^* (x) g(x) = \summ{n = 1}{\infty} \underbrace{\intt{a}{b} \textrm{d}x\ f_m^* (x)f_n(x)}_{= \delta_{nm}} = a_m$$
\begin{align*}
 	g(x) &= \sum_n a_n f_n (x) = \sum_n (f_n,g)f_n(x)\\
 	&= \sum_n \intt{a}{b} \textrm{d}x'\ f_n^* (x') g(x') f_n(x)\\
 	&= \intt{a}{b} \textrm{d}x'\ g(x') \underbrace{\summ{n = 1}{\infty} f_n(x)f_n^*(x')}_{= \delta(x - x')}
\end{align*}
da $ \int_{a}^{b} \dd x' g(x') = g(x) $
\frbox{Vollständigkeitsrelation}{$$\summ{n = 1}{\infty}f_n(x)f_n^*(x') = \delta(x - x')$$}
\noindent
\emph{Beispiele:}
\begin{enumerate}[1)]
	\item 
	\begin{equation*}
	f_n(x) = \sqrt{\frac{l}{2}} \sin\left(\frac{n \pi x}{l}\right) \qquad I = [0,l] 
	\end{equation*}
	$ (f_n,f_m) = \delta_{nm} $\\
	$ g: \quad I \rightarrow \mathbb{R} \quad g(0) ) 0 = g(l) $
	\begin{equation*}
	g(x) = \sum_{n} a_n \sqrt{\frac{l}{2}} \sin \left(\frac{n \pi x}{l}\right)
	\end{equation*}
	\item Fourierreihe: $ \{f_n\} $:\\
	$ n=0: \quad \frac{1}{\sqrt{l}} $
	\begin{equation*}
	n \in \mathbb{N}: \qquad \sqrt{\frac{2}{l}} \sin \left(\frac{n \pi x}{l}\right) \ \ ; \qquad \sqrt{\frac{2}{l}} \cos\left(\frac{n \pi x}{l}\right) \qquad I = [0,l]
	\end{equation*}
	\begin{equation*}
	g(x) = a_0 \frac{1}{\sqrt{l}} + \sum_{n=1}^{\infty} \left[a_n \sqrt{\frac{2}{l}} \sin \left(\frac{2 \pi x}{l}\right) + b_n \sqrt{\frac{2}{l}} \cos\left(\frac{n \pi x}{l}\right)\right]
	\end{equation*}
	\vspace{5pt}
\end{enumerate}

\subsection{Laplace-Gleichung in Kugelkoordinaten}

\begin{minipage}{.5\linewidth}
	%t1:
	\centering
	\begin{tikzpicture}
		\draw[->] (0,0) -- (0,2) node[anchor=south east] {$ z $};
		\draw[->] (0,0) -- (2,0) node[anchor=north west] {$ y $};
		\draw[->] (0,0) -- (-1,-1) node[anchor=north east] {$ x $};
		\draw[thick,->] (0,0) -- node[above=10pt] {$ \vec{r} $} (1.5,1);
		\draw[dashed] (0,0) -- (1.5,-1) -- (1.5,1);
		\centerarc[](0,0)(90:33:1cm);
		\centerarc[](0,0)(-135:-33:.85cm);
		\node[xshift=10pt,yshift=17pt] at (0,0) {$ \theta $};
		\node[below,yshift=-8pt] at (0,0) {$ \varphi $};
	\end{tikzpicture}
\end{minipage}
\begin{minipage}{.5\linewidth}
	\begin{equation*}
	\vec{r} = r \begin{pmatrix}
	\sin \theta \cos \varphi \\ \sin \theta \sin \varphi \\ \cos \theta
	\end{pmatrix}
	\end{equation*}\\
	$$\Phi (\vec{r}) = \Phi(r,\theta, \varphi)$$
	$$\Delta \Phi = 0$$
	\vspace{5pt}
\end{minipage}\\
$$\rightarrow \frac{1}{r} \prt{^2}{r^2}(r \Phi) + \frac{1}{r^2 \sin \theta} \prt{}{\theta} \left( \sin \theta \prt{\Phi}{\theta}\right) + \frac{1}{r^2 \sin^2 \theta} \prt{^2 \Phi}{\varphi^2} = 0$$
$$\frac{1}{r} \prt{^2}{r^2} (r \Phi) = \frac{1}{r^2} \prt{}{r} \left(r^2 \prt{\Phi}{r}\right)$$
\subsubsection{Separationsansatz:}
$$\Phi(r,\theta,\varphi) = \frac{U(r)}{r} P(\cos\theta)Q(\varphi)$$
1. Term:
$$\frac{1}{r} \prt{^2}{r^2} \left(\cancel{r} \frac{U(r)}{\cancel{r}} P(\cos\theta)Q(\varphi)\right) = P(\cos\theta)Q(\varphi)\frac{1}{r}\frac{\textrm{d}^2 U}{\textrm{d} r^2}$$
$$\Rightarrow 0 = PQ \frac{1}{r} \frac{\textrm{d}^2 U}{\textrm{d} r^2} + UQ \frac{1}{r^3 \sin \theta}\frac{\textrm{d}}{\textrm{d}\theta} \left(\sin\theta \frac{\textrm{d} P}{\textrm{d} \theta} \right) + UP \frac{1}{r^3 \sin^2 \theta} \frac{\textrm{d}^2 Q}{\textrm{d} \varphi^2} \qquad \bigg\rvert \cdot \frac{r^3 \sin^2 \theta}{UPQ}$$
$$\Rightarrow \underbrace{- r^2 \sin^2 \theta \frac{1}{U} \frac{\textrm{d}^2 U}{\textrm{d} r^2} - \sin \theta \frac{1}{P}\frac{\textrm{d}^2}{\textrm{d} \theta}\left( \sin \theta \frac{\textrm{d} P}{\textrm{d} \theta} \right)}_{\textrm{unabhängig von } \varphi} = \underbrace{\frac{1}{Q} \frac{\textrm{d}^2 Q}{\textrm{d} \varphi^2}}_{\mathclap{\textrm{unabhängig von }r,\theta}} = \const \defeq -m^2$$
für $Q$:
\begin{enumerate}[i)]
	\item $$\frac{\textrm{d}^2 Q}{\textrm{d} \varphi^2} + m^2 Q = 0$$ Lösung: $$Q(\varphi) = e^{im\varphi} = \cos(m\varphi) + i\sin(m\varphi)$$
	 	$$Q(\varphi + 2 \pi) = Q(\varphi) \quad e^{im(\varphi + 2\pi)} = e^{im\varphi} \quad \Rightarrow \quad m = \mathbb Z$$
	 	$$\frac{r^2}{U} \frac{\textrm{d}^2 U}{\textrm{d} r^2} + \frac{1}{P\sin\theta} \frac{\textrm{d}}{\textrm{d} \theta}\left( \sin\theta \frac{\textrm{d}P}{\textrm{d}\theta} \right) = \frac{m^2}{\sin^2\theta}$$
	 	$$\ub{\frac{r^2}{U} \frac{\textrm{d}^2 U}{\textrm{d} r^2}}_{\tx{unabh. von } \theta } = - \ub{\frac{1}{P\sin\theta}\frac{\textrm{d} P}{\textrm{d} \theta}\left( \sin\theta \frac{\textrm{d}P}{\textrm{d}\theta} \right)}_{\tx{unabh. von } V} = \const. \defeq \lambda$$
	 \item $$\frac{\textrm{d}^2 U}{\textrm{d} r^2} - \frac{\lambda}{r^2} U(r) = 0$$
	 	$\rightarrow$ Lösung für $\lambda = l(l + 1) \qquad$ (Warum das so ist, ist  in iii) erklärt)
	 	$$U(r) = a_l r^{l + 1} + b_l r^{-l}$$
	 	$\rightarrow$ \underline{Spezielle Lösung für $m = 0$:}
	 	$$\Phi(r,\theta) = \frac{U(r)}{r}P_l (\cos\theta) = (a_l r^l + b_l r^{-l - 1}) P_l (\cos\theta)$$
	 	allg. Lösung: $\Delta\Phi = 0$ für $\prt{\Phi}{\varphi} = 0$
	 	$$\Phi(r,\theta) = \summ{l = 0}{\infty} (\custo{\to}{a_l}{\mathclap{\hspace{30pt}\textrm{durch Randbedingungen festgelegt}}} r^l + \custo{\to}{b_l}{} r^{-l - 1}) P_l(\cos\theta)$$
	 \item $$\frac{1}{\sin\theta} \frac{\textrm{d}}{\textrm{d} \theta}\left( \sin\theta \frac{\textrm{d}P}{\textrm{d}\theta}\right) + \left(\lambda - \frac{m^2}{\sin^2\theta}\right)P(\cos\theta) = 0$$
	 	$$x \defeq \cos\theta \quad P(x): \textrm{ DGL für } P(x) \quad \frac{\textrm{d}}{\textrm{d} \theta} P(x(\theta)) = \frac{\textrm{d}P}{\textrm{d}x} \frac{\textrm{d}x}{\textrm{d}\theta} = -\sin \theta \frac{\textrm{d}P}{\textrm{d}x}$$
	 	\begin{equation*}
	 	\dd x = - \frac{1}{\sin\theta} \prd{}{\theta}
	 	\end{equation*}
	 	$$\Rightarrow - \frac{\textrm{d}}{\textrm{d}x} \bigg( - \equalto{\sin^2\theta}{1 - x^2} \frac{\textrm{d}P}{\textrm{d}x} \bigg) + \left(\lambda - \frac{m^2}{1-x^2}\right) P(x) = 0$$
\end{enumerate}
\frbox{Zugeordnete Legendresche DGL}{$$\Rightarrow \frac{\textrm{d}}{\textrm{d}x} \left( (1-x^2) \frac{\textrm{d}P}{\textrm{d}x} \right) + \left( \lambda - \frac{m^2}{1 - x^2} \right) P(x) = 0$$}
\begin{minipage}{.6\linewidth}
	Spezialfall: \textbf{Zylindersymmetrische Probleme}: $\Phi$ unabhängig von $\varphi$\\
	$\rightarrow$ \textbf{Legendre-Polynome}
\end{minipage}
\begin{minipage}{.4\linewidth}
	%t2:
	\centering
	\begin{tikzpicture}[scale=1.25]
		\draw[->] (-1,0) -- (1,0) node[anchor=north west] {$ x $};
		\draw[->] (0,-1) -- (0,1) node[anchor=south east] {$ z $};
		\draw[thick,->] (-.8,-.5) -- node[left] {$ \rho $} (-.8,.8) node[anchor=south east] {$ \vec{E}_0 $};
		\draw[pattern=north east lines] (0,0) circle (.6cm);
	\end{tikzpicture}
\end{minipage}
$$\prt{\Phi}{\varphi}= 0, \quad Q(\varphi) = e^{im\varphi} \Rightarrow m = 0 \Rightarrow Q(\varphi) = 1$$
$$\frac{\textrm{d}}{\textrm{d}x}\left( (1 - x^2) \frac{\textrm{d}P}{\textrm{d}x} \right) + \lambda P (x) = 0$$
\frbox{Legendresche DGL}{$$(1-x^2) \frac{\textrm{d}^2P}{\textrm{d}x^2} - 2x \frac{\textrm{d}P}{\textrm{d}x} + \lambda P(x) = 0$$}
Potenzreihenansatz: $P(x) = \summ{k = 0}{\infty} a_k x^k$
\begin{itemize}
	\item[$\rightarrow$] Fließbach
	\item[$\rightarrow$] Legendre Polynome
	\item[$\rightarrow$] relevante Lösung nur für $\lambda = l(l + 1) \qquad l \in \mathbb N_0$
\end{itemize}







\end{document}
\PassOptionsToPackage{dvipsnames}{xcolor}
\documentclass[titlepage,11pt,a4paper,ngerman]{report}
\usepackage[utf8]{inputenc}
\usepackage[T1]{fontenc}
\usepackage[german]{babel}
\usepackage{graphicx}
\usepackage{wrapfig}
\usepackage{amsmath}
\usepackage{amsfonts}
\usepackage{amssymb}
\usepackage{tikz}
\usepackage{tikz-cd}
\usepackage{nicefrac}
\usepackage{mathtools}
\usepackage{enumerate}
\usepackage{cancel}
\usepackage[hidelinks]{hyperref}
\usepackage{cleveref}
\usepackage{tcolorbox}

\usepackage[margin=1in]{geometry}

\usepackage{placeins}
\usepackage{booktabs}

\usepackage{url}

\usetikzlibrary{calc}
\usetikzlibrary{decorations.pathmorphing,patterns}
\usetikzlibrary{arrows}


%Environments und Newcommands:

% Allgemein:

% zu zeigen symbol
\newcommand{\zz}{\fontfamily{cmss} \selectfont{Z\kern-.61em\raise-0.7ex\hbox{Z}:}}
% build over
\newcommand{\bov}[2]{\buildrel{#2} \over{#1}}
% better looking := (defined as)
\newcommand*{\defeq}{\mathrel{\vcenter{\baselineskip0.5ex \lineskiplimit0pt \hbox{\scriptsize.}\hbox{\scriptsize.}}}=}
\newcommand*{\eqdef}{=\mathrel{\vcenter{\baselineskip0.5ex \lineskiplimit0pt \hbox{\scriptsize.}\hbox{\scriptsize.}}}}

\newcommand{\hfw}{\color{RubineRed}\tx{ $\star$hier fehlt was$\star$ } \color{black}}
\def\checkmark{\tikz\fill[scale=0.4](0,.35) -- (.25,0) -- (1,.7) -- (.25,.15) -- cycle;} 

%Text:
\newcommand{\tx}[1]{\textrm{#1}}
\newcommand{\const}{\tx{const.}}

\newcommand{\ul}[1]{\underline{#1}}
\newcommand{\ol}[1]{\overline{#1}}
\newcommand{\ub}[1]{\underbrace{#1}}
\newcommand{\ob}[1]{\overbrace{#1}}

%Mathe:
\newcommand{\verteq}{\rotatebox{90}{$\,=$}}
\newcommand{\equalto}[2]{\underset{\scriptstyle\overset{\mkern4mu\verteq}{#2}}{#1}}
\newcommand{\equaltoup}[2]{\overset{\scriptstyle\underset{\mkern4mu\verteq}{#2}}{#1}}
\newcommand{\custo}[3]{\underset{\scriptstyle\overset{\mkern4mu\rotatebox{-90}{$\,#1$}}{#3}}{#2}}
\newcommand{\custoup}[3]{\overset{\scriptstyle\overset{\mkern4mu\rotatebox{-90}{$\,#1$}}{#3}}{#2}}
\newcommand{\casess}[4]{\left\{ \begin{array}{ll} {#1} & {#2} \\ {#3} & {#4} \end{array} \right.}


%Spezielles:

%Theo:
\newcommand{\lag}{\mathcal{L}}
\newcommand{\ham}{\mathcal{H}}
\newcommand{\gre}{\mathcal{G}}
\newcommand{\prt}[2]{\frac{\partial #1}{\partial #2}}
\newcommand{\eofr}{\vec{E}(\vec{r})}
\newcommand{\pofr}{\Phi(\vec{r})}
\renewcommand{\Phi}{\varPhi}


%LA:
\newenvironment{bew}[1]{\subsection{Bew: #1}}{\hfill$\square$}
\newcommand{\Bew}[2]{\begin{bew}{#1}#2\end{bew}}

\newcommand{\enph}{F: V \to V \textrm{ Endomorphismus}}
\newcommand{\im}{\tx{im}}
\newcommand{\spa}{\tx{span}}
\newcommand{\adj}{\tx{adj}}
\newcommand{\grad}{\tx{grad}}
\newcommand{\ord}{\tx{ord}}

\newcommand{\basis}[3]{\{#1_{#2}, \dots, #1_{#3}\}}
\newcommand{\ska}[2]{\langle #1 , #2 \rangle}
\newcommand{\dmat}[3]{\begin{pmatrix} #1_{#2}&&\\ &\ddots& \\ && #1_{#3} \end{pmatrix}}

%Ex:
\newcommand{\kq}{\frac{1}{4\pi\epsilon_0}}
\newcommand{\uind}{U_{\tx{ind}}}
\newcommand{\folie}[1]{\color{gray}[Folie: #1]\color{black}}

% ANDREZ
\newcommand{\summ}[2]{\sum_{#1}^{#2}}
\newcommand{\intt}[2]{\int_{#1}^{#2}}
\renewcommand{\vec}[1]{\boldsymbol{#1}}
\newcommand{\lcom}[1]{\color{MidnightBlue}#1\color{black}}
\renewcommand{\epsilon}{\varepsilon}
\newcommand{\vabla}{\boldsymbol{\nabla}}
\newcommand{\bei}[1]{\emph{Beispiel:}}
\newcommand{\bem}[1]{\emph{Bemerkung:}}

% Boxen:

\tcbuselibrary{theorems}

% mahlt eine box nur um den text mit tittel
\newtcbox{\fribox}[1]{nobeforeafter,colback=white,colframe=red!75!black,fonttitle=\bfseries,title=#1,sharp corners,tcbox raise base}

% mahlt eine große box um alles mit tittel
\newcommand{\frbox}[2]{\begin{tcolorbox}[colback=white,colframe=red!75!black,fonttitle=\bfseries,title=#1]#2\end{tcolorbox}}

% mahlt eine box nur um den text
\newtcbox{\ribox}{nobeforeafter,colback=white,colframe=red!75!black,sharp corners,tcbox raise base}

% mahlt eine große box um alles was drinnen ist
\newcommand{\rbox}[1]{\begin{tcolorbox}[colback=white,colframe=red!75!black]#1\end{tcolorbox}}

% mahlt eine box um mathe innerhalb mathmode
\newcommand{\rmbox}[1]{\tcboxmath[colback=white,colframe=red!75!black]{#1}}

% super box (looks like regular boxed but wraps around anything)
\newenvironment{supbox}{\begin{tcolorbox}[colback=white,colframe=black,sharp corners,boxrule=.5pt]}{\end{tcolorbox}}

\newcommand{\bbb}[2]{\begin{tcolorbox}[colback=white,colframe=black,fonttitle=\bfseries,title=#1,sharp corners,tcbox raise base]#2\end{tcolorbox}}

% array type box with title
\newenvironment{zebox}[1]{\begin{array}{|c|}
		\multicolumn{1}{l}{\tx{#1}} \\
		\hline
		\displaystyle
	}{\\ \hline
\end{array}}

% evtl: \renewcommand{\boxed}{\rmbox}

\tikzset{
	annotated cuboid/.pic={
		\tikzset{%
			every edge quotes/.append style={midway, auto},
			/cuboid/.cd,
			#1
		}
		\draw [every edge/.append style={pic actions, densely dashed, opacity=.5}, pic actions]
		(0,0,0) coordinate (o) -- ++(-\cubescale*\cubex,0,0) coordinate (a) -- ++(0,-\cubescale*\cubey,0) coordinate (b) edge coordinate [pos=1] (g) ++(0,0,-\cubescale*\cubez)  -- ++(\cubescale*\cubex,0,0) coordinate (c) -- cycle
		(o) -- ++(0,0,-\cubescale*\cubez) coordinate (d) -- ++(0,-\cubescale*\cubey,0) coordinate (e) edge (g) -- (c) -- cycle
		(o) -- (a) -- ++(0,0,-\cubescale*\cubez) coordinate (f) edge (g) -- (d) -- cycle;
		\path [every edge/.append style={pic actions, |-|}]
		%(b) +(0,-5pt) coordinate (b1) edge ["\cubex \cubeunits"'] (b1 -| c)
		%(b) +(-5pt,0) coordinate (b2) edge ["\cubey \cubeunits"] (b2 |- a)
		%(c) +(3.5pt,-3.5pt) coordinate (c2) edge ["\cubez \cubeunits"'] ([xshift=3.5pt,yshift=-3.5pt]e)
		;
	},
	/cuboid/.search also={/tikz},
	/cuboid/.cd,
	width/.store in=\cubex,
	height/.store in=\cubey,
	depth/.store in=\cubez,
	units/.store in=\cubeunits,
	scale/.store in=\cubescale,
	width=10,
	height=10,
	depth=10,
	units=cm,
	scale=.1,
}


\hbadness=99999

\begin{document}

%\renewcommand{\thechapter}{\Roman{chapter}}

\title{
	{\Huge Theoretische Physik II\\[3pt]Elektrodynamik}\\[1em]
	{\Large Vorlesung von Prof. Dr. Michael Thoss im Wintersemester 2018}}
\author{Markus Österle\\ Andréz Gockel}
\date{15.10.2018}
\maketitle
\tableofcontents

\setcounter{chapter}{-1}
\chapter{Einführung}

%\addcontentsline{toc}{section}{section die nur im Inhaltsverzeichniss ist}
%\bibliographystyle{plain}
%\bibliography{literature}
%\addcontentsline{toc}{section}{Literatur}

\section{Zur Vorlesung}

\begin{description}
	\item[Dozent] Michael Thoss
	\item[Übungen] Donnerstag/Freitag (ILIAS) beginnt 18./19.10.18
	\item[Übungsleiter] Jakob Bätge
	\item[Abgabe der Hausaufgaben] bus Dienstag 12:00 - Briefkasten GuMi
	\item[Klausur] 13.02.19, 10-12 Uhr, Hörsaal Anatomie (Nachklausur: 26.19, 10-12 Uhr)
	\item[Ankündigungen] ILIAS Pass: theophy2.thoss18
	\item[Angaben] Vorlesung: 4 SWS, Übung: 2 SWS, ECTS: 7 
	\item[Vorkenntnisse] Mathematik: Analysis für Physiker (Vektor Rechnung), Theoretische Physik I, Experimental Physik II. 
\end{description}

\begin{tcolorbox}[colback=white,colframe=black,fonttitle=\bfseries,title=Hinweis zu den Übungen,sharp corners,tcbox raise base]
	\begin{itemize}
		\item[-] Keine Anwesenheitspflicht.
		\item[-] Keine Punktzahl nötig für Klausurzulassung.
		\item[-] Kann auch wehrend Übungen abgegeben werden.
	\end{itemize}
\end{tcolorbox}
\noindent
Lehrbücher: 
\begin{itemize}
	\item W. Nolting, \textit{Grundkurs Theoretische Physik 3: Elektrodynamik} (Springer)
	\item D.J. Griffiths, \textit{Elektrodynamik: Eine Einführung} (Pearson) 
	\item T. Fließbach, \textit{Elektrodynamik} (Spektrum Akademischer Verlag)
	\item J.D. Jackson, \textit{Klassische Elektrodynamik} (Walter de Gruyter) \lcom{geht dieser Vorlesung hinaus}
\end{itemize}

\section{Einführung und Überblick}

Die vier fundamentalen Wechselwirkungen (WW):
\begin{itemize}
	\item Starke WW
	\item \textbf{Elektromagnetische WW}\  \lcom{Wird in dieser Vorlesung betrachtet}
	\item Schwache WW
	\item Gravitation
\end{itemize}

\subsection{Rückblick}
Theoretische Physik 1: 
\begin{itemize}
	\item Mechanik 
	\item Punktmechanik: Bahnkurven von Körpern
	\item Bewegungsgleichung: $m \vec{\ddot r} = \vec{F}$
\end{itemize}
\subsection{Elektrodynamik}
\begin{itemize}
	\item Grundlegende Größen
	\item Felder
	\item 
	\begin{equation*}
	\begin{array}{cc}
	\vec{E}(\vec{r},t)\ \ \  & \vec{B}(\vec{r},t) \\[5pt]
	\tx{elektrisches Feld \ \ \ } &  \tx{Magnetfeld}
	\end{array}
	\end{equation*}
	\item[$\boldsymbol{\rightarrow}$] Feldtheorie \lcom{sehr wichtiges Konzept}
\end{itemize}
\lcom{Wie sind Elektrische Felder definiert?}\\
Experimentelle Definition als Messgröße: Kraft auf Ladung
$$\vec{F} = q(\vec{E}(\vec{r},t) + \vec{v} \times \vec{B}(\vec{r},t))$$
Theoretische Definition ist Mathematisch: Feldgleichungen-Maxwellgleichungen
$$ \ \qquad \vec\nabla \cdot\vec{E} = \frac{1}{\epsilon_0}\rho \qquad \qquad \ \qquad  \vec\nabla\cdot\vec{B} = 0$$
$$\vec\nabla\times\vec{E} + \prt{\vec{B}}{t} = 0 \qquad \vec\nabla\times\vec{B} - \mu_0 \epsilon_0 \prt{\vec{E}}{t} = \mu_0 \vec{j}$$
Hierbei steht $\rho$ für die Ladungsdichte und $\vec{j}$ für die Stromdichte.

\section{Aufbau der Vorlesung}
\textbf{1./2.} Statische Phänomene: $ \ \prt{\vec{E}}{t} = 0 = \prt{\vec{B}}{t}$
$$\Rightarrow \vec\nabla\cdot\vec{E} = \frac{1}{\epsilon_0}\rho \qquad\ \vec\nabla \cdot \vec{B} = 0$$
$$\ \ \, \quad\underbrace{\vec\nabla\times\vec{E} = 0}_{\textrm{1. Elektrostatik}} \qquad \underbrace{\vec\nabla\times\vec{B} = 0}_{\textrm{2. Magnetostatik}}$$
\textbf{3.} Zeitabhängige magnetische/elektrische Felder\\
\textbf{4.} Relativistische Formulierung der Elektrodynamik

\chapter{Elektrostatik}

Wir beschäftigen uns in diesem Kapitel mit \textbf{ruhenden Ladungen} und \textbf{zeitunabhängigen Feldern}. Das Grundproblem besteht darin, dass wir eine Ladungsverteilung haben und das Elektrische Feld und dessen Potential bestimmen wollen.\\[5pt]
\FloatBarrier
\begin{wrapfigure}{r}{3cm}
	\vspace{-1.2cm}
	\fbox{\begin{tikzpicture}
		\draw node[circle,fill,inner sep=1pt,label=left:$q_1$](a){} -- (1,0);
		\draw[xshift=0.5cm,yshift=0.7cm] node[circle,fill,inner sep=1pt,label=right:$q_2$](a){} -- (1,0);
		\draw[xshift=0.5cm,yshift=-0.2cm] node[circle,fill,inner sep=1pt,label=right:$q_3$](a){} -- (1,0);
		\end{tikzpicture}}
\end{wrapfigure}
\FloatBarrier
$\rightarrow$ Feld $\vec{E}(\vec{r})$, el. Potential $\Phi(\vec{r})$
\section{Elektrische und Coulombsches Gesetz}
Ladung: Beobachtungstatsachen:
\begin{enumerate}[i)]
	\item Zwei Arten ,,+``, ,,-``
	\item Abgeschlossenes System: Ladung erhalten: $q = \sum_i q_i = \const$
	\item Ladung ist quantisiert in Einheiten der Elementarladung: $$q = ne,\ n \in \mathbb{Z},\ e = 1,602\cdot 10^{-19}\,\textrm{C}$$ 
	$n = -1$: für ein Elektron wäre ein Beispiel einer Punktladung
\end{enumerate}

\begin{minipage}{.6\linewidth}
	Kontinuierliche Ladungsverteilung Ladungsdichte\\
	$\rho(\vec{r}) = \frac{\textrm{Ladung}}{\textrm{Volumen}} = \frac{\Delta q}{\Delta V}$
	Gesamtladung in $V$: $$Q = \int_V d^3 r\, \rho(\vec{r})$$
\end{minipage}
\begin{minipage}{.4\linewidth}
	\hspace{30pt}
	
	%T2 done
	
	\begin{tikzpicture}
	\node at (1.5,1.5) (a) {};
	\draw  plot [smooth cycle, tension=1] coordinates { (0,0) (.5,1.5) (.5,2.5) (2,2) (3,2) (2.5,1) (1.5,0) };
	\pic at (a) {annotated cuboid={width=3,height=3,depth=3}};
	\draw[] (a) node[right,xshift=10pt] {$ \Delta V $};
	\draw[] (a) node[right,xshift=-10pt,yshift=10pt] {$ \Delta q $};
	\draw[->] (3,.5) to[out=180,in=-10] (1.5,.7);
	\node[right] at (3,.5) (b) {$ V $};
	\draw[thick,->] (-.5,-.5) -- (-.5,2) node[anchor=south east] {y};
	\draw[thick,->] (-.5,-.5) -- (2,-.5) node[anchor=north west] {x};
	\draw[thick,->] (-.5,-.5) -- (1.24,1.24) node[below,xshift=-10,yshift=-20pt] {$ \vec{r} $};
	\end{tikzpicture}
\end{minipage}

\subsection{Coulombsches Gesetz}

\begin{minipage}{.585\linewidth}
	Die Kraft, welche eine am Ort $\vec{r}_2$ lokalisierte Punktladung auf eine Punktladung am Ort $\vec{r}_1$ ausübt, ist gegeben durch:
	$$\vec{F}_{12} = k \frac{q_1 q_2}{|\vec{r}_1 - \vec{r}_2|^2} \underbrace{\frac{\vec{r}_1 - \vec{r}_2}{|\vec{r}_1 - \vec{r}_2|}}_{\vec{e}_{r_{12}}}$$
\end{minipage}
\begin{minipage}{.4\linewidth}
	
	%T3 done
	
	\hspace{30pt}
	\begin{tikzpicture}
	\node at (1,2) (a) {};
	\node at (2,1) (b) {};
	\draw[thick,->] (0,0) -- (0,2.5);
	\draw[thick,->] (0,0) -- (2.5,0);
	\draw[->] (0,0) -- (a) node[above] {$ q_1 $};
	\draw[->] (0,0) -- (b) node[right] {$ q_2 $};
	\node at (.5,1) (c) {};
	\node at (1,.5) (d) {};
	\draw[] (c) node[above,yshift=5,xshift=-3] {$ \vec{r}_1 $};
	\draw[] (d) node[below,yshift=2,xshift=5] {$ \vec{r}_2 $};
	\end{tikzpicture}
\end{minipage}

\begin{enumerate}
	\item $\vec{F}_{12} \sim q_1 q_2$
	\item $\vec{F}_{12} \sim \frac{1}{|\vec{r}_1 - \vec{r}_2|^2}$
	\item $\vec{F}_{12} \sim q_1 q_2\, \vec{e}_{r_{12}}$
	\item $\vec{F}_{12} = -\vec{F}_{21}$
\end{enumerate}
Es gilt das Superpositionsprinzip: Das heißt, durch vektorielle Addition der Kräfte kann die Gesamtkraft ermittelt werden.

$$\vec{F}_1 = k \summ{j = 2}{N} \frac{q_1 q_j}{r_{1j}^2}\vec{e}_{r_{1j}}$$

\paragraph{Zur Konstanten $k$:}
Die Konstante ist abhängig von dem verwendeten Maßsystemen.
\begin{enumerate}[i)]
	\item Gauß-System (cgs): $k \equiv 1$, dyn = $\frac{\textrm{g}\cdot \textrm{cm}}{\textrm{s}^2} = 10^{-5}\,\textrm{N}$
	1 dyn = $\frac{(1\textrm{ESE})^2}{\textrm{cm}^2} \quad 1\textrm{ESE} = \frac{\sqrt{\textrm{g}\cdot \textrm{cm}^3}}{\textrm{s}}$
	\item SI (MKSA-System): Definition von A = Amp\`ere 
	
	
	% T4 done 
	
	\vspace{-15pt}
	$$\frac{\Delta F}{\Delta l}= 2 \cdot 10^{-7}\,\frac{\textrm{N}}{\textrm{m}} \qquad \qquad \begin{tikzpicture}
	\draw (0,0) -- (3,0);
	\draw (0,1) -- (3,1);
	\draw[thick,<->] (.75,0) -- (.75,1) node[anchor=north east,yshift=-6pt] {1 m};
	\draw[thick,->] (2,0) -- (2,.3);
	\draw[thick,->] (2,1) -- (2,.7);
	\draw[->] (2,0) -- (2.5,0) node[above] {$ I $};
	\draw[->] (2,1) -- (2.5,1) node[above] {$ I $};
	\end{tikzpicture}$$
	Strom = $\frac{\textrm{Ladung}}{\textrm{Zeit}}$ $\Rightarrow$  \ 
	$
	1 \tx{A} = \frac{1 \tx{C}}{1 \tx{s}} \quad \rightarrow \quad e = 1{,}602 \cdot 10^{-19} \, \tx{C} \qquad c \approx 3 \cdot 10^{8} \, \frac{\tx{m}}{\tx{s}}
	$\\[5pt]
	$
	\frac{\Delta F}{\Delta l} = k \frac{2 I^2}{c^2 d} \qquad \rightarrow k = 2 \cdot 10^{-7} \ \frac{\tx{N}}{\tx{m}} \frac{c^2 1 \tx{m}}{2 (1 \tx{A})^2} = 10^{-7} c^2 \, \frac{\tx{N}}{\tx{A}^2}
	$
	\begin{equation*}
	k = \frac{1}{4 \pi \epsilon_0}
	\end{equation*}
	Damit erhalten wir für die Dielektrizitätskonstante des Vakuums:
	\begin{equation*}
	\epsilon_0 = 8{,}85 \cdot 10^{-12} \frac{\tx{C}^2}{\tx{N} \tx{m}^2}
	\end{equation*}
\end{enumerate}


\section{Elektrisches Feld}
\subsection{Feld eines Systems von Punktladungen}

\begin{minipage}{.5\linewidth}
	$N$-Ladungen $q_1, \dots, q_N$ ruhen an den Orten $\vec{r}_1, \dots, \vec{r}_N$. Nun bringen wir eine Testladung $q$ am Ort $\vec{r}$ mit ein. 
\end{minipage}
\begin{minipage}{.5\linewidth}
	
	%T5 done
	
	\hspace{50pt}
	\begin{tikzpicture}
	\node at (3,1) (a) {};
	\node at (2.5,2) (b) {};
	\node at (1,2) (c) {};
	\draw[thick,->] (0,0) -- (0,2.5) node[anchor=south east] {y};
	\draw[thick,->] (0,0) -- (2.5,0) node[anchor=north west] {x};
	\draw[->] (0,0) -- (a) node[above] {$ q_1 $};
	\draw[->] (0,0) -- (b) node[right] {$ q_2 $};
	\draw[->] (0,0) -- (c) node[above] {$ q $};
	\node at (1.5,.5) (c) {};
	\node at (1.25,1) (d) {};
	\node at (.5,1) (e) {};
	\draw[] (c) node[below,xshift=10pt] {$ \vec{r}_1 $};
	\draw[] (d) node[below,xshift=5pt] {$ \vec{r}_2 $};
	\draw[] (e) node[left] {$ \vec{r} $};
	\end{tikzpicture}
\end{minipage}
Kraft von $q_1,\ q_2$ auf $q$
$$\vec{F} = \frac{1}{4\pi \epsilon_0} q \summ{j = 1}{N} q_n \frac{\vec{r} - \vec{r}_j}{|\vec{r} - \vec{r}_j|^3} = q \vec{E}(\vec{r})$$
Somit ist das elektrisches Feld:
$$\rmbox{\vec{E}(\vec{r}) = \frac{1}{4 \pi \epsilon_0} \summ{j = 1}{N} q_j \frac{\vec{r} - \vec{r}_j}{|\vec{r} - \vec{r}_j|^3}}$$
\textbf{\emph{Bemerkung}}
\begin{enumerate}[i)]
	\item Testladung klein (formal: $ \lim_{q\to0}\frac{\vec{F}}{q} $)
	\item math. $ \vec{E}(\vec{r}) $ Vektorpfeil\\
	\begin{equation*}
	\tx{kartesisch:} \quad \vec{E}(\vec{r}) = \begin{pmatrix}
	E_x(\vec{r}) \\ E_y(\vec{r}) \\ E_z(\vec{r})
	\end{pmatrix}
	\end{equation*}
	\item Wechselwirkungsprozess: 2 Teile
	\begin{equation*}
	q_j \rightarrow \vec{E}(\vec{r}) \rightarrow \vec{F} = q \vec{E} (\vec{r})
	\end{equation*}
	\item Superpositionsprinzip gilt
\end{enumerate}

\subsection[Feld einer kontinuierlichen Ladungsverteilung]{Feld einer kontinuierlichen Ladungsverteilung $\rho(\vec{r})$}


\begin{minipage}{.5\linewidth}
	$$\vec{E}(\vec{r}) =\frac{1}{4 \pi \epsilon_0} \ub{\int\limits_V}_{\mathclap{\substack{\tx{schließt alle} \\ \tx{Ladungen ein}}}} d^3 r'\, \rho(\vec{r}')\frac{\vec{r} - \vec{r}_j}{|\vec{r} - \vec{r}_j|^3}$$
	$\rho(\vec{r}_j) = \frac{\Delta q_j}{\Delta V_j}$
\end{minipage}
\begin{minipage}{.4\linewidth}
	
	%T6 done
	
	\hspace{50pt}
	\begin{tikzpicture}
	\node at (1.5,1.5) (a) {};
	\draw  plot [smooth cycle, tension=1] coordinates { (0,0) (.5,1) (.5,2.5) (2,2.2) (3,2) (2.5,.5) (1.5,-.2) };
	\pic at (a) {annotated cuboid={width=3,height=3,depth=3}};
	\draw[] (a) node[right,xshift=10pt] {$ \Delta V $};
	\draw[] (a) node[right,xshift=-10pt,yshift=10pt] {$ \Delta q_j $};
	\draw[->] (3,.5) to[out=180,in=-40] (2.2,.8);
	\node[right] at (3,.5) (b) {$ V $};
	\draw[thick,->] (-.5,-.5) -- (-.5,2) node[anchor=south east] {y};
	\draw[thick,->] (-.5,-.5) -- (2,-.5) node[anchor=north west] {x};
	\draw[thick,->] (-.5,-.5) -- (.25,.25) node[anchor=south east] {z};
	\draw[] (1,0) node[above] {$ \rho(\vec{r}) $};
	\end{tikzpicture}
\end{minipage}

\begin{align*}
\vec{E}(\vec{r}) &= k\sum_j\Delta q_j\frac{\vec{r}-\vec{r}_j}{|\vec{r}-\vec{r}_j|^3}\\
&= k\sum_j\Delta V_j\rho(\vec{r}_j)\frac{\vec{r}-\vec{r}_j}{|\vec{r}-\vec  {r}_j|^3}\\
\scriptstyle{\tx{mit } \Delta V_j \to 0}\ \ &\rightarrow k\int_V d^3r'\rho(\vec{r}')\frac{\vec{r}-\vec{r}'}{|\vec{r}  -\vec{r}'|^3}
\end{align*}

%%%%%% Vorlesung 2

% Wiederholung:
% $\vec{F} = q\vec{E}$ punktladungen %
% \fbox{\begin{tikzpicture}
%	\draw node[circle,fill,inner sep=1pt,label=left:$q_1$](a){} -- (1,0);
%	\draw[xshift=0.5cm,yshift=0.7cm] node[circle,fill,inner sep=1pt,label=right:$q_2$](a){} -- (1,0);
%	\draw[xshift=0.5cm,yshift=-0.2cm] node[circle,fill,inner sep=1pt,label=right:$q_3$](a){} -- (1,0);
%\end{tikzpicture}}
%
%

\subsection{Ladungsdichte einer Punktladung}
\paragraph{Deltafunktion}


%% T1


$$\rho (\vec{r}) = q \delta(\vec{r}-\vec{r}_0)$$
Punktladung in $\vec{r}_0 \Rightarrow \rho(\vec{r}) = 0 \quad \vec{r} \neq \vec{r}_0$

Ladungsdichte divergiert in $\vec{r}_0$
$$\rho(\vec{r}_0) = \infty$$

Modell für Punktladung:
Ladung $q$ in Kugel mit Radius $\epsilon$  um $\vec{r}_0,\ \epsilon \rightarrow 0$
$$\rho_2 (\vec{r}) = \left\{ \begin{array}{cc}
\frac{q}{v_k} & |\vec{r}| \leq \epsilon\\
0 & \textrm{sonst}	
\end{array} \right\} = \frac{q}{\frac{4}{3}\pi \epsilon^3}\underbrace{\Theta}_{\hspace{20pt}\mathclap{\textrm{Stufenfunktion}}}(\epsilon - |\vec{r}|)$$

$$\rho(\vec{r}) = \lim_{\epsilon \rightarrow 0}\ \rho_\epsilon (\vec{r}) = \left\{ \begin{array}{cc}
\infty & \vec{r} = 0\\
0 & \vec{r} \neq 0
\end{array}\right.$$
Divergenz muss so sein, dass $$ \int_V d^3r\ \rho(\vec{r}) = q$$
% unter dem V: \vec{F}_0 \in V

\paragraph{Definition Delta-Funktion (Diracsche Deltafunktion)}

\begin{enumerate}
	\item 
	$$\delta (\vec{r} - \vec{r}_0) = \left\{ \begin{array}{cc}
	\infty & \vec{r} \neq \vec{r}_0 \\
	0 & \vec{r} = \vec{r}_0
	\end{array}\right.$$
	\item 
	$$\int_V d^3 r\ f(\vec{r}) \delta(\vec{r}$$ %%% hier fehlt was
\end{enumerate}


% tafel 2 nochmal ?


\paragraph{Mathematik}
Distribution - Funktional

Funktional: Abb. Funktionen $\mapsto \mathbb R, \mathbb C$
$$\delta_{\vec{r}_0}: f \mapsto f(\vec{r}_0)$$

\paragraph{Physik}

$$\int d^3 r\ f(\vec{r}) \delta (\vec{r}-\vec{r}_0) = f(\vec{r})$$
$\delta$-Fkt. als Grenzwert einer Folge von Funktionen im Integral
$$\int d^3 r\ f(\vec{r}) \delta (\vec{r} - \vec{r}_0) = \lim_{\epsilon \to 0} \quad \int d^3\ f(\vec{r} g_\epsilon(\vec{r}-\vec{r}_0) $$
mit
$$\lim_{\epsilon \to 0} g_\epsilon (\vec{r}-\vec{r}_0) =  \left\{ \begin{array}{cc}
0 & \vec{r} \neq \vec{r}_0 \\
\infty & \vec{r} = \vec{r}_0
\end{array}\right.$$
$$\int_V d^3 r\ g_\epsilon (\vec{r}-\vec{r}_0) = 1$$

Beispiel: $g_\epsilon (\vec{r}-\vec{r}_0) = \frac{\Theta(\epsilon - |\vec{r}|}{\frac{4}{3}\pi \epsilon^3}$

Mehrere Punktladungen $q_j$ in $\vec{r}_j$
$$\rho(\vec{r}) = \sum_j q_j \delta(\vec{r}-\vec{r}_j)$$
$$\Rightarrow \vec{E}(\vec{r}) = \kq \int_V d^3 r'\ \rho(\vec{r}') \frac{\vec{r} - \vec{r}'}{|\vec{r} - \vec{r}'|^3}$$
% align here 
$$= \kq \int_V d^3 r'\ \sum_j q_j \delta(\vec{r} - \vec{r}_j) \frac{\vec{r} - \vec{r}'}{|\vec{r} - \vec{r}'|^3}$$
%% hier fehlt was

\subsection{Flächenladungsdichte}
%% T2

$\sigma(\vec{r}) = \frac{\textrm{Ladung}}{\textrm{Fläche}} = \frac{\Delta q}{\Delta A}$

erzeugte elektrisches Feld:
$$\vec{E}(\vec{r}) = \kq \int_A \underbrace{df'}_{\mathclap{\textrm{Flächenelement}}}\ \sigma (\vec{r}) \frac{\vec{r} - \vec{r}'}{|\vec{r} - \vec{r}'|^3}$$
% oberflächenitegral

Beispiel: Elektrisches Feld einer homogenen Flächenladung
%% T3
$\sigma = \const$ in $x$-$y$-Ebene
$$\vec{E}(\vec{r}) = \kq \intt{-\infty}{\infty}dx'\intt{-\infty}{\infty}dy'\ \sigma \frac{\vec{r} - \vec{r}'}{|\vec{r} - \vec{r}'|^3} \qquad \vec{r}' = (x', y', 0)$$
Symmetrie: $\vec{E}$ unabhängig von $x,y \qquad \vec{r} = (0,0,z)$
$$\vec{r} - \vec{r}' = (-x',-y',z),\ |\vec{r} - \vec{r}'|^3 = (x'^2+y'^2+z^2)^{\nicefrac{3}{2}}$$
$$E_x \sim \sigma \intt{-\infty}{\infty}dx'\intt{-\infty}{\infty}dy'\ \frac{(x')}{(x'^2 + y'^2 + z'^2)^{\nicefrac{3}{2}}} = 0 = E_y$$

$\vec{E} = (0,0,E_z)$

$$E_z = \kq \sigma \intt{-\infty}{\infty}dx'\underbrace{\intt{-\infty}{\phantom{-}\infty}dy'\ \frac{(x')}{(x'^2 + y'^2 + z'^2)^{\nicefrac{3}{2}}}}_{\frac{1}{x'^2 +z'^2}\left.\frac{x'}{(x'^2 + y'^2 + z'^2)^{\nicefrac{3}{2}}} \right\rvert_{-\infty}^{\infty} = }$$
%%% hier fehlt was im underbrace
%%% make align
%%% hier fehlt der rest der tafel
%% T4
%%% rest dieser tafel
%%%% remove phantoms
%% T5

\subsection{Linenladungsdichte}
%% T6
$\lambda(\vec{r} = \frac{\textrm{Ladung}}{\textrm{Länge}} = \frac{\Delta q}{\Delta s}$
$$\vec{E}(\vec{r}) = \kq \underbrace{\int_\gamma}_{\mathclap{\textrm{Linienintegral}}} ds'\ \lambda(\vec{r}') \frac{\vec{r} - \vec{r}'}{|\vec{r} - \vec{r}'|^3}$$

Beispiel: Elektrisches Feld einer homogenen Linienladung
$\lambda = \const$
%% T7

\begin{align*}
\vec{E}(\vec{r}) &= \kq \int_\gamma ds'\ \lambda \frac{\vec{r} - \vec{r}'}{|\vec{r} - \vec{r}'|^3} \qquad \gamma: z' \mapsto \vec{r}'(z') = \begin{pmatrix} 0 \\ 0\\ z \end{pmatrix}\\
&= \frac{\lambda}{4 \pi \epsilon_0} \intt{-\infty}{\infty}dz'\ \frac{\vec{r} - \begin{pmatrix} 0 \\ 0\\ z \end{pmatrix}}{(x^2 + y^2 + (z-z')^2)^{\nicefrac{3}{2}}}
\end{align*}
$\tilde{z} = z' - z$
$$E_x = \frac{\lambda x}{4 \pi \epsilon_0}\intt{-\infty}{\infty}dz'\ \frac{1}{(x^2 + y^2 + (z-z')^2)^{\nicefrac{3}{2}}} = \frac{\lambda}{4 \pi \epsilon_0}\intt{-\infty}{\infty}d\tilde{z}\ \frac{1}{(x^2 + y^2 + \tilde{z}^2)^{\nicefrac{3}{2}}} = \frac{\lambda}{2 \pi \epsilon_0} \frac{x}{x^2 + y^2}$$
$$E_y = \frac{\lambda}{2 \pi \epsilon_0}\frac{x}{x^2 + y^2}$$
$$E_z = \frac{\lambda}{4 \pi \epsilon_0} \intt{-\infty}{\infty}dz'\ \frac{z-z'}{(x^2 + y^2 + (z-z')^2)^{\nicefrac{3}{2}}} = 0$$
$$\vec{E}(\vec{r}) \frac{\lambda}{2 \pi \epsilon_0} \frac{1}{XXX}$$
%%%% hier fehlt was

\section{Feldgleichungen und elektrostatische Potential}
$$\vec{E}(\vec{r}) = \kq \int d^3 r'\ \rho(\vec{r}') \frac{\vec{r} - \vec{r}'}{|\vec{r} - \vec{r}'|^3}$$
\subsection{Elektrostatisches Potential}
elektrische Feld ist ein Potentialfeld $\vec{E}(\vec{r}) = - \nabla\phi(\vec{r}) = - \bigg( \vec{e}_x \prt{\phi}{x} + \vec{e}_y \prt{\phi}{y} + \vec{e}_z \prt{\phi}{z}\bigg)$
$$\frac{\vec{r} - \vec{r}'}{|\vec{r} - \vec{r}'|^3} = -\nabla\frac{1}{|\vec{r}-\vec{r}'|}$$
$$-\prt{}{x} \frac{1}{[(x-x')^2 + (y-y')^2 + (z-z')^2]^{\nicefrac{1}{2}}} = - \bigg(-\frac{1}{2}\bigg) \frac{2(x-x')}{[(x-x')^2 + (y-y')^2 + (z-z')^2]^{\nicefrac{3}{2}}} = \frac{(x-x')}{|\vec{r} - \vec{r}'|^3}$$
$$\Rightarrow \vec{E}(\vec{r}) = \kq \int d^3 r'\ \rho(\vec{r}') \bigg(-\nabla_{\vec{r}} \frac{1}{|\vec{r} - \vec{r}'|} \bigg) = \nabla_F \kq \int d^3 r'\ \frac{\rho(\vec{r}')}{|\vec{r} - \vec{r}'|}$$
$\rightarrow$ elektrostatisches Potential
$$\phi(\vec{r}) = \kq \int d^3 r'\ \frac{\rho(\vec{r}')}{|\vec{r} - \vec{r}'|} + c$$
übliche Konvention: $c = 0\ (\phi(\vec{r})\buildrel{\rightarrow}\over{|\vec{r}| \to \infty} 0)$

Potential einer Punktladung in $\vec{r}_0$:
$$\rho(\vec{r}) = q \delta(\vec{r}-\vec{r}_0)$$
$$\phi(\vec{r}) = \kq \int_{\mathbb R^3} d^3 r'\ \frac{q\delta(\vec{r}' - \vec{r}_0)}{|\vec{r}-\vec{r}'|} = \kq \frac{q}{|\vec{r} - \vec{r}_0|}$$

$$\vec{E}$$
%%% hier fehlt end


%%%% Vorlesung 3

\lcom{(Funktional-analysis Siegfried Großmann Springer)\\
	(Landau-Lipschitz Buch geht weit der Vorlesung hinaus)}

\subsection{Feldgleichugn (differentielle Form)}

\paragraph{Rotation (Wirbel)}
$$\textrm{rot}\, \vec{E} \nabla \times \vec{E} = \vec{e}_x \bigg(\prt{E_z}{y} - \prt{E_x}{z}\bigg) + \vec{e}_y \bigg(\prt{E_x}{z} - \prt{E_z}{x}\bigg) + \vec{e}_z \bigg(\prt{E_y}{x} - \prt{E_x}{y}\bigg) \Rightarrow \nabla \times$$
$$\Rightarrow \nabla \times \vec{E} = - \nabla \times (\nabla\phi) = 0$$

Mathe: Es sind äquivalent 
\begin{enumerate}[i]
	\item $\vec{E} = - \nabla\phi$
	\item $\nabla \times \vec{E} = 0$ (auf einfach zusammenhäng. Gebiet)
	\item Kurvenintegral $\int_\gamma d\vec{r} \cdot \vec{E}$ ist wegunabhängig %%% T1
	$$\intt{\vec{r}_1}{\vec{r}_2} d\vec{r} \cdot \vec{E} = - \intt{\vec{r}_1}{\vec{r}_2} dt \underbrace{\frac{d\vec{r}}{dt}\times \nabla\phi(\vec{r}(t))}_{\frac{d\phi}{dt}} = \underbrace{(\phi(\vec{r}_2) - \phi(\vec{r}_1))}_{\textrm{Potentialdifferenz}} $$
\end{enumerate}

\subsection{Divergenz (Quellen)}

$$\div \vec{E} = \nabla \cdot \vec{E} = \prt{E_x}{x} + \prt{E_y}{y} + \prt{E_z}{z}$$

\begin{align*}
\nabla \cdot \vec{E}(\vec{r}) &= \nabla_{\vec{r}} \cdot \kq \int d^3r'\ \rho(\vec{r}') \frac{1}{|\vec{r} - \vec{r}'|^3}\\
&= \kq \int_V d^3 r' \rho (\vec{r}') \nabla_{\vec{r}} \cdot \frac{1}{|\vec{r} - \vec{r}'|^3}
\end{align*}

$x$-Anteil:
\begin{align*}
\frac{\partial}{\partial x}\frac{x-x'}{[(x-x')^2 + (y-y')^2 + (z-z')^2]^{\nicefrac{3}{2}}} &= \frac{1 \cdot [\dots]^{\nicefrac{3}{2}}(x - x')(x - x')^{\nicefrac{3}{2}}\cdot 2[\dots]^{\nicefrac{1}{2}}}{[\dots]^3}\\
&= \frac{[\dots]^{\nicefrac{1}{2}}((x - x')^2 + (y - y')^2 + (z - z')^2 - 3(x - x')^2)}{[\dots]^{\nicefrac{3}{2}}}\\
&= \frac{(y - y')^2 + (z - z')^2 - 2(x - x')^2}{[\dots]^{\nicefrac{3}{2}}}
\end{align*}

$$\frac{\partial}{\partial y} \frac{y-y'}{[(x-x')^2 + (y-y')^2 + (z-z')^2]^{\nicefrac{3}{2}}} = \frac{(x - x')^2 + (z - z')^2 - 2(y - y')^2}{[\dots]^{\nicefrac{3}{2}}}$$ 

$$\frac{\partial}{\partial z} \frac{z-z'}{[(x-x')^2 + (y-y')^2 + (z-z')^2]^{\nicefrac{3}{2}}} = \frac{(x - x')^2 + (y - y')^2 - 2(z - z')^2}{[\dots]^{\nicefrac{3}{2}}}$$ 

$$\nabla \cdot \frac{\vec{r} - \vec{r}'}{|\vec{r} - \vec{r}'|^3} = 0 \quad \textrm{falls} \quad \vec{r} \neq \vec{r}'$$

$$\Rightarrow \textrm{falls} \quad \vec{r} \notin V, \textrm{ d.h. } \vec{r} \textrm{ in Gebiet ohne Ladungsdichte } \rho(\vec{r}) = 0$$

$$\Rightarrow \nabla \cdot \vec{E}(\vec{r}) = 0$$ %%%

%%%%%

$\vec{r} \in V$: Grenzwertbetrachtung (Regularisierung des Integranden)\\
statt
$$\frac{\vec{r} - \vec{r}'}{|\vec{r} - \vec{r}'|^3} = \frac{\vec{r}-\vec{r}'}{[(x-x')^2 + (y-y')^2 + (z-z')^2]^{\nicefrac{3}{2}}}$$
betrachten wir:
$$\vec{f}_a (\vec{r}-\vec{r}') = \frac{\vec{r}-\vec{r}'}{[(x-x')^2 + (y-y')^2 + (z-z')^2]^{\nicefrac{3}{2}}} = \frac{\vec{r}-\vec{r}'}{[(\vec{r}-\vec{r}')^2 + a^2]^{\nicefrac{3}{2}}} \quad a \in \mathbb R,\ a>0$$

am Ende Grenzwert $\displaystyle{\lim_{a \to 0}}$
$$\nabla \cdot \vec{E} (\vec{r}) = \kq \lim_{a \to 0} \int_V d^3r'\ \rho(\vec{r}')\ \rmbox{\hspace{-10pt}\nabla_{\vec{r}}\cdot f_a(\vec{r}-\vec{r}')}$$

\begin{align*}
\prt{}{x} \frac{x-x'}{[(\vec{r}-\vec{r}')^2 + a^2]^{\nicefrac{3}{2}}} &= \frac{[\dots +a^2]^{\nicefrac{3}{2}} - (x - x') \frac{3}{2} \cdot 2(x-x') [\dots + a^2]^{\nicefrac{3}{2}}}{[\dots + a^2]^3}\\
&= \frac{(y - y')^2 + (z - z')^2 + a^2 - 2(x - x')^2}{[\dots + a^2]^{\nicefrac{3}{2}}}
\end{align*}

$$\nabla_{\vec{r}} \cdot f_a(\vec{r}-\vec{r}') = \frac{3a^2}{[(\vec{r}-\vec{r}')^2 + a^2]^{\nicefrac{5}{2}}}$$
$$\lim_{a \to 0} f_a(\vec{r}-\vec{r}') = \left\{ \begin{array}{cc}
0 & \vec{r} \neq \vec{r}'\\
\infty & \vec{r} = \vec{r}'	
\end{array} \right.$$
$$\Rightarrow \textrm{zum Integral } \int_V d^3r' \dots \textrm{ trägt (in Limes }a \to 0) \textrm{ nur der Bereich } \vec{r}' \approx \vec{r} \textrm{ bei}$$
$$K_R (\vec{r}) = \{\vec{r}' \in \mathbb R^3: |\vec{r} - \vec{r}'| \leq R\}$$

$$\lim_{a \to 0} \int_V d^3 r'\ \rho(\vec{r}') \nabla_{\vec{r}} \cdot f_a(\vec{r}- \vec{r}') = \lim_{a \to 0} \int_{K_R (F)} d^3 r' \ \rho(\vec{r}') \frac{3a^2}{[(\vec{r}-\vec{r}')^2 + a^2]^{\nicefrac{5}{2}}} + \lim_{a \to 0} $$ %%%

%%%%%%%%


% unter diesem ist alles korrekt 

\begin{enumerate}
	\item Integral:
	\begin{align*}
	\int_{K_R (0)} d^3 \tilde r\ \rho(\vec{r}) \frac{3a^2}{(\tilde{r}^2 + a^2)^{\nicefrac{5}{2}}} &= \rho(\vec{r}) \underbrace{\intt{0}{R} d\tilde r\ \frac{3a^2}{(\tilde{r}^2 + a^2)^{\nicefrac{5}{2}}}}_{\left[\frac{\tilde r^3}{(\vec{\tilde{r}}^2 + a^2)^{\nicefrac{3}{2}}}\right]_0^R} \underbrace{\int d \equaltoup{\Omega}{\mathclap{\sin \theta d \theta d \varphi}}}_{= 4\pi}\\
	&= 4 \pi \rho(\vec{r}) \frac{R^3}{(R^2 + a^2)^{\nicefrac{3}{2}}} \underset{a \to 0}{\longrightarrow} 4 \pi \rho(\vec{r})
	\end{align*} 
	\item Integral:
	$$\int_{K_R (0)} d^3 \tilde{r}\ \equalto{\vec{\tilde{r}}}{\tilde{r}\vec{e}_{\vec{\tilde{r}}}} \cdot \nabla_{\vec{r}} \rho(\vec{r}) \frac{3 a^2}{(\tilde{r}^2 + a^2)^{\nicefrac{5}{2}}} = \underbrace{\intt{0}{R} d \tilde r\ \frac{3a^2 \tilde r^3}{(\tilde r^2 + a^2)^{\nicefrac{3}{2}}}}_{\frac{2}{3}a -3a^2 \left(\frac{R^2 +\frac{2}{3}a^2}{(R^2 + a^2)^{\nicefrac{3}{2}}}\right)} \underbrace{\int d \Omega\ \vec{e}_{\tilde{r}} \cdot \nabla \rho(\vec{r})}_{\textrm{unabh. von } a} \underset{a \to 0}{\longrightarrow} 0$$
	gilt auch für alle höheren Terme
\end{enumerate}

$$\lim_{a \to 0} \int_V d^3 r' \rho(\vec{r}) \nabla_{\vec{r}}\cdot \frac{(\vec{r}-\vec{r}'}{[(\vec{r}-\vec{r}')^2 + a^2]^{\nicefrac{3}{2}}} = 4 \pi \rho (\vec{r})$$
$$\Rightarrow \nabla \cdot \vec{E}(\vec{r}) = \kq \lim_{a \to 0} '' = \frac{1}{\epsilon_0} \rho(\vec{r})$$

\rbox{$$\nabla \vec{E}(\vec{r}) = \frac{1}{\epsilon_0} \rho(\vec{r}) \quad \vec{r} \in \mathbb R^3$$}

\subsection{Zusammenfassung:}

\paragraph{Feldgleichungen der Elektrostatik}
Mathe: partielle DGL
\rbox{$$\nabla \vec{E}(\vec{r}) = \frac{1}{\epsilon_0} \rho(\vec{r}) \textrm{ inhomogene DGL}$$ 
	$$\nabla \times \vec{E}(\vec{r}) = 0\textrm{ homogene DGL}$$}
DGL für Potential $\phi$: 
$\vec{E} = - \nabla \phi$

\begin{align*}\nabla \cdot \vec{E} = \nabla \cdot (- \nabla \phi) &= - \nabla \cdot \begin{pmatrix} 
\partial_x \phi \\
\partial_y \phi \\ 
\partial_z \phi 
\end{pmatrix} \\
&= - \underbrace{\left( \frac{\partial^2 \phi}{\partial x^2} + \frac{\partial^2 \phi}{\partial y^2} + \frac{\partial^2 \phi}{\partial  z^2} \right) }_{=: \Delta \phi}\end{align*}

Partielle DGL 2. Ordnung:
\frbox{Poissongleichung}{$$\Delta \Phi(\vec{r}) = - \frac{1}{\epsilon_0} \rho(\vec{r})$$}

für Gebiete mit $\rho (\vec{r}) = 0$:
$$\Delta \phi (\vec{r}) = 0 \qquad \textrm{Laplacegleichung}$$

\paragraph{Darstellung der Deltafunktion:}

$$\lim_{a \to 0} \int_{\mathbb{R}^3} d^3r'\ \rho(\vec{r}')\underbrace{\nabla_{\vec{r}} \cdot \frac{(\vec{r} - \vec{r}')}{[(\vec{r} - \vec{r}')^2 + a^2]^{\nicefrac{3}{2}}}}_{\frac{3 a^2}{[(\vec{r} - \vec{r}')^2 + a^2]^{\nicefrac{5}{2}}} =: g_a(\vec{r}' - \vec{r})} = 4 \pi \rho(\vec{r})$$

$$\frac{1}{4 \pi} g_a \textrm{ liefert Grenzwertdarstellung der $\delta$-funktion.}$$

$$\lim_{a\to 0} \int_{\mathbb{R}^3} d^3r'\ \rho(\vec{r}') \frac{1}{4\pi}g_a(\vec{r}'-\vec{r})=\rho(\vec{r})$$

$$\lim_{a \to 0} g_a(\vec{r}' - \vec{r}) = \left\{ \begin{array}{cc}
0 		& \vec{r} \neq  \vec{r}'\\
\infty 	& \vec{r} = 	\vec{r}'
\end{array} \right.$$

$$\delta(\vec{r}) = \lim_{a \to 0} \frac{1}{4 \pi} \nabla_{\vec{r}} \cdot \frac{r^2}{(r^2+a^2)^{\nicefrac{3}{2}}}$$

$$\overset{\textrm{formal}}{=} \frac{1}{4 \pi} \nabla \cdot \underbrace{\frac{\vec{r}}{r^3}}_{\mathclap{= - \nabla \frac{1}{r}}} = - \frac{1}{4 \pi} \nabla \cdot \bigg(\nabla \frac{1}{r}\bigg) = \frac{-1}{4 \pi} \Delta \frac{1}{r} \Rightarrow \Delta \frac{1}{r} = -4 \pi \delta(\vec{r})$$

z.B. Potential einer Punktladung $\rho$ q in $\vec{r}_0$:
$$\phi(\vec{r}) = \kq \frac{q}{|\vec{r} - \vec{r}_0|}$$
$$\Delta \phi(\vec{r}) = \kq q \underbrace{\Delta_{\vec{r}} \frac{1}{|\vec{r} - \vec{r}_0|} }_{= - 4 \pi \delta(\vec{r} - \vec{r}_0)} = - \frac{1}{\epsilon_0} \underbrace{q \delta(\vec{r} - \vec{r}_0)}_{= \rho(\vec{r})} = \frac{1}{\epsilon_0} \rho(\vec{r})$$

%%%%%%%%% Vorlesung 4

\bbb{Wiederholung}{\begin{align*}
	\vec{\nabla} \cdot \vec{E}(\vec{r}) &= \frac{1}{\epsilon_0} \rho(\vec{r})\\
	\vec{\nabla} \times \vec{E}(\vec{r}) &= 0 \\[10pt]
	\Rightarrow \quad \vec{E} &= - \vec{\nabla} \Phi \\
	\Rightarrow \quad \Delta \Phi (\vec{r}) &= - \frac{1}{\epsilon_0} \rho(\vec{r})
	\end{align*}
}

\subsection{Integralsätze der Vektoranalysis}

\paragraph{1) Gaußscher Satz:} 
Sei $\vec{A}(\vec{r})$ ein Vektorfeld im Volumen $V \subset \mathbb R^3$, so gilt:
$$\int_V d^3r\ \nabla \cdot \vec{A}(\vec{r}) = \int_{\partial V} d\vec{f}\ \cdot \vec{A}(\vec{r})$$
$$\partial V \tx{ Rand von } V$$
$$d \vec{f} = \custo{\rightarrow}{\vec{n}}{\mathclap{\textrm{nach aussen orientierter Normaleneinheutsvektor}}}\ df$$
%%%% T1

\noindent
\emph{Bemerkung:}\\
\begin{enumerate}[i)]
	\item Analogie 1D: Fundamentalsatz der Integralrechnung:
	\begin{equation*}
	\int_{a}^{b} dx \frac{df}{dx} = f(b) - f(a)
	\end{equation*}
	\item Geometrische / physikalische Integration:\\
	Fluss des Vektorfeldes $ \vec{A} $ durch $ \partial V $
	\begin{equation*}
	\int_{\partial V} d\vec{f} \cdot \vec{A}
	\end{equation*}
	Integral über die Quellen von $ \vec{A} $
	\begin{equation*}
	\int_V d^3r \vec{\nabla} \cdot \vec{A}
	\end{equation*}
	\begin{equation*}
	\vec{A} = \const \rightarrow \vec{\nabla} \cdot \vec{A} = 0
	\end{equation*}
	\emph{Beispiel:} Geschwindigkeit einer Flüssigkeit: $ \vec{A}(\vec{r}) = \vec{v}(\vec{r}) $\\
	
	% tikz 1 T1
	
	$$ \vec{v} = \const \quad \vec{\nabla} \cdot \vec{v} = 0 \quad \int_{\partial V} d\vec{f} \cdot \vec{v} = 0 $$
	$ \Rightarrow $ Es gibt keine Quellen von $ \vec{v} $
	
	% tikz 2 T2
	
	$$ \vec{\nabla} \cdot \vec{r} \neq 0 \quad \int_{\partial V} d\vec{f} \cdot \vec{v} \neq 0 $$
	\item
	
	% tikz 3 T3
	
	\begin{align*}
	\int_V d^3r \vec{\nabla} \cdot \vec{A}(\vec{r})  =  \int_{0}^{\Delta x} dx \int_{0}^{\Delta y} dy \int_{0}^{\Delta z} \left(\frac{\partial A_x}{\partial x} + \frac{\partial A_y}{\partial y} + \frac{\partial A_z}{\partial z}\right)  \\
	\end{align*}
	\begin{align*}
	&\int_{0}^{\Delta z} dz \int_{0}^{\Delta y} dy \ub{\int_{0}^{\Delta x} dx \frac{\partial A_x}{\partial x}}_{A_x(\Delta x , y , z) - A_x(0,y,z)}\\
	& = \int_{0}^{\Delta z} dz \int_{0}^{\Delta y} A_x(\Delta x,y,z) - \int_{0}^{\Delta z} dz \int_{0}^{\Delta y} dy A_x(0,y,z)\\
	& = \int_{F^+_A} d\vec{f} \cdot \vec{A} + \int_{F^-_A} d\vec{f} \cdot \vec{A}
	\end{align*}
	\begin{equation*}
	F_x^+: \quad d\vec{f} = \vec{e}_x dy dz \qquad F_x^-: \quad d\vec{f} = - \vec{e}_x dy dz
	\end{equation*}
	ebenso gilt dann für die anderen Koordinaten:
	\begin{align*}
	\int_{0}^{\Delta x} dx \int_{0}^{\Delta z} dz \int_{0}^{\Delta y} dy \frac{\partial A_y}{\partial y} &= \int_{F^+_y} d\vec{f} \cdot \vec{A} + \int_{F^-_y} d\vec{f} \cdot \vec{A} \\
	\int_{0}^{\Delta x} dx \int_{0}^{\Delta y} dy \int_{0}^{\Delta z} dz \frac{\partial A_z}{\partial z} &= \int_{F^+_z} d\vec{f} \cdot \vec{A} + \int_{F^-_z} d\vec{f} \cdot \vec{A} \\
	\end{align*}
\end{enumerate}

$$\Rightarrow \int_V d^3 r \nabla \cdot \vec{A} = \int_{\partial V} d \vec{f} \cdot \vec{A}$$

\paragraph{2) Stokescher Satz}
Sei $\vec{A}(\vec{r})$ ein Vektorfeld, $F$ eine Fläche mit Randkurve $\partial F$, so gilt:
$$\int\limits_{\mathclap{\textrm{Linienintegral}\rightarrow\,\partial F \phantom{\textrm{Linienintegral}\rightarrow\,}}} d\vec{r} \vec{A}(\vec{r}) = \int\limits_{\mathclap{\phantom{\, \leftarrow \textrm{Oberflächenint.}} F\, \leftarrow \textrm{Oberflächenint.}}} d \vec{f} \cdot (\nabla \times \vec{A}(\vec{r}))$$
%%%% T5
$$d\vec{f} = \vec{n} df$$
Richtung von $d\vec{f}$ und Umlauf sinn von $\partial F$: rechte Hand Regel.

\emph{Beispiel:}
\begin{equation*}
\vec{A}(\vec{r}) = \begin{pmatrix}
-y \\ x \\ 0
\end{pmatrix}
\end{equation*}

% tikz 6

\begin{equation*}
\vec{\nabla} \times \vec{A} = \begin{pmatrix}
\frac{\partial A_z}{\partial y} - \frac{\partial A_y}{\partial z} \\
\frac{\partial A_x}{\partial z} - \frac{\partial A_z}{\partial x} \\
\frac{\partial A_y}{\partial x} - \frac{\partial A_x}{\partial y}
\end{pmatrix} = \begin{pmatrix}
0 \\ 0 \\ 1 + 1
\end{pmatrix} = 2 \vec{e}_z
\end{equation*}

% tikz 7

\begin{equation*}
\vec{r}(\varphi) = R \begin{pmatrix}
\cos \varphi \\ \sin \varphi \\ 0
\end{pmatrix} \qquad \varphi \in [0,2\pi]
\end{equation*}
\begin{equation*}
\frac{\partial \vec{r}}{\partial \varphi} = R \begin{pmatrix}
- \sin \varphi \\ \cos \varphi \\ 0
\end{pmatrix}
\end{equation*}
\begin{align*}
\int_{\partial F} d \vec{r} \cdot \vec{A}(\vec{r}) &= \int_{0}^{2 \pi} d \varphi \frac{\partial \vec{r}}{\partial \varphi} \cdot \vec{A}( \vec{r} ( \varphi)) \\
&= \int_{0}^{2 \pi} d \varphi R ( + \sin^2 \varphi + \cos^2 \varphi) = 2 \pi R^2\\
\int_F d\vec{f} \ob{\left(\vec{\nabla} \times  \vec{A}\right)}^{2 \vec{e}_z} &= 2 \pi R^2
\end{align*}

Vektorfeld ohne Wirbel z.B. $\vec{A} = \const$\\ %%%% T8
$\nabla \times \vec{A} = 0$\\ %%%% T9
\bem  %%%% T10,11

\subsection{Integrale Form der Feldgleichung}
\subsection{Gaußsches Gesetz}
\begin{equation*}
\vec{\nabla} \cdot \vec{E} = \frac{1}{\epsilon_0}
\end{equation*}

% tikz mit unförmigem volumen und einem V drinnen

\begin{align*}
\int_{V} d^3 r \vec{\nabla} \cdot \vec{E}(\vec{r}) &= \frac{1}{\epsilon_0} \int_{V} d^3 r \rho (\vec{r}) = \frac{1}{\epsilon_0} Q_V\\
&= \int_V d\vec{f} \cdot \vec{E}(\vec{r})
\end{align*}

\begin{equation*}
\rmbox{\int_{\partial V} d \vec{f} \cdot \vec{E}(\vec{r}) = \frac{1}{\epsilon} Q_V}
\end{equation*}

\subsubsection{Berechnung elektrischer Felder für hochsymmetrische Ladungsverteilungen}
\emph{Beispiel:}\\
Homogen geladene Kugel mit Radius $ R $ und Gesamtladung $ Q $. Damit ist die Ladungsdichte innerhalb der Kugel:
\begin{equation*}
\rho = \frac{Q}{V} = \frac{Q}{\frac{4}{3} \pi R^3}
\end{equation*}

% tikz T12

\begin{equation*}
\vec{E}(\vec{r}) = E_r (r) \vec{e}_r
\end{equation*}
\begin{equation*}
\vec{r} = r \begin{pmatrix}
\sin \theta \cos \varphi \\ \sin \theta \sin \varphi \\ \cos \theta
\end{pmatrix}
\end{equation*}
$ \vec{e}_r = \frac{\vec{r}}{r} $

Fluss von $\vec{E}$ durch Oberfläche einer Kugel mit Radius $r$ %%%% T13
$$d \vec{f} = \vec{e}_r r^2 \sin \theta d \theta d \varphi \Rightarrow d \vec{f} \cdot \vec{E} = E_r(r)r^2\sin \theta d \theta d \varphi $$

\begin{align*}
\int_{\partial K_r (0)} d\vec{f}\ \vec{E} &= \int_0^T d \theta \ \intt{0}{2 \pi} d \varphi E_r (r) r^2 \sin \theta\\
&= E_r (r) r^2 4 \pi\\
&= \frac{1}{\epsilon_0} Q_{K_r (0)} = \frac{1}{\epsilon_0} \int_{K_r (0)} d^3 r\ \rho(\vec{r}) = \frac{1}{\epsilon_0} \left\{ \begin{array}{cc}
Q & r > R\\
Q \frac{r^3}{R^3} & r \leq R 		
\end{array}\right.
\end{align*}

$$\Rightarrow E_r(r) = \frac{Q}{4 \pi \epsilon_0}\left\{ \begin{array}{cc}
\frac{1}{r^2} & r > R\\
\frac{r}{R^3} & r \leq R 		
\end{array}\right.$$ %%%% T14 

\subsection{Satz von Stokes}
\begin{equation*}
\vec{\nabla} \times \vec{E} = 0
\end{equation*}
\textbf{Definition:} $ \gamma = \partial F $\\
$ \int_\gamma $ ist dann ein Linienintegral über eine geschlossene Kurve
\begin{equation*}
\rmbox{\int_\gamma d \vec{r} \cdot \vec{E} = \int_F d\vec{f} \cdot (\vec{\nabla} \times \vec{E}) = 0}
\end{equation*}

\subsection{Zusammenfassung: Feldgleichungen der Elektrostatik}

\textbf{differentielle Darstellung:}
\begin{equation*}
\vec{\nabla}\cdot \vec{E} = \frac{1}{\epsilon_0} \rho \quad \vec{\nabla} \times \vec{E} = 0 \quad \rightarrow \quad \vec{E} = - \vec{\nabla} \Phi \quad \rightarrow \quad \Delta \Phi = - \frac{1}{\epsilon_0} \rho
\end{equation*}
\textbf{Integral Darstellung:}
\begin{equation*}
\int_{\partial V} d\vec{f} \cdot \vec{E} = \frac{1}{\epsilon_0} Q_V \qquad , \qquad \oint_\gamma d\vec{r} \cdot \vec{E} = 0
\end{equation*}

\section{Elektrostatische Energie}
\textbf{potentielle Energie einer Punktladung im äußeren elektrischen Feld}\\
Kraft auf Ladung $ q $:
\begin{equation*}
\vec{F} = q \vec{E}
\end{equation*}
Die Arbeit bei Verschiebung der Ladung von $ \vec{a} $ nach $ \vec{b} $
\begin{align*}
W &= - \int_{\vec{a}}^{\vec{b}} d\vec{r} \cdot \vec{F} = - q \int_{\vec{a}}^{\vec{b}} d\vec{r} \cdot \vec{E}(\vec{r})\\
&= q \int_{\vec{a}}^{\vec{b}} d\vec{r} \cdot \vec{\nabla} \Phi = q \ub{( \Phi(\vec{b}) - \Phi(\vec{a})}_{\tx{Potentialdifferenz}}
\end{align*}
Die Arbeit um $ q $ aus dem unendlichen $ \infty $ nach $ \vec{r} $ zu bringen ist dann:

% tikz T16
\begin{equation*}
W = q (\Phi(\vec{r}) - \Phi(\infty))
\end{equation*}
Zur Referenz: $ \Phi(\infty) = 0 $\\
Damit ist die Energie der Ladung $ q $ im äußeren Feld:
\begin{equation*}
\Rightarrow W = q (\Phi(\vec{r}))
\end{equation*}
\begin{equation*}
\vec{E} = - \vec{\nabla} \Phi
\end{equation*}

%29.10.18

%T -1 evtl = T16 checken

\begin{tikzpicture}
\draw[thick,->] (0,0) -- (1,1) node[below,xshift=-10pt,yshift=-10pt] {$  \vec{r} $};
\draw[<-] (1.15,1.15) to[out=60,in=120] (3,1) node[right] {$ q $};
\draw[fill=black] (1.1,1.1)circle(2pt);
\draw[->] (-.5,.5) to[out=20,in=-90] (.5,1.5);
\draw[->] (.5,-.5) to[out=70,in=180] (1.5,.5) node[below,yshift=-10pt] {$ \vec{E} $};
\end{tikzpicture}

%andrez 1:
\subsection{Elektrostatische Potentielle Energie}

\paragraph{Energie einer Verteilung von Punktladungen}

$N$ Ladungen $q$: an Orten $\vec{r}_i$\\
Zunächst: $\underbrace{i - 1}_{\mathclap{\textrm{erzeugen am Ort } \vec{r}_i}}$ Ladungen $q_j$ bei $\vec{r}_j$\\
Das Potential
$$\Phi(\vec{r}_i) = \kq \summ{j = 1}{i - 1} \frac{q_i}{|\vec{r}_j - \vec{r}_i|}$$ 

%% T1:
\begin{tikzpicture}
\draw[thick,->] (0,0) -- (1,1) node[above] {$ q_1 $};
\draw[thick,->] (0,0) -- (2,1) node[right] {$ q_2 $};
\draw[thick,->] (0,0) -- (1.8,0) node[right] {$ q_3 $};
\node[above,xshift=-5pt] at (.5,.5) {$ \vec{r}_1 $};
\node[right,yshift=-5pt] at (1,.5) {$ \vec{r}_2 $};
\node[below] at (.9,0) {$ \vec{r}_3 $};
\end{tikzpicture}

Arbeit um $ i $-te Ladung aus dem unendlichen nach $ \vec{r} $ zu bringen:
\begin{equation*}
W_i = q_i \Phi(\vec{r}_i) = \frac{1}{4 \pi \epsilon_0} \sum_{i=1}^{j-1} \frac{q_i q_j}{r_{ij}}
\end{equation*}
Somit ergibt sich die gesamte Arbeit für $ N $ Ladungen als:
\begin{align*}
W = \sum_{i=2}^{N} W_i &= \frac{1}{4 \pi \epsilon_0} \sum_{i=2}^{N} \sum_{j=1}^{i-1} \frac{q_i q_j}{r_{ij}}\\
&= \frac{1}{2} \sum_{i=1}^{N} \sum_{i=1}^{N} \frac{q_i q_j}{r_{ij}}
\end{align*}

\begin{equation*}
\Rightarrow W = \frac{1}{8 \pi \epsilon_0} \sum_{\substack{i,j\\i\neq j}} \frac{q_i q_j}{r_{ij}}
\end{equation*}


% andrez 2:
\begin{align*}
W &= \frac{1}{2} \summ{i = 1}{N} q_i \underbrace{\bigg( \sum_{\substack{j\\ j \neq i}} \kq \frac{q_i}{r_{ij}}\bigg)}_{\Phi_{/ \hspace{-3.5pt}i}(\vec{r}_i)}\\
&= \frac{1}{2} \summ{i = 1}{N} q_i \Phi_{/ \hspace{-4pt}i}(\vec{r}_i)
\end{align*}

\paragraph{Energie einer kontinuierlichen lokalisierten Ladungsverteilung}

%% T2:
\begin{tikzpicture}
\draw  plot [smooth cycle, tension=.7] coordinates { (0,0) (.25,.5) (.25,1) (.75,.75) (1,0) (.5,-.25)};
\node at (.5,.35) {$ \rho $};
\end{tikzpicture}

\begin{align*}
W &= \frac{1}{8 \pi \epsilon_0} \int d^3r\ \int d^3r'\ \frac{\rho(\vec{r}) \rho(\vec{r}')}{|\vec{r}_j - \vec{r}_i|}\\
&= \frac{1}{2} \int d^3r\ \rho(\vec{r}) \kq \underbrace{\int_{\mathbb R^3} d^3r'\ \frac{\rho(\vec{r}')}{|\vec{r}_j - \vec{r}_i|}}_{\Phi(\vec{r})}
\end{align*}

%% T3:
\begin{tikzpicture}
\draw  plot [smooth cycle, tension=.7] coordinates { (0,0) (.25,.5) (.25,1) (.75,.75) (1,0) (.5,-.25)};
\node at (.5,.35) {$ \rho $};
\draw[thick] (-.5,-1) to[out=20,in=160] (1.5,-1);
\draw[thick] (-1,-.5) to[out=70,in=-70] (-1,1.5);
\draw[thick,->] (-.5,-1) to[out=20,in=180] (.5,-.8);
\draw[thick,->] (-1,-.5) to[out=70,in=-90] (-.8,.5);
\node at (-1.4,.5) {$ \vec{E}_{\tx{ext}} $};
\end{tikzpicture}

$$W_{\textrm{ext}} = \int d^3 r\ \rho(\vec{r}) \Phi_{\textrm{ext}}(\vec{r})$$


\subsubsection{Energie $ W $ durch $ \vec{E} $ ausdrücken:}

\begin{align*}
\vec\Delta \Phi &= - \frac{1}{\epsilon_0} \rho \quad \Rightarrow \quad W = -\frac{1}{2} \int d^3r \epsilon_0 \ub{\vec\Delta \Phi(\vec{r}) \Phi(\vec{r})}_{\vec{\nabla} \cdot (\Phi \vec{\nabla} \Phi) - \equalto{(\vec{\nabla} \Phi)^2}{\vec{E}} }\\
&= -\frac{\epsilon_0}{2} \ub{\int_{\mathbb{R}^3} d^3r \vec{\nabla} \cdot (\Phi \vec{\nabla} \Phi)} + \frac{\epsilon_0}{2} \int d^3r \vec{E}(\vec{r})\\[5pt]
& \qquad \qquad \lim\limits_{R\to \infty} \int_{K_R(0)} d^3r \vec{\nabla} \cdot (\Phi \vec{\nabla} \Phi) = \lim\limits_{R \to \infty} \int_{\partial K_R(0)} \ub{d\vec{f} \cdot \ub{(\Phi \vec{\nabla}\Phi)}_{\buildrel {R\to \infty} \over \sim \frac{1}{R^3}}}_{\sim \frac{1}{R}} = 0\\
&= \frac{\epsilon_0}{2} \int d^3r \vec{E}(\vec{r})
\end{align*}
Zur Umformung oben wurde benutzt:
\begin{equation*}
\Phi \buildrel R\to \infty \over \sim \frac{1}{R} \qquad \vec{\nabla} \Phi \sim \frac{1}{R^2}\qquad d\vec{f} = \vec{n} \ub{df}_{\sim R^2}
\end{equation*}
Damit ergibt sich für die Energie einer Verteilung von Punktladungen 


% andrez 3:
$$\Rightarrow \quad \rmbox{W = \frac{\epsilon_0}{2} \int d^3 r\ \vec{E}^2(\vec{r})} \qquad \textrm{nicht für Punkladungen}$$
Energiedichte des elektrostatischen Feldes
$$\rmbox{w(\vec{r}) = \frac{\epsilon_0}{2} \vec{E}^2(\vec{r})}$$


\begin{minipage}{.6\linewidth}
	\textbf{\emph{Beispiel: Plattenkondensator}}\\
	Fläc2he $F$, Ladung $\rightarrow\ r = \frac{q}{F}\ \rightarrow\ \vec{E} = \frac{r}{\epsilon_0} \vec{e}_x$\\[5pt]
	$ \rightarrow $ Die Energiedichte ist: $ w = \frac{\epsilon_0}{2} \vec{E}^2 = \frac{\sigma^2}{2 \epsilon_0} $ (nicht für Punktladungen)\\[5pt]
	$ \rightarrow $ Die Energie beträgt: $ W = \int d^3 r w(\vec{r}) = l \cdot F \cdot \frac{\sigma^2}{2 \epsilon_0} $
\end{minipage}
\begin{minipage}{0.1\linewidth}
	$ \phantom{M} $
\end{minipage}
\begin{minipage}{.3\linewidth}
	%% T4 T5:
	\begin{tikzpicture}
	\draw[thick,->] (-1,0) -- (3,0);
	\draw (-.05,0) rectangle (.05,1.5) node[above] {$ \sigma $};
	\draw (2-.05,0) rectangle (2+.05,1.5) node[above] {$ -\sigma $};
	\node[below] at (0,0) {0};
	\node[below] at (2,0) {$ l $};
	\draw[thick,->] (.5,.4) -- (1.5,.4);
	\draw[thick,->] (.5,.8) -- (1.5,.8);
	\draw[thick,->] (.5,1.2) -- (1.5,1.2);
	\node at (1,1.8) {$ \vec{E} $};
	\end{tikzpicture}
	\begin{tikzpicture}
	\draw[->] (-1,0) -- (3,0);
	\node[above] at (0,2)  {$ \phantom{M} $};
	\draw[<-] (0,2) -- (0,-.3) node[below] {0};
	\draw (2,.2) -- (2,-.2) node[below] {$ l $};
	\draw[thick] (-1,0.01) -- (0,0.01);
	\draw[thick] (2,0.01) -- (3,0.01);
	\draw (0,1.2) -- (-.3,1.2) node[left] {$ \frac{\sigma}{\epsilon_0} $};
	\draw[thick] (2,1.2) --(0,1.2);
	\draw[thick] (2,1.21) -- (0,1.21);
	\end{tikzpicture}
\end{minipage}

\subsubsection{Potentialdifferenz - Spannung}
\begin{equation*}
\Phi(\vec{r}) - \Phi(0) = - \int_{0}^{\vec{r}} d\vec{r}' \cdot \vec{E} (\vec{r}') = - \int_{0}^{x} dx' \frac{\sigma}{\epsilon_0} = - \frac{\sigma}{\epsilon} x
\end{equation*}

% T6:
% T7:
$$
\begin{tikzpicture}
\draw[->] (0,0) -- (0,1) node[anchor=south east] {$ z $};
\draw[->] (0,0) -- (.7,.7) node[anchor=south] {$ y $};
\draw[->] (0,0) -- (1,0) node[anchor=north west] {$ x $};
\draw[dashed] (.7,.7) -- (.7+1,.7);
\draw[dashed] (1,0) -- (1+.7,.7);
\draw[thick,->] (0,0) -- (1.8,1.2) node[right,yshift=-5pt] {$ \vec{r} $};
\end{tikzpicture}
\qquad \qquad 
\begin{tikzpicture}
\draw[->] (-.5,0) -- (1.5,0) node[anchor=north west] {$ x $};
\draw[->] (0,-1) -- (0,1) node[anchor=south east] {$ \Phi $};
\draw (0,0) -- (1,-1);
\draw[dashed] (1,-1) -- (1,.2) node[anchor=north west,yshift=-5pt] {$ l $};
\end{tikzpicture}
$$

\noindent
Die Spannung zwischen zwei Kondensatorplatten ist dann:
\begin{equation*}
U = \Phi(0) - \Phi(l) = \frac{\sigma}{\epsilon_0} l = \frac{q}{\epsilon_0 F} l
\end{equation*}
Die Kapazität ist also:
\begin{equation*}
C = \frac{q}{U} = \frac{\epsilon_0 F}{l}
\end{equation*}

% andrez 4:
\noindent
\lcom{Was ist die Energie bei einer Verteilung von Punktladungen und bei einer kontinuierlichen Ladungsverteilung. Bei einer kontinuierlichen Ladungsverteilung haben wir herausgefunden:}
$$W = \frac{1}{8 \pi \epsilon_0} \sum_{\substack{i,j\\ i \neq j}} \frac{q_i q_j}{r_{ij}} \qquad \textrm{ für Punktladungen}$$
\lcom{Die Energie der Punktladung selbst steckt hier nicht drinnen. Man muss dabei aufpassen, welche Gleichung man für welches Modell benutzt.}
$$\vec{E} = \kq q \frac{\vec{r}}{r^3} \qquad \int d^3 r\ \vec{E}^2 = \int d^3 r\ \frac{1}{r^4} = \infty$$

\section{Verhalten des el. Feldes an Grenzflächen mit Flächenladung}
$ \rightarrow $ Diskontinuitäten von $ \vec{E} $\\
\emph{Beispiel:} Wir betrachten eine homogene Flächenladung.\\[10pt]
\begin{minipage}{.4\linewidth}
	%%% dat tafel dude
	% T 8 - 11 auf blatt:
	\begin{tikzpicture}
	\draw (-1,-1) to[out=45,in=180] (1.2,0);
	\draw (-1.2,-.4) to[out=45,in=180] (1,.6);
	\draw (-.8,-1.6) to[out=45,in=180] (1.4,-.6);
	\draw (0,1) to[out=-30,in=90] (1,-1);
	\draw (-.7,.7) to[out=-30,in=90] (.3,-1.3);
	\draw (-1.4,.2) to[out=-30,in=90] (-.4,-1.6);
	\draw[thick,<-] (.5,.3) to[out=60,in=180] (1.5,1) node[right] {$ \sigma(\vec{r}) = \frac{\tx{Ladung}}{\tx{Fläche}} $};
	\end{tikzpicture}
\end{minipage}
\begin{minipage}{.3\linewidth}
	\begin{tikzpicture}
	\draw[thick,->] (0,0) -- (0,.75) node[anchor=south east] {$ z $};
	\draw[thick,->] (0,0) -- (1,0) node[anchor=north west] {$ y $};
	\draw[thick,->] (0,0) -- (-.75,-.75) node[anchor=south east] {$ x $};
	\draw (-.2,-.2) -- (-.2+1,-.2);
	\draw (-.4,-.4) -- (-.4+1,-.4);
	\draw (-.6,-.6) -- (-.6+1,-.6);
	\node at (.5,-.75) {$ \sigma $};
	\end{tikzpicture}
\end{minipage}\nolinebreak
\begin{minipage}{.3\linewidth}
	\begin{tikzpicture}
	\draw[thick,->] (0,-1.2) -- (0,1.2) node[above] {$ E_z $};
	\draw[thick,->] (-2,0) -- (2,0) node[anchor=north west] {$ z $};
	\draw[thick] (2,.8) -- (0,.8) ;
	\draw (0,.8) -- (-.15,.8) node[left] {$ \frac{\sigma}{2 \epsilon_0} $};
	\draw[thick] (-2,-.8) -- (0,-.8);
	\draw (0,-.8) -- (.15,-.8) node[right] {$\frac{\sigma}{2 \epsilon_0}$};
	\end{tikzpicture}
\end{minipage}
\begin{equation*}
\Rightarrow \vec{E} = \frac{\sigma}{2 \epsilon_0} \tx{sgn}(z) \vec{e}_z
\end{equation*}
\begin{align*}
\vec{E}_\perp &= \pm \frac{\sigma}{2 \epsilon_0} \vec{e}_z\\
\vec{E}_\parallel &= 0
\end{align*}

\begin{minipage}{.6\linewidth}
	Das elektrische Feld $ \vec{E}_\parallel $ ist gleich der Ableitung des elektrischen Potentials:\\
	Das elektrische Potential ist also stetig.
\end{minipage}
\begin{minipage}{.4\linewidth}
	\hspace{30pt}
	\begin{tikzpicture}
	\draw[thick,->] (0,-.7) -- (0,.7) node[anchor=north west] {$ z $};
	\draw[thick,->] (-1,0) -- (1.2,0) node[anchor=south east] {$ \Phi(z) $};
	\draw (-1,-.7) -- (0,0) -- (1,-.7);
	\end{tikzpicture}
\end{minipage}





%andrez 5:
\paragraph{Normalkomponente $\vec{E}_\perp$}

%% T12

Gaußscher Satz für $V$:
\begin{align*}
\int_V d^3 r'\ \vabla \cdot \vec{E}(\vec{r}) &= \int_{\partial V} d\vec{f}'\ \vec{E}(\vec{r})\\
&=\custo{\rotatebox{90}{$\scriptscriptstyle{\Delta z \to 0}$}\atop \longrightarrow}{\int_{\textrm{Mantel}} d\vec{f}'\ \vec{E}}{\hspace{-23.5pt}0} + \custo{\rotatebox{90}{$\scriptscriptstyle{\Delta z \to 0}$}\atop \longrightarrow}{\int_{\partial V_+} d\vec{f}'\ \vec{E}(\vec{r})}{\int_F df'\ \vec{n} \cdot \vec{E}_+} + \custo{\rotatebox{90}{$\scriptscriptstyle{\Delta z \to 0}$}\atop \longrightarrow}{\int_{\partial V_-} d\vec{f}'\ \vec{E}}{- \int_F df'\ \vec{n} \cdot \vec{E}_-}
\end{align*}
$ \vec{E}_{\pm} $ ist das Feld auf beiden Seiten der Grenzfläche

% eaualto zwischen Zeile 1 und 2 der equations ...
\begin{equation*}
\int_{\partial V} d\vec{f}' \vec{E} \buildrel \Delta z \to 0 \over \longrightarrow \int_{F} df \vec{n} \cdot (\vec{E}_+ - \vec{E}_-) \buildrel F \to 0 \over \longrightarrow F\  \vec{n} \cdot \big(\vec{E}_+(\vec{r}) - \equalto{\vec{E}_-(\vec{r})}{}\big)
\end{equation*}
\begin{equation*}
\int_V d^2r' \equalto{\vec{\nabla} \cdot \vec{E}(\vec{r}')}{\frac{1}{\epsilon_0} \rho(\vec{r}')} = \frac{1}{\epsilon_0} \int_V d^3r' \rho(\vec{r}) = \frac{1}{\epsilon_0} \int_F df'  \sigma(\vec{r}') \buildrel F \to 0 \over \longrightarrow \frac{1}{\epsilon_0} F \sigma(\vec{r})
\end{equation*}
\begin{equation*}
\Rightarrow \vec{n} \cdot \left(\vec{E}_+(\vec{r}) - \vec{E}_-(\vec{r}) \right) = \frac{1}{\epsilon_0} \sigma(\vec{r})
\end{equation*}
\begin{equation*}
E_{\perp_\pm} = \vec{n} \cdot \vec{E}_{\pm} \qquad E_{\perp_+}(\vec{r}) - E_{\perp_-} (\vec{r}) = \frac{1}{\epsilon_0} \sigma(\vec{r})
\end{equation*}



%T 13 und Tafel vor M4 


\begin{equation*}
0 = \oint_\gamma d\vec{r}' \cdot \vec{E} \buildrel \Delta z \to 0 \over \longrightarrow \int_{-\frac{L}{2}}^{-\frac{L}{2}} ds \vec{t} \cdot \left(\vec{E}_+ - \vec{E}_-\right) \buildrel L \to 0 \over \longrightarrow L \vec{t} \cdot \left(\vec{E}_+(\vec{r}) - \vec{E}_-(\vec{r})\right) = 0
\end{equation*}
\begin{equation*}
\rightarrow \vec{t} \cdot (\vec{E}_+ (\vec{r}) - \vec{E}_- (\vec{r}) = 0
\end{equation*}
$ \rightarrow $ Die Tangentialkomponente ist stetig
\begin{equation*}
E_{\parallel_{+}} = E_{\parallel_{-}}
\end{equation*}
Insgesamt ergibt sich damit:
\begin{equation*}
\vec{E}_+(\vec{r}) - \vec{E}_-(\vec{r}) = \frac{\sigma}{\epsilon_0} \vec{n}
\end{equation*}
Das elektrische Potential $ \Phi $ ist damit stetig.
\begin{equation*}
\ub{\Phi(\vec{r}_b) - \Phi(\vec{r}_a)}_{\Phi_+(\vec{r}) - \Phi_-(\vec{r})} = \int_{\vec{r}_a}^{\vec{r}_b} d\vec{r}' \cdot \vec{E} \quad \buildrel \Delta z \to 0 \over \longrightarrow 0
\end{equation*}

% T14 

% andrez 6 oder 5:
\paragraph{Randbedingungen an el. Leitern}

Leiter: Material mit freibeweglichen Ladungsträgern (Metall)\\
%% T15
Eigenschaften von $\vec{E}$ im Leiter:
\begin{enumerate}[i)]
	\item $\vec{E} = 0$
	\item $0 = \vabla \cdot \vec{E} = \frac{1}{\epsilon_0} \rho, \qquad \rho(\vec{r}) = 0$
	\item Nettoladung befinden sich an Oberfläche %% T16
	\item Potential $\Phi(\vec{r}_b) - \Phi(\vec{r}_a) = 0 \quad \rightarrow \quad \Phi(\vec{r}) = \const$ %% T17
\end{enumerate}

\paragraph{Randbedingungen}
%% T18
$$\vec{E}_+ - \vec{E}_- = \frac{\sigma^-}{\epsilon_0} \vec{n}$$
$$\vec{E}_- = 0$$
$$\rightarrow \vec{E}_+(\vec{r}) = \frac{\sigma(\vec{r})}{\epsilon_0} \vec{n} {(\vec{r})}$$
\folie{Ladung an Oberfläche eines Leiters}







% für tikz13 
\begin{equation*}
\int\limits_{\begin{tikzpicture}[scale=.5]
	\draw (0,0) -- (1,0);
	\draw (1,0) -- (1,1);
	\draw (1,1) -- (0,1);
	\draw (0,1) -- (0,0);
	\end{tikzpicture}}
\end{equation*}

% 5.11.18

\section{Randwertprobleme (RWP) der Elektrostatik und \\ Lösungsmethoden}

\subsection{Formulierung des Randwertproblems}

Das elektrische Potential: $ \Phi(\vec{r}): \quad \vec{E}(\vec{r}) = - \vec{\nabla} \Phi(\vec{r}) $\\
$$ \vec{\Delta} \Phi (\vec{r}) = -\frac{1}{\epsilon_0} \rho(\vec{r})  \qquad \tx{Poisson-Gleichung}$$
Für eine gegebene lokale Ladungsverteilung $ \rho $ gilt:
\begin{equation*}
\Phi(\vec{r}) = \frac{1}{4 \pi \epsilon_0} \int d^3 r' \frac{\rho(\vec{r}')}{|\vec{r} - \vec{r}'|}
\end{equation*}
\begin{equation*}
\rightarrow \Phi(\vec{r}) \buildrel |\vec{r}| \rightarrow 0 \over \longrightarrow 0
\end{equation*}
Typische Problemstellung:\\
Ladungsverteilung $\rho$ + Werte des Potentials auf Randfläche

\bei\\ % T1

Randwertproblem: Gegeben: $\rho(\vec{r}')$ im Raumbereich $V$\\
$\Phi(\vec{r})$ oder $\vec{E}(\vec{r})$ auf Randfläche $\partial V$

Gesucht: $\Phi(\vec{r}),\ \vec{E}(\vec{r})$ überall in $V$

\noindent
Zwei Fälle:
\begin{enumerate}[i)]
	\item $ \Phi(\vec{r}) $ ist auf der Randfläche gegeben\\
	$ \rightarrow $ \textbf{Dirichlet-Randbedingung}
	\item $ \vec{E}(\vec{r}) $ ist auf der Randfläche gegeben\\
	$ \rightarrow $ \textbf{Neumannsche Randbedingung}
\end{enumerate}

%T2
\lcom{Wir beschränken uns vorwiegend auf den ersten Fall. }
\lcom{Zur Lösung dieser Probleme gibt es einige Methoden. Zum Einstieg und zur Wiederholung betrachten wir zunächst die Methode der Spiegelladung. }

\subsection{Methode der Bildladung (Spiegelladung)}

\subsubsection{Punktladung vor leitender, geerdeter Metallplatte}

%T3 

\begin{equation*}
\vec{\Delta} \Phi(\vec{r}) = - \frac{1}{\epsilon_0} \rho (\vec{r}) = - \frac{q}{\epsilon_0} \delta(\vec{r} - \vec{r}_0)
\end{equation*}
$ \vec{r} \in V  \qquad \vec{r}_0 = (d,0,0) \qquad V = \{\vec{r} \in \mathbb{R}^3 , x > 0\}$\\
Randbedingungen:
\begin{equation*}
\pofr = 0 \quad \tx{für} \quad \vec{r} \in \partial V ,\quad \quad \tx{d.h.} \ \vec{r} = (0,y,z)
\end{equation*}
\textbf{Idee:} Ersetze ursprüngliche Problem durch ''Fiktives'' Problem mit zusätzlichen Ladungen außerhalb von $V$, welche die Randbedingungen simulieren.\\[5pt]
Potential der Punkladungen in $\vec{r}_0$:
$$\Phi_q (\vec{r}) = \kq \frac{q}{|\vec{r} - \vec{r}_0|}$$ 
addiere Ladung $-q$ in $\vec{r}'_0 = (-d, 0, 0) = -\vec{r}_0$\\[5pt]

\begin{equation*}
\pofr = \kq \left( \frac{q}{|\vec{r} - \vec{r}_0|} - \frac{q}{|\vec{r}+\vec{r}_0|} \right)
\end{equation*}
Schauen wir nun nach ob dies die Poisson-GLeichung erfüllt:
\begin{align*}
\vec{\Delta} \Phi &= \frac{q}{4 \pi \epsilon_0} \Bigg( \ub{\Delta \frac{1}{|\vec{r} - \vec{r}_0|}}_{= - 4 \pi \delta(\vec{r} - \vec{r}_0)} - \ub{- \Delta \frac{1}{|\vec{r} + \vec{r}_0|}}_{= - 4 \pi \delta(\vec{r} + \vec{r}_0)}\Bigg)\\
&= \underset{\checkmark}{- \frac{q}{\epsilon_0} \delta(\vec{r} - \vec{r}_0)} + \frac{q}{\epsilon_0} \ub{\delta(\vec{r} + \vec{r}_0)}_{\mathclap{\qquad = 0 \ \tx{für} \ \vec{r} \neq - \vec{r}_0 \ \checkmark \quad \forall \vec{r} \in V}}
\end{align*}

\subsubsection{Diskussion der Lösung}
\begin{enumerate}[i)]
	\item \textbf{Struktur}
	$$\pofr = \underbrace{\kq \frac{1}{|\vec{r}-\vec{r}_0|}}_{\eqdef\ \Phi_{\textrm{s}}(\vec{r})} + \underbrace{\frac{(-q)}{4 \pi \epsilon_0} \frac{1}{|\vec{r}-\vec{r}_0|}}_{\eqdef\ \Phi_{\textrm{hom}}(\vec{r})}$$ 
	$ \vec{r} \in V $
	\begin{equation*}
	\Delta \Phi_\tx{s}(\vec{r}) = - \frac{1}{\epsilon_0} \rho(\vec{r}) \quad \tx{Poisson-Gleichung}
	\end{equation*}
	\begin{equation*}
	\Delta \Phi_{\tx{hom}} (\vec{r}) = 0 \qquad \tx{Laplace-Gleichung}
	\end{equation*}
	Mathematisch: Lösung inhomogener DGL
	\begin{equation*}
	\pofr = \Phi_\tx{s}(\vec{r}) + \Phi_{\tx{hom}}(\vec{r})
	\end{equation*}
	$ \Phi_{\tx{hom}} $ wird so gewählt, dass die Randbedingungen erfüllt werden:
	\begin{equation*}
	\vec{r} \in \partial V: \quad \Phi_\tx{o} (\vec{r}) = \Phi_\tx{s}(\vec{r}) + \Phi_{\tx{hom}}(\vec{r})
	\end{equation*}
	\item \textbf{Elektrisches Feld}
	\begin{equation*}
	\vec{E} = - \vec{\nabla} \Phi = \frac{q}{4 \pi \epsilon_0} \left(\frac{(x-d,y,z)}{|\vec{r} - \vec{r}_0|^3} - \frac{(x+d,y,z)}{|\vec{r} + \vec{r}_0|^3}\right)
	\end{equation*}
	An der Oberfläche $ x \to 0 $, $ x \ge 0 $\\
	$ |\vec{r} \pm \vec{r}_0|^3 \rightarrow (d^2 + y^2 + z^2) $\\
	\begin{equation*}
	\eofr \bigg|_{\vec{r} \in \partial V} = - \frac{qd}{2 \pi \epsilon_0} \frac{1}{(d^2 + y^2 + z^2) ^{3/2}} \vec{e}_x
	\end{equation*}
	\lcom{Durch das externe elektrische Feld verschieben sich die Ladungsträger im Metall und es entsteht eine Influenzladung an der Oberfläche.}
	\item \textbf{Influenzladung auf Metalloberfläche}
	$$\vec{E}_+ - \equalto{\vec{E}_-}{0} = \frac{\sigma}{\epsilon_0} \vec{n} \qquad \vec{n} = \vec{e}_x$$
	$\vec{r} \in \partial V$:
	$$\sigma (\vec{r}) = \epsilon_0 \vec{E}_+(\vec{r}) = - \frac{qd}{2 \pi(d^2 + y^2 + z^2)^{\nicefrac{3}{2}}}$$
	gesamte influenzierte Ladung
	$$q_i = \int_{\partial V} df\ \sigma(\vec{r}) = \dots = -q$$
	\item \textbf{Kraft zwischen Punktladungen und Metallplatte}
	$$\vec{F} = q \vec{\tilde{E}}(\vec{r}_0) = \frac{-q^2}{4 \pi \epsilon_0 (2d)^2}\vec{e}_x$$
\end{enumerate}

% andrez 5:
\subsubsection{Eindeutigkeit der Lösung des Randwertproblems}
Dirichlet-Randwertproblem: 
\begin{align*}
\Delta \pofr = -\frac{1}{\epsilon_0} \rho(\vec{r}) &\qquad \vec{r} \in V\\
\pofr = \Phi_0 (\vec{r}) \quad \; &\qquad \vec{r} \in \partial V \quad \;
\end{align*}
Annahme: $\Phi_1,\ \Phi_2$ lösen RWP
\begin{align*}
\textrm{d.h. }\qquad \Delta\Phi_1(\vec{r}) = \frac{1}{\epsilon_0} \rho(\vec{r}) = \Delta \Phi_2 (\vec{r}) &\qquad \vec{r} \in V\\
\Phi_1 (\vec{r}) = \Phi_0(\vec{r}) = \Phi_2(\vec{r}) &\qquad \vec{r} \in \partial V
\end{align*}
Setze :
\begin{equation*}
\psi(\vec{r}) \defeq \Phi_1(\vec{r}) - \Phi_2(\vec{r})
\end{equation*}
\begin{equation*}
\Delta \Phi(\vec{r}) = 0 \quad \vec{r} \in V
\end{equation*}
\begin{equation*}
\vec{r} \in \partial V \quad \psi(\vec{r}) = \Phi_1(\vec{r}) - \Phi_2(\vec{r}) = 0
\end{equation*}

\subsubsection{Greensche Identität:}
$ g,h $ Funktionen an V:
\begin{align*}
&\ \ \ \, \int_V d^3 r \left[\left(\vec{\nabla} (\vec{r})\right) \cdot \left(\vec{\nabla} h(\vec{r}))\right) + g(\vec{r}) \Delta h(\vec{r})\right]\\
&= \int_{\partial V} d\vec{f} \cdot \left(g(\vec{r}) \vec{\nabla} h(\vec{r}\right)\\
&= \int_{\partial V} d\vec{f} g(\vec{r}) \ub{\vec{n} \cdot \vec{\nabla} h(\vec{r})}_{= \frac{\partial h}{\partial n} (\vec{r})}
\end{align*}

% andrez 6 % lost board von jan holen:
$h = g = \psi$
$$\Rightarrow \quad \int_V d^3 r\ ((\vabla \psi)^2 + \psi(\vec{r}) \underbrace{\Delta \psi(\vec{r})}_{=0}) = \int_{\partial V} df\ \underbrace{\psi(\vec{r})}_{=0} \frac{\partial \psi(\vec{r})}{\partial n} $$
\begin{equation*}
\Rightarrow \int_V d^3 r \left(\vec{\nabla} \psi(\vec{r}) \right)^2 = 0 \Rightarrow \vec{\nabla} \psi(\vec{r}) = 0 \qquad \vec{r} \in V
\end{equation*}
\begin{equation*}
\psi(\vec{r}) = \const \qquad \psi(\vec{r}) = 0 \ \tx{in} \ V \ \ \Rightarrow \ \ \Phi_1(\vec{r}) = \Phi_2(\vec{r})
\end{equation*}


\subsection{Formale Lösungen des elektrostatischen Randwertproblems mit \\ Greenschen Funktionen}
GF: generelle Methode um inhomogene DGL zu lösen
\begin{equation*}
\Delta \pofr = - \frac{1}{\epsilon_0} \rho(\vec{r})
\end{equation*}
Greensche Funktionen der Poisson-Gleichung: $ \gre(\vec{r},\vec{r}') $ mit
\frbox{Greensche Funktionen der Poisson-Gleichung}{\begin{equation*}
\Delta_{\vec{r}} \gre(\vec{r},\vec{r}') = - \frac{1}{\epsilon_0} \delta(\vec{r} - \vec{r}')
\end{equation*}}
\lcom{Diese Gleichung geht vor einer Punktladung mit $ q=1 $ aus, ist hier aber zunächst einmal eine Definition.}

% andrez 7:
$\mathcal G$ bekannt 
$$\rightarrow \pofr = \int d^3 r'\ \mathcal G(\vec{r}, \vec{r}') \rho(\vec{r}')$$
$$\Delta_{\vec{r}} \pofr = \int d^3 r'\ \underbrace{(\Delta_{\vec{r}} \mathcal G(\vec{r}, \vec{r}'))}_{= \frac{1}{\epsilon_0 \delta(\vec{r}-\vec{r}')}} \rho (\vec{r}) = - \frac{1}{\epsilon_0} \rho (\vec{r})\ \checkmark$$

\begin{equation*}
\Delta_{\vec{r}} \frac{1}{|\vec{r} - \vec{r}'|} = 4 \pi \delta(\vec{r} - \vec{r}')
\end{equation*}
\begin{equation*}
\gre(\vec{r},\vec{r}') = \frac{1}{4 \pi \epsilon_0} \frac{1}{|\vec{r} - \vec{r}'|}
\end{equation*}
\begin{equation*}
\rightarrow \Delta_{\vec{r}} \gre(\vec{r}, \vec{r}') = - \frac{1}{\epsilon_0} \delta(\vec{r} - \vec{r}')
\end{equation*}
\begin{equation*}
\pofr = \frac{1}{4 \pi \epsilon_0} \int d^3 r \frac{\rho(\vec{r}')}{|\vec{r} - \vec{r}'|}
\end{equation*}
\begin{equation*}
\pofr \buildrel |\vec{r}| \to \infty \over \longrightarrow 0
\end{equation*}


% andrez 8:
$$\rightarrow\ \Delta_{\vec{r}} \mathcal G(\vec{r}, \vec{r}') = - \frac{1}{\epsilon_0} \delta(\vec{r} - \vec{r}')$$
$$\mathcal G(\vec{r}, \vec{r}') \underset{|\vec{r}| \to \infty} \longrightarrow 0 $$

\subsubsection{Dirichlet-Randwertproblem}

$$\Delta \pofr = -\frac{1}{\epsilon_0} \rho(\vec{r}) \quad \vec{r} \in V$$
$$\pofr = \Phi_0(\vec{r}) \quad \vec{r} \in \partial V$$
GF: $$\Delta_{\vec{r}} \mathcal G(\vec{r}, \vec{r}') = \ \frac{1}{\epsilon_0} \delta(\vec{r} - \vec{r}') \quad \vec{r},\vec{r}' \in V$$
$$\mathcal G(\vec{r}, \vec{r}') = 0 \textrm{ für } \substack{\vec{r} \in \partial V \\ \vec{r}' \in V}$$
\lcom{Hiermit haben wir das Grenzwertproblem auf eine Integration zurückgeführt. Dies werden wir nun Beweisen:}\\[5pt]
Die 2. Greensche Identität lautet: 
\begin{align*}
&\ \ \ \, \int_V d^3 r' \left( g(\vec{r}') \Delta_{\vec{r}'} h(\vec{r}')  - h(\vec{r}') \Delta_{\vec{r}'} g(\vec{r}') \right)\\
&= \int_{\partial V} d\vec{f}' \cdot \left(g(\vec{r}') \vec{\nabla}_{\vec{r}'} h(\vec{r}') - h(\vec{r}') \vec{\nabla} _{\vec{r}'} g(\vec{r}') \right)
\end{align*}
\begin{equation*}
g(\vec{r}') \defeq \Phi(\vec{r}') \qquad h(\vec{r}') \defeq \gre(\vec{r}',\vec{r})
\end{equation*}
\begin{align*}
\Rightarrow & \ \ \ \, \int_V d^3 r' \Bigg[ \Phi(\vec{r}') \ub{\Delta_{\vec{r}'} \gre(\vec{r}',\vec{r})}_{= -\frac{1}{\epsilon_0} \delta(\vec{r}' - \vec{r})} - \gre(\vec{r}',\vec{r}) \ub{\Delta_{\vec{r}'} \Phi(\vec{r}')}_{= - \frac{1}{\epsilon_0} \rho(\vec{r}')} \Bigg]\\
&= \int_{\partial V} d\vec{f}' \Bigg[ \ub{\Phi(\vec{r}')}_{= \Phi_0(\vec{r}')} \vec{\nabla}_{\vec{r}'} \gre(\vec{r}',\vec{r}) - \ub{\gre(\vec{r}',\vec{r})}_{= 0} \vec{\nabla}_{\vec{r}'} \Phi(\vec{r}') \Bigg]\\ % check if last phi(r') is right % andrez 9:
\Rightarrow &= - \frac{1}{\epsilon_0} \pofr + \frac{1}{\epsilon_0} \int_V d^3 r'\ \mathcal G(\vec{r}, \vec{r}') \rho(\vec{r})\\
&= \int_{\partial V} d \vec{f}'\ \Phi_0 (\vec{r}') \vabla_{\vec{r}'} \mathcal G(\vec{r}, \vec{r}')\\
&= \int_{\partial V} d \vec{f}'\ \Phi_0 (\vec{r}') \prt{\mathcal G}{n'}(\vec{r}, \vec{r}')
\end{align*}
\begin{equation*}
\Rightarrow \pofr = \int_V d^3r'\ \mathcal G(\vec{r}, \vec{r}') \rho(\vec{r}') - \epsilon_0 \int_{\partial V} df'\ \Phi_0 (\vec{r}') \prt{\mathcal G}{n'}(\vec{r}, \vec{r}')
\end{equation*}

Es gilt (HA):
\begin{equation*}
\gre(\vec{r},\vec{r}') = \gre(\vec{r}',\vec{r}) \qquad \tx{Reziprozität}
\end{equation*}
\begin{equation*}
\rightarrow \vec{\nabla}_{\vec{r}'} \gre(\vec{r},\vec{r}') = \vec{\nabla}_{\vec{r}'} \gre(\vec{r}',\vec{r})
\end{equation*}
\begin{equation*}
\Delta_{\vec{r}} \gre(\vec{r},\vec{r}') = \Delta_{\vec{r}'} \gre(\vec{r},\vec{r}')
\end{equation*}

\rbox{\begin{align*}
\Phi(\vec{r}) &= \int_V d^3 r' \gre(\vec{r},\vec{r}') \rho(\vec{r}')\\
&\ \ \ \, - \epsilon_0 \int_{\partial V} df' \Phi_0(\vec{r}') \frac{\partial \gre}{\partial n'} (\vec{r},\vec{r}')
\end{align*}}

% bild





\end{document}
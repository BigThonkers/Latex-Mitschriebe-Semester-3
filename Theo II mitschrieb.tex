\PassOptionsToPackage{dvipsnames}{xcolor}
\documentclass[titlepage,11pt,a4paper,ngerman]{report}
\usepackage[utf8]{inputenc}
\usepackage[T1]{fontenc}
\usepackage[german]{babel}
\usepackage{graphicx}
\usepackage{wrapfig}
\usepackage{amsmath}
\usepackage{amsfonts}
\usepackage{amssymb}
\usepackage{tikz}
\usepackage{tikz-cd}
\usepackage{nicefrac}
\usepackage{mathtools}
\usepackage{enumerate}
\usepackage{cancel}
\usepackage[hidelinks]{hyperref}
\usepackage{cleveref}
\usepackage{tcolorbox}
\usepackage{bm}
\usepackage[shortlabels]{enumitem}

\usepackage[margin=1in]{geometry}

\usepackage{placeins}
\usepackage{booktabs}
\usepackage{wasysym}

\usepackage{url}

\usetikzlibrary{calc}
\usetikzlibrary{decorations.pathmorphing,patterns}
\usetikzlibrary{arrows}
\usetikzlibrary{decorations.pathreplacing}


%Environments und Newcommands:

% Allgemein:

% zu zeigen symbol
\newcommand{\zz}{\fontfamily{cmss} \selectfont{Z\kern-.61em\raise-0.7ex\hbox{Z}:}}
% build over
\newcommand{\bov}[2]{\buildrel{#2} \over{#1}}
% better looking := (defined as)
\newcommand*{\defeq}{\mathrel{\vcenter{\baselineskip0.5ex \lineskiplimit0pt \hbox{\scriptsize.}\hbox{\scriptsize.}}}=}
\newcommand*{\eqdef}{=\mathrel{\vcenter{\baselineskip0.5ex \lineskiplimit0pt \hbox{\scriptsize.}\hbox{\scriptsize.}}}}
% integral differential d
\newcommand{\dif}{\mathop{}\!\mathrm{d}}
\newcommand{\difi}[1]{\mathrm{d}#1\mathop{}\!}

\newcommand{\hfw}{\color{RubineRed}\tx{ $\star$hier fehlt was$\star$ } \color{black}}
\def\checkmark{\tikz\fill[scale=0.4](0,.35) -- (.25,0) -- (1,.7) -- (.25,.15) -- cycle;} 

%Text:
\newcommand{\tx}[1]{\textrm{#1}}
\newcommand{\const}{\tx{const.}}

\newcommand{\ul}[1]{\underline{#1}}
\newcommand{\ol}[1]{\overline{#1}}
\newcommand{\ub}[1]{\underbrace{#1}}
\newcommand{\ob}[1]{\overbrace{#1}}

\newcommand{\dd}{\tx{d}}

% arrow list
\newlist{arrowlist}{itemize}{1}
\setlist[arrowlist]{label=$\Rightarrow$}

%Mathe:
\newcommand{\verteq}{\rotatebox{90}{$\,=$}}
\newcommand{\equalto}[2]{\underset{\scriptstyle\overset{\mkern4mu\verteq}{#2}}{#1}}
\newcommand{\equaltoup}[2]{\overset{\scriptstyle\underset{\mkern4mu\verteq}{#2}}{#1}}
\newcommand{\custo}[3]{\underset{\scriptstyle\overset{\mkern4mu\rotatebox{-90}{$\,#1$}}{#3}}{#2}}
\newcommand{\custoup}[3]{\overset{\scriptstyle\underset{\mkern4mu\rotatebox{-90}{$\hspace{-3pt} #1$}}{#3}}{#2}}
\newcommand{\casess}[4]{\left\{ \begin{array}{ll} {#1} & {#2} \\ {#3} & {#4} \end{array} \right.}


%Spezielles:

%Theo:
\newcommand{\lag}{\mathcal{L}}
\newcommand{\ham}{\mathcal{H}}
\newcommand{\gre}{\mathcal{G}}
\newcommand{\prt}[2]{\frac{\partial #1}{\partial #2}}
\newcommand{\prd}[2]{\frac{\tx{d} #1}{\tx{d} #2}}
\newcommand{\eofr}{\vec{E}(\vec{r})}
\newcommand{\pofr}{\Phi(\vec{r})}
\renewcommand{\Phi}{\varPhi}
\newcommand{\grr}{\mathcal G(\vec{r},\vec{r}')}

%LA:
\newenvironment{bew}[1]{\subsection{Bew: #1}}{\hfill$\square$}
\newcommand{\Bew}[2]{\begin{bew}{#1}#2\end{bew}}

\newcommand{\enph}{F: V \to V \textrm{ Endomorphismus}}
\newcommand{\im}{\tx{im}}
\newcommand{\spa}{\tx{span}}
\newcommand{\adj}{\tx{adj}}
\newcommand{\grad}{\tx{grad}}
\newcommand{\ord}{\tx{ord}}

\newcommand{\basis}[3]{\{#1_{#2}, \dots, #1_{#3}\}}
\newcommand{\ska}[2]{\langle #1 , #2 \rangle}
\newcommand{\dmat}[3]{\begin{pmatrix} #1_{#2}&&\\ &\ddots& \\ && #1_{#3} \end{pmatrix}}

%Ex:
\newcommand{\kq}{\frac{1}{4\pi\epsilon_0}}
\newcommand{\uind}{U_{\tx{ind}}}
\newcommand{\folie}[1]{\color{gray}[Folie: #1]\color{black}}
\newcommand{\versuch}[1]{\color{red!50!black} Versuch: \color{black} #1\\ }

% ANDREZ
\newcommand{\summ}[2]{\sum_{#1}^{#2}}
\newcommand{\intt}[2]{\int_{#1}^{#2}}
\renewcommand{\vec}[1]{\bm{#1}}
\newcommand{\lcom}[1]{\color{MidnightBlue}#1\color{black}}
\renewcommand{\epsilon}{\varepsilon}
\newcommand{\vabla}{\bm{\nabla}}
\newcommand{\bei}{\emph{Beispiel: }}
\newcommand{\bem}{\emph{Bemerkung:}}
\renewcommand{\paragraph}[1]{\subsubsection{#1}}

\newcommand{\mau}{$\buildrel \mathcal{O} \over{\textbf{.}}$}

% Boxen:

\tcbuselibrary{theorems}

% mahlt eine box nur um den text mit tittel
\newtcbox{\fribox}[1]{nobeforeafter,colback=white,colframe=red!75!black,fonttitle=\bfseries,title=#1,sharp corners,tcbox raise base}

% mahlt eine große box um alles mit tittel
\newcommand{\frbox}[2]{\begin{tcolorbox}[colback=white,colframe=red!75!black,fonttitle=\bfseries,title=#1]#2\end{tcolorbox}}

% mahlt eine box nur um den text
\newtcbox{\ribox}{nobeforeafter,colback=white,colframe=red!75!black,sharp corners,tcbox raise base}

% mahlt eine große box um alles was drinnen ist
\newcommand{\rbox}[1]{\begin{tcolorbox}[colback=white,colframe=red!75!black]#1\end{tcolorbox}}

% mahlt eine box um mathe innerhalb mathmode
\newcommand{\rmbox}[1]{\tcboxmath[colback=white,colframe=red!75!black]{#1}}

% super box (looks like regular boxed but wraps around anything)
\newenvironment{supbox}{\begin{tcolorbox}[colback=white,colframe=black,sharp corners,boxrule=.5pt]}{\end{tcolorbox}}

\newcommand{\bbb}[2]{\begin{tcolorbox}[colback=white,colframe=black,fonttitle=\bfseries,title=#1,sharp corners,tcbox raise base]#2\end{tcolorbox}}

% array type box with title
\newenvironment{zebox}[1]{\begin{array}{|c|}
		\multicolumn{1}{l}{\tx{#1}} \\
		\hline
		\displaystyle
	}{\\ \hline
\end{array}}

% evtl: \renewcommand{\boxed}{\rmbox}

\def\centerarc[#1](#2)(#3:#4:#5)% Syntax: [draw options] (center) (initial angle:final angle:radius)
{ \draw[#1] ($(#2)+({#5*cos(#3)},{#5*sin(#3)})$) arc (#3:#4:#5); }

\tikzset{
	annotated cuboid/.pic={
		\tikzset{%
			every edge quotes/.append style={midway, auto},
			/cuboid/.cd,
			#1
		}
		\draw [every edge/.append style={pic actions, densely dashed, opacity=.5}, pic actions]
		(0,0,0) coordinate (o) -- ++(-\cubescale*\cubex,0,0) coordinate (a) -- ++(0,-\cubescale*\cubey,0) coordinate (b) edge coordinate [pos=1] (g) ++(0,0,-\cubescale*\cubez)  -- ++(\cubescale*\cubex,0,0) coordinate (c) -- 
		(o) -- ++(0,0,-\cubescale*\cubez) coordinate (d) -- ++(0,-\cubescale*\cubey,0) coordinate (e) edge (g) -- (c) -- cycle
		(o) -- (a) -- ++(0,0,-\cubescale*\cubez) coordinate (f) edge (g) -- (d) -- cycle;
		\path [every edge/.append style={pic actions, |-|}]
		%(b) +(0,-5pt) coordinate (b1) edge ["\cubex \cubeunits"'] (b1 -| c)
		%(b) +(-5pt,0) coordinate (b2) edge ["\cubey \cubeunits"] (b2 |- a)
		%(c) +(3.5pt,-3.5pt) coordinate (c2) edge ["\cubez \cubeunits"'] ([xshift=3.5pt,yshift=-3.5pt]e)
		;
	},
	/cuboid/.search also={/tikz},
	/cuboid/.cd,
	width/.store in=\cubex,
	height/.store in=\cubey,
	depth/.store in=\cubez,
	units/.store in=\cubeunits,
	scale/.store in=\cubescale,
	width=10,
	height=10,
	depth=10,
	units=cm,
	scale=.1,
}

\hbadness=99999

\begin{document}

\title{
	{\Huge Theoretische Physik II\\[3pt]Elektrodynamik}\\[1em]
	{\Large Vorlesung von Prof. Dr. Michael Thoss im Wintersemester 2018}}
\author{Markus Österle \hspace{5pt} Andréz Gockel  \hspace{5pt} Damian Lanzenstiel \hspace{5pt} Patrick Munnich}
\date{ \today}%15.10.2018}
\maketitle
\tableofcontents

\setcounter{chapter}{-1}
\chapter{Einführung}

%\addcontentsline{toc}{section}{section die nur im Inhaltsverzeichniss ist}
%\bibliographystyle{plain}
%\bibliography{literature}
%\addcontentsline{toc}{section}{Literatur}

\section{Zur Vorlesung}

\begin{description}
	\item[Dozent] Michael Thoss
	\item[Übungen] Donnerstag/Freitag (ILIAS) beginnt 18./19.10.18
	\item[Übungsleiter] Jakob Bätge
	\item[Abgabe der Hausaufgaben] bus Dienstag 12:00 - Briefkasten GuMi
	\item[Klausur] 13.02.19, 10-12 Uhr, Hörsaal Anatomie (Nachklausur: 26.19, 10-12 Uhr)
	\item[Ankündigungen] ILIAS Pass: theophy2.thoss18
	\item[Angaben] Vorlesung: 4 SWS, Übung: 2 SWS, ECTS: 7 
	\item[Vorkenntnisse] Mathematik: Analysis für Physiker (Vektor Rechnung), Theoretische Physik I, Experimental Physik II. 
\end{description}

\begin{tcolorbox}[colback=white,colframe=black,fonttitle=\bfseries,title=Hinweis zu den Übungen,sharp corners,tcbox raise base]
	\begin{itemize}
		\item[-] Keine Anwesenheitspflicht.
		\item[-] Keine Punktzahl nötig für Klausurzulassung.
		\item[-] Kann auch wehrend Übungen abgegeben werden.
	\end{itemize}
\end{tcolorbox}
\noindent
Lehrbücher: 
\begin{itemize}
	\item W. Nolting, \textit{Grundkurs Theoretische Physik 3: Elektrodynamik} (Springer)
	\item D.J. Griffiths, \textit{Elektrodynamik: Eine Einführung} (Pearson) 
	\item T. Fließbach, \textit{Elektrodynamik} (Spektrum Akademischer Verlag)
	\item J.D. Jackson, \textit{Klassische Elektrodynamik} (Walter de Gruyter) \lcom{geht dieser Vorlesung hinaus}
\end{itemize}

\section{Einführung und Überblick}

Die vier fundamentalen Wechselwirkungen (WW):
\begin{itemize}
	\item Starke WW
	\item \textbf{Elektromagnetische WW}\  \lcom{Wird in dieser Vorlesung betrachtet}
	\item Schwache WW
	\item Gravitation
\end{itemize}

\subsection{Rückblick}
Theoretische Physik 1: 
\begin{itemize}
	\item Mechanik 
	\item Punktmechanik: Bahnkurven von Körpern
	\item Bewegungsgleichung: $m \vec{\ddot r} = \vec{F}$
\end{itemize}
\subsection{Elektrodynamik}
\begin{itemize}
	\item Grundlegende Größen
	\item Felder
	\item 
	\begin{equation*}
	\begin{array}{cc}
	\vec{E}(\vec{r},t)\ \ \  & \vec{B}(\vec{r},t) \\[5pt]
	\tx{elektrisches Feld \ \ \ } &  \tx{Magnetfeld}
	\end{array}
	\end{equation*}
	\item[$\boldsymbol{\rightarrow}$] Feldtheorie \lcom{sehr wichtiges Konzept}
\end{itemize}
\lcom{Wie sind Elektrische Felder definiert?}\\
Experimentelle Definition als Messgröße: Kraft auf Ladung
$$\vec{F} = q(\vec{E}(\vec{r},t) + \vec{v} \times \vec{B}(\vec{r},t))$$
Theoretische Definition ist Mathematisch: Feldgleichungen-Maxwellgleichungen
$$ \ \qquad \vec\nabla \cdot\vec{E} = \frac{1}{\epsilon_0}\rho \qquad \qquad \ \qquad  \vec\nabla\cdot\vec{B} = 0$$
$$\vec\nabla\times\vec{E} + \prt{\vec{B}}{t} = 0 \qquad \vec\nabla\times\vec{B} - \mu_0 \epsilon_0 \prt{\vec{E}}{t} = \mu_0 \vec{j}$$
Hierbei steht $\rho$ für die Ladungsdichte und $\vec{j}$ für die Stromdichte.

\section{Aufbau der Vorlesung}
\textbf{1./2.} Statische Phänomene: $ \ \prt{\vec{E}}{t} = 0 = \prt{\vec{B}}{t}$
$$\Rightarrow \vec\nabla\cdot\vec{E} = \frac{1}{\epsilon_0}\rho \qquad\ \vec\nabla \cdot \vec{B} = 0$$
$$\ \ \, \quad\underbrace{\vec\nabla\times\vec{E} = 0}_{\textrm{1. Elektrostatik}} \qquad \underbrace{\vec\nabla\times\vec{B} = 0}_{\textrm{2. Magnetostatik}}$$
\textbf{3.} Zeitabhängige magnetische/elektrische Felder\\
\textbf{4.} Relativistische Formulierung der Elektrodynamik

\chapter{Elektrostatik}

Wir beschäftigen uns in diesem Kapitel mit \textbf{ruhenden Ladungen} und \textbf{zeitunabhängigen Feldern}. Das Grundproblem besteht darin, dass wir eine Ladungsverteilung haben und das Elektrische Feld und dessen Potential bestimmen wollen.\\[5pt]
\FloatBarrier
\begin{wrapfigure}{r}{3cm}
	\vspace{-1.2cm}
	%t1:
	\fbox{\begin{tikzpicture}
		\draw node[circle,fill,inner sep=1pt,label=left:$q_1$](a){} -- (1,0);
		\draw[xshift=0.5cm,yshift=0.7cm] node[circle,fill,inner sep=1pt,label=right:$q_2$](a){} -- (1,0);
		\draw[xshift=0.5cm,yshift=-0.2cm] node[circle,fill,inner sep=1pt,label=right:$q_3$](a){} -- (1,0);
		\end{tikzpicture}}
\end{wrapfigure}
\FloatBarrier
$\rightarrow$ Feld $\vec{E}(\vec{r})$, el. Potential $\Phi(\vec{r})$
\section{Elektrische Ladung und Coulomb'sches Gesetz}
Ladung: Beobachtungstatsachen:
\begin{enumerate}[i)]
	\item Zwei Arten ,,+``, ,,-``
	\item Abgeschlossenes System: Ladung erhalten: $q = \sum_i q_i = \const$
	\item Ladung ist quantisiert in Einheiten der Elementarladung: $$q = ne,\ n \in \mathbb{Z},\ e = 1,602\cdot 10^{-19}\,\textrm{C}$$ 
	$n = -1$: für ein Elektron wäre ein Beispiel einer Punktladung
\end{enumerate}

\noindent
\begin{minipage}{.6\linewidth}
	Kontinuierliche Ladungsverteilung Ladungsdichte\\
	$\rho(\vec{r}) = \frac{\textrm{Ladung}}{\textrm{Volumen}} = \frac{\Delta q}{\Delta V}$
	Gesamtladung in $V$: $$Q = \int_V \dd^3 r\, \rho(\vec{r})$$
\end{minipage}%
\begin{minipage}{.4\linewidth}
	\centering
	%t2:
	\begin{tikzpicture}
	\node at (1.5,1.5) (a) {};
	\draw  plot [smooth cycle, tension=1] coordinates { (0,0) (.5,1.5) (.5,2.5) (2,2) (3,2) (2.5,1) (1.5,0) };
	\pic at (a) {annotated cuboid={width=3,height=3,depth=3}};
	\draw[] (a) node[right,xshift=10pt] {$ \Delta V $};
	\draw[] (a) node[right,xshift=-10pt,yshift=10pt] {$ \Delta q $};
	\draw[->] (3,.5) to[out=180,in=-10] (1.5,.7);
	\node[right] at (3,.5) (b) {$ V $};
	\draw[thick,->] (-.5,-.5) -- (-.5,2) node[anchor=south east] {y};
	\draw[thick,->] (-.5,-.5) -- (2,-.5) node[anchor=north west] {x};
	\draw[thick,->] (-.5,-.5) -- (1.24,1.24) node[below,xshift=-10,yshift=-20pt] {$ \vec{r} $};
	\end{tikzpicture}
\end{minipage}%

\subsection{Coulombsches Gesetz}

\begin{minipage}{.585\linewidth}
	Die Kraft, welche eine am Ort $\vec{r}_2$ lokalisierte Punktladung auf eine Punktladung am Ort $\vec{r}_1$ ausübt, ist gegeben durch:
	$$\vec{F}_{12} = k \frac{q_1 q_2}{|\vec{r}_1 - \vec{r}_2|^2} \underbrace{\frac{\vec{r}_1 - \vec{r}_2}{|\vec{r}_1 - \vec{r}_2|}}_{\vec{e}_{r_{12}}}$$
\end{minipage}
\begin{minipage}{.4\linewidth}
	%t3:
	\hspace{30pt}
	\begin{tikzpicture}
	\node at (1,2) (a) {};
	\node at (2,1) (b) {};
	\draw[thick,->] (0,0) -- (0,2.5);
	\draw[thick,->] (0,0) -- (2.5,0);
	\draw[->] (0,0) -- (a) node[above] {$ q_1 $};
	\draw[->] (0,0) -- (b) node[right] {$ q_2 $};
	\node at (.5,1) (c) {};
	\node at (1,.5) (d) {};
	\draw[] (c) node[above,yshift=5,xshift=-3] {$ \vec{r}_1 $};
	\draw[] (d) node[below,yshift=2,xshift=5] {$ \vec{r}_2 $};
	\end{tikzpicture}
\end{minipage}

\begin{enumerate}
	\item $\vec{F}_{12} \sim q_1 q_2$
	\item $\vec{F}_{12} \sim \frac{1}{|\vec{r}_1 - \vec{r}_2|^2}$
	\item $\vec{F}_{12} \sim q_1 q_2\, \vec{e}_{r_{12}}$
	\item $\vec{F}_{12} = -\vec{F}_{21}$
\end{enumerate}
Es gilt das Superpositionsprinzip: Das heißt, durch vektorielle Addition der Kräfte kann die Gesamtkraft ermittelt werden.

$$\vec{F}_1 = k \summ{j = 2}{N} \frac{q_1 q_j}{r_{1j}^2}\vec{e}_{r_{1j}}$$

\paragraph{Zur Konstanten $k$:}

Die Konstante ist abhängig von dem verwendeten Maßsystemen.
\begin{enumerate}[i)]
	\item Gauß-System (cgs): $k \equiv 1$, dyn = $\frac{\textrm{g}\cdot \textrm{cm}}{\textrm{s}^2} = 10^{-5}\,\textrm{N}$\\
	1 dyn = $\frac{(1\textrm{ESE})^2}{\textrm{cm}^2} \quad 1\textrm{ESE} = \frac{\sqrt{\textrm{g}\cdot \textrm{cm}^3}}{\textrm{s}}$
	\item SI (MKSA-System): Definition von A = Amp\`ere 
	%t4:
	\vspace{-15pt}
	$$\frac{\Delta F}{\Delta l}= 2 \cdot 10^{-7}\,\frac{\textrm{N}}{\textrm{m}} \qquad \qquad \begin{tikzpicture}
	\draw (0,0) -- (3,0);
	\draw (0,1) -- (3,1);
	\draw[thick,<->] (.75,0) -- (.75,1) node[anchor=north east,yshift=-6pt] {1 m};
	\draw[thick,->] (2,0) -- (2,.3);
	\draw[thick,->] (2,1) -- (2,.7);
	\draw[->] (2,0) -- (2.5,0) node[above] {$ I $};
	\draw[->] (2,1) -- (2.5,1) node[above] {$ I $};
	\end{tikzpicture}$$
	Strom = $\frac{\textrm{Ladung}}{\textrm{Zeit}}$ $\Rightarrow$  \ 
	$
	1 \tx{A} = \frac{1 \tx{C}}{1 \tx{s}} \quad \rightarrow \quad e = 1{,}602 \cdot 10^{-19} \, \tx{C} \qquad c \approx 3 \cdot 10^{8} \, \frac{\tx{m}}{\tx{s}}
	$\\[5pt]
	$
	\frac{\Delta F}{\Delta l} = k \frac{2 I^2}{c^2 d} \qquad \rightarrow k = 2 \cdot 10^{-7} \ \frac{\tx{N}}{\tx{m}} \frac{c^2 1 \tx{m}}{2 (1 \tx{A})^2} = 10^{-7} c^2 \, \frac{\tx{N}}{\tx{A}^2}
	$
	\begin{equation*}
	k = \frac{1}{4 \pi \epsilon_0}
	\end{equation*}
	Damit erhalten wir für die Dielektrizitätskonstante des Vakuums:
	\begin{equation*}
	\epsilon_0 = 8{,}85 \cdot 10^{-12} \frac{\tx{C}^2}{\tx{N} \tx{m}^2}
	\end{equation*}
\end{enumerate}

\section{Elektrisches Feld}
\subsection{Feld eines Systems von Punktladungen}

\begin{minipage}{.5\linewidth}
	$N$-Ladungen $q_1, \dots, q_N$ ruhen an den Orten $\vec{r}_1, \dots, \vec{r}_N$. Nun bringen wir eine Testladung $q$ am Ort $\vec{r}$ mit ein. 
\end{minipage}%
\begin{minipage}{.5\linewidth}
	%t5:
	\hspace{50pt}
	\begin{tikzpicture}
	\node at (3,1) (a) {};
	\node at (2.5,2) (b) {};
	\node at (1,2) (c) {};
	\draw[thick,->] (0,0) -- (0,2.5) node[anchor=south east] {y};
	\draw[thick,->] (0,0) -- (2.5,0) node[anchor=north west] {x};
	\draw[->] (0,0) -- (a) node[above] {$ q_1 $};
	\draw[->] (0,0) -- (b) node[right] {$ q_2 $};
	\draw[->] (0,0) -- (c) node[above] {$ q $};
	\node at (1.5,.5) (c) {};
	\node at (1.25,1) (d) {};
	\node at (.5,1) (e) {};
	\draw[] (c) node[below,xshift=10pt] {$ \vec{r}_1 $};
	\draw[] (d) node[below,xshift=5pt] {$ \vec{r}_2 $};
	\draw[] (e) node[left] {$ \vec{r} $};
	\end{tikzpicture}
\end{minipage}%
\\
Kraft von $q_1,\ q_2$ auf $q$
$$\vec{F} = \frac{1}{4\pi \epsilon_0} q \summ{j = 1}{N} q_j \frac{\vec{r} - \vec{r}_j}{|\vec{r} - \vec{r}_j|^3} = q \vec{E}(\vec{r})$$
Somit ist das elektrisches Feld:
$$\rmbox{\vec{E}(\vec{r}) = \frac{1}{4 \pi \epsilon_0} \summ{j = 1}{N} q_j \frac{\vec{r} - \vec{r}_j}{|\vec{r} - \vec{r}_j|^3}}$$
\textbf{\emph{Bemerkung}}
\begin{enumerate}[i)]
	\item Testladung klein (formal: $ \lim_{q\to0}\frac{\vec{F}}{q} $)
	\item math. $ \vec{E}(\vec{r}) $ Vektorpfeil\\
	\begin{equation*}
	\tx{kartesisch:} \quad \vec{E}(\vec{r}) = \begin{pmatrix}
	E_x(\vec{r}) \\ E_y(\vec{r}) \\ E_z(\vec{r})
	\end{pmatrix}
	\end{equation*}
	\item Wechselwirkungsprozess: 2 Teile
	\begin{equation*}
	q_j \rightarrow \vec{E}(\vec{r}) \rightarrow \vec{F} = q \vec{E} (\vec{r})
	\end{equation*}
	\item Superpositionsprinzip gilt
\end{enumerate}

\subsection[Feld einer kontinuierlichen Ladungsverteilung]{Feld einer kontinuierlichen Ladungsverteilung $\rho(\vec{r})$}

\begin{minipage}{.7\linewidth}
	$$\rmbox{\vec{E}(\vec{r}) =\frac{1}{4 \pi \epsilon_0} \int\limits_V \dd^3 r'\, \rho(\vec{r}')\frac{\vec{r} - \vec{r}'}{|\vec{r} - \vec{r}'|^3}}$$
	$ \ub{\int\limits_V \dd ^3 r'}_{\mathclap{\substack{\tx{schließt alle} \\ \tx{Ladungen ein}}}} \qquad \rho(\vec{r}_j) = \frac{\Delta q_j}{\Delta V_j}$
\end{minipage}%
\begin{minipage}{.3\linewidth}
	\flushright
	%t6:
	\begin{tikzpicture}
	\node at (1.5,1.5) (a) {};
	\draw  plot [smooth cycle, tension=1] coordinates { (0,0) (.5,1) (.5,2.5) (2,2.2) (3,2) (2.5,.5) (1.5,-.2) };
	\pic at (a) {annotated cuboid={width=3,height=3,depth=3}};
	\draw[] (a) node[right,xshift=10pt] {$ \Delta V_j $};
	\draw[] (a) node[right,xshift=-10pt,yshift=10pt] {$ \Delta q_j $};
	\draw[->] (3,.5) to[out=180,in=-40] (2.2,.8);
	\node[right] at (3,.5) (b) {$ V $};
	\draw[thick,->] (-.5,-.5) -- (-.5,2) node[anchor=south east] {y};
	\draw[thick,->] (-.5,-.5) -- (2,-.5) node[anchor=north west] {x};
	\draw[thick,->] (-.5,-.5) -- (.25,.25) node[anchor=south east] {z};
	\draw[] (1,0) node[above] {$ \rho(\vec{r}) $};
	\end{tikzpicture}
\end{minipage}%

\begin{align*}
\vec{E}(\vec{r}) &= k\sum_j\Delta q_j\frac{\vec{r}-\vec{r}_j}{|\vec{r}-\vec{r}_j|^3}\\
&= k\sum_j\Delta V_j\rho(\vec{r}_j)\frac{\vec{r}-\vec{r}_j}{|\vec{r}-\vec  {r}_j|^3}\\
\scriptstyle{\tx{mit } \Delta V_j \to 0}\ \ &= k\int_V \dd^3r'\rho(\vec{r}')\frac{\vec{r}-\vec{r}'}{|\vec{r}  -\vec{r}'|^3}
\end{align*}

% Vorlesung 2 %18.10.18

\subsection{Ladungsdichte einer Punktladung}

\begin{minipage}{.6\linewidth}
	\paragraph{Deltafunktion}
	
	$$\rho (\vec{r}) = q \delta(\vec{r}-\vec{r}_0)$$
	Punktladung in $\vec{r}_0 \Rightarrow \rho(\vec{r}) = 0 \quad \vec{r} \neq \vec{r}_0$\\
	Ladungsdichte divergiert in $\vec{r}_0$
	$$\rho(\vec{r}_0) = \infty$$
\end{minipage}%
\begin{minipage}{.4\linewidth}
	%t1:
	\centering
	\begin{tikzpicture}
	\draw[->] (0,0) -- (0,2);
	\draw[->] (0,0) -- (2,0);
	\draw[->] (0,0) -- (1.5,1.5) node [right] {$q$};
	\draw (1,1) node [above,left] {$\vec{r}_0$};
	\end{tikzpicture}
\end{minipage}%
\\
Modell für Punktladung:\\
Ladung $q$ in Kugel mit Radius $\epsilon$  um $\vec{r}_0,\ \epsilon \rightarrow 0$
$$\rho_2 (\vec{r}) = \left\{ \begin{array}{cc}
\frac{q}{v_k} & |\vec{r}| \leq \epsilon\\
0 & \textrm{sonst}	
\end{array} \right\} = \frac{q}{\frac{4}{3}\pi \epsilon^3}\underbrace{\Theta (\epsilon - |\vec{r}|)}_{\mathclap{\textrm{Stufenfunktion}}}$$
%
%
%
% I1 T?
%
%
%
$$\rho(\vec{r}) = \lim_{\epsilon \rightarrow 0}\ \rho_\epsilon (\vec{r}) = \left\{ \begin{array}{cc}
\infty & \vec{r} = 0\\
0 & \vec{r} \neq 0
\end{array}\right.$$
Divergenz muss so sein, dass $$ \int\limits_{\substack{V \\ \vec{r}_0 \in V}} \dd^3r\ \rho(\vec{r}) = q$$

\paragraph{Definition Delta-Funktion (Diracsche Deltafunktion)}

\begin{enumerate}
	\item 
	$$\delta (\vec{r} - \vec{r}_0) = \left\{ \begin{array}{cc}
		0 & \vec{r} \neq \vec{r}_0 \\
		\infty & \vec{r} = \vec{r}_0
	\end{array}\right.$$
	\item 
	$$\int_V \dd^3 r\ f(\vec{r}) \delta(\vec{r} - \vec{r}_0) = \left\{ \begin{array}{cc}
		f(\vec{r}_0) & \vec{r}_0 \in V\\
		0 & \vec{r}_0 \notin V
	\end{array}\right.$$ 
\end{enumerate}

\paragraph{Mathematik}
Distribution - Funktional\\[5pt]
Funktional: Abb. Funktionen $\mapsto \mathbb R, \mathbb C$
$$\delta_{\vec{r}_0}: f \mapsto f(\vec{r}_0)$$

\paragraph{Physik}

$$\int \dd^3 r\ f(\vec{r}) \delta (\vec{r}-\vec{r}_0) = f(\vec{r})$$
$\delta$-Fkt. als Grenzwert einer Folge von Funktionen im Integral
$$\int \dd^3 r\ f(\vec{r}) \delta (\vec{r} - \vec{r}_0) = \lim_{\epsilon \to 0} \quad \int \dd^3 r\ f(\vec{r}) g_\epsilon(\vec{r}-\vec{r}_0) $$
mit
$$\lim_{\epsilon \to 0} g_\epsilon (\vec{r}-\vec{r}_0) =  \left\{ \begin{array}{cc}
0 & \vec{r} \neq \vec{r}_0 \\
\infty & \vec{r} = \vec{r}_0
\end{array}\right.$$
$$\int_V \dd^3 r\ g_\epsilon (\vec{r}-\vec{r}_0) = 1$$
\emph{Beispiel:} $g_\epsilon (\vec{r}-\vec{r}_0) = \frac{\Theta(\epsilon - |\vec{r}|)}{\frac{4}{3}\pi \epsilon^3}$\\
Mehrere Punktladungen $q_j$ in $\vec{r}_j$
$$\rho(\vec{r}) = \sum_j q_j \delta(\vec{r}-\vec{r}_j)$$

\begin{align*}
 	\Rightarrow \vec{E}(\vec{r}) &= \kq \int_V \dd^3 r'\ \rho(\vec{r}') \frac{\vec{r} - \vec{r}'}{|\vec{r} - \vec{r}'|^3}\\
	&= \kq \int_V \dd^3 r'\ \sum_j q_j \delta(\vec{r} - \vec{r}_j) \frac{\vec{r} - \vec{r}'}{|\vec{r} - \vec{r}'|^3}\\
	&= \kq \sum_j q_j \int_V \dd^3 r'\ \delta(\vec{r} - \vec{r}_j) \frac{\vec{r} - \vec{r}'}{|\vec{r} - \vec{r}'|^3}\\
	&= \kq \sum_j q_j \frac{\vec{r} - \vec{r}_j}{|\vec{r} - \vec{r}_j|^3} \qquad \checkmark\\
\end{align*}

\subsection{Flächenladungsdichte}

\begin{minipage}{.5\linewidth}
	$\sigma(\vec{r}) = \frac{\textrm{Ladung}}{\textrm{Fläche}} = \frac{\Delta q}{\Delta A}$
\end{minipage}%
\begin{minipage}{.5\linewidth}
	%t2 eig %t1:
	\centering
	\begin{tikzpicture}
		\draw[->] (0,0) -- (0,1);
		\draw[->] (0,0) -- (1,0);
		\draw[->] (0,0) -- (-.5,-.5);
		\draw (.5,.2) -- ++(30:1.3) -- ++(90:1) -- ++(-150:1.3) -- (.5,.2);
		\draw (.9,.7) -- ++(30:.3) -- ++(90:.3) -- ++ (-150:.3) -- (.9,.7);
		\draw[thick,->] (0,0) -- (.9,.7) node[yshift=-7pt,xshift=-18pt] {$ \vec{r} $};
		\draw ($ (.9,.7) + (30:.3) + (90:.3) + (-120:.15) $) to[out=65,in=150] (2,2) node[right] {$ \Delta A $};
		\draw ($ (.5,.2) + (33:1.1) $) to[out=-50,in=-150] (2,.5) node[right] {$ A $};
		\draw ($ (.9,.7) + (30:.3) + (90:.3) + (-100:.2) $) to[out=20,in=180] (2.2,1.35) node[right] {$ \Delta q $};
	\end{tikzpicture}
\end{minipage}%
\\
erzeugtes elektrisches Feld:
$$\vec{E}(\vec{r}) = \kq \int\limits_A \underbrace{\dd f'}_{\mathclap{\substack{\textrm{Flächen-} \\ \tx{element}}}}\ \sigma (\vec{r}) \frac{\vec{r} - \vec{r}'}{|\vec{r} - \vec{r}'|^3}$$
\begin{minipage}{.65\linewidth}
	\bei Elektrisches Feld einer homogenen Flächenladung
\end{minipage}%
\begin{minipage}{.35\linewidth}
	%t3 eig %t2:
	\centering
	\begin{tikzpicture}
		\draw[->] (0,0) -- (0,1) node[anchor=south east] {$ z $};
		\draw[->] (0,0) -- (2,0) node[anchor=north west] {$ y $};
		\draw[->] (0,0) -- (-1,-1) node[anchor=north east] {$ x $};
		\node at (2,1) {$ \substack{\sigma = \const \tx{in} \\ x\tx{-}y \tx{-Ebene} } $};
		\draw (-.3,-.3) -- ++(1.8,0);
		\draw (-.6,-.6) -- ++(1.8,0);
		\draw (-.9,-.9) -- ++(1.8,0);
		\draw (1.8,0) -- ++(-.9,-.9);
		\draw (1.2,0) -- ++(-.9,-.9);
		\draw (.6,0) -- ++(-.9,-.9);
	\end{tikzpicture}
\end{minipage}%
$$\vec{E}(\vec{r}) = \kq \intt{-\infty}{+\infty}\dd x'\intt{-\infty}{+\infty}\dd y'\ \sigma \frac{\vec{r} - \vec{r}'}{|\vec{r} - \vec{r}'|^3} \qquad \vec{r}' = (x', y', 0)$$
Symmetrie: $\vec{E}$ unabhängig von $x,y \qquad \vec{r} = (0,0,z)$
$$\vec{r} - \vec{r}' = (-x',-y',z),\ |\vec{r} - \vec{r}'|^3 = (x'^2+y'^2+z^2)^{\nicefrac{3}{2}}$$
$$E_x \sim \sigma \intt{-\infty}{+\infty}\dd x'\intt{-\infty}{+\infty}\dd y'\ \frac{(-x')}{(x'^2 + y'^2 + z^2)^{\nicefrac{3}{2}}} = 0 = E_y$$

$\vec{E} = (0,0,E_z)$

\begin{align*}
 	E_z &= \kq \sigma_z \intt{-\infty}{+\infty}\dd x'\underbrace{\intt{-\infty}{+\infty}\dd y'\ \frac{(x')}{(x'^2 + y'^2 + z'^2)^{\nicefrac{3}{2}}}}\\
 	& \qquad \frac{1}{x'^2 +z^2}\left.\frac{y'}{(x'^2 + y'^2 + z^2)^{\nicefrac{3}{2}}} \right\rvert_{-\infty}^{+\infty} = \frac{1}{x'^2 +z^2}\left.\frac{\textrm{sgn}(y')}{\sqrt{1 + \frac{x'^2 + z^2}{y'^2}}} \right\rvert_{-\infty}^{+\infty} = \frac{2}{x'^2 +z^2}\\
 	&= \frac{1}{2 \pi \epsilon_0} \sigma_z \underbrace{\intt{-\infty}{+\infty} \dd x'\ \frac{1}{x'^2 + z^2}}_{\left. \frac{1}{z} \arctan \left(\frac{x'}{2}\right)\right\rvert_{-\infty}^{+\infty} = \frac{1}{z} \textrm{sgn}(z)\pi}\\
 	E_z &= \frac{\sigma}{2 \epsilon_0} \textrm{sgn}(z)
\end{align*}
\begin{minipage}{.6\linewidth}
	Grenzfläche: $z \rightarrow 0$
	$$\vec{E} \underset{z \to 0} \longrightarrow \left\{ \begin{array}{cc}
	\frac{\sigma}{2 \epsilon_0} \vec{e}_z & z > 0\\
	-\frac{\sigma}{2 \epsilon_0} \vec{e}_z & z < 0
	\end{array}\right.$$
	$$\vec{E}_{\perp_+} - \vec{E}_{\perp_-} = \frac{\sigma}{\epsilon_0}, \qquad \vec{E}_\parallel = 0$$
\end{minipage}%
\begin{minipage}{.4\linewidth}
	%t3:
	\centering
	\begin{tikzpicture}
		\draw[->] (0,-1.2) -- (0,1.2) node[above] {$ E_z $};
		\draw[->] (-2,0) -- (2,0) node[anchor=north west] {$ z $};
		\draw[thick] (2,.8) -- (0,.8) ;
		\draw (0,.8) -- (-.15,.8) node[left] {$ \frac{\sigma}{2 \epsilon_0} $};
		\draw[thick] (-2,-.8) -- (0,-.8);
		\draw (0,-.8) -- (.15,-.8) node[right] {$\frac{\sigma}{2 \epsilon_0}$};
	\end{tikzpicture}
\end{minipage}%
%t4: %t5:
\begin{center}
\begin{tikzpicture}
	\coordinate (o) at (-8,0);
	\draw[thick,->] ($ (o) + (-.5,0)$) -- ++(2.5,0) node[anchor=north west] {$ z $};
	\draw[very thick] ($ (o) + (0,-1) $) -- ++(0,2) node[above] {$ \sigma $};
	\draw[very thick] ($ (o) + (1.5,-1) $) -- ++(0,2) node[above] {$ -\sigma $};
	
	\draw[->] (-2,0) -- (2,0) node[anchor=north west] {$ z $};
	\draw[->] (0,-1) -- (0,1.5) node[anchor=south east] {$ E_z $};
	\draw[very thick] (-2,0) -- (-.75,0);
	\draw[very thick] (.75,0) -- (2,0);
	\draw[very thick] (-.75,1) -- (.75,1) node[anchor=south west] {$ \frac{\sigma}{\epsilon_0} $};
	\draw[ultra thick,draw=orange,dashed] (-2,.5) -- (.75,.5);
	\draw[ultra thick,draw=orange,dashed] (.75,-.5) -- (2,-.5) node[right] {$ - \sigma $};
	\draw[very thick,draw=black!25!green,dashed] (-.75,-.5) -- (-2,-.5) node[left] {$ \sigma $};
	\draw[very thick,draw=black!25!green,dashed] (2,.5) -- (-.75,.5);
	\draw (.75,-.1) -- (.75,.1);
	\draw (-.75,-.1) -- (-.75,.1);
	\draw (-.1,.5) -- (.1,.5) node[right,xshift=55pt] {$ \frac{\sigma}{2 \epsilon_0} $};
	\draw (-.1,-.5) -- (.1,-.5) node[right,xshift=-2pt] {$ \frac{- \sigma}{2 \epsilon_0} $};
\end{tikzpicture}
\end{center}

\subsection{Linenladungsdichte}

\begin{minipage}{.5\linewidth}
	$$\lambda(\vec{r}) = \frac{\textrm{Ladung}}{\textrm{Länge}} = \frac{\Delta q}{\Delta s}$$
	$$\vec{E}(\vec{r}) = \kq \underbrace{\int\limits_\gamma}_{\mathclap{\textrm{Linienintegral}}} \dd s'\ \lambda(\vec{r}') \frac{\vec{r} - \vec{r}'}{|\vec{r} - \vec{r}'|^3}$$
\end{minipage}%
\begin{minipage}{.5\linewidth}
	%t6:
	\centering
	\begin{tikzpicture}
		\draw[->] (0,0) -- (0,1);
		\draw[->] (0,0) -- (1,0);
		\coordinate (m) at (2,3);
		\draw[thick,->] (0,0) -- (m) node[xshift=-30pt,yshift=-35pt] {$ \vec{r} $};
		\draw ($ (m) + (45:0.2) $) -- ++(180:.3) node[anchor=south east] {$ \Delta s $} -- ++(-90:.3) -- ++(0:.3) -- ++(90:.3) node[anchor=north west] {$ \Delta q $};
		\draw[thick] (-.5,1.5) to[out=80,in=-170] (1,2.7) to[out=10,in=-160] (m) to[out=20,in=-100] (3,4) node[right] {$ \gamma $};
	\end{tikzpicture}
	\vspace{15pt}
\end{minipage}%
\\
\emph{Beispiel:} Elektrisches Feld einer homogenen Linienladung $ \lambda = \const $\\
\begin{minipage}{.2\linewidth}
	%t7:
	\centering
	\begin{tikzpicture}
		\draw[->] (0,0) -- (1,0) node[anchor=north west] {$ x $};
		\draw[->] (0,0) -- (.75,.75) node[anchor=south west] {$ y $};
		\draw[->] (0,0) -- (0,3) node[anchor=south west] {$ z $};
		\draw[very thick,draw=red] (0,0) -- (0,2.5) node[left,yshift=-30pt] {$ \lambda $};
	\end{tikzpicture}
\end{minipage}%
\begin{minipage}{.8\linewidth}
	\begin{align*}
	\vec{E}(\vec{r}) &= \kq \int_\gamma \dd s'\ \lambda \frac{\vec{r} - \vec{r}'}{|\vec{r} - \vec{r}'|^3} \qquad \gamma: z' \mapsto \vec{r}'(z') = \begin{pmatrix} 0 \\ 0 \\ z' \end{pmatrix}\\
	% I2:
	&= \frac{\lambda}{4 \pi \epsilon_0} \intt{-\infty}{\infty}\dd z'\ \frac{\vec{r} - \begin{pmatrix} 0 , 0 , z \end{pmatrix}^\top}{(x^2 + y^2 + (z-z')^2)^{\nicefrac{3}{2}}}
	\end{align*}
\end{minipage}%
\\
$\tilde{z} = z' - z$:
$$E_x = \frac{\lambda x}{4 \pi \epsilon_0}\intt{-\infty}{\infty} \dd z'\ \frac{1}{(x^2 + y^2 + (z-z')^2)^{\nicefrac{3}{2}}} = \frac{\lambda x}{4 \pi \epsilon_0}\underbrace{\intt{-\infty}{\infty}\dd\tilde{z}\ \frac{1}{(x^2 + y^2 + \tilde{z}^2)^{\nicefrac{3}{2}}}}_{\frac{2}{x^2 + y^2}} = \frac{\lambda}{2 \pi \epsilon_0} \frac{x}{x^2 + y^2}$$
Wegen der Symmetrie genau so:
$$E_y = \frac{\lambda}{2 \pi \epsilon_0}\frac{x}{x^2 + y^2}$$
$$E_z = \frac{\lambda}{4 \pi \epsilon_0} \intt{-\infty}{\infty}\dd z'\ \frac{z-z'}{(x^2 + y^2 + (z-z')^2)^{\nicefrac{3}{2}}} = 0$$
$$\vec{E}(\vec{r}) = \frac{\lambda}{2 \pi \epsilon_0} \frac{1}{x^2 + y^2} \begin{pmatrix}x\\ y\\ 0\end{pmatrix}$$
$$\rho = \sqrt{x^2 + y^2} \quad \vec{E}(\vec{r}) = \frac{\lambda}{2 \pi \epsilon_0} \frac{1}{\rho}\vec{e}_\rho, \qquad \vec{e}_\rho = \begin{pmatrix}\cos \varphi\\ \sin \varphi\\ 0\end{pmatrix}$$

\section{Feldgleichungen und elektrostatisches Potential}

$$\vec{E}(\vec{r}) = \kq \int \dd^3 r'\ \rho(\vec{r}') \frac{\vec{r} - \vec{r}'}{|\vec{r} - \vec{r}'|^3}$$

\subsection{Elektrostatisches Potential}

elektrische Feld ist ein Potentialfeld $\vec{E}(\vec{r}) = - \vabla\phi(\vec{r}) = - \bigg( \vec{e}_x \prt{\phi}{x} + \vec{e}_y \prt{\phi}{y} + \vec{e}_z \prt{\phi}{z}\bigg)$\\
Nebenrechnung:
$$\frac{\vec{r} - \vec{r}'}{|\vec{r} - \vec{r}'|^3} = -\vabla\frac{1}{|\vec{r}-\vec{r}'|}$$
zur Überprüfung hier die $ x $-Komponente berechnet:
$$-\prt{}{x} \frac{1}{[(x-x')^2 + (y-y')^2 + (z-z')^2]^{\nicefrac{1}{2}}} =  \frac{- \left(-\frac{1}{2}\right) \quad 2(x-x')}{[(x-x')^2 + (y-y')^2 + (z-z')^2]^{\nicefrac{3}{2}}} = \frac{(x-x')}{|\vec{r} - \vec{r}'|^3}$$
Somit erhalten wir für das $ \vec{E} $-Feld:
$$\Rightarrow \vec{E}(\vec{r}) = \kq \int \dd^3 r'\ \rho(\vec{r}') \bigg(-\vabla_{\vec{r}} \frac{1}{|\vec{r} - \vec{r}'|} \bigg) = (-) \vabla_{\vec{r}} \kq \int \dd^3 r'\ \frac{\rho(\vec{r}')}{|\vec{r} - \vec{r}'|}$$
% I3:
$\rightarrow$ elektrostatisches Potential
$$\phi(\vec{r}) = \kq \int \dd^3 r'\ \frac{\rho(\vec{r}')}{|\vec{r} - \vec{r}'|} + c$$
übliche Konvention: $c = 0 \quad (\phi(\vec{r})\buildrel{|\vec{r}| \to \infty}\over{\rightarrow} 0)$\\[5pt]
\emph{Beispiel:} Potential einer Punktladung in $\vec{r}_0$:
$$\rho(\vec{r}) = q \delta(\vec{r}-\vec{r}_0)$$
$$\phi(\vec{r}) = \frac{1}{4 \pi \epsilon_0} \int_{\mathbb R^3} \dd^3 r'\ \frac{q\delta(\vec{r}' - \vec{r}_0)}{|\vec{r}-\vec{r}'|} = \kq \frac{q}{|\vec{r} - \vec{r}_0|}$$
$$\eofr = - \vabla \phi = \kq q \vabla \frac{1}{|\vec{r} - \vec{r}_0|} = \kq q \frac{\vec{r} - \vec{r}_0}{|\vec{r} - \vec{r}_0|^3}$$

% Vorlesung 3 22.10.18

\noindent
\lcom{(Funktional-Analysis Siegfried Großmann Springer)\\
(Landau-Lipschitz Buch geht weit der Vorlesung hinaus)}
%
%
%
% I4 T äquipotentialflächen jan
%
%
%
\subsection{Feldgleichung (differentielle Form)}

\paragraph{Rotation (Wirbel)}
$$\textrm{rot}\, \vec{E} =  \vabla \times \vec{E} = \vec{e}_x \bigg(\prt{E_z}{y} - \prt{E_y}{z}\bigg) + \vec{e}_y \bigg(\prt{E_x}{z} - \prt{E_z}{x}\bigg) + \vec{e}_z \bigg(\prt{E_y}{x} - \prt{E_x}{y}\bigg)$$
$$\Rightarrow \vabla \times \vec{E} = - \vabla \times (\vabla\phi) = 0$$
Mathe: Sie sind äquivalent
\begin{enumerate}[i)]
	\item $\vec{E} = - \vabla\phi$
	\item $\vabla \times \vec{E} = 0$ (auf einfach zusammenhängendem Gebiet)
	\item Kurvenintegral $\int_\gamma \dd\vec{r} \cdot \vec{E}$ ist Wegunabhängig 
	%
	%
	%
	% I5 T1
	%
	%
	%
	$$\intt{\vec{r}_1}{\vec{r}_2} \dd\vec{r} \cdot \vec{E} = - \intt{\vec{r}_1}{\vec{r}_2} \dd t \underbrace{\frac{\dd\vec{r}}{\dd t}\times \vabla\phi(\vec{r}(t))}_{\frac{\dd\phi}{\dd t}} = \underbrace{(\phi(\vec{r}_2) - \phi(\vec{r}_1))}_{\textrm{Potentialdifferenz}} $$
\end{enumerate}

\paragraph{Divergenz (Quellen)}

$$\tx{div} \vec{E} = \vabla \cdot \vec{E} = \prt{E_x}{x} + \prt{E_y}{y} + \prt{E_z}{z}$$
\begin{align*}
\vabla \cdot \vec{E}(\vec{r}) &= \vabla_{\vec{r}} \cdot \kq \int\dd^3r'\ \rho(\vec{r}') \frac{1}{|\vec{r} - \vec{r}'|^3}\\
&= \kq \int_V \dd^3 r' \rho (\vec{r}') \vabla_{\vec{r}} \cdot \frac{1}{|\vec{r} - \vec{r}'|^3}
\end{align*}
$x$-Anteil:
\begin{align*}
\frac{\partial}{\partial x}\frac{x-x'}{[(x-x')^2 + (y-y')^2 + (z-z')^2]^{\nicefrac{3}{2}}} &= \frac{1 \cdot [\dots]^{\nicefrac{3}{2}}(x - x')(x - x')^{\nicefrac{3}{2}}\cdot 2[\dots]^{\nicefrac{1}{2}}}{[\dots]^3}\\
&= \frac{[\dots]^{\nicefrac{1}{2}}((x - x')^2 + (y - y')^2 + (z - z')^2 - 3(x - x')^2)}{[\dots]^{\nicefrac{3}{2}}}\\
&= \frac{(y - y')^2 + (z - z')^2 - 2(x - x')^2}{[\dots]^{\nicefrac{3}{2}}}
\end{align*}
$$\frac{\partial}{\partial y} \frac{y-y'}{[(x-x')^2 + (y-y')^2 + (z-z')^2]^{\nicefrac{3}{2}}} = \frac{(x - x')^2 + (z - z')^2 - 2(y - y')^2}{[\dots]^{\nicefrac{3}{2}}}$$ 
$$\frac{\partial}{\partial z} \frac{z-z'}{[(x-x')^2 + (y-y')^2 + (z-z')^2]^{\nicefrac{3}{2}}} = \frac{(x - x')^2 + (y - y')^2 - 2(z - z')^2}{[\dots]^{\nicefrac{3}{2}}}$$ 
$$\vabla \cdot \frac{\vec{r} - \vec{r}'}{|\vec{r} - \vec{r}'|^3} = 0 \quad \textrm{falls} \quad \vec{r} \neq \vec{r}'$$
$$\Rightarrow \textrm{falls} \quad \vec{r} \notin V, \textrm{ d.h. } \vec{r} \textrm{ in Gebiet ohne Ladungsdichte } \rho(\vec{r}) = 0$$
$$\Rightarrow \vabla \cdot \vec{E}(\vec{r}) = 0$$ 
%
%
%
% I6 T?
%
%
%
$\vec{r} \in V$: Grenzwertbetrachtung (Regularisierung des Integranden)\\
statt
$$\frac{\vec{r} - \vec{r}'}{|\vec{r} - \vec{r}'|^3} = \frac{\vec{r}-\vec{r}'}{[(x-x')^2 + (y-y')^2 + (z-z')^2]^{\nicefrac{3}{2}}}$$
betrachten wir:
$$f_a (\vec{r}-\vec{r}') = \frac{\vec{r}-\vec{r}'}{[(x-x')^2 + (y-y')^2 + (z-z')^2]^{\nicefrac{3}{2}}} = \frac{\vec{r}-\vec{r}'}{[(\vec{r}-\vec{r}')^2 + a^2]^{\nicefrac{3}{2}}} \quad a \in \mathbb R,\ a>0$$
am Ende Grenzwert $\displaystyle{\lim_{a \to 0}}$
$$\vabla \cdot \vec{E} (\vec{r}) = \kq \lim_{a \to 0} \int_V \dd^3r'\ \rho(\vec{r}')\ \rmbox{\hspace{-10pt}\vabla_{\vec{r}}\cdot f_a(\vec{r}-\vec{r}')}$$

\begin{align*}
\prt{}{x} \frac{x-x'}{[(\vec{r}-\vec{r}')^2 + a^2]^{\nicefrac{3}{2}}} &= \frac{[\dots +a^2]^{\nicefrac{3}{2}} - (x - x') \frac{3}{2} \cdot 2(x-x') [\dots + a^2]^{\nicefrac{3}{2}}}{[\dots + a^2]^3}\\
&= \frac{(y - y')^2 + (z - z')^2 + a^2 - 2(x - x')^2}{[\dots + a^2]^{\nicefrac{3}{2}}}
\end{align*}

$$\vabla_{\vec{r}} \cdot f_a(\vec{r}-\vec{r}') = \frac{3a^2}{[(\vec{r}-\vec{r}')^2 + a^2]^{\nicefrac{5}{2}}}$$
$$\lim_{a \to 0} f_a(\vec{r}-\vec{r}') = \left\{ \begin{array}{cc}
0 & \vec{r} \neq \vec{r}'\\
\infty & \vec{r} = \vec{r}'	
\end{array} \right.$$
%
%
%
% I7  muss beim limes ein vabla_r \cdot hin ?
%
%
%
$$\Rightarrow \textrm{zum Integral } \int_V \dd^3r' \dots \textrm{ trägt (im Limes }a \to 0) \textrm{ nur der Bereich } \vec{r}' \approx \vec{r} \textrm{ bei}$$
$$K_R (\vec{r}) = \{\vec{r}' \in \mathbb R^3: |\vec{r} - \vec{r}'| \leq R\}$$
%
%
%
% I8 T zum integral jan
%
%
%
\begin{align*}
 	\hspace{100pt} &\lim_{a \to 0} \int_V \dd^3 r'\ \rho(\vec{r}') \vabla_{\vec{r}} \cdot f_a(\vec{r}- \vec{r}')\\
	&= \lim_{a \to 0} \int_{K_R (\vec{r})} \dd^3 r' \ \rho(\vec{r}') \frac{3a^2}{[(\vec{r}-\vec{r}')^2 + a^2]^{\nicefrac{5}{2}}}\\
 	&+ \underbrace{\lim_{a \to 0} \int_{V / K_R (\vec{r})} \dd^3 r' \rho(\vec{r}') \frac{3a^2}{[(\vec{r}-\vec{r}')^2 + a^2]^{\nicefrac{5}{2}}}}_{= 0}
\end{align*}
Wähle $R$ klein genug, dass man innerhalb $K_R (\vec{r})$, $\rho(\vec{r}')$ in Taylorreihe um $\vec{r}$ entwickeln kann.
$$\vec{\tilde{r}} = \vec{r}' - \vec{r},\ \dd^3 r' = \dd^3 \tilde{r}$$
$$\int_{K_R (\vec{r})} \dd^3 r' \ \rho(\vec{r}') \frac{3a^2}{[(\vec{r}-\vec{r}')^2 + a^2]^{\nicefrac{5}{2}}} = \int_{K_R (0)} \dd^3 \tilde{r} \ \rho(\vec{r} + \vec{\tilde{r}}) \frac{3a^2}{[\vec{\tilde{r}}^2 + a^2]^{\nicefrac{5}{2}}}$$
Taylorentwicklung von $\rho(\vec{r} + \vec{\tilde{r}})$ um $\vec{\tilde{r}} = 0$
$$\rho(\vec{r} + \vec{\tilde{r}}) = \rho(\vec{r}) + \vec{\tilde{r}} \cdot \vabla \rho(\vec{r}) + \dots$$
$$\hspace{8.1cm} = \int_{K_R (0)} \dd^3 \tilde{r} \ (\rho(\vec{r}) + \vec{\tilde{r}} \cdot \vabla \rho(\vec{r}) + \dots) \frac{3a^2}{[\vec{\tilde{r}}^2 + a^2]^{\nicefrac{5}{2}}}$$
\begin{enumerate}
	\item Integral:
	\begin{align*}
	\int_{K_R (0)} \dd^3 \tilde r\ \rho(\vec{r}) \frac{3a^2}{(\tilde{r}^2 + a^2)^{\nicefrac{5}{2}}} &= \rho(\vec{r}) \underbrace{\intt{0}{R} \dd\tilde r\ \frac{3a^2}{(\tilde{r}^2 + a^2)^{\nicefrac{5}{2}}}}_{\left[\frac{\tilde r^3}{(\vec{\tilde{r}}^2 + a^2)^{\nicefrac{3}{2}}}\right]_0^R} \underbrace{\int \dd \equaltoup{\Omega}{\mathclap{\sin \theta \dd \theta \dd \varphi}}}_{= 4\pi}\\
	&= 4 \pi \rho(\vec{r}) \frac{R^3}{(R^2 + a^2)^{\nicefrac{3}{2}}} \quad \underset{a \to 0}{\longrightarrow} \quad 4 \pi \rho(\vec{r})
	\end{align*} 
	\item Integral:
	$$\int_{K_R (0)} \dd^3 \tilde{r}\ \equalto{\vec{\tilde{r}}}{\tilde{r}\vec{e}_{\vec{\tilde{r}}}} \cdot \vabla_{\vec{r}} \rho(\vec{r}) \frac{3 a^2}{(\tilde{r}^2 + a^2)^{\nicefrac{5}{2}}} = \underbrace{\intt{0}{R} \dd \tilde r\ \frac{3a^2 \tilde r^3}{(\tilde r^2 + a^2)^{\nicefrac{3}{2}}}}_{\frac{2}{3}a -3a^2 \left(\frac{R^2 +\frac{2}{3}a^2}{(R^2 + a^2)^{\nicefrac{3}{2}}}\right)} \underbrace{\int \dd \Omega\ \vec{e}_{\tilde{r}} \cdot \vabla \rho(\vec{r})}_{\textrm{unabh. von } a} \quad \underset{a \to 0}{\longrightarrow} \quad 0$$
	Das gilt auch für alle höheren Terme. Alle höheren Terme werden im Limit $ \lim_{a \to 0} $$ = 0 $.
	%
	%
	%
	% I9 wo kommt oben das \tilde{r} ^3 her ?
	%
	%
	%
\end{enumerate}

$$\lim_{a \to 0} \int_V \dd^3 r' \rho(\vec{r}) \vabla_{\vec{r}}\cdot \frac{(\vec{r}-\vec{r}')}{[(\vec{r}-\vec{r}')^2 + a^2]^{\nicefrac{3}{2}}} = 4 \pi \rho (\vec{r})$$
$$\Rightarrow \vabla \cdot \vec{E}(\vec{r}) = \kq \lim_{a \to 0} = \frac{1}{\epsilon_0} \rho(\vec{r})$$
\begin{center}
	\begin{minipage}{.5\linewidth}
		\rbox{$$\vabla \cdot \vec{E}(\vec{r}) = \frac{1}{\epsilon_0} \rho(\vec{r}) \quad \vec{r} \in \mathbb R^3$$}
	\end{minipage}
\end{center}

\subsection{Zusammenfassung:}

\paragraph{Feldgleichungen der Elektrostatik}
Mathe: partielle DGL
\rbox{
$$\vabla \cdot \vec{E}(\vec{r}) = \frac{1}{\epsilon_0} \rho(\vec{r}) \textrm{ inhomogene DGL}$$
$$\vabla \times \vec{E}(\vec{r}) = 0\textrm{ homogene DGL}$$}
\noindent
DGL für Potential $\phi$: 
$\vec{E} = - \vabla \phi$
\begin{align*}\vabla \cdot \vec{E} = \vabla \cdot (- \vabla \phi) &= - \vabla \cdot \begin{pmatrix} 
\partial_x \phi \\
\partial_y \phi \\ 
\partial_z \phi 
\end{pmatrix} \\
&= - \underbrace{\left( \frac{\partial^2 \phi}{\partial x^2} + \frac{\partial^2 \phi}{\partial y^2} + \frac{\partial^2 \phi}{\partial  z^2} \right) }_{=: \Delta \phi}\end{align*}
Partielle DGL 2. Ordnung:
\frbox{Poissongleichung}{$$\Delta \Phi(\vec{r}) = - \frac{1}{\epsilon_0} \rho(\vec{r})$$}
\noindent
für Gebiete mit $\rho (\vec{r}) = 0$:
\begin{center}
	\begin{minipage}{.4\linewidth}
		\frbox{Laplacegleichung}{
			$$\Delta \phi (\vec{r}) = 0$$}
	\end{minipage}
\end{center}

\paragraph{Darstellung der Deltafunktion:}

$$\lim_{a \to 0} \int_{\mathbb{R}^3} \dd^3r'\ \rho(\vec{r}')\underbrace{\vabla_{\vec{r}} \cdot \frac{(\vec{r} - \vec{r}')}{[(\vec{r} - \vec{r}')^2 + a^2]^{\nicefrac{3}{2}}}}_{\frac{3 a^2}{[(\vec{r} - \vec{r}')^2 + a^2]^{\nicefrac{5}{2}}} =: g_a(\vec{r}' - \vec{r})} = 4 \pi \rho(\vec{r})$$
%
%
%
% I10 Stimmt das g_a definiert als 3a^2 / (shit)^5/2 statt r / (shit)^3/2
%
%
%
$$\frac{1}{4 \pi} g_a \textrm{ liefert Grenzwertdarstellung der $\delta$-funktion.}$$
$$\lim_{a\to 0} \int_{\mathbb{R}^3} \dd^3r'\ \rho(\vec{r}') \frac{1}{4\pi}g_a(\vec{r}'-\vec{r})=\rho(\vec{r})$$
$$\lim_{a \to 0} g_a(\vec{r}' - \vec{r}) = \left\{ \begin{array}{cc}
0 		& \vec{r} \neq  \vec{r}'\\
\infty 	& \vec{r} = 	\vec{r}'
\end{array} \right.$$
$$\delta(\vec{r}) = \lim_{a \to 0} \frac{1}{4 \pi} \vabla_{\vec{r}} \cdot \frac{r^2}{(r^2+a^2)^{\nicefrac{3}{2}}}$$
$$\hspace{3.6cm} \overset{\textrm{formal}}{=} \frac{1}{4 \pi} \vabla \cdot \underbrace{\frac{\vec{r}}{r^3}}_{\mathclap{= - \vabla \frac{1}{r}}} = - \frac{1}{4 \pi} \vabla \cdot \bigg(\vabla \frac{1}{r}\bigg) = \frac{-1}{4 \pi} \Delta \frac{1}{r}$$ $$\Rightarrow \quad \rmbox{\Delta \frac{1}{r} = -4 \pi \delta(\vec{r})} \qquad 
\Rightarrow \quad \rmbox{\Delta \frac{1}{|\vec{r} - \vec{r}_0|} = - 4 \pi \delta(\vec{r} - \vec{r}_0)}$$
z.B. Potential einer Punktladung $\rho$ q in $\vec{r}_0$:
$$\phi(\vec{r}) = \kq \frac{q}{|\vec{r} - \vec{r}_0|}$$
$$\Delta \phi(\vec{r}) = \kq q \underbrace{\Delta_{\vec{r}} \frac{1}{|\vec{r} - \vec{r}_0|} }_{= - 4 \pi \delta(\vec{r} - \vec{r}_0)} = - \frac{1}{\epsilon_0} \underbrace{q \delta(\vec{r} - \vec{r}_0)}_{= \rho(\vec{r})} = \frac{1}{\epsilon_0} \rho(\vec{r})$$

% Vorlesung 4 25.10.18

\bbb{Wiederholung}{\begin{align*}
	\vec{\nabla} \cdot \vec{E}(\vec{r}) &= \frac{1}{\epsilon_0} \rho(\vec{r})\\
	\vec{\nabla} \times \vec{E}(\vec{r}) &= 0 \\[10pt]
	\Rightarrow \qquad \quad \vec{E} &= - \vec{\nabla} \Phi \\
	\Rightarrow \quad \Delta \Phi (\vec{r}) &= - \frac{1}{\epsilon_0} \rho(\vec{r})
	\end{align*}
}

\subsection{Integralsätze der Vektoranalysis}

\paragraph{1) Gaußscher Satz:} 
Sei $\vec{A}(\vec{r})$ ein Vektorfeld im Volumen $V \subset \mathbb R^3$, so gilt:
$$\int_V \dd^3r\ \vabla \cdot \vec{A}(\vec{r}) = \int_{\partial V} \dd\vec{f}\ \cdot \vec{A}(\vec{r})$$
$$\partial V \tx{ Rand von } V$$
$$\dd \vec{f} = \custo{\rightarrow}{\vec{n}}{\mathclap{\substack{\textrm{nach außen orientierter} \\ \tx{Normaleneinheitsvektor}}}}\ \dd f$$
%
%
%
% I11 T1
%
%
%
\noindent
\emph{Bemerkung:}\\
\begin{enumerate}[i)]
	\item Analogie 1D: Fundamentalsatz der Integralrechnung:
	\begin{equation*}
	\int_{a}^{b} \dd x \frac{\dd f}{\dd x} = f(b) - f(a)
	\end{equation*}
	\item Geometrische / physikalische Interpretation:\\
	Fluss des Vektorfeldes $ \vec{A} $ durch $ \partial V $
	\begin{equation*}
	\int_{\partial V} \dd\vec{f} \cdot \vec{A}
	\end{equation*}
	Integral über die Quellen von $ \vec{A} $
	\begin{equation*}
	\int_V \dd^3r \vec{\nabla} \cdot \vec{A}
	\end{equation*}
	\begin{equation*}
	\vec{A} = \const \quad \rightarrow \quad \vec{\nabla} \cdot \vec{A} = 0
	\end{equation*}
	\emph{Beispiel:} Geschwindigkeit einer Flüssigkeit: $ \vec{A}(\vec{r}) = \vec{v}(\vec{r}) $\\
	%
	%
	%
	% I12 T1 az
	%
	%
	%
	$$ \vec{v} = \const \quad \vec{\nabla} \cdot \vec{v} = 0 \quad \int_{\partial V} \dd\vec{f} \cdot \vec{v} = 0 $$
	$ \Rightarrow $ Es gibt keine Quellen von $ \vec{v} $
	%
	%
	%
	% I13 T2 az
	%
	%
	%
	$$ \vec{\nabla} \cdot \vec{r} \neq 0 \quad \int_{\partial V} \dd\vec{f} \cdot \vec{v} \neq 0 $$
	%
	%
	%
	% I14 irgendetwas ist hier sehr falsch (kim)
	%
	%
	%
	\item
	%
	%
	%
	% I15 T3   (T4 fehlt. evtl jan aufschrieb oder kim)
	%
	%
	%
	\begin{align*}
	\int_V \dd^3r \vec{\nabla} \cdot \vec{A}(\vec{r})  =  \int_{0}^{\Delta x} \dd x \int_{0}^{\Delta y} \dd y \int_{0}^{\Delta z} \dd z \left(\frac{\partial A_x}{\partial x} + \frac{\partial A_y}{\partial y} + \frac{\partial A_z}{\partial z}\right)  \\
	\end{align*}
	\begin{align*}
	&\int_{0}^{\Delta z} \dd z \int_{0}^{\Delta y} \dd y \ub{\int_{0}^{\Delta x} \dd x \frac{\partial A_x}{\partial x}}_{A_x(\Delta x , y , z) - A_x(0,y,z)}\\
	& = \int_{0}^{\Delta z} \dd z \int_{0}^{\Delta y} \dd y A_x(\Delta x,y,z) - \int_{0}^{\Delta z} \dd z \int_{0}^{\Delta y} \dd y A_x(0,y,z)\\
	& = \int_{F^+_x} \dd\vec{f} \cdot \vec{A} + \int_{F^-_x} \dd\vec{f} \cdot \vec{A}
	\end{align*}
	\begin{equation*}
	F_x^+: \quad \dd\vec{f} = \vec{e}_x \dd y \dd z \qquad F_x^-: \quad \dd\vec{f} = - \vec{e}_x \dd y \dd z
	\end{equation*}
	ebenso gilt dann für die anderen Koordinaten:
	\begin{align*}
	\int_{0}^{\Delta x} \dd x \int_{0}^{\Delta z} \dd z \int_{0}^{\Delta y} \dd y \frac{\partial A_y}{\partial y} &= \int_{F^+_y} \dd\vec{f} \cdot \vec{A} + \int_{F^-_y} \dd\vec{f} \cdot \vec{A} \\
	\int_{0}^{\Delta x} \dd x \int_{0}^{\Delta y} \dd y \int_{0}^{\Delta z} \dd z \frac{\partial A_z}{\partial z} &= \int_{F^+_z} \dd\vec{f} \cdot \vec{A} + \int_{F^-_z} \dd\vec{f} \cdot \vec{A} \\
	\end{align*}
\end{enumerate}

$$\Rightarrow \int_V \dd^3 r \vabla \cdot \vec{A} = \int_{\partial V} \dd \vec{f} \cdot \vec{A}$$

\paragraph{2) Stokescher Satz}
Sei $\vec{A}(\vec{r})$ ein Vektorfeld, $F$ eine Fläche mit Randkurve $\partial F$, so gilt:
$$\int\limits_{\mathclap{\textrm{Linienintegral}\rightarrow\,\partial F \phantom{\textrm{Linienintegral}\rightarrow\,}}} \dd\vec{r} \cdot \vec{A}(\vec{r}) = \int\limits_{\mathclap{\phantom{\, \leftarrow \textrm{Oberflächenint.}} F\, \leftarrow \textrm{Oberflächenint.}}} \dd \vec{f} \cdot (\vabla \times \vec{A}(\vec{r}))$$
%
%
%
% I16 T5 jan Bild 6793
%
%
%
$$\dd\vec{f} = \vec{n} \ \dd f$$
Richtung von $\dd\vec{f}$ und Umlauf sinn von $\partial F$: \textbf{rechte Hand Regel}.\\
%
%
%
% I30 wie ist das mit der rechten hand regel gemeint ???
%
%
%
\emph{Beispiel:}
\begin{equation*}
\vec{A}(\vec{r}) = \begin{pmatrix}
-y \\ x \\ 0
\end{pmatrix}
\end{equation*}
%
%
%
% I17 T6 jan Bild 6793
%
%
%
\begin{equation*}
\vec{\nabla} \times \vec{A} = \begin{pmatrix}
\frac{\partial A_z}{\partial y} - \frac{\partial A_y}{\partial z} \\
\frac{\partial A_x}{\partial z} - \frac{\partial A_z}{\partial x} \\
\frac{\partial A_y}{\partial x} - \frac{\partial A_x}{\partial y}
\end{pmatrix} = \begin{pmatrix}
0 \\ 0 \\ 1 + 1
\end{pmatrix} = 2 \vec{e}_z
\end{equation*}
%
%
%
% I18 T7 Photo von andrez 25.10.18
%
%
%
\begin{equation*}
\vec{r}(\varphi) = R \begin{pmatrix}
\cos \varphi \\ \sin \varphi \\ 0
\end{pmatrix} \qquad \varphi \in [0,2\pi]
\end{equation*}
\begin{equation*}
\frac{\partial \vec{r}}{\partial \varphi} = R \begin{pmatrix}
- \sin \varphi \\ \cos \varphi \\ 0
\end{pmatrix}
\end{equation*}
\begin{align*}
\int_{\partial F} \dd \vec{r} \cdot \vec{A}(\vec{r}) &= \int_{0}^{2 \pi} \dd \varphi \frac{\partial \vec{r}}{\partial \varphi} \cdot \vec{A}( \vec{r} ( \varphi)) \\
&= \int_{0}^{2 \pi} \dd \varphi R ( + \sin^2 \varphi + \cos^2 \varphi) = 2 \pi R^2\\
\int_F \dd\vec{f} \ob{\left(\vec{\nabla} \times  \vec{A}\right)}^{2 \vec{e}_z} &= 2 \pi R^2
\end{align*}
Vektorfeld ohne Wirbel z.B. $\vec{A} = \const$\\
%
%
%
% I19 T8
%
%
%
$\vabla \times \vec{A} = 0$\\
%
%
%
% I20 T9
%
%
%
\bem
%
%
%
% I21 T10
%
%
%
%\
%
%
% I22 T11
%
%
%
\subsection{Integrale Form der Feldgleichung}

\subsection{Gaußsches Gesetz}

\begin{equation*}
\vec{\nabla} \cdot \vec{E} = \frac{1}{\epsilon_0} \rho(\vec{r})
\end{equation*}
%
%
%
% I23 T tikz mit unförmigem volumen und einem V drinnen
%
%
%
\begin{align*}
\int_{V} \dd^3 r \vec{\nabla} \cdot \vec{E}(\vec{r}) &= \frac{1}{\epsilon_0} \int_{V} \dd^3 r \rho (\vec{r}) = \frac{1}{\epsilon_0} Q_V\\
&= \int_{\partial V} \dd\vec{f} \cdot \vec{E}(\vec{r})
\end{align*}
\begin{equation*}
\rmbox{\int_{\partial V} \dd \vec{f} \cdot \vec{E}(\vec{r}) = \frac{1}{\epsilon_0} Q_V}
\end{equation*}

\subsubsection{Berechnung elektrischer Felder für hochsymmetrische Ladungsverteilungen}

\emph{Beispiel:}\\
Homogen geladene Kugel mit Radius $ R $ und Gesamtladung $ Q $. Damit ist die Ladungsdichte innerhalb der Kugel:
\begin{equation*}
\rho = \frac{Q}{V} = \frac{Q}{\frac{4}{3} \pi R^3}
\end{equation*}
%
%
%
% I24 T12
%
%
%
\begin{equation*}
\vec{E}(\vec{r}) = E_r (r) \vec{e}_r
\end{equation*}
\begin{equation*}
\vec{r} = r \begin{pmatrix}
\sin \theta \cos \varphi \\ \sin \theta \sin \varphi \\ \cos \theta
\end{pmatrix}
\end{equation*}
$ \vec{e}_r = \frac{\vec{r}}{r} $\\
Fluss von $\vec{E}$ durch Oberfläche einer Kugel mit Radius $r$
%
%
%
% I25 T13
%
%
%
$$\dd \vec{f} = \vec{e}_r r^2 \sin \theta \dd \theta \dd \varphi \quad \Rightarrow \quad \dd \vec{f} \cdot \vec{E} = E_r(r)r^2\sin \theta \dd \theta \dd \varphi $$
\begin{align*}
\int_{\partial K_r (0)} \dd\vec{f}\ \vec{E} &= \int_0^\pi \dd \theta \ \intt{0}{2 \pi} \dd \varphi E_r (r) r^2 \sin \theta\\
&= E_r (r) r^2 4 \pi\\
&= \frac{1}{\epsilon_0} Q_{K_r (0)} = \frac{1}{\epsilon_0} \int_{K_r (0)} \dd^3 r\ \rho(\vec{r}) = \frac{1}{\epsilon_0} \left\{ \begin{array}{cc}
Q & r > R\\
Q \frac{r^3}{R^3} & r \leq R 		
\end{array}\right.
\end{align*}
$$\Rightarrow E_r(r) = \frac{Q}{4 \pi \epsilon_0}\left\{ \begin{array}{cc}
\frac{1}{r^2} & r > R\\
\frac{r}{R^3} & r \leq R 		
\end{array}\right.$$
%
%
%
%  I26 T14
%
%
%
\subsection{Satz von Stokes}

\begin{equation*}
\vec{\nabla} \times \vec{E} = 0
\end{equation*}
\textbf{Definition:} $ \gamma = \partial F $\\
$ \int_\gamma $ ist dann ein Linienintegral über eine geschlossene Kurve
\begin{equation*}
\rmbox{\int_\gamma \dd \vec{r} \cdot \vec{E} = \int_F \dd\vec{f} \cdot (\vec{\nabla} \times \vec{E}) = 0}
\end{equation*}

\subsection{Zusammenfassung: Feldgleichungen der Elektrostatik}

\textbf{differentielle Darstellung:}
\begin{equation*}
\vec{\nabla}\cdot \vec{E} = \frac{1}{\epsilon_0} \rho \quad \vec{\nabla} \times \vec{E} = 0 \quad \rightarrow \quad \vec{E} = - \vec{\nabla} \Phi \quad \rightarrow \quad \Delta \Phi = - \frac{1}{\epsilon_0} \rho
\end{equation*}
\textbf{Integral Darstellung:}
\begin{equation*}
\int_{\partial V} \dd\vec{f} \cdot \vec{E} = \frac{1}{\epsilon_0} Q_V \qquad , \qquad \oint_\gamma \dd\vec{r} \cdot \vec{E} = 0
\end{equation*}

\section{Elektrostatische Energie}

\textbf{potentielle Energie einer Punktladung im äußeren elektrischen Feld}\\
Kraft auf Ladung $ q $:
\begin{equation*}
\vec{F} = q \vec{E}
\end{equation*}
Die Arbeit bei Verschiebung der Ladung von $ \vec{a} $ nach $ \vec{b} $
\begin{align*}
W &= - \int_{\vec{a}}^{\vec{b}} \dd\vec{r} \cdot \vec{F} = - q \int_{\vec{a}}^{\vec{b}} \dd\vec{r} \cdot \vec{E}(\vec{r})\\
&= q \int_{\vec{a}}^{\vec{b}} \dd\vec{r} \cdot \vec{\nabla} \Phi = q \ub{( \Phi(\vec{b}) - \Phi(\vec{a}))}_{\tx{Potentialdifferenz}}
\end{align*}
Die Arbeit um $ q $ aus dem unendlichen $ \infty $ nach $ \vec{r} $ zu bringen ist dann:
% tikz T16 evtl ursprünglich hier:
\begin{equation*}
W = q (\Phi(\vec{r}) - \Phi(\infty))
\end{equation*}
Zur Referenz: $ \Phi(\infty) = 0 $\\
\begin{minipage}{.6\linewidth}
	Damit ist die Energie der Ladung $ q $ im äußeren Feld:
	\begin{equation*}
	\Rightarrow \quad \rmbox{W = q \ \Phi(\vec{r})}
	\end{equation*}
	\begin{equation*}
	\vec{E} = - \vec{\nabla} \Phi
	\end{equation*}
\end{minipage}%
\begin{minipage}{.4\linewidth}
	
	%29.10.18
	
	%T -1 evtl = T16 checken:
	\flushright
	\begin{tikzpicture}
	\draw[thick,->] (0,0) -- (1,1) node[below,xshift=-10pt,yshift=-10pt] {$  \vec{r} $};
	\draw[<-] (1.15,1.15) to[out=60,in=120] (3,1) node[right] {$ q $};
	\draw[fill=black] (1.1,1.1)circle(2pt);
	\draw[->] (-.5,.5) to[out=20,in=-90] (.5,1.5);
	\draw[->] (.5,-.5) to[out=70,in=180] (1.5,.5) node[below,yshift=-10pt] {$ \vec{E} $};
	\end{tikzpicture}
\end{minipage}%
\\
%andrez 1:
\subsection{Elektrostatische Potentielle Energie}

\paragraph{Energie einer Verteilung von Punktladungen}

\begin{minipage}{.6\linewidth}
	$N$ Ladungen $q$: an Orten $\vec{r}_i$\\
	Zunächst: $\underbrace{i - 1}_{\mathclap{\textrm{erzeugen am Ort } \vec{r}_i}}$ Ladungen $q_j$ bei $\vec{r}_j$
\end{minipage}%
\begin{minipage}{.4\linewidth}
	\flushright
	%t1:
	\begin{tikzpicture}
		\draw[thick,->] (0,0) -- (1,1) node[above] {$ q_1 $};
		\draw[thick,->] (0,0) -- (2,1) node[right] {$ q_2 $};
		\draw[thick,->] (0,0) -- (1.8,0) node[right] {$ q_3 $};
		\node[above,xshift=-5pt] at (.5,.5) {$ \vec{r}_1 $};
		\node[right,yshift=-5pt] at (1,.5) {$ \vec{r}_2 $};
		\node[below] at (.9,0) {$ \vec{r}_3 $};
	\end{tikzpicture}
\end{minipage}%
\\
Das Potential
$$\Phi(\vec{r}_i) = \kq \summ{j = 1}{i - 1} \frac{q_i}{|\vec{r}_j - \vec{r}_i|}$$ 
Arbeit um $ i $-te Ladung aus dem unendlichen nach $ \vec{r} $ zu bringen:
\begin{equation*}
W_i = q_i \Phi(\vec{r}_i) = \frac{1}{4 \pi \epsilon_0} \sum_{j=1}^{i-1} \frac{q_i q_j}{r_{ij}}
\end{equation*}
Somit ergibt sich die gesamte Arbeit für $ N $ Ladungen als:
\begin{align*}
W = \sum_{i=2}^{N} W_i &= \frac{1}{4 \pi \epsilon_0} \sum_{i=2}^{N} \sum_{j=1}^{i-1} \frac{q_i q_j}{r_{ij}}\\
&= \kq \frac{1}{2} \sum_{i=1}^{N} \sum_{\substack{j=1 \\ j\neq i}}^{N} \frac{q_i q_j}{r_{ij}}
\end{align*}
\begin{equation*}
\Rightarrow W = \frac{1}{8 \pi \epsilon_0} \sum_{\substack{i,j\\i\neq j}} \frac{q_i q_j}{r_{ij}}
\end{equation*}
% andrez 2:
\begin{align*}
W &= \frac{1}{2} \summ{i = 1}{N} q_i \underbrace{\bigg( \sum_{\substack{j\\ j \neq i}} \kq \frac{q_j}{r_{ij}}\bigg)}_{\Phi_{/ \hspace{-3.5pt}i}(\vec{r}_i)} = \frac{1}{2} \summ{i = 1}{N} q_i \Phi_{/ \hspace{-4pt}i}(\vec{r}_i)
\end{align*}

\paragraph{Energie einer kontinuierlichen lokalisierten Ladungsverteilung}

\begin{minipage}{.3\linewidth}
	%t2:
	\centering
	\begin{tikzpicture}
	\draw  plot [smooth cycle, tension=.7] coordinates { (0,0) (.25,.5) (.25,1) (.75,.75) (1,0) (.5,-.25)};
	\node at (.5,.35) {$ \rho $};
	\end{tikzpicture}
\end{minipage}%
\begin{minipage}{.7\linewidth}
	\begin{align*}
	W &= \frac{1}{8 \pi \epsilon_0} \int \dd^3r\ \int \dd^3r'\ \frac{\rho(\vec{r}) \rho(\vec{r}')}{|\vec{r}_j - \vec{r}_i|}\\
	&= \frac{1}{2} \int \dd^3r\ \rho(\vec{r}) \underbrace{ \kq \int_{\mathbb R^3} \dd^3r'\ \frac{\rho(\vec{r}')}{|\vec{r}_j - \vec{r}_i|}}_{\Phi(\vec{r})}
	\end{align*}
\end{minipage}%

\noindent
\begin{minipage}{.3\linewidth}
	%t3:
	\centering
	\begin{tikzpicture}
	\draw  plot [smooth cycle, tension=.7] coordinates { (0,0) (.25,.5) (.25,1) (.75,.75) (1,0) (.5,-.25)};
	\node at (.5,.35) {$ \rho $};
	\draw[thick] (-.5,-1) to[out=20,in=160] (1.5,-1);
	\draw[thick] (-1,-.5) to[out=70,in=-70] (-1,1.5);
	\draw[thick,->] (-.5,-1) to[out=20,in=180] (.5,-.8);
	\draw[thick,->] (-1,-.5) to[out=70,in=-90] (-.8,.5);
	\node at (-1.4,.5) {$ \vec{E}_{\tx{ext}} $};
	\end{tikzpicture}
\end{minipage}%
\begin{minipage}{.7\linewidth}
	$$W_{\textrm{ext}} = \int \dd^3 r\ \rho(\vec{r}) \Phi_{\textrm{ext}}(\vec{r})$$
\end{minipage}%

\subsubsection{Energie $ W $ durch $ \vec{E} $ ausdrücken:}

\begin{align*}
\vec\Delta \Phi &= - \frac{1}{\epsilon_0} \rho \quad \Rightarrow \quad W = -\frac{1}{2} \int \dd^3r \epsilon_0 \ub{\vec\Delta \Phi(\vec{r}) \Phi(\vec{r})}_{\vec{\nabla} \cdot (\Phi \vec{\nabla} \Phi) - \equalto{(\vec{\nabla} \Phi)^2}{\vec{E}} }\\
&= -\frac{\epsilon_0}{2} \ub{\int_{\mathbb{R}^3} \dd^3r \vec{\nabla} \cdot (\Phi \vec{\nabla} \Phi)} + \frac{\epsilon_0}{2} \int \dd^3r \vec{E}^2(\vec{r})\\[5pt]
& \qquad \qquad \lim\limits_{R\to \infty} \int_{K_R(0)} \dd^3r \vec{\nabla} \cdot (\Phi \vec{\nabla} \Phi) = \lim\limits_{R \to \infty} \int_{\partial K_R(0)} \ub{\dd\vec{f} \cdot \ub{(\Phi \vec{\nabla}\Phi)}_{\buildrel {R\to \infty} \over \sim \frac{1}{R^3}}}_{\sim \frac{1}{R}} = 0\\
&= \frac{\epsilon_0}{2} \int \dd^3r \vec{E}^2(\vec{r})
\end{align*}
Zur Umformung oben wurde benutzt:
\begin{equation*}
\Phi \buildrel R\to \infty \over \sim \frac{1}{R} \qquad \vec{\nabla} \Phi \sim \frac{1}{R^2}\qquad \dd\vec{f} = \vec{n} \ub{\dd f}_{\sim R^2}
\end{equation*}
Damit ergibt sich für die Energie einer Verteilung von Punktladungen 
% andrez 3:
$$\Rightarrow \quad \rmbox{W = \frac{\epsilon_0}{2} \int \dd^3 r\ \vec{E}^2(\vec{r})} \qquad \textrm{nicht für Punkladungen selbst!!!}$$
\begin{minipage}{.6\linewidth}
	Energiedichte des elektrostatischen Feldes
	$$\rmbox{w(\vec{r}) = \frac{\epsilon_0}{2} \vec{E}^2(\vec{r})}$$
\end{minipage}%
\begin{minipage}{.4\linewidth}
	\flushright
	%t4:
	\begin{tikzpicture}
		\draw[thick,->] (-1,0) -- (3,0);
		\draw (-.05,0) rectangle (.05,1.5) node[above] {$ \sigma $};
		\draw (2-.05,0) rectangle (2+.05,1.5) node[above] {$ -\sigma $};
		\node[below] at (0,0) {0};
		\node[below] at (2,0) {$ l $};
		\draw[thick,->] (.5,.4) -- (1.5,.4);
		\draw[thick,->] (.5,.8) -- (1.5,.8);
		\draw[thick,->] (.5,1.2) -- (1.5,1.2);
		\node at (1,1.8) {$ \vec{E} $};
	\end{tikzpicture}
\end{minipage}%

\noindent
\begin{minipage}{.6\linewidth}
	\emph{Beispiel:} \textbf{Plattenkondensator}\\
	Fläche $F$, Ladung $\rightarrow\ \sigma = \frac{q}{F}\ \rightarrow\ \vec{E} = \frac{\sigma}{\epsilon_0} \vec{e}_x$\\[5pt]
	$ \rightarrow $ Die Energiedichte ist: $ w = \frac{\epsilon_0}{2} \vec{E}^2 = \frac{\sigma^2}{2 \epsilon_0} $ (nicht für Punktladungen)\\[5pt]
	$ \rightarrow $ Die Energie beträgt: $ W = \int \dd^3 r w(\vec{r}) = l \cdot F \cdot \frac{\sigma^2}{2 \epsilon_0} $
\end{minipage}%
\begin{minipage}{0.1\linewidth}
	$ \phantom{M} $
\end{minipage}%
\begin{minipage}{.3\linewidth}
	\flushright
	%t5:
	\begin{tikzpicture}
	\draw[->] (-1,0) -- (3,0);
	\node[above] at (0,2)  {$ \phantom{M} $};
	\draw[<-] (0,2) -- (0,-.3) node[below] {0};
	\draw (2,.2) -- (2,-.2) node[below] {$ l $};
	\draw[thick] (-1,0.01) -- (0,0.01);
	\draw[thick] (2,0.01) -- (3,0.01);
	\draw (0,1.2) -- (-.3,1.2) node[left] {$ \frac{\sigma}{\epsilon_0} $};
	\draw[thick] (2,1.2) --(0,1.2);
	\draw[thick] (2,1.21) -- (0,1.21);
	\end{tikzpicture}
\end{minipage}%

\subsubsection{Potentialdifferenz - Spannung}
\begin{equation*}
\Phi(\vec{r}) - \Phi(0) = - \int_{0}^{\vec{r}} \dd\vec{r}' \cdot \vec{E} (\vec{r}') = - \int_{0}^{x} \dd x' \frac{\sigma}{\epsilon_0} = - \frac{\sigma}{\epsilon_0} x
\end{equation*}
%t6: %t7:
$$
\begin{tikzpicture}
\draw[->] (0,0) -- (0,1) node[anchor=south east] {$ z $};
\draw[->] (0,0) -- (.7,.7) node[anchor=south] {$ y $};
\draw[->] (0,0) -- (1,0) node[anchor=north west] {$ x $};
\draw[dashed] (.7,.7) -- (.7+1,.7);
\draw[dashed] (1,0) -- (1+.7,.7);
\draw[thick,->] (0,0) -- (1.8,1.2) node[right,yshift=-5pt] {$ \vec{r} $};
\end{tikzpicture}
\qquad \qquad 
\begin{tikzpicture}
\draw[->] (-.5,0) -- (1.5,0) node[anchor=north west] {$ x $};
\draw[->] (0,-1) -- (0,1) node[anchor=south east] {$ \Phi $};
\draw[thick] (0,0) -- (1,-1);
\draw[dashed] (1,-1) -- (1,.2) node[anchor=north west,yshift=-5pt] {$ l $};
\end{tikzpicture}
$$
Die Spannung zwischen zwei Kondensatorplatten ist dann:
\begin{equation*}
U = \Phi(0) - \Phi(l) = \frac{\sigma}{\epsilon_0} l = \frac{q}{\epsilon_0 F} l
\end{equation*}
Die Kapazität ist also:
\begin{equation*}
C = \frac{q}{U} = \frac{\epsilon_0 F}{l}
\end{equation*}
% andrez 4:
\lcom{Was ist die Energie bei einer Verteilung von Punktladungen und bei einer kontinuierlichen Ladungsverteilung. Bei einer kontinuierlichen Ladungsverteilung haben wir herausgefunden:}
$$W = \frac{1}{8 \pi \epsilon_0} \sum_{\substack{i,j\\ i \neq j}} \frac{q_i q_j}{r_{ij}} \qquad \textrm{ für Punktladungen}$$
\lcom{Die Energie der Punktladung selbst steckt hier nicht drinnen. Man muss dabei aufpassen, welche Gleichung man für welches Modell benutzt.}
$$\vec{E} = \kq q \frac{\vec{r}}{r^3} \qquad \int \dd^3 r\ \vec{E}^2 = \int \dd^3 r\ \frac{1}{r^4} = \infty$$

\section{Verhalten des el. Feldes an Grenzflächen mit Flächenladung}
$ \rightarrow $ Diskontinuitäten von $ \vec{E} $\\
\emph{Beispiel:} Wir betrachten eine\textbf{ homogene Flächenladung}.\\[10pt]
\begin{minipage}{.4\linewidth}
	%t8: %t9: %t10: %t11:
	\begin{tikzpicture}
	\draw (-1,-1) to[out=45,in=180] (1.2,0);
	\draw (-1.2,-.4) to[out=45,in=180] (1,.6);
	\draw (-.8,-1.6) to[out=45,in=180] (1.4,-.6);
	\draw (0,1) to[out=-30,in=90] (1,-1);
	\draw (-.7,.7) to[out=-30,in=90] (.3,-1.3);
	\draw (-1.4,.2) to[out=-30,in=90] (-.4,-1.6);
	\draw[thick,<-] (.5,.3) to[out=60,in=180] (1.5,1) node[right] {$ \sigma(\vec{r}) = \frac{\tx{Ladung}}{\tx{Fläche}} $};
	\end{tikzpicture}
\end{minipage}%
\begin{minipage}{.3\linewidth}
	\begin{tikzpicture}
	\draw[thick,->] (0,0) -- (0,.75) node[anchor=south east] {$ z $};
	\draw[thick,->] (0,0) -- (1,0) node[anchor=north west] {$ y $};
	\draw[thick,->] (0,0) -- (-.75,-.75) node[anchor=north east] {$ x $};
	\draw (-.2,-.2) -- (-.2+1,-.2);
	\draw (-.4,-.4) -- (-.4+1,-.4);
	\draw (-.6,-.6) -- (-.6+1,-.6);
	\node at (.5,-.75) {$ \sigma $};
	\end{tikzpicture}
\end{minipage}\nolinebreak%
\begin{minipage}{.3\linewidth}
	\begin{tikzpicture}
	\draw[thick,->] (0,-1.2) -- (0,1.2) node[above] {$ E_z $};
	\draw[thick,->] (-2,0) -- (2,0) node[anchor=north west] {$ z $};
	\draw[thick] (2,.8) -- (0,.8) ;
	\draw (0,.8) -- (-.15,.8) node[left] {$ \frac{\sigma}{2 \epsilon_0} $};
	\draw[thick] (-2,-.8) -- (0,-.8);
	\draw (0,-.8) -- (.15,-.8) node[right] {$\frac{\sigma}{2 \epsilon_0}$};
	\end{tikzpicture}
\end{minipage}%
\begin{equation*}
\Rightarrow \vec{E} = \frac{\sigma}{2 \epsilon_0} \tx{sgn}(z) \vec{e}_z
\end{equation*}
\begin{align*}
\vec{E}_\perp &= \pm \frac{\sigma}{2 \epsilon_0} \vec{e}_z\\
\vec{E}_\parallel &= 0
\end{align*}
\begin{minipage}{.6\linewidth}
	Das elektrische Feld $ \vec{E}_\parallel $ ist gleich der Ableitung des elektrischen Potentials:\\
	Das elektrische Potential ist also stetig.
\end{minipage}%
\begin{minipage}{.4\linewidth}
	\hspace{30pt}
	\begin{tikzpicture}
	\draw[thick,->] (0,-.7) -- (0,.7) node[anchor=south east] {$ z $};
	\draw[thick,->] (-1,0) -- (1.2,0) node[anchor=north west] {$ \Phi(z) $};
	\draw (-1,-.7) -- (0,0) -- (1,-.7);
	\end{tikzpicture}
	\vspace{5pt}
\end{minipage}%

\noindent
\begin{minipage}{.4\linewidth}
	%andrez 5:
	
	\paragraph{Normalkomponente $\vec{E}_\perp$}
	
	Gaußscher Satz für $V$:
	\vspace{50pt}
\end{minipage}%
\begin{minipage}{.6\linewidth}
	%t12:
	\centering
	\begin{tikzpicture}[scale=.8]
	\coordinate (o) at (0,0);
	\coordinate (O) at (-4,-2);
	\draw[fill=gray!7.5] (-3,-1.4) -- (1.5,-1.4) -- (3,1.4) -- (-1.5,1.4) -- (-3,-1.4);
	\draw[->] (O) -- ++(.4,0);
	\draw[->] (O) -- ++(0,.4);
	\draw[->] (O) -- ++(.3,.3);
	\draw[red,fill=red!5] (o) ellipse (1cm and .5cm);
	\draw[red,fill=red!5] ($ (o) + (0,2) $) ellipse (1cm and .5cm);
	\draw[red,dashed,fill=red!5] ($ (o) + (0,-2) $) ellipse (1cm and .5cm);
	\draw[red] (-1,2) -- (-1,0);
	\draw[red] (1,2) -- (1,0);
	\draw[red,dashed] (-1,-2) -- (-1,0);
	\draw[red,dashed] (1,-2) -- (1,0);
	\draw[red,thick,->] (o) -- ++(0,3) node[anchor=west] {$ \vec{n}_+ $};
	\draw[blue,thick,->] (o) -- ++(0,-3) node[anchor=east] {$ \vec{n}_- $};
	\draw[thick,->] (O) -- (o) node[xshift=-70pt,yshift=-20pt] {$ \vec{r} $};
	\node[right] at (o) {$ F $};
	\node[anchor=north west] at ($ (o) + (0,1.2) $) {$ \partial V_+ $};
	\node[anchor=north west] at ($ (o) + (0,-.8) $) {$ \partial V_- $};
	\draw[decorate, decoration={brace,amplitude=10pt,mirror,raise=2pt}, xshift=-10pt,red]  (-1,2) -- node[left=20pt] {$\Delta z$}  (-1,-2);
	\node at (2.2,1) {$ \sigma(\vec{r}) $};
	\draw ($ (o) + (.5,.3) $) to[out=30,in=150] (2.7,0) node[right] {$ V = \Delta z \cdot F $};
	\node[circle,fill=blue,inner sep=1pt,minimum size=1pt] at (0,-2) {};
	\node[circle,fill=red,inner sep=1pt,minimum size=1pt] at (0,2) {};
	\end{tikzpicture}
\end{minipage}%
\vspace{-10pt}
\begin{align*}
\int_V \dd^3 r'\ \vabla \cdot \vec{E}(\vec{r'}) &= \int_{\partial V} \dd\vec{f}'\ \vec{E}(\vec{r})\\
&=\custo{\overset{\rotatebox{90}{$\scriptscriptstyle{\Delta z \to 0}$}}{\longrightarrow}}{\int_{\textrm{Mantel}} \dd\vec{f}'\ \vec{E}}{\hspace{-23.5pt}0} + \custo{\overset{\rotatebox{90}{$\scriptscriptstyle{\Delta z \to 0}$}}{\longrightarrow}}{\int_{\partial V_+} \dd\vec{f}'\ \vec{E}(\vec{r})}{\int_F \dd f'\ \vec{n} \cdot \vec{E}_+} + \custo{\overset{\rotatebox{90}{$\scriptscriptstyle{\Delta z \to 0}$}}{\longrightarrow}}{\int_{\partial V_-} \dd\vec{f}'\ \vec{E}}{- \int_F \dd f'\ \vec{n} \cdot \vec{E}_-}
\end{align*}
$ \vec{E}_{\pm} $ ist das Feld auf beiden Seiten der Grenzfläche
% eaualto zwischen Zeile 1 und 2 der equations ...
\begin{equation*}
\int_{\partial V} \dd\vec{f}' \vec{E} \quad \buildrel \Delta z \to 0 \over \longrightarrow \quad \int_{F} \dd f \vec{n} \cdot (\vec{E}_+ - \vec{E}_-) \quad \buildrel F \to 0 \over \longrightarrow \quad F \ \vec{n} \cdot \big(\vec{E}_+(\vec{r}) - \equalto{\vec{E}_-(\vec{r})}{}\big)
\end{equation*}
\begin{equation*}
\int_V \dd^3r' \equalto{\vec{\nabla} \cdot \vec{E}(\vec{r}')}{\frac{1}{\epsilon_0} \rho(\vec{r}')} = \frac{1}{\epsilon_0} \int_V \dd^3r' \rho(\vec{r}) = \frac{1}{\epsilon_0} \int_F \dd f'  \sigma(\vec{r}') \quad \buildrel F \to 0 \over \longrightarrow \quad \frac{1}{\epsilon_0} F \sigma(\vec{r})
\end{equation*}
\begin{equation*}
\Rightarrow \vec{n} \cdot \left(\vec{E}_+(\vec{r}) - \vec{E}_-(\vec{r}) \right) = \frac{1}{\epsilon_0} \sigma(\vec{r})
\end{equation*}
\begin{equation*}
E_{\perp_\pm} = \vec{n} \cdot \vec{E}_{\pm} \qquad E_{\perp_+}(\vec{r}) - E_{\perp_-} (\vec{r}) = \frac{1}{\epsilon_0} \sigma(\vec{r})
\end{equation*}

%t13: und Tafel vor M4:
\noindent
\begin{minipage}{.4\linewidth}
	
	\paragraph{Tangentialkomponente $ E_\parallel $}
	
	Satz von Stokes:
	\vspace{50pt}
\end{minipage}%
\begin{minipage}{.6\linewidth}
	\centering
	\begin{tikzpicture}[scale=1.15]
		\draw (-2,1) to[out=-10,in=135] (0,0) to[out=-45,in=100] (1,-2);
		%\draw (-.3,1.25) -- (1.25,.3) -- (.3,-1.25) -- (-1.25,-.3) -- (-.3,1.25);
		\draw[->] ($ (0,0) + (-.25,1.25) + (-45:2cm) + (-135:1.5cm) $) -- ($ (0,0) + (-.25,1.25) + (-45:2cm) + (-135:.6cm) $);
		\draw[->] ($ (0,0) + (-.25,1.25) + (-45:2cm) $) -- ($ (0,0) + (-.25,1.25) + (-45:1cm) $);
		\draw[->] ($ (0,0) + (-.25,1.25) $) -- ($ (0,0) + (-.25,1.25) + (-135:.75cm) $);
		\draw[->] ($ (0,0) + (-.25,1.25) + (-135:1.5cm) $) -- ($ (0,0) + (-.25,1.25) + (-135:1.5cm) + (-45:1cm) $);
		\draw[draw=none,fill=gray,opacity=.1] ($ (0,0) + (-.25,1.25) $) -- ++(-135:1.5cm) -- ++(-45:2cm) -- ++(45:1.5cm) -- ++(135:2cm);
		\draw ($ (0,0) + (-.25,1.25) $) -- ++(-135:1.5cm) -- ++(-45:2cm) -- ++(45:1.5cm) -- ++(135:2cm);
		\draw[decorate, decoration={brace,amplitude=10pt,raise=2pt},red] ($ (0,0) + (-.25,1.25) $) -- node[above=10pt,xshift=13pt] {$ L $} ($ (0,0) + (-.25,1.25) + (-45:2cm) $);
		\draw[decorate, decoration={brace,amplitude=10pt,mirror,raise=2pt},red] ($ (0,0) + (-.25,1.25) $) -- node[above=10pt,xshift=-12pt] {$ \Delta z $} ($ (0,0) + (-.25,1.25) + (-135:1.5cm) $);
		\node at (1,-1) {$ \gamma $};
		\coordinate (O) at (-1.5,-1.5);
		\draw[->] (O) -- ++(.4,0);
		\draw[->] (O) -- ++(0,.4);
		\draw[->] (O) -- ++(.25,.25);
		\node[anchor=north east] at (O) {$ O $};
		\draw[thick,->] (O) -- (0,0) node[anchor=north east,yshift=-15pt,xshift=-25pt] {$ \vec{r} $};
		\draw[thick,->] (0,0) -- ++(135:.75cm) node[anchor=south west,yshift=-10pt,xshift=5pt] {$ \vec{E} $};
		\node at (-2,.3) {$ \sigma $};
	\end{tikzpicture}
\end{minipage}%
\begin{equation*}
0 = \oint_r \dd\vec{r} \vec{E}(\vec{r}') = \ub{
	\int\limits_{\begin{tikzpicture}[scale=.7]
		\draw[dashed] (0,0) -- (1,0) -- (1,.7) -- (0,.7) -- (0,0);
		\draw[thick,->] (1,0) -- (1,.7);
		\end{tikzpicture}} \dots + 
	\int\limits_{\begin{tikzpicture}[scale=.7]
		\draw[dashed] (0,0) -- (1,0) -- (1,.7) -- (0,.7) -- (0,0);
		\draw[thick,->] (0,0) -- (0,.7);
		\end{tikzpicture}} \dots}_{= 0 \tx{ für } \Delta z \to 0} + \ub{
	\int\limits_{\begin{tikzpicture}[scale=.7]
		\draw[dashed] (0,0) -- (1,0) -- (1,.7) -- (0,.7) -- (0,0);
		\draw[thick,->] (1,.7) -- (0,.7);
		\end{tikzpicture}} d\vec{r}' \vec{E} + 
	\int\limits_{\begin{tikzpicture}[scale=.7]
		\draw[dashed] (0,0) -- (1,0) -- (1,.7) -- (0,.7) -- (0,0);
		\draw[thick,->] (0,0) -- (1,0);
		\end{tikzpicture}} d\vec{r}' \vec{E} }_{
	\int\limits_{\begin{tikzpicture}[scale=.4]
		\draw[dashed] (0,0) -- (1,0) -- (1,.7) -- (0,.7) -- (0,0);
		\draw[thick,->] (1,0) -- (0,0);
		\end{tikzpicture}} \dd\vec{r}' (\vec{E}_+ -  \vec{E}_-)}
\end{equation*}
\begin{equation*}
0 = \oint_\gamma \dd\vec{r}' \cdot \vec{E} \quad \buildrel \Delta z \to 0 \over \longrightarrow \quad \int_{-\frac{L}{2}}^{-\frac{L}{2}} \dd s \vec{t} \cdot \left(\vec{E}_+ - \vec{E}_-\right) \quad \buildrel L \to 0 \over \longrightarrow \quad L \ \vec{t} \cdot \left(\vec{E}_+(\vec{r}) - \vec{E}_-(\vec{r})\right) = 0
\end{equation*}
\begin{equation*}
\rightarrow \vec{t} \cdot (\vec{E}_+ (\vec{r}) - \vec{E}_- (\vec{r})) = 0
\end{equation*}
$\rightarrow $ Die Tangentialkomponente ist stetig
\begin{equation*}
E_{\parallel_{+}} = E_{\parallel_{-}}
\end{equation*}
Insgesamt ergibt sich damit:
\begin{equation*}
\vec{E}_+(\vec{r}) - \vec{E}_-(\vec{r}) = \frac{\sigma}{\epsilon_0} \vec{n}
\end{equation*}
\begin{minipage}{.6\linewidth}
	Das elektrische Potential $ \Phi $ ist damit stetig.
	\begin{equation*}
	\ub{\Phi(\vec{r}_b) - \Phi(\vec{r}_a)}_{\Phi_+(\vec{r}) - \Phi_-(\vec{r})} = \int_{\vec{r}_a}^{\vec{r}_b} \dd\vec{r}' \cdot \vec{E} \quad \buildrel \Delta z \to 0 \over \longrightarrow 0
	\end{equation*}
	Und hiermit auch auf beiden Seiten der Flächenladung symmetrisch.
\end{minipage}%
\begin{minipage}{.4\linewidth}
	%t14:
	\centering
	\begin{tikzpicture}
		\draw[fill=gray!10] (-1.2,-.7) -- (.8,-.7) -- (1.2,.7) -- (-.8,.7) -- (-1.2,-.7);
		\draw[thick,<-] (0,0) -- (1.5,-1) node[anchor=west] {$ \vec{r} $};
		\node[anchor=south west] at (1,.7) {$ \sigma $};
		\draw[thick,->] (0,0) -- (0,1) node[anchor=south west] {$ \vec{r}_n $};
		\draw[thick,dashed] (0,0) -- (0,-.7);
		\draw[thick,->] (0,-.7) -- (0,-1) node[anchor=north west] {$ \vec{r}_a $};
		%\draw[decorate, decoration={brace,amplitude=10pt,raise=2pt},red] ($ (0,0) + (-.25,1.25) $) -- node[above=10pt,xshift=13pt] {$ L $} ($ (0,0) + (-.25,1.25) + (-45:2cm) $);
		\draw[decorate, decoration={brace,amplitude=10pt,mirror,raise=2pt+10pt}] (0,1) -- node[left,xshift=-28pt] {$ \Delta z $} (0,-1);
	\end{tikzpicture}
\end{minipage}%

% andrez 6 oder 5:
\subsection{Randbedingungen an el. Leitern}

\begin{minipage}{.6\linewidth}
	Leiter: Material mit freibeweglichen Ladungsträgern (Metall)\\
	Eigenschaften von $\vec{E}$ im Leiter:
\end{minipage}%
\begin{minipage}{.4\linewidth}
	%t15:
	\centering
	\begin{tikzpicture}
		\draw[thick] (0,0) arc (90:0:2.5cm);
		\draw[draw=none,pattern=north east lines] (0,0) arc (90:0:2.5cm) -- ++(-2.5,0) -- (0,0);
		\node[right] at (1.5,0) {Vakuum};
		\node[fill=white,opacity=.8,rounded corners] at (1,-2) {Leiter};
		\node[fill=white,opacity=.8,rounded corners] at (1,-1) {$ \vec{E} = 0 $};
	\end{tikzpicture}
\end{minipage}%
\\[10pt]
\begin{minipage}{.6\linewidth}
	\begin{enumerate}[i)]
		\item $\vec{E} = 0$
		\item $0 = \vabla \cdot \vec{E} = \frac{1}{\epsilon_0} \rho, \qquad \rho(\vec{r}) = 0$
		\item Nettoladung befinden sich an Oberfläche %t16
		\item Potential $\Phi(\vec{r}_b) - \Phi(\vec{r}_a) = 0 \quad \rightarrow \quad \Phi(\vec{r}) = \const$ % t17
	\end{enumerate}
\end{minipage}%
\begin{minipage}{.4\linewidth}
		%t16:
		\centering
		\begin{tikzpicture}[scale=.8]
		\draw[pattern=north east lines] (0,0) circle (1cm);
		\draw[fill=white,opacity=.6,draw=none] (0,0) circle (.995cm);
		\foreach \x/\xtext in {0/$+$,45/$+$,90/$+$,135/$+$,180/$+$,225/$+$,270/$+$,315/$+$}
		\node at (\x:.8cm) {\xtext};
		\draw[<-] (-45:1.2cm) -- ++(.5,-.2) node[right] {$ \substack{\tx{Flächenladung} \\ \sigma} $};
		\end{tikzpicture}\\
		%t17:
		\begin{tikzpicture}[scale=.8]
		\draw[pattern=north east lines] (0,0) circle (1cm);
		\draw[fill=white,opacity=.7,draw=none] (0,0) circle (.995cm);
		\node[circle,fill=black,inner sep=1pt,minimum size=1pt] (b) at (-.7,.25) {};
		\node[circle,fill=black,inner sep=1pt,minimum size=1pt] (a) at (.7,.25) {};
		\draw (a) --(b);
		\node[below] at (a) {$ \vec{r_a} $};
		\node[below] at (b) {$ \vec{r_b} $};
		\end{tikzpicture}\hspace{66pt}
\end{minipage}%

\paragraph{Randbedingungen}
%
%
%
% I27 T18
%
%
%
$$\vec{E}_+ - \vec{E}_- = \frac{\sigma^-}{\epsilon_0} \vec{n}$$
$$\vec{E}_- = 0$$
$$\rightarrow \vec{E}_+(\vec{r}) = \frac{\sigma(\vec{r})}{\epsilon_0} \vec{n} {(\vec{r})}$$
\folie{Ladung an Oberfläche eines Leiters}


% 5.11.18

\section{Randwertprobleme (RWP) der Elektrostatik und \texorpdfstring{\\}{newline} Lösungsmethoden}

\subsection{Formulierung des Randwertproblems}

Das elektrische Potential: $ \Phi(\vec{r}): \quad \vec{E}(\vec{r}) = - \vec{\nabla} \Phi(\vec{r}) $\\
$$ \vec{\Delta} \Phi (\vec{r}) = -\frac{1}{\epsilon_0} \rho(\vec{r})  \qquad \tx{Poisson-Gleichung}$$
Für eine gegebene lokale Ladungsverteilung $ \rho $ gilt:
\begin{equation*}
\Phi(\vec{r}) = \frac{1}{4 \pi \epsilon_0} \int \dd^3 r' \frac{\rho(\vec{r}')}{|\vec{r} - \vec{r}'|}
\end{equation*}
%
%
%
% I28 evtl |r| gegen 0 aber |r| gegen unendlich macht mehr sinn
%
%
%
\begin{equation*}
\rightarrow \Phi(\vec{r}) \buildrel |\vec{r}| \rightarrow \infty \over \longrightarrow 0
\end{equation*}
Typische Problemstellung:\\
Ladungsverteilung $\rho$ + Werte des Potentials auf Randfläche\\[5pt]
\bei\\
\begin{minipage}{.6\linewidth}
	Randwertproblem: Gegeben: $\rho(\vec{r}')$ im Raumbereich $V$\\
	$\Phi(\vec{r})$ oder $\vec{E}(\vec{r})$ auf Randfläche $\partial V$\\
	Gesucht: $\Phi(\vec{r}),\ \vec{E}(\vec{r})$ überall in $V$
\end{minipage}%
\begin{minipage}{.4\linewidth}
	%t1:
	\centering
	\begin{tikzpicture}
		\draw[pattern=north east lines] (0,0) circle (.7cm);
		\draw[thick] (-.7,0) -- ++(-.5,0) -- ++(0,-1);
		\coordinate (e) at (-1.2,-1);
		\draw[thick] ($ (e) + (-.25,0) $) -- ++(.5,0);
		\draw[thick] ($ (e) + (-.125,0) + (0,-.1) $) -- ++(.25,0);
		\draw[thick] ($ (e) + (-.0625,0) + (0,-.2) $) -- ++(.125,0);
		\draw (.25,.15) to[out=55,in=190] (1.2,.8) node[right] {$ \Phi = 0 $};
		\node[fill=black,circle,inner sep=1pt,minimum size=1pt] (q) at (1.5,0) {};
		\node[right] at (q) {$ q $};
	\end{tikzpicture}
\end{minipage}%
\\
Zwei Fälle:
\begin{enumerate}[i)]
	\item $ \Phi(\vec{r}) $ ist auf der Randfläche gegeben\\
	$ \rightarrow $ \textbf{Dirichlet-Randbedingung}
	\item $ \vec{E}(\vec{r}) $ ist auf der Randfläche gegeben\\
	$ \rightarrow $ \textbf{Neumannsche Randbedingung}
\end{enumerate}
\noindent
\begin{minipage}{.6\linewidth}
	Gegeben sei: $ \vec{n} \cdot \vec{E} $ dies ist gleich der \textbf{Normalenableitung}:
	\begin{equation*}
	\vec{n} \cdot \vec{E} = - \vec{n} \vabla \Phi = - \prt{\Phi}{n}
	\end{equation*}
\end{minipage}%
\begin{minipage}{.4\linewidth}
	%t2:
	\centering
	\begin{tikzpicture}[scale=.8]
		\draw[draw=none,pattern=north east lines] (0,0) -- (120:1.5cm) arc (120:-20:1.5cm) -- (0,0);
		\draw ($ (0,0) + (120:1.5cm)$) arc (120:-20:1.5cm);
		\draw[thick,->] ($ (50:1.5cm) $) -- ++(50:1cm) node[xshift=-15pt,yshift=-5pt] {$ \vec{n} $};
	\end{tikzpicture}
	\vspace{5pt}
\end{minipage}%
\\
\lcom{Wir beschränken uns vorwiegend auf den ersten Fall. }
\lcom{Zur Lösung dieser Probleme gibt es einige Methoden. Zum Einstieg und zur  betrachten wir zunächst die Methode der Spiegelladung. }

\subsection{Methode der Bildladung (Spiegelladung)}

\subsubsection{Punktladung vor leitender, geerdeter Metallplatte}

\begin{minipage}{.6\linewidth}
	\begin{equation*}
	\vec{\Delta} \Phi(\vec{r}) = - \frac{1}{\epsilon_0} \rho (\vec{r}) = - \frac{q}{\epsilon_0} \delta(\vec{r} - \vec{r}_0)
	\end{equation*}
	$ \vec{r} \in V  \qquad \vec{r}_0 = (d,0,0) \qquad V = \{\vec{r} \in \mathbb{R}^3 , x > 0\}$\\[5pt]
	Randbedingungen:
	\begin{equation*}
	\pofr = 0 \quad \tx{für} \quad \vec{r} \in \partial V ,\quad \quad \tx{d.h.} \ \vec{r} = (0,y,z)
	\end{equation*}
\end{minipage}%
\begin{minipage}{.4\linewidth}
	%t3:
	\centering
	\begin{tikzpicture}
		\draw[thick,->] (-.7,0) -- (2,0) node[anchor=north west] {$ x $};
		\draw ($ (-.1,0) + (0,-1.5) $) -- ++(0,3);
		\draw ($ (.1,0) + (0,-1.5) $) -- ++(0,3);
		\draw[draw=none,pattern=north east lines] ($ (-.1,0) + (0,-1.5)  $) rectangle ($ (.1,0) + (0,1.5) $);
		\draw[thick] (0,-.1) -- (0,.1) node[anchor=north east] {$ 0 $};
		\node[fill=black,circle,inner sep=1pt,minimum size=1pt] (q) at (1.5,0) {};
		\node[above] at (q) {$ q $};
		\draw[decorate, decoration={brace,amplitude=5pt,mirror,raise=2pt}, yshift=-4pt,black] (0,0) -- node[below=10pt] {$ d $} (1.5,0);
		\coordinate (e) at (-1,-1.5);
		\draw[thick] (e) -- ++(0,.7) -- ++(.9,0);
		\draw[thick] ($ (e) + (-.25,0) $) -- ++(.5,0);
		\draw[thick] ($ (e) + (-.125,0) + (0,-.1) $) -- ++(.25,0);
		\draw[thick] ($ (e) + (-.0625,0) + (0,-.2) $) -- ++(.125,0);
		\draw (0,.5) to[out=120,in=10] (-1.2,.8) node[left] {$ \Phi = 0 $};
		\node at (1.8,1.2) {$ V \ \ \Phi(\vec{r}) = \ ? $};
	\end{tikzpicture}
	\vspace{10pt}
\end{minipage}%
\\
\textbf{Idee:} Ersetze ursprüngliche Problem durch ,,Fiktives`` Problem mit zusätzlichen Ladungen außerhalb von $V$, welche die Randbedingungen simulieren.\\[5pt]
Potential der Punkladungen in $\vec{r}_0$:
$$\Phi_q (\vec{r}) = \kq \frac{q}{|\vec{r} - \vec{r}_0|}$$
\noindent
\begin{minipage}{.5\linewidth}
	addiere Ladung $-q$ in $\vec{r}'_0 = (-d, 0, 0) = -\vec{r}_0$\\[5pt]
	\begin{equation*}
	\pofr = \kq \left( \frac{q}{|\vec{r} - \vec{r}_0|} - \frac{q}{|\vec{r}+\vec{r}_0|} \right)
	\end{equation*}
\end{minipage}%
\begin{minipage}{.5\linewidth}
	%t4:
	\centering
	\begin{tikzpicture}
		\draw[->] (-2,0) -- (2,0) node[anchor=north west] {$ x $};
		\draw[->] (0,-1) -- (0,1) node[anchor=south east] {$ y $};
		\coordinate (q) at (1.5,0);
		\coordinate (-q) at (-1.5,0);
		\node[circle,fill=black,inner sep=1pt,minimum size=1pt] at (-q) {};
		\node[circle,fill=black,inner sep=1pt,minimum size=1pt] at (q) {};
		\node[above] at (-q) {$ -q $};
		\node[above] at (q) {$ q $};
		\draw[decorate,decoration={brace,amplitude=5pt,raise=2pt},yshift=-4pt,black] (0,0) -- node[below=10pt] {$ d $} (-1.5,0);
		\draw[decorate,decoration={brace,amplitude=5pt,mirror,raise=2pt},yshift=-4pt,black] (0,0) -- node[below=10pt] {$ d $} (1.5,0);
	\end{tikzpicture}
\end{minipage}%
\\
Schauen wir nun nach, ob dies die Poisson-Gleichung erfüllt:
\begin{align*}
\vec{\Delta} \Phi &= \frac{q}{4 \pi \epsilon_0} \Bigg( \ub{\Delta \frac{1}{|\vec{r} - \vec{r}_0|}}_{= - 4 \pi \delta(\vec{r} - \vec{r}_0)} - \ub{\Delta \frac{1}{|\vec{r} + \vec{r}_0|}}_{= - 4 \pi \delta(\vec{r} + \vec{r}_0)}\Bigg)\\
&= \underset{\checkmark}{- \frac{q}{\epsilon_0} \delta(\vec{r} - \vec{r}_0)} + \frac{q}{\epsilon_0} \ub{\delta(\vec{r} + \vec{r}_0)}_{\mathclap{\qquad = 0 \ \tx{für} \ \vec{r} \neq - \vec{r}_0 \ \checkmark \quad \forall \vec{r} \in V}}
\end{align*}

\subsubsection{Diskussion der Lösung}
\begin{enumerate}[i)]
	\item \textbf{Struktur}
	$$\pofr = \underbrace{\frac{q}{4 \pi \epsilon_0} \frac{1}{|\vec{r}-\vec{r}_0|}}_{\eqdef\ \Phi_{\textrm{s}}(\vec{r})} + \underbrace{\frac{(-q)}{4 \pi \epsilon_0} \frac{1}{|\vec{r}+\vec{r}_0|}}_{\eqdef\ \Phi_{\textrm{hom}}(\vec{r})}$$ 
	$ \vec{r} \in V $
	\begin{equation*}
	\Delta \Phi_\tx{s}(\vec{r}) = - \frac{1}{\epsilon_0} \rho(\vec{r}) \quad \tx{Poisson-Gleichung}
	\end{equation*}
	\begin{equation*}
	\Delta \Phi_{\tx{hom}} (\vec{r}) = 0 \qquad \tx{Laplace-Gleichung}
	\end{equation*}
	Mathematisch: Lösung inhomogener DGL
	\begin{equation*}
	\pofr = \Phi_\tx{s}(\vec{r}) + \Phi_{\tx{hom}}(\vec{r})
	\end{equation*}
	$ \Phi_{\tx{hom}} $ wird so gewählt, dass die Randbedingungen erfüllt werden:
	\begin{equation*}
	\vec{r} \in \partial V: \quad \Phi_\tx{o} (\vec{r}) = \Phi_\tx{s}(\vec{r}) + \Phi_{\tx{hom}}(\vec{r})
	\end{equation*}	
	\item \textbf{Elektrisches Feld}
	\begin{equation*}
	\vec{E} = - \vec{\nabla} \Phi = \frac{q}{4 \pi \epsilon_0} \left(\frac{(x-d,y,z)}{|\vec{r} - \vec{r}_0|^3} - \frac{(x+d,y,z)}{|\vec{r} + \vec{r}_0|^3}\right)
	\end{equation*}
	An der Oberfläche $ x \to 0 $, $ x \ge 0 $\\
	$ |\vec{r} \pm \vec{r}_0|^3 \rightarrow (d^2 + y^2 + z^2) $\\
	\begin{minipage}{.5\linewidth}
		\begin{equation*}
		\eofr \bigg|_{\vec{r} \in \partial V} = - \frac{qd}{2 \pi \epsilon_0} \frac{1}{(d^2 + y^2 + z^2) ^{3/2}} \vec{e}_x
		\end{equation*}
		\lcom{Durch das externe elektrische Feld verschieben sich die Ladungsträger im Metall und es entsteht eine Influenzladung an der Oberfläche.}
	\end{minipage}%
	\begin{minipage}{.5\linewidth}
		\centering
		%t5: \draw  plot [smooth cycle, tension=1] coordinates { (0,0) (.5,1.5) (.5,2.5) (2,2) (3,2) (2.5,1) (1.5,0) };
		\begin{tikzpicture}
			\draw[->] (-.5,0) -- (2.5,0) node[anchor=north west] {$ x $};
			\coordinate (q) at (1.5,0);
			\node[circle,fill=black,inner sep=1pt,minimum size=2pt] at (q) {};
			\node[yshift=15pt,xshift=30pt] at (q) {$ q > 0 $};
			\draw[->] (0,-2) -- (0,2) node[anchor=south west] {$ z $};
			\draw[draw=none,pattern=north west lines] (-.3,1.5) node[above,xshift=-5pt] {$ \sigma(z) $} rectangle (0,-1.5);
			\draw[thick,->] (q) to[out=180,in=0] (0,0);
			\draw[thick,->] (q) to[out=155,in=0] (0,.20);
			\draw[thick,->] (q) to[out=90,in=0] (0,.50);
			\draw[thick,->,looseness=1.3] (q) to[out=65,in=0] (0,.85);
			\draw[thick,->,looseness=1.6] (q) to[out=45,in=0] (0,1.3);
			\draw[thick,->,looseness=1.3] (q) to[out=25,in=-45] ($ (q) + (.4,1) $);
			\draw[thick,->] (q) to[out=-155,in=0] (0,-.20);
			\draw[thick,->] (q) to[out=-90,in=0] (0,-.50);
			\draw[thick,->,looseness=1.3] (q) to[out=-65,in=0] (0,-.85);
			\draw[thick,->,looseness=1.6] (q) to[out=-45,in=0] (0,-1.3);
			\draw[thick,->,looseness=1.3] (q) to[out=-25,in=45] ($ (q) + (.4,-1) $);
			\coordinate (n) at (-.5,0);
			\node[rounded corners,align=center,left] at (n) {$ \vec{E}_{\parallel} = 0 $ \\ \\ $ \vec{E}_{\perp} \neq 0 $};
		\end{tikzpicture}
	\end{minipage}%
	\item \textbf{Influenzladung auf Metalloberfläche}
	$$\vec{E}_+ - \equalto{\vec{E}_-}{0} = \frac{\sigma}{\epsilon_0} \vec{n} \qquad \vec{n} = \vec{e}_x$$
	$\vec{r} \in \partial V$:
	$$\sigma (\vec{r}) = \epsilon_0 \vec{E}_+(\vec{r}) = - \frac{qd}{2 \pi(d^2 + y^2 + z^2)^{\nicefrac{3}{2}}}$$
	gesamte influenzierte Ladung
	$$q_i = \int_{\partial V} \dd f\ \sigma(\vec{r}) = \dots = -q$$
	\item \textbf{Kraft zwischen Punktladungen und Metallplatte}
	$$\vec{F} = q \vec{\tilde{E}}(\vec{r}_0) = \frac{-q^2}{4 \pi \epsilon_0 (2d)^2}\vec{e}_x$$
\end{enumerate}

% andrez 5:
\subsubsection{Eindeutigkeit der Lösung des Randwertproblems}

\begin{minipage}{.6\linewidth}
	\textbf{Dirichlet-Randwertproblem:}
	\begin{align*}
	\Delta \pofr = -\frac{1}{\epsilon_0} \rho(\vec{r}) &\qquad \vec{r} \in V\\
	\pofr = \Phi_0 (\vec{r}) \quad \; &\qquad \vec{r} \in \partial V \quad \;
	\end{align*}
\end{minipage}%
\begin{minipage}{.4\linewidth}
	\centering
	\begin{tikzpicture}
		\draw (0,1) -- (0,-1);
		\draw[draw = none,pattern = north east lines] (-.3,1) rectangle (0,-1);
		\node[fill=black,circle,inner sep=1pt,minimum size=1pt] (p) at (-.1,0) {};
		\node[fill=black,circle,inner sep=1pt,minimum size=1pt] (q) at (1,0) {};
		\node[right] at (q) {$ q $};
		\node at (1,.5) {$ V $};
		\node at (0,1.3) {$ \partial V $};
		\draw[thick] (p) to[out=135,in=-10] (-.5,.3) node[left] {$ \Phi_0 $};
	\end{tikzpicture}
\end{minipage}%
\\
Annahme: $\Phi_1,\ \Phi_2$ lösen RWP
\begin{align*}
\textrm{d.h. }\qquad \Delta\Phi_1(\vec{r}) = -\frac{1}{\epsilon_0} \rho(\vec{r}) = \Delta \Phi_2 (\vec{r}) &\qquad \vec{r} \in V\\
\Phi_1 (\vec{r}) = \Phi_0(\vec{r}) = \Phi_2(\vec{r}) \quad \ &\qquad \vec{r} \in \partial V
\end{align*}
Setze:
\begin{equation*}
\varPsi(\vec{r}) \defeq \Phi_1(\vec{r}) - \Phi_2(\vec{r})
\end{equation*}
\begin{equation*}
\Delta \Phi(\vec{r}) = 0 \quad \vec{r} \in V 
\end{equation*}
\begin{equation*}
\vec{r} \in \partial V \quad \varPsi(\vec{r}) = \Phi_1(\vec{r}) - \Phi_2(\vec{r}) = 0
\end{equation*}

\subsubsection{Greensche Identität:}
$ g,h $ Funktionen an V:
\begin{align*}
&\ \ \ \, \int_V \dd^3 r \left[(\left(\vec{\nabla} g (\vec{r})\right) \cdot \left(\vec{\nabla} h(\vec{r}))\right) + g(\vec{r}) \Delta h(\vec{r})\right]\\
&= \int_{\partial V} \dd\vec{f} \cdot \left(g(\vec{r}) \vec{\nabla} h(\vec{r})\right)\\
&= \int_{\partial V} \dd f g(\vec{r}) \ub{\vec{n} \cdot \vec{\nabla} h(\vec{r})}_{= \frac{\partial h}{\partial n} (\vec{r})}
\end{align*}

% andrez 6:
$h = g = \varPsi$
$$\Rightarrow \quad \int_V \dd^3 r\ ((\vabla \varPsi)^2 + \varPsi(\vec{r}) \underbrace{\Delta \varPsi(\vec{r})}_{=0}) = \int_{\partial V} \dd f\ \underbrace{\varPsi(\vec{r})}_{=0} \frac{\partial \varPsi(\vec{r})}{\partial n} $$
\begin{equation*}
\Rightarrow \int_V \dd^3 r \left(\vec{\nabla} \varPsi(\vec{r}) \right)^2 = 0 \quad \Rightarrow \quad \vec{\nabla} \varPsi(\vec{r}) = 0 \qquad \vec{r} \in V
\end{equation*}
\begin{equation*}
\varPsi(\vec{r}) = \const \qquad \varPsi(\vec{r}) = 0 \ \tx{in} \ V \ \ \Rightarrow \ \ \Phi_1(\vec{r}) = \Phi_2(\vec{r})
\end{equation*}

\subsection{Formale Lösungen des elektrostatischen Randwertproblems mit \texorpdfstring{\\}{newline} Greenschen Funktionen (GF)}
GF: generelle Methode um inhomogene DGL zu lösen
\begin{equation*}
\Delta \pofr = - \frac{1}{\epsilon_0} \rho(\vec{r})
\end{equation*}
Greensche Funktionen der Poisson-Gleichung: $ \gre(\vec{r},\vec{r}') $ mit
\frbox{Greensche Funktionen der Poisson-Gleichung}{\begin{equation*}
\Delta_{\vec{r}} \gre(\vec{r},\vec{r}') = - \frac{1}{\epsilon_0} \delta(\vec{r} - \vec{r}')
\end{equation*}}
\noindent
\lcom{Diese Gleichung geht vor einer Punktladung mit $ q=1 $ aus, ist hier aber zunächst einmal eine Definition.}

% andrez 7:
\noindent
$\mathcal G$ bekannt 
$$\rightarrow \pofr = \int \dd^3 r'\ \mathcal G(\vec{r}, \vec{r}') \rho(\vec{r}')$$
$$\Delta_{\vec{r}} \pofr = \int \dd^3 r'\ \underbrace{(\Delta_{\vec{r}} \mathcal G(\vec{r}, \vec{r}'))}_{= -\frac{1}{\epsilon_0} \delta(\vec{r}-\vec{r}')} \rho (\vec{r}') = - \frac{1}{\epsilon_0} \rho (\vec{r})\ \checkmark$$
\begin{center}
	\begin{minipage}{.45
			\linewidth}
		\rbox{
			\begin{equation*}
			\Delta_{\vec{r}} \frac{1}{|\vec{r} - \vec{r}'|} = - 4 \pi \delta(\vec{r} - \vec{r}')
			\end{equation*}
			\begin{equation*}
			\gre(\vec{r},\vec{r}') = \frac{1}{4 \pi \epsilon_0} \frac{1}{|\vec{r} - \vec{r}'|}
			\end{equation*}
			\begin{equation*}
			\rightarrow \quad \Delta_{\vec{r}} \gre(\vec{r}, \vec{r}') = - \frac{1}{\epsilon_0} \delta(\vec{r} - \vec{r}') \phantom{\quad \rightarrow}
			\end{equation*}
		}
	\end{minipage}
\end{center}
\begin{equation*}
\pofr = \frac{1}{4 \pi \epsilon_0} \int \dd^3 r \frac{\rho(\vec{r}')}{|\vec{r} - \vec{r}'|}
\end{equation*}
\begin{equation*}
\pofr \buildrel |\vec{r}| \to \infty \over \longrightarrow 0
\end{equation*}

% andrez 8:
$$\rightarrow\ \Delta_{\vec{r}} \mathcal G(\vec{r}, \vec{r}') = - \frac{1}{\epsilon_0} \delta(\vec{r} - \vec{r}')$$
$$\mathcal G(\vec{r}, \vec{r}') \underset{|\vec{r}| \to \infty} \longrightarrow 0 $$

\subsubsection{Dirichlet-Randwertproblem}

\begin{minipage}{.6\linewidth}
	\begin{align*}
	\Delta \pofr = -\frac{1}{\epsilon_0} \rho(\vec{r}) &\qquad \vec{r} \in V\\
	\pofr = \Phi_0 (\vec{r}) \quad \; &\qquad \vec{r} \in \partial V \quad \;
	\end{align*}
\end{minipage}%
\begin{minipage}{.4\linewidth}
	\centering
	\begin{tikzpicture}
	\draw (0,1) -- (0,-1);
	\draw[draw = none,pattern = north east lines] (-.3,1) rectangle (0,-1);
	\node[fill=black,circle,inner sep=1pt,minimum size=1pt] (p) at (-.1,0) {};
	\node[fill=black,circle,inner sep=1pt,minimum size=1pt] (q) at (1,0) {};
	\node[right] at (q) {$ q $};
	\node at (1,.5) {$ V $};
	\node at (0,1.3) {$ \partial V $};
	\draw[thick] (p) to[out=135,in=-10] (-.5,.3) node[left] {$ \Phi_0 $};
	\end{tikzpicture}
\end{minipage}%
\\
\textbf{Green'sche Funktionen (GF):}
\begin{align*}
\Delta_{\vec{r}} \mathcal G(\vec{r}, \vec{r}') &= - \frac{1}{\epsilon_0} \delta(\vec{r} - \vec{r}') & \vec{r},\vec{r}' \in V\\
\mathcal G(\vec{r}, \vec{r}') &= 0 & \textrm{ für } \qquad \vec{r} \in \partial V \ \  \vec{r}' \in V
\end{align*}
\lcom{Hiermit haben wir das Grenzwertproblem auf eine Integration zurückgeführt. Dies werden wir nun Beweisen:}\\[5pt]
\textbf{Beiweis:}\\
Die 2. Greensche Identität lautet: 
\begin{align*}
&\ \ \ \, \int_V \dd^3 r' \left( g(\vec{r}') \Delta_{\vec{r}'} h(\vec{r}')  - h(\vec{r}') \Delta_{\vec{r}'} g(\vec{r}') \right)\\
&= \int_{\partial V} \dd\vec{f}' \cdot \left(g(\vec{r}') \vec{\nabla}_{\vec{r}'} h(\vec{r}') - h(\vec{r}') \vec{\nabla} _{\vec{r}'} g(\vec{r}') \right)
\end{align*}
\begin{equation*}
g(\vec{r}') \defeq \Phi(\vec{r}') \qquad h(\vec{r}') \defeq \gre(\vec{r}',\vec{r})
\end{equation*}
\begin{align*}
\Rightarrow & \ \ \ \, \int_V \dd^3 r' \Bigg[ \Phi(\vec{r}') \ub{\Delta_{\vec{r}'} \gre(\vec{r}',\vec{r})}_{= -\frac{1}{\epsilon_0} \delta(\vec{r}' - \vec{r})} - \gre(\vec{r}',\vec{r}) \ub{\Delta_{\vec{r}'} \Phi(\vec{r}')}_{= - \frac{1}{\epsilon_0} \rho(\vec{r}')} \Bigg]\\
&= \int_{\partial V} \dd\vec{f}' \Bigg[ \ub{\Phi(\vec{r}')}_{= \Phi_0(\vec{r}')} \vec{\nabla}_{\vec{r}'} \gre(\vec{r}',\vec{r}) - \ub{\gre(\vec{r}',\vec{r})}_{= 0} \vec{\nabla}_{\vec{r}'} \Phi(\vec{r}') \Bigg]\\
% andrez 9:
\Rightarrow &= - \frac{1}{\epsilon_0} \pofr + \frac{1}{\epsilon_0} \int_V \dd^3 r'\ \mathcal G(\vec{r}, \vec{r}') \rho(\vec{r})\\
&= \int_{\partial V} \dd \vec{f}'\ \Phi_0 (\vec{r}') \vabla_{\vec{r}'} \mathcal G(\vec{r}, \vec{r}')\\
&= \int_{\partial V} \dd f'\ \Phi_0 (\vec{r}') \prt{}{n'} \gre(\vec{r}, \vec{r}')
\end{align*}
\begin{equation*}
\Rightarrow \pofr = \int_V \dd^3r'\ \mathcal G(\vec{r}, \vec{r}') \rho(\vec{r}') - \epsilon_0 \int_{\partial V} \dd f'\ \Phi_0 (\vec{r}') \prt{}{n'} \gre (\vec{r}, \vec{r}')
\end{equation*}

Es gilt (HA):
\begin{equation*}
\gre(\vec{r},\vec{r}') = \gre(\vec{r}',\vec{r}) \qquad \tx{Reziprozität}
\end{equation*}
\begin{equation*}
\rightarrow \vec{\nabla}_{\vec{r}'} \gre(\vec{r},\vec{r}') = \vec{\nabla}_{\vec{r}'} \gre(\vec{r}',\vec{r})
\end{equation*}
\begin{equation*}
\phantom{\rightarrow} \ \ \Delta_{\vec{r}} \gre(\vec{r},\vec{r}') = \Delta_{\vec{r}'} \gre(\vec{r},\vec{r}')
\end{equation*}

\vspace{10pt}
\frbox{Potential bei Randwertproblem}{
\begin{equation}
\pofr = \int_{V} \dd^3 r' \grr \rho(\vec{r}') - \epsilon_0 \int_{\partial V} \dd f' \Phi_0(\vec{r}') \prt{}{n'} \grr \tag{1}
\label{pot}
\end{equation}}

% Vorlesung 8.11.18

\bbb{Wiederholung}{
	\begin{minipage}{.6\linewidth}
		\begin{align*}
		\Delta \pofr = -\frac{1}{\epsilon_0} \rho(\vec{r}) &\qquad \vec{r} \in V\\
		\pofr = \Phi_0 (\vec{r}) \quad \; &\qquad \vec{r} \in \partial V \quad \;
		\end{align*}
	\end{minipage}%
	\begin{minipage}{.4\linewidth}
		\centering
		\begin{tikzpicture}
		\draw (0,1) -- (0,-1);
		\draw[draw = none,pattern = north east lines] (-.3,1) rectangle (0,-1);
		\node[fill=black,circle,inner sep=1pt,minimum size=1pt] (p) at (-.1,0) {};
		\node[fill=black,circle,inner sep=1pt,minimum size=1pt] (q) at (1,0) {};
		\node[right] at (q) {$ q $};
		\node at (1,.5) {$ V $};
		\node at (0,1.3) {$ \partial V $};
		\draw[thick] (p) to[out=135,in=-10] (-.5,.3) node[left] {$ \Phi_0 $};
		\end{tikzpicture}
	\end{minipage}%
	\\
	Green'sche Funktionen:
	\begin{align*}
	\Delta_{\vec{r}} \mathcal G(\vec{r}, \vec{r}') &= - \frac{1}{\epsilon_0} \delta(\vec{r} - \vec{r}') &\vec{r},\vec{r}' \in V\\
	\mathcal G(\vec{r}, \vec{r}') &= 0 &\textrm{ für } \qquad \vec{r} \in \partial V \ \  \vec{r}' \in V
	\end{align*}
	Wenn die Green'sche Funktion $ \gre $ die Bedingungen erfüllt, können wir das Potential so schreiben wie in Gleichung \eqref{pot}.
}

\pagebreak
\noindent
\emph{Bemerkungen:}
\begin{enumerate}[i)]
	\item Spezialfälle:
	\begin{enumerate}[1)]
		\begin{minipage}{.5\linewidth}
			\item $ V $ Ladungsfrei ($ \rho(\vec{r}) = 0 $ in $ V $)
			\begin{equation*}
			\rightarrow \Phi(\vec{r}) = - \epsilon_0 \int_{\partial V} \dd f' \Phi_0(\vec{r}) \frac{\partial \mathcal{G}}{\partial n'} (\vec{r},\vec{r}')
			\end{equation*}
		\end{minipage}%
		\begin{minipage}{.5\linewidth}
			%t1:
			\centering
			\begin{tikzpicture}[scale=1.8]
				\draw[draw=none,pattern=north east lines,rounded corners] ($(-.7,.5) + (-.1,.1)$) rectangle ($(.7,-.5) + (.1,-.1)$);
				\draw[fill=gray!10] (-.7,.5) rectangle (.7,-.5);
				\draw (-.7,-.5) -- (.7,-.5) -- (.7,.5) -- (-.7,.5) -- (-.7,-.5);
				\node[left] at (-1,0) {$ \Phi_1 $};
				\node[right] at (1,0) {$ \Phi_1 $};
				\node[above] at (0,.7) {$ \Phi_2 $};
				\node[below] at (0,-.7) {$ \Phi_2 $};
				\node at (0,0) {$ V $};
			\end{tikzpicture}
		\end{minipage}%
		\\
		\begin{minipage}{.4\linewidth}
			\centering
			%t2:
			\begin{tikzpicture}
				\draw[scale=1.4,looseness=1.5,draw=none,pattern=north east lines] (-.5,0) to[out=90,in=180] (0,1.3) to[out=0,in=70] (.5,0) to[out=-110,in=0] (-.3,-.9) to[out=180,in=-90] (-.5,0);
				\draw[looseness=1.5,fill=white] (-.5,0) to[out=90,in=180] (0,1.5) to[out=0,in=70] (.5,0) to[out=-110,in=0] (-.3,-1) to[out=180,in=-90] (-.5,0);
				\node at (0,0) {$ V $};
				\draw[thick] (.5,0) to[out=50,in=180] (1,.3) node[right] {$ \Phi_0 = \const $};
			\end{tikzpicture}
		\end{minipage}%
		\begin{minipage}{.6\linewidth}
			\begin{align*}
			\Rightarrow \Phi(\vec{r}) &= - \epsilon_0 \Phi_0 \ub{\int_{\partial V} \dd f' \frac{\partial \gre}{\partial n'} (\vec{r},\vec{r}') }_{\int \dd f' \vec{n} \cdot \vec{\nabla}_{\vec{r}'} \gre}\\
			&= - \epsilon_0 \Phi_0 \int \dd \vec{f}' \cdot \vec{\nabla}_{\vec{r}'} \gre \\
			&\buildrel \mathclap{\tx{S.v.G.}} \over = - \epsilon_0 \Phi_0 \int_V \dd^3 \vec{r}' \ub{\vec{\nabla}_{\vec{r}'} \cdot (\vec{\nabla}_{\vec{r}'} \mathcal{G})}_{\Delta_{\vec{r}'} \cdot \mathcal{G} = - \frac{1}{\epsilon_0} \delta(\vec{r} - \vec{r}')}\\
			&= - \epsilon_0 \Phi_0 \cdot (- \frac{1}{\epsilon_0})\\
			\Rightarrow \Phi(\vec{r}) &= \Phi_0
			\end{align*}
		\end{minipage}%
		\item $ V = \mathbb{R}^3 $, lokalisierte Ladungsverteilung $ \rho $
		\begin{equation*}
		\Phi(\vec{r}) = \frac{1}{4 \pi \epsilon_0} \int \dd^3 \vec{r}' \frac{\rho(\vec{r})}{|\vec{r} - \vec{r}'|}
		\end{equation*}
		\begin{equation*}
		\rmbox{ \gre(\vec{r},\vec{r}') = \kq \frac{1}{|\vec{r} - \vec{r}'|}} \qquad \int_{\partial V} \dots \rightarrow 0
		\end{equation*}
		eine \textbf{spezielle Lösung} für $ \gre $
	\end{enumerate}
	\item $ \gre $ ist auch die Lösung einer inhomogenen partiellen DGL
	\begin{equation*}
	\gre(\vec{r},\vec{r}') = \ub{\gre_s(\vec{r},\vec{r}')}_{\substack{\tx{spezielle} \\ \tx{Lösung der} \\ \tx{inhomogenen} \\ \tx{DGL}}} + \ub{F(\vec{r},\vec{r}')}_{\substack{\tx{Lösung} \\ \tx{zugehörigen} \\ \tx{homogenen} \\ \tx{DGL}}}
	\end{equation*}
	\begin{align*}
	\Delta_{\vec{r}'} \gre_s(\vec{r},\vec{r}') &= - \frac{1}{\epsilon_0} \delta(\vec{r} - \vec{r}')\\
	\Delta_{\vec{r}'} F(\vec{r},\vec{r}') \, &= 0
	\end{align*}
	\begin{equation*}
	\gre_j(\vec{r},\vec{r}') = \kq \frac{1}{|\vec{r} - \vec{r}'|} \qquad \tx{Laplace anwenden !}
	\end{equation*}
	%
	%
	%
	% I29 gre_ j soll wahrscheinlich gre_s sein
	%
	%
	%
	\begin{equation*}
	\gre(\vec{r},\vec{r}') = \ub{\kq \frac{1}{|\vec{r} - \vec{r}'|}}_{\tx{immer zur Lösung}} \quad + \ub{F(\vec{r},\vec{r}')}_{\substack{\tx{so wählen, dass} \\ \tx{Randbedingungen erfüllt}}}
	\end{equation*}
	$ F(\vec{r},\vec{r}') $ so wählen, dass die Randbedingungen erfüllt sind: $ \gre(\vec{r},\vec{r}') = 0 \quad \vec{r} \in \partial V $.
\end{enumerate}

\subsection{Greensche Funktion des Dirichlet Randwertproblems einer Ebene}

\begin{minipage}{.55\linewidth}
	\begin{align*}
	\Delta_{\vec{r}'} \gre(\vec{r},\vec{r}') &= - \frac{1}{\epsilon_0} \delta(\vec{r} - \vec{r}') & \vec{r},\vec{r}' \in V \\
	\gre(\vec{r},\vec{r}') &= 0 & \vec{r} \in \partial V \, \tx{ \scriptsize{(z=0)}} , \quad \vec{r}' \in V
	\end{align*}
	\begin{equation*}
	V=\{\vec{r} \in \mathbb{R}^3 | z < 0\}
	\end{equation*}
\end{minipage}%
\begin{minipage}{.45\linewidth}
	\flushright
	%t3:
	\begin{tikzpicture}[scale=1.1]
		\draw[draw=none,pattern=north east lines] (-.7,0) -- (2.7,0) -- ++(-1.2,-1.2) -- ++(-3.4,0);
		\draw[draw=none,fill=white,opacity=.6] (-.7,0) -- (2.7,0) -- ++(-1.2,-1.2) -- ++(-3.4,0);
		\draw[->] (0,0) -- (0,1.5) node[anchor=south east] {$ z $};
		\draw[->] (0,0) -- (-1.5,-1.5) node[anchor=north east] {$ x $};
		\draw[->] (-1,0) -- (3,0) node[anchor=north west] {$ y $};
		\node at (2,-.4) {$ \partial V $};
		\draw[thick,->] (0,0) -- (.7,.5) node[left,yshift=3pt] {$ \vec{r}' $};
		\draw[thick,->] (0,0) -- (.7,-1);
		\node[fill=white,opacity=.8,rounded corners] at (0,-.7) {$ \tilde{\vec{r}}' $};
		\node[fill=black,circle,inner sep=1pt,minimum size=1pt] at (.7,-.3) {};
		\node[fill=black,circle,inner sep=1pt,minimum size=1.5pt] (q1) at (.7,.5) {};
		\node[fill=black,circle,inner sep=1pt,minimum size=1.5pt] (q2) at (.7,-1) {};
		\node[anchor=south west] at (q1) {$ q = 1 $};
		\node[anchor=north west,yshift=-3pt] at (q2) {$ \substack{ \tx{\normalsize{$ - q $}} \\ \mathclap{\tx{\scriptsize{Spiegelladung}}}} $};
		\draw[thick,dashed] (.7,.5) -- (.7,-1);
		\draw[thick,->] (0,0) -- (-.7,.9) node[left] {$ \vec{r} $};
	\end{tikzpicture}
\end{minipage}%
\\
Analog: Punktladung ,,$ q = 1 $`` in $ \vec{r}' $ vor leitender Ebene mit Potential 0
\begin{equation*}
\Phi(\vec{r}) = \frac{q}{4 \pi \epsilon_0} \left(\frac{1}{|\vec{r} - \vec{r}'|} - \frac{1}{|\vec{r} - \tilde{\vec{r}}'|}\right) \qquad \tilde{\vec{r}}' = (x',y',-z')
\end{equation*}
\begin{equation*}
\rmbox{ \gre(\vec{r},\vec{r}') = \frac{1}{q} \Phi(\vec{r}) = \kq \left(\frac{1}{|\vec{r} - \vec{r}'|} - \frac{1}{|\vec{r} - \tilde{\vec{r}}'|}\right)}
\end{equation*}

\noindent
\emph{Beweis:}
\begin{equation*}
\Delta_{\vec{r}} \grr = \kq \bigg(\equalto{\Delta_{\vec{r}} \frac{1}{|\vec{r} - \vec{r}'|}}{- 4 \pi \delta(\vec{r} - \vec{r}')} - \equalto{\Delta_{\vec{r}} \frac{1}{|\vec{r} - \tilde{\vec{r}}'|}}{- 4 \pi \delta(\vec{r} - \tilde{\vec{r}}') = 0 }\bigg) = - \frac{1}{\epsilon_0} \delta(\vec{r} - \vec{r}')
\end{equation*}
1. Teil: $ \vec{r} \in \partial V : \quad z = 0 \ ,\qquad$ 2. Teil = 0: $ \quad \tilde{\vec{r}}' \notin V $ !!!
\begin{align*}
\frac{1}{|\vec{r} - \vec{r}'|} &= \frac{1}{\sqrt{(x-x')^2 + (y - y')^2 + (z')^2}} = \frac{1}{\sqrt{(x-x')^2 + (y-y')^2 + (-z)^2}}\\
&= \frac{1}{|\vec{r} - \tilde{\vec{r}}'|}
\end{align*}
$ \gre(\vec{r},\vec{r}') = 0 \quad \vec{r} \in \partial V $\\[5pt]
\emph{Bemerkung:}
\begin{enumerate}[i)]
	\item $ \hspace{1.5pt} \gre(\vec{r},\vec{r}') = \phantom{-} \kq \frac{1}{|\vec{r} - \vec{r}'|} + F(\vec{r},\vec{r}') $\\
	$ F(\vec{r},\vec{r}') = - \kq \frac{1}{|\vec{r} - \tilde{\vec{r}}'|} $\\
	$ \Delta_{\vec{r}} F(\vec{r},\vec{r}') = 0 \qquad \qquad $ (da $ \Delta_{\vec{r}'} F(\vec{r},\vec{r}') = - 4 \pi \delta (\vec{r} - \tilde{\vec{r}}') \quad $ und $ \tilde{\vec{r}}' \notin V $ !!!)
	\item Symmetrie der Greenschen Funktion (Reziprozitätsrelation):
	\begin{equation*}
	\gre(\vec{r},\vec{r}') = \gre(\vec{r}',\vec{r})
	\end{equation*}
	$ \rightarrow $ formale Lösung des Randwertproblems für eine beliebige Ladungsverteilung und Randwerte $ \Phi_0(\vec{r}) $ in der Ebene:\\
	\begin{minipage}{.55\linewidth}
			\begin{equation*}
			\Phi(\vec{r}) = \int_V \dd^3 r' \gre(\vec{r},\vec{r}') \rho(\vec{r}') - \epsilon_0 \int_{\partial V} \dd f' \Phi_0(\vec{r}') \prt{\gre}{n'}
			\end{equation*}
			%
			%
			%
			% I31 in der zeichnung soll es \rho \equiv 0 sein statt \Phi = 0 weil in der Rechnung wird \rho = 0 benutzt und ein minus fehlt
			%
			%
			%
			\begin{equation*}
			\rho \equiv 0 \ \ \ \Rightarrow \ \ \ \Phi(\vec{r}) = \epsilon_0 \hspace{-15pt} \int\limits_{\sqrt{x^2 + y^2} \le R} \hspace{-15pt} \dd y' \dd x' \Phi_0(x',y',0) \prt{\gre}{n'}
			\end{equation*}
	\end{minipage}%
	\begin{minipage}{.45\linewidth}
		%t4:
		\centering
		\begin{tikzpicture}
			\draw[draw=none,pattern=north east lines] ($ (-1.3,0) + (.6,.6) $) -- ++(3.1,0) -- ++(-1.4,-1.4) -- ++(-3.1,0);
			\draw[draw=none,fill=white,opacity=.7]  ($ (-1.3,0) + (.6,.6) $) -- ++(3.1,0) -- ++(-1.4,-1.4) -- ++(-3.1,0);
			\draw[fill=white] (0,0)ellipse (1cm and .5cm);
			\draw[->] (0,0) -- (0,1) node[anchor=south east] {$ z $};
			\draw[->] (.7,.7) -- (-1,-1) node[anchor=north east] {$ x $};
			\draw[->] (-1.5,0) -- (2,0) node[anchor=north west] {$ y $};
			\draw[pattern=north west lines] (0,0) ellipse (1cm and .5cm);
			\draw[thick] (0,0) -- (.45,-.45) node[fill=black,circle,inner sep=1pt,minimum size=1pt] {};
			\node[below,fill=white,opacity=.8,inner sep=2pt,rounded corners] at (.3,-.5) {$ R $};
			\node[fill=white,opacity=.8,inner sep=2pt,rounded corners] at (2,.5) {$ \Phi \equiv 0 $};
			\draw (-.5,.3) to[out=100,in=-10] (-1,1) node[left] {$ \Phi_0(x,y,0) $};
		\end{tikzpicture}
	\end{minipage}%
\end{enumerate}

\subsection{Separation der Variablen und Entwicklung nach orthogonalen Funktionen}

\textbf{Eine allgemeine Methode zur Lösung partieller DGL.}\\[5pt]
Zur Vereinfachung: \textbf{Laplace.Gl $ \boldsymbol{ \Delta \Phi = 0 \ \quad +} \quad $ Randbedingung}\\
Es soll also immer gelten $ \rho = 0 $\\[5pt]
Verbindung zur Poisson-Gl: $ \Delta \Phi(\vec{r}) = - \frac{1}{\epsilon_0} \rho(\vec{r}) $
\begin{equation*}
\Phi(\vec{r}) = \Phi_s(\vec{r}) + \Phi_{\tx{hom}} \qquad \Phi(\vec{r}) = \kq \int \dd^3 r' \frac{\rho(\vec{r}')}{|\vec{r} - \vec{r}'|} + \Phi_{\tx{hom}}
\end{equation*}

\subsubsection{Motivation: 1-Dim Randwertproblem}

\begin{minipage}{.5\linewidth}
	%t5:
	\centering
	\begin{tikzpicture}
		\coordinate (a) at (-1,0);
		\coordinate (b) at (1,0);
		\draw[->] (-2,0) -- (2,0) node[anchor=north west] {$ x $};
		\draw ($ (a) + (0,-1) $) -- ++(0,2) node[anchor=south east] {$ \Phi_1 $};
		\draw ($ (b) + (0,-1) $) -- ++(0,2) node[anchor=south west] {$ \Phi_2 $};
		\node[anchor=north west] at (a) {$ 0 $};
		\node[anchor=north east] at (b) {$ l_x $};
		\draw[draw=none,pattern=north east lines] ($ (a) + (0,-1) $) rectangle ++ (-.3,2);
		\draw[draw=none,pattern=north east lines] ($ (b) + (0,-1) $) rectangle ++ (.3,2);
		\draw[thick] ($ (a) + (45:.1) $) -- ++(-135:.2);
		\draw[thick] ($ (a) + (-45:.1) $) -- ++(135:.2);
		\draw[thick] ($ (b) + (45:.1) $) -- ++(-135:.2);
		\draw[thick] ($ (b) + (-45:.1) $) -- ++(135:.2);
	\end{tikzpicture}
	\vspace{5pt}
\end{minipage}%
\begin{minipage}{.5\linewidth}
	$$ \Phi(x) = ? \qquad \rho = 0 $$
	\begin{equation*}
	\Delta \Phi(x) = \frac{\dd^2 \Phi}{\dd x^2} = 0
	\end{equation*}
	\begin{equation*}
	\Rightarrow \Phi(x) = c_1 + c_2 x
	\end{equation*}
\end{minipage}%
\\
\textbf{Randbedingungen:}\\
$$ \Phi(0) = c_1 = \Phi_1  \qquad \Phi(l_x) = \Phi_1 + c_2 l_x = \Phi_2 $$
\begin{equation*}
\rightarrow c_2 = \frac{\Phi_2 - \Phi_1}{l_x} \quad \rightarrow \quad \Phi(x) = \Phi_1 + \frac{\Phi_2 - \Phi_1}{l_x} x
\end{equation*}
\begin{equation*}
\Rightarrow \vec{E} = - \vec{\nabla} \Phi = - \frac{\Phi_2 - \Phi_1}{l_x} \vec{e}_x
\end{equation*}

\subsubsection{2-Dim Randwertproblem} \label{tikz}

\begin{minipage}{.5\linewidth}
	%t6:
	\centering
	\begin{tikzpicture}
		\coordinate (x) at (3.5,0);
		\coordinate (y) at (0,2.5);
		\draw[draw=none,pattern=north east lines,rounded corners] ($ (0,0) + (-.3,-.3) $) rectangle ($ (x) + (y) + (.3,.3)$);
		\draw[draw=none,fill=white,opacity=.5,rounded corners] ($ (0,0) + (-.3,-.3) $) rectangle ($ (x) + (y) + (.3,.3)$);
		\draw[draw=none,fill=white] (0,0) rectangle ($ (x) + (y) $);
		\node[xshift=-10pt,left] at ($ (0,0)!.5!(y) $) {$ \Phi = 0 $};
		\node[xshift=10pt,right] at ($ (x) + (y) - (0,0)!.5!(y) $) {$ \Phi = 0 $};
		\node[yshift=-10pt,below] at ($ (0,0)!.5!(x) $) {$ \Phi = 0 $};
		\node[yshift=10pt,above] at ($ (y) + (0,0)!.5!(x) $) {$ \Phi_R(x) $};
		\draw[->] (-.5,0) -- ($ (x) + (.6,0) $) node[anchor=north west] {$ x $};
		\draw[->] (0,-.5) -- ($ (y) + (0,.6) $) node[anchor=south east] {$ y $};
		\draw ($ (x) + (0,.1)$) -- ++(0,-.3) node[below,yshift=-5pt] {$ l_x $};
		\draw ($ (y) + (.1,0) $) -- ++(-.3,0) node[left,xshift=-5pt] {$ l_y $};
		\draw (y) -- ($ (x) + (y) $);
		\draw (x) -- ($ (x) + (y) $);
		\node at ($ (0,0)!.5!(x) + (0,0)!.5!(y) $) {$ V $};
	\end{tikzpicture}
\end{minipage}%
\begin{minipage}{.5\linewidth}
	Wir suchen: $ \Phi = \Phi(x,y) \quad $ mit $ \rho = 0 $
	\begin{equation*}
	0 = \Delta \Phi = \frac{\partial ^2 \Phi}{\partial x^2} + \frac{\partial^2 \Phi}{\partial y^2}
	\end{equation*}
\end{minipage}%
\\
\textbf{Randbedingungen:}\\
\begin{enumerate}[i)]
	\item $ \Phi(\vec{r}) = 0 \phantom{\Phi_R(x)} \qquad y = 0 $
	\item $ \Phi(\vec{r}) = 0 \phantom{\Phi_R(x)} \qquad x = 0 $
	\item $ \Phi(\vec{r}) = 0 \phantom{\Phi_R(x)} \qquad x = l_x $
	\item $ \Phi(\vec{r}) = \Phi_R(x) \phantom{0} \qquad y = l_y $
\end{enumerate}
\textbf{Separationsansatz:}
$ \Phi(x,y) = f(x) g(y) $
\begin{align*}
0 = \Delta \Phi &= \left(\prt{^2}{x^2} + \prt{^2}{y^2}\right) f(x) g(y)\\
&= \prt{^2 f}{x^2} g(y) + f(x) \prt{^2 g}{y^2}\\
&= \Delta \Phi = \prd{^2 f}{x^2} g(y) + f(x) \prd{^2 g}{y^2}
\end{align*}
\begin{equation*}
\rmbox{0 = \Delta \Phi = \prd{^2 f}{x^2} g(y) + f(x) \prd{^2 g}{y^2} } \qquad \left\rvert \cdot \, \frac{1}{f\,g} \right.
\end{equation*}
umformen:
\begin{equation*}
\Rightarrow \ub{\frac{1}{f(x)} \prd{^2 f}{x^2}}_{\tx{Fkt. von} x} = - \ub{\frac{1}{g(y)} \prd{^2 g}{y^2}}_{\tx{Fkt. von} y} = \const = - \alpha ^2
\end{equation*}
\begin{equation*}
\prd{^2 f}{x^2} = - \alpha^2 f(x) \quad \tx{mit } e^{i\alpha x} \qquad \prd{^2 g}{y^2} = \alpha^2 g(y) \quad \tx{mit } e^{\alpha y}
\end{equation*}
\begin{equation*}
e^{i \alpha x} \Rightarrow f(x) = a \sin(\alpha x) + b \cos(\alpha x) \qquad e^{\alpha y} \Rightarrow g(x) = c \sinh(\alpha y) + d \cosh(\alpha y)
\end{equation*}
$ \Phi(x,y) = f(x) \cdot g(y) $\\[5pt]
\textbf{Randbedingungen:}
\begin{enumerate}[i)]
	\item $ 0 = \Phi(x,0) = f(x) \cdot d \quad \Rightarrow d = 0 $
	\item $ 0 = \Phi(0,y) = b \cdot g(y) \quad \Rightarrow b = 0 $
	\begin{equation*}
	\Rightarrow \Phi(x,y) = a \sin(\alpha x) c \sinh(\alpha y) = \equalto{A}{a\cdot c} \sin(\alpha x) \sinh(\alpha y)
	\end{equation*}
	\item $ 0 = \Phi(l_x,y) = A \sin(\alpha l_x) \sinh(\alpha y) $
	$ \rightarrow \sin(\alpha l_x) = 0 \quad \Rightarrow \quad \alpha = \frac{n \pi}{l_x} \qquad n \in \mathbb{Z} (\tx{oder } n \in \mathbb{N}) $
	\begin{equation*}
	\rightarrow \Phi_n(x,y) = A_n \sin\left(\frac{n \pi x}{l_x}\right) \sinh\left(\frac{n \pi y}{l_x}\right)
	\end{equation*}
	\item $ \Phi(x,l_y) = \Phi_R(x) $
	\begin{equation*}
	\Rightarrow \Phi_R(x) \custo{\leftarrow}{=}{} A_n \sin\left(\frac{n \pi x}{l_x}\right) \sinh\left(\frac{n \pi l_y}{l_x}\right) \qquad \forall x \in [0,l_x]
	\end{equation*}
	im allgemeinen ist dies nicht möglich, aber da es sich um eine lineare DGL ($ \Delta \Phi = 0 $) handelt:\\
	$ \rightarrow $ Linearkombinationen von Lösungen sind auch Lösungen
\end{enumerate}
\textbf{Ansatz für allgemeine Lösung:}
\begin{equation*}
\rmbox{\Phi(x,y) = \sum_{n=1}^{\infty} A_n \sin \left(\frac{n \pi x}{l_x}\right) \sinh \left(\frac{n \pi y}{l_x}\right)}
\end{equation*}
Der Ansatz erfüllt $ \Delta \Phi = 0 $ und erfüllt die Randbedingungen i), ii), iii). Um iv) zu erfüllen fordern wir:
\begin{equation*}
\rmbox{\Phi_R(x) \buildrel ! \over = \ub{\sum_{n=1}^{\infty} A_n \sin\left(\frac{n \pi x}{l_x}\right)}_{\tx{Entwicklung}} \ub{\sinh\left(\frac{n \pi l_y}{l_x}\right)}_{\const}}
\end{equation*}
Der erste Teil des Ausdrucks entspricht der Entwicklung von $ \Phi_R(x) $ nach Funktionen $ \sin \left(\frac{n \pi x}{l_x}\right) $ also einer Fourier-Reihe.\\[5pt]
\begin{comment}
\textbf{Bestimmung von $ A_n $:}
Multipliziere mit $ \sin\left(\frac{n \pi x}{l_x}\right) \quad m \in \mathbb{N} $ und danach Integration:
\begin{align*}
\int_{0}^{l_x} \dd x \sin\left(\frac{n \pi x}{l_x}\right) \Phi_R (x) &=\right)  \sum_{n=1}^{\infty} A_n \sinh\left(\frac{m \pi l_y}{l_x}\right) \int_{0}^{l_x} \dd x \ub{\custoup{\leftarrow}{\sin\left(\frac{m \pi x}{l_x}\right)}{\qquad \qquad \quad \mathclap{\tx{zueinander orthogonale Vektoren}}} \custoup{\leftarrow}{\sin\left(\frac{n \pi x}{l_x}\right)}{}}_{= \frac{l_x}{2} \delta_{nm}}\\
&= A_m \frac{l_x}{2} \sinh\left(\frac{n \pi l_y}{l_x}\right)
\end{align*}
\begin{equation*}
A_m = \frac{2}{l_x \sinh\left(\frac{n \pi l_y}{l_x}\right)} \int_{0}^{l_x} \dd x \sin \left(\frac{n \pi x}{l_x}\right) \Phi_R(x)
\end{equation*}
in $ \Phi(x,y) $ einsetzen\\[5pt]
\end{comment}
\textbf{Bestimmung von $ A_n $:}
Multipliziere mit $ \sin\left(\frac{m \pi x}{l_x}\right) \quad m \in \mathbb{N} $ und danach Integration:
\begin{align*}
\int_{0}^{l_x} \dd x \sin\left(\frac{m \pi x}{l_x}\right) \Phi_R (x) &= \sum_{n=1}^{\infty} A_n \sinh\left(\frac{n \pi l_y}{l_x}\right) \ub{\int_{0}^{l_x} \dd x \ \custoup{\leftarrow}{\sin\left(\frac{m \pi x}{l_x}\right)}{\qquad \qquad \quad \mathclap{\tx{zueinander orthogonale Vektoren}}} \custoup{\leftarrow}{\sin\left(\frac{n \pi x}{l_x}\right)}{}}_{= \frac{l_x}{2} \delta_{nm}}\\
&= A_m \frac{l_x}{2} \sinh\left(\frac{m \pi l_y}{l_x}\right)
\end{align*}
\begin{equation*}
A_m = \frac{2}{l_x \sinh\left(\frac{m \pi l_y}{l_x}\right)} \int_{0}^{l_x} \dd x \sin \left(\frac{m \pi x}{l_x}\right) \Phi_R(x)
\end{equation*}
in $ \Phi(x,y) $ einsetzen\\[5pt]

% 12.11.18

\bbb{Wiederholung}{$$\Delta \Phi(\vec{r}) = 0 \quad + \quad \textrm{Randbedingungen}$$
$$\Phi = \Phi(x,y) = f(x)g(y)$$
$$\Phi_n(x,y) = A_n \sin\left(\frac{n \pi x}{l_x}\right) \sinh\left(\frac{n \pi y}{l_x}\right)$$
$$n \in \mathbb N$$
$$\Phi(x,y) = \sum_n A_n \sin\left(\frac{n \pi x}{l_x}\right) \sinh\left(\frac{n \pi y}{l_x}\right)$$
$$A_n = \frac{2}{l_x \sinh\left(\frac{n \pi l_y}{l_x}\right)} \intt{0}{l_x}\textrm{d}x\ \sin\left(\frac{n \pi x}{l_x}\right) \Phi_R (x)$$
Zu dem Problem gehört die Skizze aus Abschnitt \ref{tikz}: 2-Dim Randwertproblem.}

\subsection{Vollständige Orthonormale Funktionensysteme (VONS)}
Betrachte Funktionen $g(x),h(x)$ auf $I = [a,b] \subset \mathbb R$
$$g,h: \quad I \to \mathbb R\ (\mathbb C)$$
\textbf{Skalarprodukt}: $(g,h) = \int_a^b \textrm{d}x\ g^* (x) h(x)$\\
$(g,h) = 0$: $g$ und $h$ \textbf{orthogonal}, $(g,g) = 1:\ g$ \textbf{normiert}\\
Norm: $||g|| = \sqrt{(g,g)}$

\noindent
Ein abzählbarer Satz von Funktionen $\{f_n\} = \{f_1, f_2, \dots \}$\\
Heißt orthonormiert falls: $(f_m,f_n) = \delta_{nm}\ \rightarrow$ \textbf{Orthonormalsystem}\\
\textbf{Vollständigkeit:} Ein Satz von Funktionen heißt vollständig (VONS) falls \textbf{jede} quadratintegrable\footnote{$\textrm{Falls }\int \textrm{d}x\ |g(x)|^2 \textrm{ existiert}$} Funktion $g: I \to \mathbb R (\mathbb C)$ in der Form $g(x) = \summ{n = 1}{\infty} a_n f_n(x)$ dargestellt werden kann.\\
Genauer: $\lim\limits_{n \to \infty} \intt{a}{b} \textrm{d}x\ |\ g(x) - \summ{n = 1}{\infty} a_n f_n(x)| = 0$\\[5pt]
Bestimmung der Koeffizient $ a_n $:
$$g(x) = \sum_n a_n f_n(x)\ \qquad \left| \int \textrm{d}x\ f_m^*(x) \right.$$
$$\intt{a}{b}\textrm{d}x\ f_m^* (x) g(x) = \summ{n = 1}{\infty} \underbrace{\intt{a}{b} \textrm{d}x\ f_m^* (x)f_n(x)}_{= \delta_{nm}} = a_m$$
\begin{align*}
 	g(x) &= \sum_n a_n f_n (x) = \sum_n (f_n,g)f_n(x)\\
 	&= \sum_n \intt{a}{b} \textrm{d}x'\ f_n^* (x') g(x') f_n(x)\\
 	&= \intt{a}{b} \textrm{d}x'\ g(x') \underbrace{\summ{n = 1}{\infty} f_n(x)f_n^*(x')}_{= \delta(x - x')}
\end{align*}
da $ \int_{a}^{b} \dd x' g(x') = g(x) $
\frbox{Vollständigkeitsrelation}{$$\summ{n = 1}{\infty}f_n(x)f_n^*(x') = \delta(x - x')$$}
\noindent
\emph{Beispiele:}
\begin{enumerate}[1)]
	\item 
	\begin{equation*}
	f_n(x) = \sqrt{\frac{l}{2}} \sin\left(\frac{n \pi x}{l}\right) \qquad I = [0,l] 
	\end{equation*}Bedeutung der einzelnen Terme
	$ (f_n,f_m) = \delta_{nm} $\\
	$ g: \quad I \rightarrow \mathbb{R} \quad g(0) = 0 = g(l) $
	\begin{equation*}
	g(x) = \sum_{n} a_n \sqrt{\frac{l}{2}} \sin \left(\frac{n \pi x}{l}\right)
	\end{equation*}
	\item Fourierreihe: $ \{f_n\} $:\\
	$ n=0: \quad \frac{1}{\sqrt{l}} $
	\begin{equation*}
	n \in \mathbb{N}: \qquad \sqrt{\frac{2}{l}} \sin \left(\frac{n \pi x}{l}\right) \ \ ; \qquad \sqrt{\frac{2}{l}} \cos\left(\frac{n \pi x}{l}\right) \qquad I = [0,l]
	\end{equation*}
	\begin{equation*}
	g(x) = a_0 \frac{1}{\sqrt{l}} + \sum_{n=1}^{\infty} \left[a_n \sqrt{\frac{2}{l}} \sin \left(\frac{n \pi x}{l}\right) + b_n \sqrt{\frac{2}{l}} \cos\left(\frac{n \pi x}{l}\right)\right]
	\end{equation*}
	\vspace{5pt}
\end{enumerate}

% weis nicht ob das folgende hier hin muss oder wo anders

% benutze big math mode oder normal math mode anstatt in-text mode

\begin{center}
	\begin{tabular*}{.9\textwidth}{@{\extracolsep{\fill}}lll}
		\hline
		Vektoren & Bezeichnung & Funktionen\\
		\hline
		$ \vec{r} $ & Vektor & $ g(x) $ \\
		$ \{\vec{e}_n\} $ & Basis & $ \{f_n(x)\} $\\
		$ (\vec{e}_n \cdot \vec{e}_{n'}) = \delta_{nn'} $ & Orthonormierung & $ (f_n, f_{n'}) = \delta(\vec{r} - \vec{r}_0)_{nn'} $\\
		${\displaystyle \vec{r} = \sum_{n=1}^{3} a_n \vec{e}_n }$ & Entwicklung & $ {\displaystyle g(x) = \sum_{n=1}^{\infty} a_n f_n(x) }$\\
		$ a_n = (\vec{e}_n \cdot \vec{r}) $ & \hspace{-8.5pt} $ \begin{array}{l}
			\tx{Entwicklungs-}\\
			\tx{koeffizienten}
		\end{array}$ & $ a_n = (f_n,g) $\\
		$ \vec{r} \defeq \begin{pmatrix}
		a_1 \\ a_2 \\ a_3
		\end{pmatrix} $ & \hspace{-8.5pt} $ \begin{array}{l}
		\tx{Darstellung durch}\\
		\tx{Spaltenvektor}
		\end{array} $ & $ g(x) \defeq \begin{pmatrix}
		a_1 \\ a_2 \\ a_3 \\ \vdots
		\end{pmatrix} $
	\end{tabular*}
\end{center}

\subsection{Laplace-Gleichung in Kugelkoordinaten}

\begin{minipage}{.5\linewidth}
	%t1:
	\centering
	\begin{tikzpicture}
		\draw[->] (0,0) -- (0,2) node[anchor=south east] {$ z $};
		\draw[->] (0,0) -- (2,0) node[anchor=north west] {$ y $};
		\draw[->] (0,0) -- (-1,-1) node[anchor=north east] {$ x $};
		\draw[thick,->] (0,0) -- node[above=10pt] {$ \vec{r} $} (1.5,1);
		\draw[dashed] (0,0) -- (1.5,-1) -- (1.5,1);
		\centerarc[](0,0)(90:33:1cm);
		\centerarc[](0,0)(-135:-33:.85cm);
		\node[xshift=10pt,yshift=17pt] at (0,0) {$ \theta $};
		\node[below,yshift=-8pt] at (0,0) {$ \varphi $};
	\end{tikzpicture}
\end{minipage}%
\begin{minipage}{.5\linewidth}
	\begin{equation*}
	\vec{r} = r \begin{pmatrix}
	\sin \theta \cos \varphi \\ \sin \theta \sin \varphi \\ \cos \theta
	\end{pmatrix}
	\end{equation*}\\
	$$\Phi (\vec{r}) = \Phi(r,\theta, \varphi)$$
	$$\Delta \Phi = 0$$
	\vspace{5pt}
\end{minipage}%
\\
$$\rightarrow \frac{1}{r} \prt{^2}{r^2}(r \Phi) + \frac{1}{r^2 \sin \theta} \prt{}{\theta} \left( \sin \theta \prt{\Phi}{\theta}\right) + \frac{1}{r^2 \sin^2 \theta} \prt{^2 \Phi}{\varphi^2} = 0$$
$$\frac{1}{r} \prt{^2}{r^2} (r \Phi) = \frac{1}{r^2} \prt{}{r} \left(r^2 \prt{\Phi}{r}\right)$$

\subsubsection{Separationsansatz:}

$$\Phi(r,\theta,\varphi) = \frac{U(r)}{r} P(\cos\theta)Q(\varphi)$$
1. Term:
$$\frac{1}{r} \prt{^2}{r^2} \left(\cancel{r} \frac{U(r)}{\cancel{r}} P(\cos\theta)Q(\varphi)\right) = P(\cos\theta)Q(\varphi)\frac{1}{r}\frac{\textrm{d}^2 U}{\textrm{d} r^2}$$
$$\Rightarrow 0 = PQ \frac{1}{r} \frac{\textrm{d}^2 U}{\textrm{d} r^2} + UQ \frac{1}{r^3 \sin \theta}\frac{\textrm{d}}{\textrm{d}\theta} \left(\sin\theta \frac{\textrm{d} P}{\textrm{d} \theta} \right) + UP \frac{1}{r^3 \sin^2 \theta} \frac{\textrm{d}^2 Q}{\textrm{d} \varphi^2} \qquad \bigg\rvert \cdot \frac{r^3 \sin^2 \theta}{UPQ}$$
$$\Rightarrow \underbrace{- r^2 \sin^2 \theta \frac{1}{U} \frac{\textrm{d}^2 U}{\textrm{d} r^2} - \sin \theta \frac{1}{P}\frac{\textrm{d}^2}{\textrm{d} \theta}\left( \sin \theta \frac{\textrm{d} P}{\textrm{d} \theta} \right)}_{\textrm{unabhängig von } \varphi} = \underbrace{\frac{1}{Q} \frac{\textrm{d}^2 Q}{\textrm{d} \varphi^2}}_{\mathclap{\textrm{unabhängig von }r,\theta}} = \const \defeq -m^2$$
für $Q$:
\begin{enumerate}[i)]
	\item $$\frac{\textrm{d}^2 Q}{\textrm{d} \varphi^2} + m^2 Q = 0$$ Lösung: $$Q(\varphi) = e^{im\varphi} = \cos(m\varphi) + i\sin(m\varphi)$$
	 	$$Q(\varphi + 2 \pi) = Q(\varphi) \quad \tx{da} \quad e^{im(\varphi + 2\pi)} = e^{im\varphi} \quad \Rightarrow \quad m = \mathbb Z$$
	 	$$\frac{r^2}{U} \frac{\textrm{d}^2 U}{\textrm{d} r^2} + \frac{1}{P\sin\theta} \frac{\textrm{d}}{\textrm{d} \theta}\left( \sin\theta \frac{\textrm{d}P}{\textrm{d}\theta} \right) = \frac{m^2}{\sin^2\theta}$$
	 	$$\ub{\frac{r^2}{U} \frac{\textrm{d}^2 U}{\textrm{d} r^2}}_{\tx{unabh. von } \theta } = - \ub{\frac{1}{P\sin\theta}\frac{\textrm{d}}{\textrm{d} \theta}\left( \sin\theta \frac{\textrm{d}P}{\textrm{d}\theta} \right)}_{\tx{unabh. von } V} = \const \defeq \lambda$$
	 \item $$\frac{\textrm{d}^2 U}{\textrm{d} r^2} - \frac{\lambda}{r^2} U(r) = 0$$
	 	$\rightarrow$ Lösung für $\lambda = l(l + 1) \qquad$ (Warum das eine Lösung ist, wird  in iii) erklärt)
	 	$$U(r) = a_l r^{l + 1} + b_l r^{-l}$$
	 	$\rightarrow$ \underline{Spezielle Lösung für $m = 0$:}
	 	$$\Phi(r,\theta) = \frac{U(r)}{r}P_l (\cos\theta) = (a_l r^l + b_l r^{-l - 1}) P_l (\cos\theta)$$
	 	allg. Lösung: $\Delta\Phi = 0$ für $\prt{\Phi}{\varphi} = 0$
	 	$$\rmbox{ \Phi(r,\theta) = \summ{l = 0}{\infty} (\custo{\to}{a_l}{\mathclap{\hspace{30pt}\textrm{durch Randbedingungen festgelegt}}} r^l + \custo{\to}{b_l}{} r^{-l - 1}) P_l(\cos\theta)}$$
	 \item $$\frac{1}{\sin\theta} \frac{\textrm{d}}{\textrm{d} \theta}\left( \sin\theta \frac{\textrm{d}P}{\textrm{d}\theta}\right) + \left(\lambda - \frac{m^2}{\sin^2\theta}\right)P(\cos\theta) = 0$$
	 	$$x \defeq \cos\theta \quad P(x): \textrm{ DGL für } P(x) \quad \frac{\textrm{d}}{\textrm{d} \theta} P(x(\theta)) = \frac{\textrm{d}P}{\textrm{d}x} \frac{\textrm{d}x}{\textrm{d}\theta} = -\sin \theta \frac{\textrm{d}P}{\textrm{d}x}$$
	 	\begin{equation*}
	 	\dd x = - \frac{1}{\sin\theta} \prd{}{\theta}
	 	\end{equation*}
	 	$$\Rightarrow - \frac{\textrm{d}}{\textrm{d}x} \bigg( \equalto{ - \sin^2\theta}{1 - x^2} \frac{\textrm{d}P}{\textrm{d}x} \bigg) + \left(\lambda - \frac{m^2}{1-x^2}\right) P(x) = 0$$
\end{enumerate}
\frbox{Zugeordnete Legendresche DGL}{$$\Rightarrow \frac{\textrm{d}}{\textrm{d}x} \left( (1-x^2) \frac{\textrm{d}P}{\textrm{d}x} \right) + \left( \lambda - \frac{m^2}{1 - x^2} \right) P(x) = 0$$}
\begin{minipage}{.7\linewidth}
	Spezialfall: \textbf{Zylindersymmetrische Probleme}: $\Phi$ unabhängig von $\varphi$\\
	$\rightarrow$ \textbf{Legendre-Polynome}
\end{minipage}%
\begin{minipage}{.3\linewidth}
	%t2:
	\flushright
	\begin{tikzpicture}[scale=1.25]
		\draw[->] (-1,0) -- (1,0) node[anchor=north west] {$ x $};
		\draw[->] (0,-1) -- (0,1) node[anchor=south east] {$ z $};
		\draw[thick,->] (-.8,-.5) -- node[left] {$ \rho $} (-.8,.8) node[anchor=south east] {$ \vec{E}_0 $};
		\draw[pattern=north east lines] (0,0) circle (.6cm);
	\end{tikzpicture}
\end{minipage}%
$$\prt{\Phi}{\varphi}= 0, \quad Q(\varphi) = e^{im\varphi} \Rightarrow m = 0 \Rightarrow Q(\varphi) = 1$$
$$\frac{\textrm{d}}{\textrm{d}x}\left( (1 - x^2) \frac{\textrm{d}P}{\textrm{d}x} \right) + \lambda P (x) = 0$$
\frbox{Legendresche DGL}{$$(1-x^2) \frac{\textrm{d}^2P}{\textrm{d}x^2} - 2x \frac{\textrm{d}P}{\textrm{d}x} + \lambda P(x) = 0$$}
Potenzreihenansatz: $P(x) = \summ{k = 0}{\infty} a_k x^k$
\begin{itemize}
	\item[$\rightarrow$] Fließbach
	\item[$\rightarrow$] Legendre Polynome
	\item[$\rightarrow$] relevante Lösung nur für $\lambda = l(l + 1) \qquad l \in \mathbb N_0$
\end{itemize}


% 15.10.18

\bbb{Wiederholung}{
\textbf{Laplace-Gleichung in Kugelkoordinaten}
\begin{equation*}
\Delta \Phi = 0 \qquad \Phi(r,\theta,\varphi)
\end{equation*}
\begin{equation*}
\Phi(r,\theta,\varphi) = \frac{U(r)}{r} P(\cos\theta) Q(\varphi)
\end{equation*}
\begin{enumerate}[i)]
	\item \begin{equation*}
	\prd{^2Q}{\varphi^2} + m^2 Q = 0 \qquad m \in \mathbb{Z}
	\end{equation*}
	\begin{equation*}
	\rightarrow Q(\varphi) = e ^{i m \varphi}
	\end{equation*}
	\item \begin{equation*}
	\prd{^2 U}{r^2} - \frac{\lambda}{r^2} U = 0 \qquad \lambda = l (l+1) \quad l \in \mathbb{N}_0
	\end{equation*}
	\begin{equation*}
	U(r) = a_l r^{l+1} + b_l r^{-l}
	\end{equation*}
	\item \begin{equation*}
	\frac{1}{\sin \theta} \prd{}{\theta} \left(\sin \theta \prd{P}{\theta} \right) + \left(\lambda - \frac{m^2}{\sin^2 \theta}\right) P(\cos\theta) = 0
	\end{equation*}
\end{enumerate}
\textbf{Zylindersymmetrische Probleme:}
$ \prd{\Phi}{\varphi} = 0 \quad \rightarrow \quad m = 0 $\\
$ \rightarrow P_l(\cos\theta) : $ Legendre-Polynome\\
$ \rightarrow $ allgemeine Lösung:
\begin{equation*}
\rmbox{\Phi(r,\theta) = \sum_{l=0}^{\infty} \left(a_l r^l + b_l r^{-l-1}\right) P_l(\cos\theta)}
\end{equation*}
}
\noindent
\subsubsection{\emph{Beispiel:} \textbf{Leitende Kugel im homogenen Feld}}

\begin{minipage}{.5\linewidth}
	%t1:
	%\begin{wrapfigure}{r}{0.2\textwidth}
	%\vspace{-90pt}
	%\centering
	%\fbox{
	\centering
	\begin{tikzpicture}
	\draw[shade, ball color=gray!50!white] (0,0) circle (1cm) node[yshift=.5cm,right] {$\Phi_0$};
	\draw[->] (0,0) -- (0,2) node[above] {$z$};
	\draw[->] (0,0) -- (2,0) node[right] {$x$};
	\draw[-] (0,0) -- ++(200:1) node[xshift=.5cm] {$R$};
	\draw[->,thick] (-1.2,-1) -- (-1.2,1) node[above] {$\vec{E}_\textrm{ext}$};
	\draw[fill] (0,0) circle (0.02cm);
	\end{tikzpicture}
	\vspace{5pt}
	%}
	%\caption{Leitende Kugel im Homogenen Feld}
	%\end{wrapfigure}
	%\vspace{50pt}
\end{minipage}%
\begin{minipage}{.5\linewidth}
	\begin{equation*}
	\vec{E}_{\tx{ext}} = E_0 \vec{e}_z
	\end{equation*}
	\begin{equation*}
	\Phi(\vec{r}) = \Phi_0 \qquad |\vec{r}| \le R
	\end{equation*}
\end{minipage}%
\\[5pt]
Die Frage ist jetzt was ist das äußere Potential und das äußere $ \vec{E} $-Feld:
\begin{equation*}
\Phi(\vec{r}) \quad \tx{für} \quad |\vec{r}| > R \qquad \rightarrow \vec{E}(\vec{r})
\end{equation*}
Lösung des Randwertproblems $ \Delta \Phi(\vec{r}) = 0 $ für $ |\vec{r}| > R $ mit der \textbf{Randbedingungen}:
$$ \Phi(\vec{r}) = \Phi_0 \tx{ für }  |\vec{r}| = R $$
$$ \Phi(\vec{r}) \buildrel |\vec{r}| \to \infty \over \longrightarrow - E_0 z + \const = - E_0 r \cos\theta + \Phi_1 $$
Aufgrund der Zylindersymmetrie des Problems ist $ \Phi $ eine Funktion von $ \theta $ und $ r $: $ \Phi(r,\theta) $
\begin{equation*}
\rightarrow \Phi(r,\theta) = \sum_{l=0}^{\infty} \left(a_l r^l + b_l r^{-l-1}\right) P_l(\cos\theta)
\end{equation*}
\begin{enumerate}[i)]
	\item $ r = R $
	\begin{align*}
	\Phi(R,\theta) &= \sum_{l=0}^{\infty} \left(a_l R^l + b_l R^{-l-1}\right) P_l(\cos\theta)\\
	&\overset{!}{=} \Phi_0 \cdot 1 = \Phi_0 P_0(\cos\theta) 
	\end{align*}
	an Beide Seiten Multiplizieren wir$ \int_{-1}^{1} \dd (\cos \theta) P_n (\cos\theta) $ für $ n = 0,1,2,3,\dots $
	\begin{align*}
	\sum_{l=0}^{\infty} \left(a_l R^l + \frac{b_l}{R^{l+1}}\right) \ub{\int_{-1}^{1} \dd (\cos\theta) P_n(\cos\theta) P_l(\cos\theta)}_{\delta_{nl} \frac{2}{2n+1}} &= \Phi_0 \ub{\int_{-1}^{1} \dd (\cos\theta) P_n(\cos\theta) P_0(\cos\theta)}_{\delta_{n0} \frac{2}{2n+1} = 2 \delta_{n_0}}\\
	\Rightarrow \summ{l = 0}{\infty} \frac{2}{2n + 1}\left(a_l R^l + \frac{b_l}{R^{l+1}}\right) \delta_{nl} &= 2 \Phi_0 \delta_{n0} \qquad n = 0, 1, 2,\dots
	\end{align*}
	% :andrez 1 teil 2 des aligns:
	$$\begin{array}{lcl}
		\underline{n = 0:} & 2\left(a_0 R^0 + \displaystyle\frac{b_0}{R}\right) = 2 \Phi_0 &\Rightarrow b_0 = R(\Phi_0 - a_0)\\
		&&\\
		\underline{n \neq 0:} & \displaystyle\frac{2}{2n+1}\left(a_n R^n + \frac{b_n}{R^{n+1}}\right) = 0 &\Rightarrow b_n = R^{2n+1}a_n
	\end{array}$$
	\item $ r \to \infty $
	\begin{align*}
	\pofr &\rightarrow -E_0 r \cos\theta + \Phi_1\\
	&= - E_0 r P_1(\cos\theta) + \Phi_1 P_0(\cos\theta) \quad \buildrel r \to \infty \over \longleftarrow \quad \sum_{l=0}^{\infty} \left(a_l r^l + \frac{b_l}{R^{l+1}}\right) P_l(\cos\theta)
	\end{align*}
	\begin{equation*}
	\Rightarrow \sum_{l=0}^{\infty} \left(a_l r^l + \frac{b_l}{R^{l+1}}\right) \delta_{nl} \frac{2}{2n+1} \quad \buildrel r \to \infty \over \longrightarrow \quad 2 \Phi_1 \delta_{n0} - E_0 r \delta_{n1} \frac{2}{2n+1} = 2 \Phi_1 \delta_{n0} - \frac{2}{3} E_0 r \delta_{n1}
	\end{equation*}
	für $ n = 0,1,2,\dots $
	\begin{align*}
	&\underline{n = 0:} \quad \left(a_0 + \frac{b_0}{r}\right) 2 \buildrel r \to \infty \over \longrightarrow 2 \Phi_1 \qquad \quad \ \ \Rightarrow \qquad a_0 = \Phi_1\\
	&\underline{n = 1:} \quad \left(a_1 r + \frac{b_1}{r^2}\right)\cancel{\frac{2}{3}} \buildrel r \to \infty \over \longrightarrow - \cancel{\frac{2}{3}} E_0 r \quad \ \Rightarrow \ \, \quad \ \  a_1 = E_0\\
	&\underline{n > 1:} \quad \left(a_n r^n + \frac{b_n}{r^{n+1}}\right) \frac{2}{2n+1} \buildrel r \to \infty \over \longrightarrow \quad \Rightarrow \qquad \: a_n = 0
	\end{align*}
	\begin{align*}
	\rightarrow \ \ b_0 &= R(\Phi_0 - \Phi_1)\\
	b_1 &= - R^3 \qquad \ \, a_1 = E_0 R^3\\
	n>1: \ \ b_n &= - R^{2n+1} \quad a_n = 0
	\end{align*}
	\begin{equation*}
	\Phi(r,\theta) = \left[\Phi_1 + \frac{R(\Phi_0 - \Phi_1)}{r}\right] \ub{P_0(\cos\theta)}_{=1} + \left[-E_0 r + \frac{E_0 R^3}{r^2}\right] \ub{P_1(\cos\theta)}_{\cos\theta}
	\end{equation*}
	\frbox{Potential einer Kugel im homogenen $ \vec{E} $-Feld}{\begin{equation*}
	\rightarrow\Phi(r,\theta) = \Phi_1 + (\Phi_0 - \Phi_1) \frac{R}{r} - E_0 r\cos\theta + E_0\frac{R^3}{r^2} \cos\theta
	\end{equation*}}
\end{enumerate}
Diskussion der \textbf{Bedeutung der einzelnen Terme}:
\begin{itemize}
	\item 	$ \Phi_1 $ ist eine Konstante die auf das Potential keine physikalische Auswirkung hat.
	\item $ -E_0 r \cos\theta $ ist das Potential des äußeren Feldes.
	\item $ \Phi_0 - \Phi_1 $ ist das Potential einer möglichen Gesamtladung auf der Kugel.
	\item $ E_0\frac{R^3}{r^2} \cos \theta $ ist der Beitrag der Ladungsverschiebung auf der Kugel. Also das Potential der Influenzierten Ladungen.
\end{itemize}
\begin{minipage}{.7\linewidth}
	Eine Kugel mit Ladung $ Q $ ohne äußeres Feld ($ E_0 = 0 $):
	\begin{align*}
	\rightarrow\pofr &= \frac{1}{4 \pi \epsilon_0} \frac{Q}{r} + \const\\
	&\overset{!}{=} \Phi_1 + (\Phi_0 - \Phi_1) \frac{R}{r}
	\end{align*}
	\begin{equation*}
	\rightarrow \const = \Phi_1 \qquad \Phi_0 - \Phi_1 = \frac{Q}{4 \pi \epsilon_0 R}
	\end{equation*}
	\begin{equation*}
	\Rightarrow \quad \Phi_1 = \Phi_0 - \frac{Q}{4 \pi \epsilon_0 R}
	\end{equation*}
	\vspace{5pt}
\end{minipage}%
\begin{minipage}{.3\linewidth}
	%t2:
	\centering
	\begin{tikzpicture}
		\draw[shade, ball color=gray!20!white] (0,0) circle (1cm);
		\draw[->,thick] (-1.1,-1) -- (-1.1,1) node[above] {$E_0\vec{e}_z$};
		\draw (0,0) node[white!10!blue,xshift=-.5cm,yshift=-.5cm] {$-$};
		\draw (0,0) node[white!10!blue,yshift=-.6cm] {$-$};
		\draw (0,0) node[white!10!blue,xshift=.5cm,yshift=-.5cm] {$-$};
		\draw (0,0) node[red,xshift=-.5cm,yshift=.5cm] {$+$};
		\draw (0,0) node[red,yshift=.6cm] {$+$};
		\draw (0,0) node[red,xshift=.5cm,yshift=.5cm] {$+$};
	\end{tikzpicture}
\end{minipage}%
\\
Eine ungeladene Kugel: $ Q = 0 \quad \rightarrow \Phi_1 = \Phi_0 $
\begin{equation*}
\rightarrow \rmbox{\Phi(r,\theta) = \Phi_0 - E_0 r \cos\theta + E_0 \frac{R^3}{r^2} \cos \theta}
\end{equation*}

\subsubsection{Lösung für $ m \neq 0 $ (Potenzreihenansatz)}

% andrez 2 allgemeiner fall:
$\rightarrow$ Zugeordnete Legendre-Polynome
$$P^m_l (x) \qquad x = \cos\theta$$
$\bullet$ Allgemeine Struktur:
$$P_l^m \sim (1-x^2)^{|m|/2} \times \textrm{ Polynom } (l-|m|)\textrm{-ten Grades}$$
Zusammenfassung der Funktionen:
$$ P, \theta \textrm{ in Produkt: } P_l^m (\cos\theta) Q_m (\varphi)$$
\frbox{$\Rightarrow$ Kugelflächenfunktionen}{
$$\mathcal Y_{lm} (\theta,\varphi) = \sqrt{\frac{2l+1}{4\pi}\frac{(l-m)!}{(l+m)!}} P_l^m (\cos\theta)e^{im\varphi}$$
}
$$l = 0,1,2,\dots \qquad m = 0,\pm 1,\pm 2,\dots,\pm l \qquad \theta \in [0,\pi] \qquad \varphi \in [0,2\pi]$$
\frbox{Allgemeine Lösung der Laplace-Gleichung in Kugelkoordinaten:}{
\begin{equation*}
\Phi(r,\theta,\varphi) = \sum_{l=0}^{\infty} \sum_{m=-l}^{l} \left(a_{lm} r^l + \frac{b_{lm}}{r^{l+1}}\right) \mathcal{Y}_{lm} (\theta,\varphi)
\end{equation*}
}
$ \Delta \Phi = 0 $

% Theo Vorlesung 19.11.2018

\section{Multipolentwicklung}

\begin{minipage}{.7\linewidth}
	Beliebige endlich große Ladungsverteilung
	\[q=\int\dd^3r'\rho(\vec{r}')\]
	\[\vec{r}\gg R\ \quad \Phi(\vec{r})\approx\frac{1}{4\pi\epsilon_0}\frac{q}{r}\]
\end{minipage}%
\begin{minipage}{.3\linewidth}
	%t1:
	\centering
	\begin{tikzpicture}
	\draw[fill=gray!10]  plot [smooth cycle, tension=.7] coordinates { (-.5,.3) (-.8,1) (-.3,1.5) (.8,2) (.6,1) (.9,-.5) (0,-.6) (-.6,-.2) };
	\node[circle,fill=black,inner sep=1pt,minimum size=1pt] (O) at (0,0) {};
	\draw[thick,->] (O) -- node[above] {$ \vec{r} $} (2,1);
	\node at (.2,1) {$ \rho $};
	\node[left] at (O) {$ O $};
	\end{tikzpicture}
\end{minipage}%
\\
\begin{minipage}{.7\linewidth}
	statische, lokalisierte Ladungsverteilung:
	\[\rho(\vec{r})=\begin{cases}\textrm{beliebig}&r<R\\0&r>R\end{cases}\]
	\[\Phi(\vec{r})=\frac{1}{4\pi\epsilon_0}\int_V\dd^3r'\frac{\rho(\vec{r}')}{|\vec{r}-\vec{r}'|}\]
\end{minipage}%
\begin{minipage}{.3\linewidth}
	%t2:
	\flushleft
	\begin{tikzpicture}[scale=1.2]
	\draw[fill=gray!10] plot [smooth cycle,tension=.8] coordinates { (.5,-1) (.1,-.9) (-.2,-.2) (-.8,.7) (.9,.9) (.6,-.2) };
	\draw[->] (0,0) -- (0,2) node[anchor=south east] {$ z $};
	\draw[->] (0,0) -- (2,0) node[anchor=north west] {$ x $};
	\draw[thick,->] (0,0) -- (-1,-1) node[anchor=south west,above=12pt,xshift=10pt] {$ R $};
	\draw[dashed] (0,0) circle (1.4cm);
	\draw[thick,->] (0,0) -- node[above] {$ \vec{r}' $} (.7,.8);
	\draw[thick,->] (0,0) -- node[above=5pt,xshift=10pt] {$ \vec{r} $} (2,1);
	\node at (.2,-.5) {$ \rho $};
	\node at (-.4,-1) {$ V $};
	\end{tikzpicture}
	\vspace{10pt}
\end{minipage}%
\\
F"ur $r>R:\ |\vec{r}'|<|\vec{r}|\rightarrow$ Taylorentwicklung von $\frac{1}{|\vec{r}-\vec{r}'|}$ in $\vec{r'} \rightarrow$ d.h. in $ x_1',x_2',x_3'$ \\
\[\frac{1}{|\vec{r}-\vec{r}'|}=\frac{1}{\sqrt{(x_1-x_1'^2)+(x_2-x_2'^2)+(x_3-x_3'^2)}} \qquad \qquad \qquad \vec{r}=\begin{pmatrix}x_1\\x_2\\x_3\end{pmatrix}\]
Taylorentwicklung:
\[f(\vec{r}')=f(x_1',x_2',x_3')=f(0,0,0)+\sum_{i=1}^3x_i'\frac{\partial f}{\partial x_i}(0)+\frac{1}{2}\sum_{i,j=1}^3x_i'x_j'\frac{\partial^2f}{\partial x_i'\partial x_j'}(0)+\ldots\]
Zuerst berechnen wir die einzelnen Terme:
\[f(\vec{r}')=\frac{1}{|\vec{r}-\vec{r}'|}:\ f(0)=\frac{1}{r} \quad \ \ \ \ \  \frac{\partial f}{\partial x_i'}=\frac{(x_i-x_i')}{|\vec{r}-\vec{r}'|^3}\bigg|_0=\frac{x_i}{r^3} \quad \ \ \ \ \ \frac{\partial^2f}{\partial x_i'\partial x_j'}(0)=\ldots=\frac{3x_ix_j-r^2 \delta_{ij}}{r^5}\]
Für $|\vec{r}|<|\vec{r}'|$:
\vspace{-15pt} \[\Phi(\vec{r})=\frac{1}{4\pi\epsilon_0}\int\dd^3r\rho(\vec{r}')\Bigg\{\frac{1}{r}+\sum_{i=1}^3\frac{x_i'x_i}{r^3}+\frac{1}{2}\sum_{i,j}x_i'x_j'\frac{3x_ix_j-\custoup{\rightarrow}{r^2\delta_{ij}}{}}{r^5}+\ldots\Bigg\}\]
Den letzten (mit einem Pfeil markierten) Term schauen wir uns jetzt noch einmal genauer an.
\[\sum_{i,j}x_i'x_j'r^2\delta_{ij}=r^2\underbrace{\sum_ix_i'^2}_{=r'^2}=r^2r'^2=r'^2\sum_ix_i^2=\sum_{i,j}r'^2x_ix_j\delta_{ij}\]
Somit können wir den letzten Term umschreiben als:
\[\sum_{i,j}x_i'x_j'\frac{(3x_ix_j-r^2\delta_{ij})}{r^5}=\sum_{i,j}x_ix_j\frac{(3x_i'x_j'-r'^3\delta_{ij})}{r^5}\]
Das Potential unserer Ladungsverteilung im externen $ \vec{E} $-Feld ergibt sich dann als
\begingroup\makeatletter\def\f@size{10}\check@mathfonts
\def\maketag@@@#1{\hbox{\m@th\large\normalfont#1}}
\begin{equation*}
\rmbox{\Phi(\vec{r})=k\Bigg\{\frac{1}{r} \underbrace{\int\dd^3r'\rho(\vec{r}')}_{\mathclap{\substack{ q \textrm{ Gesamtladung} \\  \tx{(Monopol)}}}} +\sum_{i=1}^3\frac{x_i}{r^3} \underbrace{\int\dd^3r'x_i'\rho(\vec{r}')}_{\mathclap{\substack{p_i \textrm{ Dipolmoment} \\ \vec{p}=(p_1,p_2,p_3)}}}  +\frac{1}{2}\sum_{i,j}\frac{x_ix_j}{r^5}\underbrace{\int\dd^3r'\rho(\vec{r}')(3x_i'x_j'-r'^2\delta_{ij})}_{\mathclap{\eqdef\ Q_{ij}\textrm{ Quadrupolmoment}}}+\ldots\Bigg\}}
\end{equation*}\endgroup
\[\quad \rmbox{\Rightarrow\Phi(\vec{r})=\frac{1}{4\pi\epsilon_0}\left\{\frac{q}{r}+\frac{\vec{r}\cdot\vec{p}}{r^3}+\frac{1}{2}\sum_{i,j}\frac{x_ix_j}{r^5}Q_{ij}+\ldots\right\}}\]
\textbf{Diskussion}:
\begin{enumerate}[i)]
	\item \textbf{Monopol} \[\Phi_M(\vec{r})=\frac{1}{4\pi\epsilon_0}\frac{q}{r} \qquad \propto\frac{1}{r}\textrm{ dominiert f"ur }q\neq0\]
	\[\rightarrow\vec{E}_M(\vec{r})=-\vec{\nabla}\Phi_M=\frac{q}{4\pi\epsilon_0}\frac{\vec{r}}{r^3} \qquad \propto\frac{1}{r^2}\]
	\item \textbf{Dipol} \[\Phi_D(\vec{r})=\frac{1}{4\pi\epsilon_0}\frac{\vec{r}\cdot\vec{p}}{r^3} \qquad \propto\frac{1}{r^2}\]
	Das elektrische Feld:
	\[\vec{E}_D=-\vec{\nabla}\Phi_D\]
	\begin{align*}
	\vec{\nabla}_{\vec{r}} \left(\vec{p}\cdot\frac{\vec{r}}{r^3}\right) : \ \ \frac{\partial}{\partial x} \left(p_x\frac{x}{r^3}+p_y\frac{y}{r^3}+p_z\frac{z}{r^3}\right) &= \left(\frac{p_x}{r^3} -3 p_x \frac{xx}{r^5}-3p_y\frac{yx}{r^5} -3p_z \frac{zx}{r^5} \right)\\
	&= \frac{p_x}{r^3} -3 \frac{\vec{p}\cdot\vec{r}}{r^5}x\\
	\frac{\partial}{\partial y}(\ldots) &= \frac{p_y}{r^3}-3\frac{\vec{p}\cdot\vec{r}}{r^5}y \\
	\frac{\partial}{\partial z} (\ldots) &= \frac{p_z}{r^3} -3 \frac{\vec{p}\cdot\vec{r}}{r^5} z 
	\end{align*}
	\[\Rightarrow\vec{\nabla}_{\vec{r}}\left(\vec{p}\cdot\frac{\vec{r}}{r^3}\right)=\frac{\vec{p}}{r^3}-3\frac{\vec{p}\cdot\vec{r}}{r^5}\vec{r}\]
	$$\vec{E}_{D} = - \vec{\nabla} \Phi_D= \frac{1}{4\pi\epsilon_0} \left[\frac{3(\vec{p}\cdot\vec{r})\vec{r}}{r^5}-\frac{\vec{p}}{r^3}\right] \qquad \propto \frac{1}{r^3} \quad \vec{r}\neq0$$
	\begin{minipage}{.5\linewidth}
		\emph{Beispiel:} für die Realisierung eines Dipols\\
		\textbf{Punktladungen:} $q_0,-q_0$ in $\vec{r}_+,\vec{r}_-$\\
		Gesamtladung: $q=q_0-q_0=0$
	\end{minipage}%
	\begin{minipage}{.5\linewidth}
		%t3:
		\flushright
		\begin{tikzpicture}
			\node[circle,fill=black,inner sep=1pt,minimum size=1pt] (q1) at (-1,2) {};
			\node[circle,fill=black,inner sep=1pt,minimum size=1pt] (q2) at (2,1) {};
			\node[left] at (q1) {$ q_0 $};
			\node[right] at (q2) {$ -q_0 $};
			\draw[->] (0,0) -- node[below,xshift=-5pt] {$\vec{r}_+$} (q1);
			\draw[->] (0,0) -- node[below,xshift=5pt] {$\vec{r}_-$} (q2);
			\draw[thick,->] (q2) -- node[above=5pt,xshift=10pt] {$\vec{d} = \vec{r}_+ - \vec{r}_-$} (q1);
		\end{tikzpicture}
	\end{minipage}%
	\[\rightarrow\Phi_M(\vec{r})\equiv0\]
	\[\Phi(\vec{r})=\frac{1}{4\pi\epsilon_0}\left(\frac{q_0}{|\vec{r}-\vec{r}_+|}-\frac{q_0}{|\vec{r}-\vec{r}_-|}\right)=\frac{1}{4\pi\epsilon_0}\left(\frac{\vec{r}\cdot\vec{p}}{r^3}+\ldots\right)\]
	\[\rho(\vec{r}')=q_0\delta(\vec{r}'-\vec{r}_+)-q_0\delta(\vec{r}'-\vec{r}_-)\]
	\[\vec{p}=\int\dd^3r'\rho(\vec{r}')\vec{r}'=q_0\vec{r}_+-q_0\vec{r}_-=q_0\vec{d} \qquad \rmbox{\vec{p} = q_0 \vec{d}}\]
	mehrere Punktladungen $q_i$ in $\vec{r}_i$\\
	\[\rightarrow\vec{p}=\sum_iq_i\vec{r}_i\]
	\item \textbf{Quadrupolmoment}
	\[Q_{ij}=\int\dd^3r'\rho(\vec{r}')(3x_i'x_j'-r'^2\delta_{ij})\]
	Quadrupoltensor $Q=\begin{pmatrix}Q_{11}& \color{black!50!green} Q_{12}& \color{black!30!red} Q_{13}\\\color{black!50!green} Q_{21} &Q_{22}& \color{black!30!blue} Q_{23}\\\color{black!30!red} Q_{31} & \color{black!30!blue} Q_{32} & Q_{33}\end{pmatrix}$\\ 
	\textbf{Eigenschaften}
	\begin{enumerate}[i)]
		\item Spurfrei: $\mathrm{tr}(Q)=\sum_iQ_{ii}=0 \quad \Rightarrow $ 2 unabhängige Elemente
		\item Symmetrisch: $Q_{ij}=Q_{ji} \quad \Rightarrow 3 $ unabhängige Elemente
	\end{enumerate}
	$\Rightarrow$ 5 unabhängige Elemente\\[5pt]
	
	Ableitung in Kugelkoordinaten\\
	$\Rightarrow$ Sphärische Multipolmomente
\end{enumerate}

\subsection{Multipolentwicklung der Energie der Ladungsverteilung im äußeren Feld}

\begin{minipage}{.6\linewidth}
	Energie:
	\begin{equation}
	W=\int\dd^3r\rho(\vec{r})\Phi(\vec{r})
	\label{W}
	\end{equation}
	Wir stellen uns vor, das $ \vec{E} $-Feld wird von sehr weit entfernten Ladungen erzeugt. Wir machen somit also die Annahme, dass sich $\Phi(\vec{r})$ in dem Gebiet, wo $ \rho(\vec{r}) $ ist, sich nur wenig ändert.
\end{minipage}%
\begin{minipage}{.4\linewidth}
	%t4:
	\centering
	\begin{tikzpicture}
		\draw[fill=gray!10] plot [smooth cycle, tension=.8] coordinates { (.5,-1.5) (0,-1) (-.7,-.5) (-.7,.7) (.7,1) (.7,0) (1.2,-.7) };
		\draw[->] (0,0) -- (0,2) node[anchor=south east] {$ z $};
		\draw[->] (0,0) -- (2,0) node[anchor=north west] {$ x $};
		\draw[->] (0,0) -- (-1,-1) node[anchor=north east] {$ y $};
		\node at (.5,-.7) {$ \rho $};
		\coordinate (a) at (1.5,1.7);
		\coordinate (b) at ($ (a) + (-55:.5) $);
		\coordinate (c) at ($ (a) + (-52:1) $);
		\coordinate (d) at ($ (a) + (-50:1.5) $);
		\draw[->] (a) to[out=-150,in=10] ++(-160:4);
		\draw[->] (b) to[out=-145,in=15] ++(-155:4.5);
		\draw[->] (c) to[out=-145,in=35] ++(-145:4.5);
		\draw[->] (d) to[out=-145,in=55] ++(-135:4);
		\node at (2,-1.5) {$ \vec{E} = - \vec{\nabla} \Phi $};
	\end{tikzpicture}
\end{minipage}%
\\
$\rightarrow$ Taylorentwicklung von $\Phi(\vec{r})$ um $\vec{r}=0$\\
\begin{align*}
\Phi(\vec{r}) &= \Phi(0) + \vec{r} \cdot \vec{\nabla} \Phi(0) + \frac{1}{2} \sum_{i,j} x_i x_j \frac{\partial^2\Phi}{\partial x_i\partial x_j} (0) + \ldots \\
&= \Phi(0) - \vec{r} \cdot \vec{E} (0) - \underbrace{\frac{1}{2} \sum_{i,j} x_i x_j \frac{\partial E_j}{\partial x_i} (0)}_{} + \ldots\\
& \qquad \qquad \qquad \qquad = \frac{1}{6} \sum_{i,j} (3 x_i x_j - r^2 \delta_{ij}) \frac{\partial E_j}{\partial x_i} (0) 
\end{align*}
Dies gilt, da:
$$\sum_{i,j}r^2\delta_{ij}\frac{\partial E_j}{\partial x_i}(0)=r^2\underbrace{\sum_i\frac{\partial E_i}{\partial x_i}(0)}_{\mathclap{\vec{\nabla}\cdot\vec{E}(0)=0}}$$
$ \vec{\nabla} \cdot \vec{E} = 0 $ gilt, da $ \vec{E} $ ein äußeres Feld ist.
Damit erhalten wir dann für die Energie mit Formel \eqref{W}:
\begin{align*}
\rightarrow W &= \Phi(0)\underbrace{\int\dd^3r\rho(\vec{r})}_{=q}-\vec{E}(0)\cdot\underbrace{\int\dd^3r \ \vec{r}\rho(\vec{r})}_{=\vec{p}}-\frac{1}{6}\sum_{i,j}\frac{\partial E_j}{\partial x_i}(0)\underbrace{\int\dd^3r(3x_ix_j-r^2\delta_{ij})\rho(\vec{r})}_{=Q_{ij}}+\ldots \\
&= \custo{\substack{\vspace{10pt}\\ \nwarrow} }{q}{\mathclap{\tx{Ladung}}\qquad} \custo{\substack{\leftarrow}}{\Phi(0)}{\ \ \mathclap{\tx{Potential}}} \quad - \quad \custo{\substack{\leftarrow}}{\vec{p}}{\ \ \mathclap{\tx{Dipol}}} \cdot \custo{\substack{\swarrow\\ \vspace{5pt}}}{\vec{E}(0)}{\qquad \mathclap{\tx{Feld}}} \quad - \quad \frac{1}{6} \sum_{i,j} 
\ \custo{\substack{\vspace{7pt} \\ \nwarrow}}{Q_{ij}}{\mathclap{\substack{\\[5pt]\tx{Quadrupol}}} \quad} \  \custo{\substack{\swarrow \\ \vspace{5pt}}}{\frac{\partial E_j}{\partial x_i} (0)}{\qquad \quad \mathclap{\tx{Feldgradient}}} + \ldots
\end{align*}

\noindent
\begin{minipage}{.5\linewidth}
	
	\subsubsection{Wechselwirkungsenergie zweier Dipole}
	
	Betrachte 2 Punktdipole $\vec{p}_1,\vec{p}_2$ in $\vec{r}_1,\vec{r}_2$.\\
	$\vec{p}_2$ erzeugt am Ort $\vec{r}_1$ das "au\ss ere Feld:
\end{minipage}%
\begin{minipage}{.5\linewidth}
	%t5:
	\centering
	\begin{tikzpicture}
	\coordinate (q1) at (1.5,1);
	\coordinate (q2) at (-.7,1.5);
	\draw[->] (0,0) -- node[below,xshift=-5pt] {$\vec{r}_1$} (q2);
	\draw[->] (0,0) -- node[below,xshift=5pt] {$\vec{r}_2$} (q1);
	\draw[thick,->] (q1) -- node[above] {$ \vec{r}_{12} $} (q2);
	\draw[very thick,->] (q2) -- node[left] {$\vec{p}_1$} ++(70:.7);
	\draw[very thick,->] (q1) -- node[right] {$\vec{p}_2$} ++(110:.7);
	\end{tikzpicture}
	\vspace{10pt}
\end{minipage}%
\[\vec{E}(\vec{r}_1)=\frac{1}{4\pi\epsilon_0}\left[\frac{3(\vec{p}_2\cdot\vec{r}_{12})\vec{r}_{12}}{r_{12}^5}-\frac{\vec{p}_2}{r_{12}^3}\right]\]
\[\rightarrow W=-\vec{p}_1\cdot\vec{E}(\vec{r}_1)=\frac{1}{4\pi\epsilon_0}\left[\frac{\vec{p}_1\cdot\vec{p}_2}{r_{12}^3}-\frac{3(\vec{p}_2\cdot\vec{r}_{12})(\vec{p}_1\cdot\vec{r}_{12})}{r_{15}^5}\right] \qquad \propto\frac{1}{r_{12}^3}\]
Je nach Orientierung der Dipole ist diese Wechselwirkung anziehend oder absto\ss end.\\
z.B.: $\vec{p}_1,\vec{p}_2\perp\vec{r}_{12}$
\[\rightarrow W=\frac{1}{4\pi\epsilon_0}\frac{\vec{p}_1\cdot\vec{p}_2}{r_{12}^3}\begin{cases}>0&\uparrow\uparrow \qquad \tx{abstoßend}\\<0&\uparrow\downarrow \qquad \tx{anziehend}\end{cases}\]


% Vorlesung 22.11.18

\section{Elektrostatik in Materie - Dielektrika}

\textbf{Definition: Dielektrika}\\
Nichtleitende Substanzen (Gase, Flüssigkeiten, Festkörper). Die Ladungsträger sind also fest gebunden.\\[5pt]
äußere Felder $ \Rightarrow $ Polarisation\\[5pt]
Mechanismen
\begin{enumerate}[i)]
	\item \textbf{Verschiebungspolarisation} (Deformationspolarisation)\\
	neutrales Atom
	%T1 
	\item \textbf{Orientierungspolarisation}\\
	Molekül mit permanentem Dipolmoment z.B. Wasser
	%T2
	%T3
\end{enumerate}
Phänomenologie: Experimentalphysik\\
Plattenkondensator:\\[10pt]
\begin{minipage}{.5\linewidth}
	\flushleft
	\textbf{ohne Medium:}\\
	%t4:
	\centering
	\begin{tikzpicture}
		\draw[very thick] (-1.3,-2) -- (-1.3,2) node[above] {$ \sigma $};
		\draw[very thick] (1.3,-2) -- (1.3,2) node[above] {$ - \sigma $};
		\draw[->] (-1.3,-2.5) -- ++(4,0) node[right] {$ x $};
		\draw ($ (1.3,-2.5) + (0,.1) $) -- ($ (1.3,-2.5) + (0,-.1) $) node[below] {$ d $};
		\draw ($ (-1.3,-2.5) + (0,.1) $) -- ($ (-1.3,-2.5) + (0,-.1) $) node[below] {$ 0 $};
		\foreach \y in {-1.8,-.6,.6,1.8}
		\draw[->] (-1.1,\y) -- (1.1,\y);
		\foreach \y in {-1.8,-.6,.6,1.8}
		%\node at (-1.8,\y) {$ + $};
		\draw[fill = white!20!red] (-1.6,\y) circle (.2cm) node[] {$ \color{white} + \color{white} $};
		\foreach \y in {-1.8,-.6,.6,1.8}
		%\node at (1.8,\y) {$ - $};
		\draw[fill = white!20!blue] (1.6,\y) circle (.2cm) node[] {$ \color{white} - \color{white} $};
		\node[above] at (0,2) {$ \vec{E} $};
	\end{tikzpicture}
\end{minipage}%
\begin{minipage}{.5\linewidth}
	\flushleft
	\textbf{mit Medium:}\\
	%t5:
	\centering
	\begin{tikzpicture}
		\draw[very thick] (-1.5,-2) -- (-1.5,2) node[above] {$ \sigma $};
		\draw[very thick] (1.5,-2) -- (1.5,2) node[above] {$ - \sigma $};
		\draw[->] (-1.5,-2.5) -- ++(4,0) node[right] {$ x $};
		\draw ($ (1.5,-2.5) + (0,.1) $) -- ($ (1.5,-2.5) + (0,-.1) $) node[below] {$ d $};
		\draw ($ (-1.5,-2.5) + (0,.1) $) -- ($ (-1.5,-2.5) + (0,-.1) $) node[below] {$ 0 $};
		\draw[fill=gray!10] (-1.1,-2) rectangle (1.1,2); 
		\foreach \y in {-1.8,-.6,.6,1.8}
		\draw[->] (-1.3,\y) -- (1.3,\y);
		\foreach \y in {-1.8,-.6,.6,1.8}
		%\node at (-1.8,\y) {$ + $};
		\draw[fill = white!20!red] (-1.8,\y) circle (.2cm) node[] {$ \color{white} + \color{white} $};
		\foreach \y in {-1.8,-.6,.6,1.8}
		%\node at (1.8,\y) {$ - $};
		\draw[fill = white!20!blue] (1.8,\y) circle (.2cm) node[] {$ \color{white} - \color{white} $};
		\node[above] at (0,2) {$ \vec{\tilde{E}} $};
		\coordinate (ar) at (0,-.7);
		\draw[->] ($ (ar) + (-.3,0) $) -- node[above] {$\vec{P}$} ($ (ar) + (.3,0) $);
		\foreach \y in {-1.2,0,1.2}
		\shadedraw[right color=white!10!red, left color=white!20!blue,draw=black] (-.55,\y) ellipse (.45cm and .2cm) node[] {$ \color{white} - \ + \color{black} $};
		\foreach \y in {-1.2,0,1.2}
		\shadedraw[right color=white!10!red, left color=white!20!blue,draw=black] (.55,\y) ellipse (.45cm and .2cm) node[] {$ \color{white} - \ + \color{black} $};
	\end{tikzpicture}
\end{minipage}%

\noindent
\begin{minipage}{.5\linewidth}
	\begin{equation*}
	E = \frac{\sigma}{\epsilon_0} = \frac{q}{\epsilon_0 F}
	\end{equation*}
	\begin{equation*}
	\Phi = - \frac{\sigma x }{\epsilon_0}
	\end{equation*}
	\begin{equation*}
	U = \frac{q d}{\epsilon_0 F}
	\end{equation*}
	\begin{equation*}
	C = \frac{q}{U} = \epsilon_0 \frac{F}{d}
	\end{equation*}
\end{minipage}%
\begin{minipage}{.5\linewidth}
	\begin{equation*}
	\tilde{C} > C \qquad \tilde{C} = \epsilon \frac{F}{d} \quad \epsilon > \epsilon_0
	\end{equation*}
	$$ q = \const $$
	\begin{equation*}
	\tilde{U} = \frac{qd}{\epsilon F} < U
	\end{equation*}
	\begin{equation*}
	\tilde{E} = \frac{q}{\epsilon F} = \frac{\sigma}{\epsilon} < E
	\end{equation*}
	\begin{equation*}
	\tilde{E} = E + E_p = E - \frac{1}{\epsilon_0} P
	\end{equation*}
\end{minipage}%

\subsection{Makroskopische Feldgleichungen der Elektrostatik}

Ausgangspunkt: allgemeine (mikroskopische) Feldgleichungen
\begin{equation*}
\left.\begin{array}{c}
\vec{\nabla} \cdot \vec{E} = \frac{1}{\epsilon_0} \rho \ \\[10pt]
\vec{\nabla} \times \vec{E} = 0 \quad \ \, \: 
\end{array}\right\} \underset{\substack{\rho \\ \tx{bekannt}}}{\rightarrow} \begin{array}{l}
\vec{E}(\vec{r}) = \kq \int \dd ^3 r' \rho(\vec{r}') \frac{\vec{r} - \vec{r}'}{|\vec{r} - \vec{r}'|^3} \\[10pt]
\, \Phi(\vec{r}) = \kq \int \dd ^3 r' \frac{\rho(\vec{r}')}{|\vec{r} - \vec{r}'|}
\end{array}
\end{equation*}
Makroskopische Messungen: $ \approx 10^{23} $ Teilchen\\
$ \rightarrow $ Mittlung über mikroskopische Details.

\subsection{Mittelung von Funktionen}

Wir haben eine physikalische Größe $ A(\vec{r}) $ und wollen diese Mitteln.
\begin{align*}
\langle A \rangle (\vec{r}) &\defeq \int_{\mathbb{R}^3} \dd ^3 r' f(\vec{r} - \vec{r}') A(\vec{r}') \\
&= \int_{\mathbb{R}^3} \dd^3 r' f(\vec{r}') A(\vec{r} - \vec{r}')
\end{align*}
$ f : $ legt Bereich fest, über den gemittelt wird

\noindent
\begin{minipage}{.5\linewidth}
	Eigenschaften:
	\begin{enumerate}[i)]
		\item $ \int \dd^3 r' f(\vec{r}') = 1 $
		\item $ f(\vec{r}) \ge 0 $
		\item Eine glatte Funktion, die sich auf molekularer Skala (nm) wenig ändert.
	\end{enumerate}
\end{minipage}%
\begin{minipage}{.5\linewidth}
	%t6:
	\flushright
	\begin{tikzpicture}
	\draw[->] (0,-.3) -- (0,2.5) node[anchor=south east] {$ y $};
	\draw[->] (-3,0) -- (3,0) node[anchor=north west] {$ x $};
	\draw[thick] (-2.5,0) to[out=0,in=-100] (-2,.8) to[out=80,in=180] (-1.5,1.6) -- (1.5,1.6) to[out=0,in=100] (2,.8) to[out=-80,in=180] (2.5,0);
	\draw[<->] (-1.5,2) -- node[above,xshift=10pt] {$ L $} ++(3,0);
	\foreach \x in {-2.5,1.5}
	\draw[<->] (\x,.7) -- ++(1,0) node[above] {$ \Delta L $};
	\foreach \x\y in {.5/.7,.675/.4,.95/.6}
	\node[circle,fill=black,inner sep=1pt,minimum size=1pt] at (\x,\y) {};
	\draw (.5,.7) -- (.675,.4) -- (.95,.6);
	\draw[<->] ($ (.5,.7) + (0,.2) $) -- node[above] {$ a $} ($ (.95,.6) + (.06,.16) $);
	\draw (1.855,1.355) to[out=60,in=-170] (3,2) node[right] {$ f(\vec{r}) $};
	\end{tikzpicture}\\
	\flushleft
	$ \qquad \, $ mit $ L, \Delta L \gg a $
	\vspace{10pt}
\end{minipage}%
\\
Wir schauen uns nun an wie die Ableitung einer gemittelten Funktion aussieht.
\begin{align*}
\prt{}{x_i} \langle A \rangle (\vec{r}) &= \prt{}{x_i} \int \dd ^3 r' f(\vec{r}') A(\underset{\downarrow}{\vec{r}} - \vec{r}') \\
&= \int \dd ^3 r' f(\vec{r}') \prt{A}{x_i} (\vec{r} - \vec{r}') \\
&= \langle \prt{A}{x_i} \rangle (\vec{r})
\end{align*}
\begin{align*}
\vabla \cdot \vec{E} = \frac{1}{\epsilon_0} \rho \quad &\Rightarrow \quad \vabla \cdot \langle \vec{E} \rangle = \frac{1}{\epsilon_0} \langle \rho \rangle \\
\vabla \times \vec{E} = 0 \quad &\Rightarrow \quad \vabla \times \langle \vec{E} \rangle = 0
\end{align*}
\begin{equation*}
\rightarrow \quad \langle \vec{E} \rangle = - \vabla \langle \Phi \rangle
\end{equation*}

\subsection{Bestimmung von \texorpdfstring{$ \langle \vec{\rho} \rangle $}{mittel rho}}

\begin{minipage}{.5\linewidth}
	Aufteilung der Materie in Untereinheiten:\\
	\textbf{Festkörper:}\\
	Elementarzellen:
\end{minipage}%
\begin{minipage}{.5\linewidth}
	\centering
	%t7:
	\begin{tikzpicture}[scale=.75]
		\foreach \x in {.5,1.5,2.5}
		\draw (\x,0) -- ++(0,3);
		\foreach \y in {.5,1.5,2.5}
		\draw (0,\y) -- ++(3,0);
		\draw[draw=none,pattern=north east lines] (.5,.5) rectangle (1.5,1.5);
	\end{tikzpicture}
	\vspace{10pt}
\end{minipage}%

\noindent
\begin{minipage}{.5\linewidth}
	\textbf{Gas:}\\
	Moleküle
\end{minipage}%
\begin{minipage}{.5\linewidth}
	\centering
	%t8:
	\begin{tikzpicture}[scale=.75]
		\draw (0,0) circle (1cm);
		\node[circle,fill=black,inner sep=1pt,minimum size=1pt] (O) at (0,0) {};
		\coordinate (o) at (2,-1.2);
		\draw[thick,->] (o) -- node[anchor=south west] {$ \vec{r}_n $ Schwerpunkt} (O);
		\draw[thick,->] (o) -- node[below] {$ \vec{r} $} (-.5,-.5);
		\draw (-.55,-.5) to[out=150,in=60] (-2,-.7) node[below] {$ \vec{r} - \vec{r}_n $};
		\draw (.5,.7) to[out=80,in=200] (1,1.5) node[right] {$ \rho_n $};
	\end{tikzpicture}
	\vspace{10pt}
\end{minipage}%
\\
$ \rho_n : $ Ladungsdichte des $ n $-ten Moleküls bzgl. des Schwerpunktes $ \vec{r}_n $. Die gesamte Ladungsdichte ist somit:
\begin{equation*}
\rho_g(\vec{r}) = \sum_{n} \rho_n (\vec{r} - \vec{r}_n)
\end{equation*}
$ \rho_g (\vec{r}) $ sind hierbei alle gebundenen Ladungen.\\
Zusätzlich gibt es möglicherweise freie Ladungsträger $ \rho_f(\vec{r}) $\\
$ \Rightarrow $ gesamte Ladungsdichte
\begin{equation*}
\rho(\vec{r}) = \rho_f(\vec{r}) + \rho_g(\vec{r})
\end{equation*}
Mittlung von $ \rho_g $ über einen makroskopisch kleinen aber mikroskopisch großen Bereich:
\begin{align*}
\langle \rho_g \rangle (\vec{r}) &= \int \dd^3 r' f(\vec{r}') \rho_g (\vec{r} - \vec{r}') \\
&= \int \dd^3 r' f(\vec{r}') \sum_{n} \rho_n (\vec{r} - \vec{r}' - \vec{r}_n) \\
&= \sum_{n} \int_{\mathbb{R}^3} \dd^3 r' f(\vec{r}') \rho_n (\vec{r} - \vec{r}' - \vec{r}_n)
\end{align*}
Nun Betrachten wir den letzten Term:
\begin{align*}
&\int \dd^3 r' f(\vec{r}') \rho_n (\ub{\vec{r} - \vec{r}' - \vec{r}_n}_{= \tilde{\vec{r}}})\\
=& \int \dd ^3 r' \ub{f(\vec{r} - \vec{r}_n - \tilde{\vec{r}})}_{\substack{\tx{ändert sich} \\ \tx{wenig auf} \\ \tx{mol. Skala}}} \ub{\rho_n(\tilde{\vec{r}})}_{\substack{\tx{lokalisiert auf} \\ \tx{molekularer Skala a} \\ \rho_n(\tilde{\vec{r}})\approx 0 \tx{ für } |\tilde{\vec{r}}| \gg a }}\\
\tx{Taylorentwicklung in } \tilde{\vec{r}}: &\quad f(\vec{r} - \vec{r}_n - \tilde{\vec{r}}) = f(\vec{r} - \vec{r}_n) - \tilde{\vec{r}} \cdot \vabla f(\vec{r} - \vec{r}_n) + \dots \\
=& f(\vec{r} - \vec{r}_n) \ub{\int \dd^3 \tilde{r} \rho_n(\tilde{\vec{r}})}_{= q_n} - \vabla f(\vec{r} - \vec{r}_n) \cdot \ub{\int \dd ^3 r' \tilde{\vec{r}} \rho_n(\tilde{\vec{r}})}_{= \vec{p}_m \tx{Dipolmoment}} + \dots \\
& \tx{Höhere Terme werden vernachlässigen z.B. das Quadrupolmoment.} \\
=& f(\vec{r} - \vec{r}_n) q_n - \ub{\vabla f(\vec{r} - \vec{r}_n) \cdot \vec{p}_m}_{= \vabla \cdot (\vec{p}_n f(\vec{r} - \vec{r}_n))} + \dots \\
=& \int \dd^3r' q_n \delta(\vec{r}' - \vec{r}_n) f(\vec{r} - \vec{r}') - \vabla \cdot \int \dd^3r' \vec{p}_n \delta(\vec{r}' - \vec{r}_n) f(\vec{r} - \vec{r}') + \dots \\
=& \langle \custo{\Rightarrow}{q_n}{\mathclap{\tx{Gesamtladung im SP}}} \delta(\vec{r}' - \vec{r}_n) \rangle (\vec{r}) - \vabla \cdot \langle \custo{\Rightarrow}{\vec{p}_n}{\mathclap{\tx{Dipolmoment im SP}}} \delta(\vec{r}' - \vec{r}_n) \rangle (\vec{r}) + \dots
\end{align*}


% oben aufhübschen unbedingt


\noindent
\begin{align*}
\langle \rho_g \rangle &= \sum_n \int \dots \\
&= \langle \sum_n q_n \delta(\vec{r}' - \vec{r}_n) \rangle (\vec{r}) - \vabla \cdot \langle \sum_n \vec{p}_n \delta(\vec{r}' - \vec{r}_n) \rangle(\vec{r}) + \dots
\end{align*}
Der \textbf{erste Term} steht für die mittlere Gesamtladung der gegebenen Ladungen $ = 0 $ für:
\begin{enumerate}[i)]
	\item neutrale Untereinheiten
	\item makroskopisch neutraler Körper
\end{enumerate}
Der \textbf{zweite Term} wird Definiert als das makroskopische Dipolmoment $ \eqdef \vec{P} (\vec{r}) = \frac{\tx{Dipolmoment}}{\tx{Volumen}} $
\begin{align}
\vec{P}(\vec{r}) &= \sum_n \vec{p}_n \int \dd^3r' \delta(\vec{r}' - \vec{r}_n) f(\vec{r} - \vec{r}')\\
&= \sum_n \vec{p}_n f(\vec{r} - \vec{r}_n) \approx \frac{1}{V} \sum_{m,\vec{r}_n \in V} \vec{p}_n
\end{align}
\begin{minipage}{.5\linewidth}
	\begin{equation*}
	f(\vec{r} - \vec{r}_n) \approx \left\{ \begin{array}{cc}
	\frac{1}{V} & |\vec{r} - \vec{r}_n| \le L \\[5pt]
	0 & \tx{sonst}
	\end{array}\right.
	\end{equation*}
\end{minipage}%
\begin{minipage}{.5\linewidth}
	\centering
	%t9:
	\begin{tikzpicture}
		\draw[->] (0,-.3) -- (0,2.5) node[anchor=south east] {$ y $};
		\draw[->] (-3,0) -- (3,0) node[anchor=north west] {$ x $};
		\draw[thick] (-2.5,0) to[out=0,in=-100] (-2,.8) to[out=80,in=180] (-1.5,1.6) -- (1.5,1.6) to[out=0,in=100] (2,.8) to[out=-80,in=180] (2.5,0);
		\draw[<->] (-1.5,2) -- node[above,xshift=30pt] {$ L \ \ \rightarrow V $} ++(3,0);
		\draw (-1.855,1.355) to[out=120,in=-10] (-3,2) node[left] {$ f(\vec{r}) $};
	\end{tikzpicture}
\end{minipage}%
\\
$ \rightarrow $ gemittelte (makroskopische) Ladungsdichte:
\begin{align*}
\langle \rho \rangle (\vec{r}) &= \langle \rho_f \rangle(\vec{r}) + \ub{\langle \sum_n q_n \delta(\vec{r}' - \vec{r}_n) \rangle(\vec{r})}_{\substack{\tx{Gesamtladung} \\ \tx{oft } = 0}} - \vabla \cdot \vec{P}(\vec{r}) + \dots
\end{align*}
\frbox{Gemittelte makroskopische Ladungsverteilung}{
\begin{equation*}
\langle \rho \rangle (\vec{r}) = \langle \rho_f \rangle (\vec{r}) - \vabla \cdot \vec{P}(\vec{r}) + \dots
\end{equation*}
}

\noindent
Daraus folgt:
\begin{equation*}
\rightarrow \quad \vabla \cdot \langle\vec{E} \rangle (\vec{r}) = \frac{1}{\epsilon_0} \langle\rho \rangle (\vec{r}) = \frac{1}{\epsilon_0} \langle \rho_f \rangle (\vec{r}) - \frac{1}{\epsilon_0} \vabla \cdot \vec{P}(\vec{r}) + \dots
\end{equation*}
Dies können wir umformen in etwas, das der Maxwellgleichung ähnelt:
\begin{equation*}
\vabla \cdot \ub{(\epsilon_0 \langle \vec{E} \rangle + \vec{P} + \dots)}_{\defeq \vec{D}(\vec{r})} = \langle \rho_f \rangle (\vec{r})
\end{equation*}
$ \vec{D}(\vec{r}) \defeq $ dielektrische Verschiebung.
\begin{equation*}
\rightarrow \quad \vabla \cdot \vec{D}(\vec{r}) = \langle \rho_f \rangle (\vec{r})
\end{equation*}

\subsection{Makroskopische Feldgleichungen der Elektrostatik \tiny(Wiederholung)}

\begin{equation*}
\vabla \cdot \vec{D}(\vec{r}) = \rho_f (\vec{r})
\end{equation*}
\begin{equation*}
\vabla \times \vec{E}(\vec{r}) = 0
\end{equation*}
\begin{equation*}
\vec{D} = \epsilon_0 \vec{E} + \vec{P} + \dots
\end{equation*}
Diese Gleichungen können wir nun mit dem Satz von Gauß und dem Satz von Stokes auch in Integraler Form schreiben:

\noindent
\begin{minipage}{.6\linewidth}
	\begin{equation*}
	\oint_{\partial V} \dd\vec{f} \cdot \vec{D} = \int_{V} \dd^3r \vabla \cdot \vec{D} =  \int_{V} \dd^3 r \rho_f(\vec{r}) = q_{f_V}
	\end{equation*}
	\begin{equation*}
	\int_{\partial V} \dd \vec{f} \ \vabla \times \vec{E} = \oint_{\partial F} \dd\vec{r} \ \vec{E} = 0
	\end{equation*}
\end{minipage}%
\begin{minipage}{.4\linewidth}
	\centering
	%t10:
	\begin{tikzpicture}[scale=.7]
	\node at (-2.3,0) {Gauß};
	\draw[fill=gray!30!white]  plot [smooth cycle, tension=1] coordinates { (-.3,-2) (-.3,-1) (-1,0) (-.3,1) (1,1.3) (.6,0) (1.3,-1) (.5,-1.6) };
	\node at (0,0) {$ V $};
	\end{tikzpicture}
	%t11:
	\begin{tikzpicture}[scale=.7]
	\node at (-2.3,0) {Stokes};
	\draw[fill=gray!30!white] (0,0) circle (1cm);
	\draw[->] (1,0) -- (1,.01) node[right] {$ \partial F $};
	\end{tikzpicture}
	\vspace{5pt}
\end{minipage}%

% Vorlesung 26.11.18

\bbb{Wiederholung}{
\textbf{Makroskopische Feldgleichungen der Elektrostatik}
\begin{equation*}
\vabla \times \vec{E}(\vec{r}) = 0 \qquad \langle E \rangle (\vec{r})
\end{equation*}
\begin{equation*}
\vabla \cdot \vec{D} (\vec{r}) = \rho_f(\vec{r})
\end{equation*}
\begin{equation*}
\vec{D} = \epsilon_0 \vec{E} + \vec{P}
\end{equation*}
\lcom{Wir haben hier jetzt zwei Feldgleichungen für zwei Vektorfelder. Dies reicht nicht aus um beide Vektorfelder eindeutig zu bestimmen. Hierfür müssen wir die Wirbel und Quellen beider Felder beschreiben. $ \vec{E} $ und $ \vec{D} $ sind also nicht unabhängig sondern miteinander verknüpft.}
}
\noindent
\emph{Bemerkung:} \textbf{(Schlussfolgerungen aus den Feldgleichungen der Elektrostatik)}
\begin{enumerate}[i)]
	\item \lcom{Es sieht so aus als ob $ \vec{D} $ nur von der freien Ladungsdichte abhängt, dies ist aber nur in manchen fällen so (Plattenkondensator).}
	
	\noindent
	\begin{minipage}{.5\linewidth}
		Es gilt nur wenn $ \vec{\nabla} \times \vec{D} = 0 $\\
		\textbf{Gegenbeispiel:}
		homogen polarisierte Kugel:
		\begin{equation*}
		\vec{E} = - \frac{1}{3 \epsilon_0} \vec{P} \qquad \tx{in der Kugel}
		\end{equation*}
		\begin{equation*}
		\rightarrow \quad \vec{D} = \epsilon_0 \vec{E} + \vec{P} = \frac{2}{3} \vec{P}
		\end{equation*}
	\end{minipage}%
	\begin{minipage}{.5\linewidth}
		\centering
		%t1:
		\begin{tikzpicture}
			\draw (0,0) circle (1.5 cm);
			\draw[->] (.2,-.5) -- node[left] {$ \vec{P} $} ++(0,1);
			\draw[->] (2,.5) -- node[right] {$ \vec{E} $} ++(0,-1);
			\foreach \y in {-.6,.6}
			\shadedraw[top color=white!10!red, bottom color=white!20!blue,draw=black] (-.55,\y) ellipse (.2cm and .45cm) node[] {$ \substack{ \color{white} + \color{black} \\[5pt] \color{white} - \color{black}} $};
			\foreach \x in {-.6,.6}
			\shadedraw[top color=white!10!red, bottom
			color=white!20!blue,draw=black] (+.55,\x) ellipse (.2cm and .45cm) 	node[] {$ \substack{ \color{white} + \color{black} \\[5pt] \color{white} - \color{black}} $};
			\draw[->] (.9,-.4) -- node[right] {$ \vec{D} $} ++(0,.8);
		\end{tikzpicture}
	\end{minipage}%
	\item \begin{equation*}
	\vec{E} = \frac{1}{\epsilon_0} (\vec{D} - \vec{P})
	\end{equation*}
	$ \vec{E} $ hängt über die Polarisation direkt von dem Medium ab.
	\begin{align*}
	\vabla \cdot \vec{E} &= \frac{1}{\epsilon_0} \vabla \cdot \vec{D} - \frac{1}{\epsilon_0} \vec{\nabla} \cdot \vec{P} \\
	&= \frac{1}{\epsilon_0} \rho_f - \frac{1}{\epsilon_0} \vabla \cdot \vec{P}
	\end{align*}
	$ \rightarrow $ Polarisationsladungsdichte $ \rho_p = - \vec{\nabla} \cdot \vec{P} $
	\begin{equation*}
	\Rightarrow \quad \vabla \cdot \vec{E} = \frac{1}{\epsilon_0} (\rho_f + \rho_p)
	\end{equation*}
	\item Die Polarisation wirkt wie ein inneres Zusatzfeld, das sich mit  dem durch $ \rho_f $ erzeugten Feld $ \vec{E}_0 $ überlagert.
	$ \vec{E} = \vec{E}_0 + \vec{E}_p $\\
	\begin{minipage}{.5\linewidth}
		Im Plattenkondensator:
		\begin{equation*}
		\vec{E}_p = - \frac{1}{\epsilon_0} \vec{P}
		\end{equation*}
		\begin{equation*}
		\vec{E} = \vec{E}_0 - \frac{1}{\epsilon_0} \vec{P}
		\end{equation*}
	\end{minipage}%
	\begin{minipage}{.5\linewidth}
		\centering
		%t2:
		\begin{tikzpicture}
			\draw[very thick] (-1.5,-2) -- (-1.5,2) node[above] {$ \sigma $};
			\draw[very thick] (1.5,-2) -- (1.5,2) node[above] {$ - \sigma $};
			\node[above] at (1.5,2.4) {$ - \rho_f $};
			\node[above] at (-1.5,2.4) {$ \rho_f $};
			%axis:
			%\draw[->] (-1.5,-2.5) -- ++(4,0) node[right] {$ x $};
			%\draw ($ (1.5,-2.5) + (0,.1) $) -- ($ (1.5,-2.5) + (0,-.1) $) node[below] {$ d $};
			%\draw ($ (-1.5,-2.5) + (0,.1) $) -- ($ (-1.5,-2.5) + (0,-.1) $) node[below] {$ 0 $};
			\draw[fill=gray!10] (-1.1,-2) rectangle (1.1,2); 
			\foreach \y in {-1.8,-.6,.6,1.8}
			\draw[->] (-1.3,\y) -- (1.3,\y);
			\foreach \y in {-1.8,-.6,.6,1.8}
			%\node at (-1.8,\y) {$ + $};
			\draw[fill = white!20!red] (-1.8,\y) circle (.2cm) node[] {$ \color{white} + \color{white} $};
			\foreach \y in {-1.8,-.6,.6,1.8}
			%\node at (1.8,\y) {$ - $};
			\draw[fill = white!20!blue] (1.8,\y) circle (.2cm) node[] {$ \color{white} - \color{white} $};
			\node[above] at (0,2) {$ \vec{E}_0 $};
			\coordinate (ar) at (0,-.5);
			\draw[->,color=blue] ($ (ar) + (.3,-.3) $) -- node[above] {$\vec{E}_P $} ++(-.6,0);
			\draw[->,red] ($ (ar) + (-.3,1) $) -- node[above] {$\vec{P} $} ++(.6,0);
			\foreach \y in {-1.2,0,1.2}
			\shadedraw[right color=white!10!red, left color=white!20!blue,draw=black] (-.55,\y) ellipse (.45cm and .2cm) node[] {$ \color{white} - \ + \color{black} $};
			\foreach \y in {-1.2,0,1.2}
			\shadedraw[right color=white!10!red, left color=white!20!blue,draw=black] (.55,\y) ellipse (.45cm and .2cm) node[] {$ \color{white} - \ + \color{black} $};
		\end{tikzpicture}
	\end{minipage}%
	\item Potential
	\begin{equation*}
	\vabla \times \langle \vec{E} \rangle = 0
	\end{equation*}
	\begin{equation*}
	\langle \vec{E} \rangle = - \vabla \langle \Phi \rangle
	\end{equation*}
	Einfach aber zu viel Zeitaufwand für die Vorlesung
	\begin{align*}
	\langle \Phi \rangle &= \kq \int \dd^3  r' \frac{\langle \rho \rangle (\vec{r}')}{|\vec{r} - \vec{r}'|} \\
	&= \kq \int \dd ^3 r' \frac{1}{|\vec{r} - \vec{r}'|}\left(\rho_f(\vec{r}') - \vabla \cdot \vec{P}(\vec{r}') + \dots \right)
	\end{align*}
	\item \textbf{Zusammenhang zwischen $ \vec{P} $ und $ \vec{E} $: Suszeptibilität}
	\begin{equation*}
	\vec{P} = \vec{P(\vec{E})} \qquad \vec{P}(\vec{E} = 0) = 0
	\end{equation*}
	Entwicklung von $ \vec{P} $ in Potenzen von $ \vec{E} $:
	\begin{align*}
	P_i = \sum_{j=1}^{3} \gamma_{ij} E_j + \sum_{j,k=1}^{3} \beta_{ijk} E_j E_k + \dots
	\end{align*}
	$ \gamma_{ij} $ \& $ \beta_{ijk} $ sind Materialkonstanten\\
	lineare Näherung:\\
	allgemeines \textbf{anisotropes} Dielektrikum
	\begin{equation*}
	P_i = \sum_j \gamma_{ij} E_j
	\end{equation*}
	isotropes Dielektrikum:
	\begin{equation*}
	P_i = \gamma E_i
	\end{equation*}
	\begin{equation*}
	\vec{P} = \chi_e \epsilon_0 \vec{E} \qquad \chi_e \epsilon_0 = \gamma
	\end{equation*}
	$ \chi_e $ ist die \textbf{Dielektrische Suszeptibilität}
	\begin{align*}
	\rightarrow \quad \vec{D} &= \vec{P} + \epsilon_0 \vec{E} = (\chi_e \epsilon_0 + \epsilon_0) \vec{E} \\
	&= \ub{(1 + \chi_e)}_{\defeq \epsilon_r} \epsilon_0 \vec{E} \\
	&= \epsilon \vec{E} = \epsilon_0 \epsilon_r \vec{E}
	\end{align*}
	\begin{minipage}{.6\linewidth}
		$ \epsilon_r $ ist die relative Dielektrizitätskonstante\\
		$ \epsilon = \epsilon_r \epsilon_0 $ ist die Dielektrizitätskonstante
		\begin{equation*}
		\rmbox{\vec{D} = \epsilon \vec{E} = \epsilon_0 \epsilon_r \vec{E}}
		\end{equation*}
		$ \epsilon_0 $ ist die elektrische Feldkonstante oder auch Permittivität / Dielektrizitätskonstante des Vakuums
	\end{minipage}%
	\begin{minipage}{.4\linewidth}
		\flushleft
		Typische Werte für $ \epsilon_r $:\\[5pt]
		\centering
		\begin{tabular}{l|c}
			Medium & $ \epsilon_r $ \\ \hline 
			Vakuum: & $ \epsilon_r = 1 $ \\
			$ \tx{H}_2 $: & 1,00025 \\
			$ \tx{N}_2 $: & 1,00055 \\
			$ \tx{H}_2 \tx{O} $: & 80,1
		\end{tabular}
	\end{minipage}%
\end{enumerate}

\subsection{Feldgleichungen für lineares, isotropes Dielektrikum} % check if section is right

\begin{equation*}
\vabla \cdot \vec{D} = \rho_f
\end{equation*}
\begin{center}
	\begin{minipage}{.5\linewidth}
		\rbox{
			\vspace{-12.5pt}
			\begin{align*}
			\vabla \cdot (\epsilon \vec{E}) &= \rho_f \\ 
			\vabla \times \vec{E} &= 0
			\end{align*}
		}
	\end{minipage}
\end{center}
\textbf{homogenes Medium} $ \epsilon = \const $
\begin{equation*}
\epsilon \vabla \cdot \vec{E} = \rho_f
\end{equation*}
\begin{equation*}
\rightarrow \vabla \cdot \vec{E} = \frac{1}{\epsilon \rho_f} = \ub{\frac{1}{\epsilon_r}}_{\mathclap{\tx{Medium}}} \frac{1}{\epsilon_0} \rho_f
\end{equation*}
\begin{equation*}
\vabla \times \vec{E} = 0
\end{equation*}
\begin{align*}
\Delta \Phi &= - \vabla \cdot \vec{E} \\
&= - \frac{1}{\epsilon} \rho_f
\end{align*}
\begin{align*}
\rightarrow \quad \Delta \Phi &= - \custo{\leftarrow}{\frac{1}{\epsilon_0}}{} \ \frac{1}{\epsilon_r} \rho_f
\end{align*}

\subsection{Punktladung in homogenem Dielektrikum (lineare Näherung)} % check this too

\begin{minipage}{.6\linewidth}
	\begin{align*}
	\rho_f(\vec{r}) &= q \delta(\vec{r}) \\
	\vabla \cdot \vec{E}(\vec{r}) &= \frac{1}{\epsilon} q \delta(\vec{r}) \\
	\vabla \times \vec{E}(\vec{r}) &= 0
	\end{align*}
\end{minipage}%
\begin{minipage}{.4\linewidth}
	\centering
	%t3:
	\begin{tikzpicture}[scale=.8]
		\draw[pattern=north east lines, pattern color = black!30!white,scale=1.5]  plot [smooth cycle, tension=0.8] coordinates { (-1,0) (-1,1) (0,1) (.5,.5) (1,0) (.7,-.7) (0,-1) (-.4,-.7) };
		\coordinate (o) at (-.35,-.25);
		\draw[->] (o) -- ++(1,0) node[anchor=north west] {$ x $};
		\draw[->] (o) -- ++(0,1) node[anchor=south east] {$ y $};
		\node[circle,fill=black,inner sep=1pt,minimum size=1pt] at (o) {};
		\node[anchor=north east] at (o) {$ q $};
		\node at (-1.4,1.2) {$ \vec{\epsilon} $};
		\draw ($ (o) + (-.8,0) $) to[out=-135,in=90] (-1.4,-1.5) node[below] {Medium};
	\end{tikzpicture}
\end{minipage}%

\noindent
\begin{minipage}{.6\linewidth}
	Nun können wir das $ \vec{E} $-Feld im Vakuum bestimmen:
	\begin{align*}
	\rightarrow \quad \eofr &= \frac{1}{4 \pi \epsilon} q \frac{\vec{r}}{r^3} \\
	&= \frac{1}{\epsilon_r} \ub{\kq q \frac{\vec{r}}{r^3}}_{= \vec{E}_{\tx{vak}}}\\
	&= \frac{1}{\epsilon_r} \vec{E}_{\tx{vak}} (\vec{r}) < \vec{E}_{\tx{vak}}(\vec{r})\\[10pt]
	\rightarrow \qquad \ \vec{P} &= \chi_e \epsilon_0 \vec{E} = (\epsilon_r - 1) \epsilon_0 \vec{E}\\
	&= \frac{(\epsilon_r - 1)}{4 \pi \epsilon_r} q \frac{\vec{r}}{r^3}
	\end{align*}
	\begin{equation*}
	\rmbox{\Rightarrow \quad \vec{E} = \vec{E}_{\tx{vak}} - \frac{1}{\epsilon_0} \vec{P}}
	\end{equation*}
	\vspace{3pt}
\end{minipage}%
\begin{minipage}{.4\linewidth}
	\centering
	%t4:
	\begin{tikzpicture}
		\draw (-1.5,0) -- (1.5,0);
		\draw (0,-1.5) -- (0,1.5);
		\draw[<->] (-1,0) -- (1,0);
		\draw[<->] (0,-1) -- (0,1);
		\draw (0,0) circle (.5cm);
		\node at (1.25,1.25) {Vakuum};
		\node at (-.8,.8) {$ \vec{E} $};
	\end{tikzpicture}
\end{minipage}%
\\
\lcom{Hier erkennt man explizit, dass das Vakuum-Feld von der Polarisation vermindert wird.}
\begin{equation*}
\vec{D} = \epsilon \vec{E} = \frac{1}{4 \pi} q \frac{\vec{r}}{r^3}
\end{equation*}
\lcom{In diesem einfachen Fall ist $ \vec{D} $ vollständig durch die freie Ladung $ q $ (in der Abbildung Positiv) bestimmt.}

\subsection{Zusammenhang zwischen atomarer/molekularer Polarisierbarkeit und Suszeptibilitäten}

%Dies ist ein testtext der nur zumm zweck des testen eines textes dient ansonsten erfüllt dieser testtext keinen weeiteren %sinn Dies ist ein testtext der nur zumm zweck des testen eines textes dient ansonsten erfüllt dieser testtext keinen %weeiteren sinn Dies ist ein testtext der nur zumm zweck des testen eines textes dient ansonsten erfüllt dieser testtext %keinen weeiteren sinn
\begin{minipage}{.5\linewidth}
	\textbf{Verschiebungspolarisation:}
	\begin{equation*}
	\vec{p} = \alpha \vec{E}_{\tx{lokal}}
	\end{equation*}
	$ \Rightarrow $ Polarisation:
	\begin{equation*}
	\vec{P} = n \vec{p} = n \alpha \vec{E}_{\tx{lok}}
	\end{equation*}
\end{minipage}%
\begin{minipage}{.5\linewidth}
	\centering
	%t5:
	\begin{tikzpicture}
		\shadedraw[right color=white!10!red, left color=white!20!blue,draw=black] (1,0) ellipse (.45cm and .2cm) node[] {$ \color{white} - \ + \color{black} $};
		\shadedraw[top color=white!10!red, bottom
		color=white!20!blue,draw=black] (0,1) ellipse (.2cm and .45cm) 	node[] {$ \substack{ \color{white} + \color{black} \\[5pt] \color{white} - \color{black}} $};
		\shadedraw[right color=white!20!blue, left color=white!10!red,draw=black] (-1,0) ellipse (.45cm and .2cm) node[] {$ \color{white} + \ - \color{black} $};
		\shadedraw[top color=white!20!blue, bottom
		color=white!10!red,draw=black] (0,-1) ellipse (.2cm and .45cm) 	node[] {$ \substack{ \color{white} - \color{black} \\[5pt] \color{white} + \color{black}} $};
		\draw[fill=white!10!blue!15!red] (0,0) circle (.25cm) node[] {$\color{white} + \color{black}$};
		\node at (0,-2) {$ \substack{\tx{polarisierbares} \\ \tx{Medium}} $};
		\coordinate (p) at (-.5,1);
		\coordinate (ep) at (.5,1);
		\coordinate (d) at (1.5,1);
		\coordinate (e) at (2.5,1);
		\draw[thick,->] ($ (p) + (0,-.4) $) -- node[left] {$ \vec{P} $} ++(0,.8);
		\draw[thick,->] ($ (ep) + (0,.4) $) -- node[right] {$ \vec{E}_P $} ++(0,-.8);
		\draw[thick,->] ($ (e) + (0,-.4) $) -- node[right] {$ \vec{E} $} ++(0,.8);
		\draw[thick,->] ($ (d) + (0,-.25) $) -- node[right] {$ \vec{D} $} ++(0,.5);
	\end{tikzpicture}
\end{minipage}%

\noindent
$ n $ ist die Teilchenzahldichte.\\
Aus den makroskopischen Gleichungen haben wir erhalten:
\begin{equation*}
\vec{P} = \chi_e \epsilon_0 \equalto{\vec{E}}{\mathclap{ \substack{\tx{makroskopisches} \\ \tx{Feld} }}}
\end{equation*}
In einem verdünnten Gas gilt: $ \vec{E}_{\tx{lok}} \approx \vec{E} $
\begin{equation*}
\Rightarrow \quad \chi_e \epsilon_0 \vec{E} = n \alpha \vec{E} \quad \Rightarrow \quad \rmbox{\chi_e = \frac{n \alpha}{\epsilon_0}}
\end{equation*}
\begin{equation*}
\rmbox{\epsilon_r = 1 + \frac{n \alpha}{\epsilon_0}}
\end{equation*}

\subsection{Randwertprobleme}

\begin{equation*}
\vabla \times \vec{E} = 0 \qquad \vabla \cdot \vec{D} = \rho_f
\end{equation*}
\begin{equation*}
\vec{D} = \epsilon_0 \vec{E} + \vec{P} \qquad \vec{P} = \vec{P} (\vec{E})
\end{equation*}
lineares homogenes Dielektrikum
\begin{equation*}
\vabla \cdot \vec{E} = \frac{1}{\epsilon} \rho_f \qquad \vabla \times \vec{E} = 0
\end{equation*}
\begin{equation*}
\Delta \Phi = - \frac{1}{\epsilon} \rho_f
\end{equation*}
$ \rightarrow $ \textbf{Randwertproblem:}\\
Gegeben: $ \rho_f , \epsilon $ Randbedingungen\\
Gesucht: $ \Phi, \vec{E} $

\subsection{Randbedingungen für \texorpdfstring{$ \vec{D}, \vec{E} $}{D E} an einer Grenzschicht mit Flächenladung}

\begin{center}
	%t6:
	\begin{tikzpicture}
		\coordinate (e) at (-4,2);
		\shadedraw[top color=white!10!red, bottom
		color=white!20!blue,draw=black] (e) ellipse (.2cm and .45cm) 	node[] {$ \substack{ \color{white} + \color{black} \\[5pt] \color{white} - \color{black}} $};
		\draw[thick,->] ($ (e) + (.5,0) + (0,-.4) $) -- node[right] {$ \vec{E} $} ++(0,.8);
		%neues Bild
		\centerarc[](0,0)(80:10:3cm);
		\centerarc[thick,dashdotted](0,0)(80:10:2.9cm);
		\coordinate (o) at (45:3cm);
		\draw[thick,->] (o) -- ++(45:1cm) node[anchor=south west] {$ \vec{n} $};
		\draw[thick,->] (o) -- ++(-45:1cm) node[anchor=north west] {$ \vec{t} $};
		\node at (2.5,.8) {$ \sigma_f $};
		\node at (1,3.5) {$ \epsilon_1 $};
		\node at (.6,2.15) {$ \epsilon_2 $};
	\end{tikzpicture}
\end{center}
Erinnerung: mikroskopische Feldgleichungen
\begin{equation*}
\vabla \cdot \vec{E} = \frac{1}{\epsilon_0} \rho \quad \rightarrow \quad \vec{n} \cdot (\vec{E}_1 - \vec{E}_2) = \frac{\sigma}{\epsilon_0}
\end{equation*}
\begin{equation*}
\vabla \times \vec{E} = 0 \quad \Rightarrow \quad \vec{t} \cdot (\vec{E}_1 - \vec{E}_2) = 0
\end{equation*}
für makroskopische Feldgleichungen:
\begin{equation*}
\vabla \cdot \vec{D} = \rho_f \quad \Rightarrow \quad \vec{n} \cdot (\vec{D}_1 - \vec{D}_2) = \sigma_f
\end{equation*}
\begin{equation*}
\vabla \times \vec{D} = 0 \quad \Rightarrow \quad \vec{t} \cdot (\vec{E}_1 - \vec{E}_2) = 0
\end{equation*}
speziell lineare, homogene Dielektrika ($ \epsilon_1 = \const, \epsilon_2 = \const $):
\begin{equation*}
\vec{D}_i = \epsilon_i \vec{E}_i \qquad i = 1,2
\end{equation*}
\begin{equation*}
\vec{n} \cdot (\epsilon_1 \vec{E}_1 - \epsilon_2 \vec{E}_2) = \sigma_f
\end{equation*}
Falls $ \sigma_f = 0 $ (es gibt also \textbf{keine} Ladung an der Oberfläche);
\begin{equation*}
\Rightarrow \quad \vec{n} \vec{E}_1 = \frac{\epsilon_2}{\epsilon_1} \vec{n} \vec{E}_2
\end{equation*}
Das heißt, das $ \vec{E} $-Feld ist unstetig wenn $ \epsilon_1 \neq \epsilon_2 $ (aufgrund der \textbf{Polarisationsladung}).

% mach die tabelle oben fertig

% Vorlesung 29.11.18

\bbb{Wiederholung}{
zu \textbf{Randbedingunen für $ \vec{D} $ und $ \vec{E} $ an Grenzflächen mit Flächenladung}
\begin{align*}
\vec{n} \cdot (\vec{D}_1 - \vec{D}_2) &= \sigma_f\\
\vec{t} \cdot (\vec{E}_1 - \vec{E}_2) &= 0
\end{align*}
}

\noindent
\emph{Beispiel:} \textbf{Plattenkondensator mit Dielektrikum}\\
\begin{minipage}{.4\linewidth}
	\centering
	%t1:
	\begin{tikzpicture}
		%\fill[gray!20] (-1.5,-2) rectangle (1.5,2); 
		\draw[fill=gray!10] (-1.25,-2) rectangle (1.25,2); 
		\draw[very thick] (-1.5,-2) -- (-1.5,2) node[above] {$\sigma$};
		\draw[very thick] (1.5,-2) -- (1.5,2) node[above] {$-\sigma$};
		\node[above] at (0,1.3) {$\epsilon$};
		\node[above] at (0,0) {$\vec{E}_M = ?$};
		\node[above] at (0,-1) {$\vec{D}_M = ?$};
		\node[above] at (-2.5,-.2) {$\epsilon_0$};
		\node[above] at (2.5,-.2) {$\epsilon_0$};
		\node[above] at (2.5,-3.5) {R};
		\node[above] at (-2.5,-3.5) {L};
		\node[above] at (0,-3.5) {M};
		%axis:
		\draw[->] (-2,-2.5) -- ++(4,0) node[right] {$ x $};
		\draw ($ (1.5,-2.5) + (0,.1) $) -- ($ (1.5,-2.5) + (0,-.1) $) node[below] {$ d $};
		\draw ($ (-1.5,-2.5) + (0,.1) $) -- ($ (-1.5,-2.5) + (0,-.1) $) node[below] {$ 0 $};
		\foreach \y in {-1.8,-.6,.6,1.8}
		\draw[fill = white!20!red] (-1.8,\y) circle (.2cm) node[] {$ \color{white} + \color{white} $};
		\foreach \y in {-1.8,-.6,.6,1.8}
		\draw[fill = white!20!blue] (1.8,\y) circle (.2cm) node[] {$ \color{white} - \color{white} $};
		\coordinate (ar) at (0,-.5);
	\end{tikzpicture}
	\vspace{3pt}
\end{minipage}%
\begin{minipage}{.6\linewidth}
	\begin{equation*}
	\vec{E}_{L/R} = 0 \qquad \vec{D} = \epsilon \vec{E} \quad \rightarrow \quad \vec{D}_{L/R} = 0
	\end{equation*}
	\begin{equation*}
	\vec{E}_M = E_x \vec{e}_x \qquad \vec{D}_M = D_x \vec{e}_x \qquad \vec{n} = \vec{e}_x
	\end{equation*}
\end{minipage}%
\\
Für den linken Bereich gilt (analog auch rechts):
\begin{equation*}
\vec{n} \cdot (\vec{D}_M - \equalto{\vec{D}_L}{0}) = \sigma = \frac{q}{F}
\end{equation*}
Für den mittleren Bereich:
\begin{equation*}
\prd{D_x}{x} = \vabla \cdot \vec{D}_M = 0 % das m hier war klein aber alle anderen sind groß ... :
\end{equation*}
\begin{equation*}
\rightarrow \vec{D}_M = \sigma \vec{e}_x
\end{equation*}
\begin{minipage}{.5\linewidth}
	\begin{align*}
	\vec{E}_M &= \frac{1}{\epsilon} \vec{D}_M = \frac{\sigma}{\epsilon} \vec{e}_x\\
	&= \frac{1}{\epsilon_r} \ub{\frac{\sigma}{\epsilon_0} \vec{e}_x}_{\vec{E}_{\tx{vak}}} = \frac{1}{\epsilon_r} \vec{E}_{M_\tx{vak}} \le \vec{E}_{M_\tx{vak}}\\[10pt]
	\vec{P} &= \chi_e \epsilon_0 \vec{E} = (\epsilon_r - 1) \epsilon_0 \vec{E}\\
	&= \casess{0}{L/R}{\frac{\epsilon_r - 1}{\epsilon_r}\sigma \vec{e}_x}{M}
	\end{align*}
	\begin{equation*}
	\rmbox{\Rightarrow \quad \vec{E}_M = \vec{E}_{\tx{vak}} - \frac{1}{\epsilon_0} \vec{P}}
	\end{equation*}
\end{minipage}%
\begin{minipage}{.5\linewidth}
	\centering
	%t2:
	\begin{tikzpicture}[scale=1.2]
		\draw[very thick] (-1.5,-2) -- (-1.5,2) node[above] {$ \sigma $};
		\draw[very thick] (1.5,-2) -- (1.5,2) node[above] {$ - \sigma $};
		%\node[above] at (1.5,2.4) {$ - \rho_f $};
		%\node[above] at (-1.5,2.4) {$ \rho_f $};
		%axis:
		%\draw[->] (-1.5,-2.5) -- ++(4,0) node[right] {$ x $};
		%\draw ($ (1.5,-2.5) + (0,.1) $) -- ($ (1.5,-2.5) + (0,-.1) $) node[below] {$ d $};
		%\draw ($ (-1.5,-2.5) + (0,.1) $) -- ($ (-1.5,-2.5) + (0,-.1) $) node[below] {$ 0 $};
		\draw[fill=gray!10] (-1.1,-2) rectangle (1.1,1.5); 
		\foreach \y in {-1.8,-.6,.6,1.8}
		\draw[->] (-1.3,\y) -- (1.3,\y);
		\foreach \y in {-1.8,-.6,.6,1.8}
		%\node at (-1.8,\y) {$ + $};
		\draw[fill = white!20!red] (-1.8,\y) circle (.2cm) node[] {$ \color{white} + \color{white} $};
		\foreach \y in {-1.8,-.6,.6,1.8}
		%\node at (1.8,\y) {$ - $};
		\draw[fill = white!20!blue] (1.8,\y) circle (.2cm) node[] {$ \color{white} - \color{white} $};
		\node[above] at (0,2) {$ \vec{E}_{\textrm{vak}} $};
		\coordinate (ar) at (0,-.5);
		\draw[->,blue] ($ (ar) + (-.3,-.3) $) -- node[below] {$\vec{E}_{\textrm{M}} $} ++(.6,0);
		\draw[->,red] ($ (ar) + (-.3,1) $) -- node[below] {$\vec{P} $} ++(.6,0);
		\foreach \y in {-1.5,-.3,1}
		\shadedraw[right color=white!10!red, left color=white!20!blue,draw=black] (-.55,\y) ellipse (.45cm and .2cm) node[] {$ \color{white} - \ + \color{black} $};
		\foreach \y in {-1.5,-.3,1}
		\shadedraw[right color=white!10!red, left color=white!20!blue,draw=black] (.55,\y) ellipse (.45cm and .2cm) node[] {$ \color{white} - \ + \color{black} $};
	\end{tikzpicture}
\end{minipage}%

\subsubsection{Spannung und Kapazität}

\begin{equation*}
\vec{E} = - \vabla \cdot \Phi \qquad \Phi(x) = - \frac{\sigma}{\epsilon} x
\end{equation*}
\begin{equation*}
U = \Phi(0) - \Phi(d) = \frac{\sigma d}{\epsilon} = \frac{q}{\epsilon F} d = \frac{1}{\epsilon_r} \ub{\frac{q}{\epsilon_0 F} d}_{U_{\tx{vak}}} \le U_{\tx{vak}}
\end{equation*}
\begin{equation*}
C = \frac{q}{U} = \frac{\epsilon F}{d} = \epsilon_r \ub{\frac{\epsilon_0 F}{d}}_{C_\tx{vak}} \ge C_{\tx{vak}}
\end{equation*}
\lcom{Dies gilt für den Fall eines Kondensators mit \textbf{\ul{fester Ladung}} aud den Platten.}

\subsubsection{anderes Szenario: \ul{feste Spannung}}

\begin{minipage}{.5\linewidth}
	\centering
	%t3:
	\begin{tikzpicture}
		\draw[pattern=north east lines] (-.5,-.5) rectangle (.5,.5);
		\draw[thick,-] (-.6,-.6) -- (-.6,.6);
		\draw[thick,-] (.6,-.6) -- (.6,.6);
		\draw[-] (-.6,0) -- (-1.2,0);
		\draw[-] (.6,0) -- (1.2,0);
		\draw[-] (-1.2,0) -- (-1.2,-1.2);
		\draw[-] (1.2,0) -- (1.2,-1.2);
		\draw[-] (-1.2,-1.2) -- (-0.1,-1.2);
		\draw[-] (1.2,-1.2) -- (0.1,-1.2);
		\draw[-] (0.1,-1) -- (0.1,-1.4);
		\draw[-] (-0.1,-1.1) -- (-0.1,-1.3) node[yshift=-.3cm,xshift=.1cm] {$U$};
	\end{tikzpicture}
\end{minipage}%
\begin{minipage}{.5\linewidth}
	\begin{equation*}
	U = \frac{\sigma}{\epsilon} d \overset{!}{=} U_{\tx{vak}} = \frac{\sigma_0}{\epsilon_0} d
	\end{equation*}
	\begin{equation*}
	\rightarrow \quad \sigma = \frac{\epsilon}{\epsilon_0} \sigma_0 \ge \sigma_0
	\end{equation*}
	\begin{equation*}
	q = \frac{\epsilon}{\epsilon_0} q_0 \ge q_0
	\end{equation*}
	\vspace{5pt}
\end{minipage}%
\\
\lcom{Hier muss deshalb Ladung in den Kondensator fließen um das $ \vec{E} $-Feld konstant zu halten. Dadurch steigt die Kapazität.}
\begin{equation*}
C = \frac{q}{U} = \frac{\epsilon F}{d}
\end{equation*}
\begin{equation*}
E = \frac{\sigma}{\epsilon} = \frac{\epsilon}{\epsilon_0} \sigma_0 \frac{1}{\epsilon} = \frac{\sigma_0}{\epsilon_0} = E_{\tx{vak}}
\end{equation*}
\begin{equation*}
D = \epsilon E = \frac{\epsilon}{\epsilon_0} \ub{\epsilon_0 \equaltoup{E}{E_{\tx{vak}}}}_{D_{\tx{vak}}} = \epsilon_r D_{\tx{vak}}
\end{equation*}

\subsection{Elektrostatische Energie in Dielektrika}

im Vakuum:
\begin{equation*}
W = \frac{\epsilon_0}{2} \int_V \dd^3r \left(\vec{E}(\vec{r})\right)^2
\end{equation*}
(Bei komplexem Feld Betragsquadrat nehmen $ |\vec{E}(\vec{r})|^2 $. Dies ist nur ein technischer Trick, da $ \vec{E} $-Felder Reell sind)\\[5pt]
makroskopisches Feld in Medien:
\begin{equation*}
W = \frac{1}{2} \int_V \dd^3r \equalto{\vec{D}}{\epsilon \vec{E}} \cdot \vec{E} = \frac{\epsilon}{2} \int_V \dd^3 r \left(\vec{E}(\vec{r})\right)^2
\end{equation*}
Plattenkondensator: $ C = \frac{\epsilon F}{d} \quad U = E d $\\
Energie:
$$ E = \frac{1}{2} CU^2 = \frac{1}{2} \frac{\epsilon F}{d} E^2 d^2 = \frac{1}{2} \ob{\epsilon \vec{E}}^{\vec{D}} \cdot \vec{E} \ob{F d}^{V} = \frac{1}{2} \vec{D} \cdot \vec{E} \cdot V $$
$ \rightarrow $ Energiedichte:
\begin{equation*}
\frac{W}{V} = \frac{1}{2} \vec{D} \cdot \vec{E}
\end{equation*}


\chapter{Magnetostatik}

\textbf{Elektrostatik:}\\
ruhende Ladungen $ \Rightarrow $ es wirken Zeitunabhängige elektrische Felder $ \vec{E}(\vec{r}) $\\[5pt]
\textbf{Magnetostatik:}\\
magnetische Felder entstehen aus \textbf{bewegten Ladungen}\\
Kraft auf bewegte Ladung:
\begin{equation*}
\vec{F} = q (\vec{E} + \vec{v} \times \vec{B})
\end{equation*}
\lcom{Magnetfelder von Bewegten Ladungen sind zeitlich verändert und daher kompliziert zu beschreiben. Daher verwenden wir hier ersteinmal statische Ströme die konstande Magnetfelder erzeugen.}\\[3pt]
Magnetostatik:
$$ \begin{array}{c}
\tx{stationäre}\\ \tx{Ströme}
\end{array} \Rightarrow \begin{array}{c}
\tx{zeitunabhängige} \\ \tx{Magnetfelder} \\ \vec{B}(\vec{r})
\end{array} $$
\lcom{Zunächst müssen wir erst einige Dinge Definieren:}

\section{Strom, Stromdichte und Kontinuitätsgleichung}

\subsection{Strom}

\begin{minipage}{.6\linewidth}
	metallischer Leiter:
	\begin{equation*}
	I = \frac{\tx{Ladung}}{\tx{Zeit}} = \frac{\Delta q}{\Delta t} \qquad [I] = 1 \, \tx{A} = \tx{C s}^{-1}
	\end{equation*}
\end{minipage}%
\begin{minipage}{.4\linewidth}
	\centering
	%t4:
	\begin{tikzpicture}
		\draw[-,thick] (-2,.7) -- (3,.7);
		\draw[-,thick] (-2,-.7) -- (3,-.7);
		\draw (.5,.7) arc[start angle=90,end angle=270, x radius=0.3, y radius=.7];
		\draw[dashed] (.5,0.7) arc[start angle=90,end angle=-90, x radius=0.3, y radius=.7] node[yshift=.7cm] {$F$};
		\draw[fill = white!20!blue] (-1.8,.3) circle (.2cm) node[] {$ \color{white} - \color{white} $};
		\draw[fill = white!20!blue] (-1.5,-.3) circle (.2cm) node[] {$ \color{white} - \color{white} $};
		\draw[->] (-1.5,.3) -- (-1.1,.3);
		\draw[->] (-1.2,-.3) -- (-.8,-.3);
	\end{tikzpicture}
	\vspace{5pt}
\end{minipage}%
\\[10pt]
\emph{Beispiel:} \textbf{Stationärer Strom}\\[5pt]
\begin{minipage}{.6\linewidth}
	Ladungsträger mit:\\
	$ v : $ Geschwindigkeit ($ \const $)\\
	$ n : $ homogene Dichte\\
	$ q : $ Ladung
\end{minipage}%
\begin{minipage}{.4\linewidth}
	%t5:
	\centering
	\begin{tikzpicture}
		\draw[dashed] (-2,.7) arc[start angle=90,end angle=-90, x radius=0.3, y radius=.7];
		\draw[pattern = north east lines,pattern color = black!30!white] (-2,.7) -- (.5,.7) arc[start angle=90, end angle=270, x radius=.3, y radius=.7] -- (-2,-.7) arc[start angle=270, end angle=90, x radius=.3, y radius=.7];
		\draw[draw=none, fill = black!10!white] (.5,0) ellipse (0.3cm and 0.7cm);
		\draw[-,thick] (-3,.7) -- (2,.7);
		\draw[-,thick] (-3,-.7) -- (2,-.7);
		\draw (.5,.7) arc[start angle=90,end angle=270, x radius=0.3, y radius=.7];
		\draw[dashed] (.5,0.7) arc[start angle=90,end angle=-90, x radius=0.3, y radius=.7] node[yshift=.7cm] {$F$};
		\draw[fill = white!20!blue] (-1.8,.3) circle (.2cm) node[] {$ \color{white} - \color{white} $};
		\draw[fill = white!20!blue] (-1.5,-.3) circle (.2cm) node[] {$ \color{white} - \color{white} $};
		\draw[->] (-1.5,.3) -- (-1.1,.3);
		\draw[->] (-1.2,-.3) -- (-.8,-.3);
		\draw[decorate, decoration={brace,amplitude=10pt,mirror,raise=2pt}]  (-2,-.7) -- node[below=12pt] {$\Delta x = v \Delta t$}  (.5,-.7);
	\end{tikzpicture}
\end{minipage}%
\\
Leiter mit:
$ F : $ Querschnittsfläche\\[5pt]
in $ \Delta t : \quad n \cdot F \cdot v \cdot \Delta t $ Ladung durch $ F $\\[5pt]
Ladung: $ \Delta q = q n F v \Delta t $\\[5pt]
Strom: $ I = \frac{\Delta q}{\Delta t} = q \cdot n \cdot v \cdot F $

\subsection{Stromdichte:}

\begin{minipage}{.5\linewidth}
	\begin{equation*}
	\vec{j} = \frac{\tx{Strom}}{\tx{Fläche}} = \frac{I}{F}
	\end{equation*}
\end{minipage}%
\begin{minipage}{.5\linewidth}
	\centering
	%t6:
	\begin{tikzpicture}
		\coordinate (e) at (0,0);
		\draw[fill = white!20!blue] (e) circle (.2cm) node[] {$ \color{white} - \color{black} $};
		\draw[->] ($ (e) + (35:.3) $) -- ++(35:.4);
		\draw (-1,0) to[out=90,in=180] (0,1) to[out=0,in=-135] (3,2);
		\draw (0,-1) to[out=70,in=200] (1.5,-.5) to[out=20,in=-170] (3,1) to[out=10,in=135] (3.5,.8);
		\draw[very thick] (2.6,1.65) -- (3,1);
		\draw ($ (2.6,1.65)!.5!(3,1) $) to[out=20,in=160] (4,1.3) node[anchor=north west] {$ F $};
		\node at (.5,2) {im Allgemeinen:};
	\end{tikzpicture}
\end{minipage}%
\\
\emph{Beispiel:} $ \vec{j} = q \cdot n \cdot v $\\
\lcom{Die Stromdichte soll eine vektorielle Größe sein um die Richtung des Stromes mit einzubeziehen.}
\begin{equation*}
\vec{j}(\vec{r},t)
\end{equation*}
\begin{equation*}
I = \int_F \dd \vec{f} \cdot \vec{j}(\vec{r},t)
\end{equation*}
Zusammenhang: $ \vec{j},\rho,\vec{r} $:\\[5pt]
\emph{Beispiel:} $ j = \ub{q \cdot n}_{\rho} \cdot v $
\begin{equation*}
\vec{j}(\vec{r},t) = \rho(\vec{r},t) \vec{v}(\vec{r},t)
\end{equation*}

\subsubsection{Stromdichte von Punktladungen}

Punktladungen $ q_i $ mit Ortsvektoren $ \vec{r}_i $ und Geschwindigkeiten $ \vec{v}_i = \dot{\vec{r}}_i(t) $
\begin{equation*}
\rho(\vec{r},t) = \sum_i q_i \delta(\vec{r} - \vec{r}_i(t)) 
\end{equation*}
\begin{equation*}
\vec{j}(\vec{r},t) = \sum_i q_i \dot{\vec{r}}_i \delta(\vec{r} - \vec{r}_i(t))
\end{equation*}
\begin{minipage}{.6\linewidth}
	\subsubsection{Linienströme}
	
	Ströme durch dünne Drähte
	$$ s \mapsto \vec{r}(s) \qquad \frac{d\vec{r}}{ds} = \frac{\vec{j}}{|\vec{j}|} $$
	beliebige Funktion $ h(\vec{r}) $.\\
	Es gilt außerdem:
	\begin{equation*}
	\dd \vec{f} = \frac{\vec{j}}{|\vec{j}|} \dd f \qquad \dd f = \dd \vec{f} \cdot \frac{\vec{j}}{|\vec{j}|}
	\end{equation*}
\end{minipage}%
\begin{minipage}{.4\linewidth}
	%t7:
	\begin{tikzpicture}
		\coordinate (r) at ($ (1.5,-2) + (-2,-.5) $);
		% gamma
		\node at (3,.4) {$ \gamma_i $};
		% rectangles and connection
		\coordinate (i) at (-.75,-.35);
		\draw ($ (i) + (-.25,-.25) $) rectangle ($ (i) + (.25,.25) $);
		\draw[fill=black!4!red!5!white] ($ (r) + (-2.25,-2.4) $) rectangle ($ (r) + (2.25,.25) $);
		\draw ($ (i) + (0,-.25) $) to[out=-60,in=70] ($ (r) + (0,.25) $);
		% thick line
		\draw[very thick] (-1.5,-1) to[out=45,in=-160] (0,0) to[out=20,in=180] (1.5,0) to[out=0,in=-120] (3,1);
		% origin and j
		\coordinate (o) at (2.5,-2);
		\draw[->] (o) -- ++(.5,0);
		\draw[->] (o) -- ++(0,.5);
		\node[anchor=north east] at (o) {$ 0 $};
		\node[circle,fill=black,inner sep=1pt,minimum size=1pt] (null) at (0,0) {};
		\draw[thick, ->] (o) -- node[right] {$ \vec{r}(s) $} (null);
		\draw[thick, ->] (0,0) -- ++(20:1cm);
		\draw ($ (0,0)!.7!(20:1cm) $) to[out=110,in=0] (-.3,.7) node[left] {$ \vec{j} , \ \prd{\vec{r}}{s} $};
		\node[circle,fill=black,inner sep=1pt,minimum size=1pt] at (o) {};
		% enlarged zylinder
		\draw ($ (r) + (-1,0) $) -- ++(2,0);
		\draw ($ (r) + (-1,-1.4) $) -- ++(2,0);
		\draw[draw=none, fill=black!15!white] ($ (r) + (.7,-.7) $) ellipse (.3cm and .7cm);
		\draw[pattern = north east lines, pattern color=black!40!white] ($ (r) + (-.7,-1.4) $) arc[start angle=-90, end angle=-270, x radius=.3, y radius=.7] -- ($ (r) + (.7,0) $) arc[start angle=90, end angle=270, x radius=.3, y radius=.7] -- cycle;
		\draw[dashed] ($ (r) + (.7,0) $) arc[start angle=90, end angle=-90, x radius=.3, y radius=.7];
		\draw[decorate, decoration={brace,amplitude=7pt,mirror,raise=2pt}] ($ (r) + (-.7,-1.4) $) -- node[below=12pt] {$ \dd s $} ($ (r) + (.7,-1.4) $);
		\draw ($ (r) + (.7,-.7) $) to[out=-25,in=120] ($ (r) + (1.4,-1.4) $) node[anchor=north west] {$ \dd f $};
		\draw ($ (r) + (-.4,-.7) $) to[out=160,in=20] ($ (r) + (-1.4,-.7) $) node[left] {$ \dd ^3 r $};
	\end{tikzpicture}
\end{minipage}%


\noindent
\begin{align*}
& \int \dd^3 r \vec{j}(\vec{r},t) h(\vec{r})\\
=& \int \dd s \dd f \vec{j}(\vec{r},t) h(\vec{r})\\
=& \int \dd s \dd \vec{f} \frac{\vec{j}}{|\vec{j}|} \vec{j} h\\
=& \int_x \ub{\dd s \frac{\vec{j}}{|\vec{j}|}}_{= \dd \vec{r}} h(\vec{r}) \ub{\int \dd \vec{f} \cdot \vec{j}(\vec{r},t)}_{= I(\vec{r},t)}\\
=& \quad \rmbox{\int_{\gamma} \dd \vec{r} h(\vec{r}) I(\vec{r},t) \underset{\mathclap{\substack{\tx{falls} \\ I = \const}}}{\ =\ } I \int_\gamma \dd \vec{r} h(\vec{r})}
\end{align*}
effektiv gilt also:
\begin{equation*}
\tx{,,} \ \vec{j} \dd ^3 r = I \dd \vec{r} \ \tx{``}
\end{equation*}

\subsection{Kontinuitätsgleichung}

\begin{minipage}{.5\linewidth}
	Ladungsdichte: $ \rho(\vec{r},t) $\\
	Ladung in $ V $: $ \int_V \dd^3 r \rho(\vec{r},t) $
\end{minipage}%
\begin{minipage}{.5\linewidth}
	\centering
	%t8:
	\begin{tikzpicture}
		\draw[thick] (-2.5,1) -- (2.5,1);
		\draw[thick] (-2.5,-1) -- (2.5,-1);
		\draw[thick] (-1.5,1) -- (-1.5,-1);
		\draw[thick] (1.5,1) -- (1.5,-1);
		\foreach \x\y in {.6/.4,.2/-.5,-.7/.2}
		\draw[fill = white!20!blue] (\x,\y) circle (.2cm) node[] {$ \color{white} - \color{white} $};
		\foreach \x\y\a in {.6/.4/a,.2/-.5/b,-.7/.2/c}
		\coordinate (\a) at (\x,\y);
		\draw[->] ($ (a) + (.3,0) $) -- node[above] {$ \vec{j} $} ($ (a) + (.7,0) $);
		\draw[->] ($ (b) + (.3,0) $) -- node[above] {$ \vec{j} $} ($ (b) + (.7,0) $);
		\draw[->] ($ (c) - (.3,0) $) -- node[above] {$ \vec{j} $} ($ (c) - (.7,0) $);
		\draw[->] (-1.5,0) -- node[above] {$ \dd \vec{f} $} ++(-1,0);
		\draw[->] (1.5,0) -- node[above] {$ \dd \vec{f} $} ++(1,0);
		\draw (0,.7) to[out=90,in=-20] (-.5,1.3) node[anchor=south east] {$ V $};
	\end{tikzpicture}
\end{minipage}%
\\
Strom von Ladungen aus $ V $ (durch $ \partial V $):
\begin{equation*}
I = \int_{\partial V} \dd \vec{f} \cdot \vec{j}(\vec{r},t)
\end{equation*}
in abgeschlossenen Systemen gilt: Die Ladung ist konstant:\\[10pt]
\begin{minipage}{.35\linewidth}
	\centering
	\begin{tikzpicture}
		\draw[fill=black!5!white] (-1.5,-1) rectangle (1.5,1) node[anchor=north east] {$ V $};
		\draw[fill=black!15!white] (-1.5,.7) .. controls (0,.8) and (0,.2) .. (1.5,.3) -- ++(0,.7) -- ++(-3,0) -- ++(0,-.3);
		\draw[fill=black!15!white] (-1.5,-.7) .. controls (0,-.8) and (0,-.2) .. (1.5,-.3) -- ++(0,-.7) -- ++(-3,0) -- ++(0,.3);
		\draw[->] (-1.2,0) -- node[above] {$ \vec{j} $} (-.3,0);
		\node at (-1.75,-.2) {$ F_1 $};
		\draw[->] (-1.5,.25) -- node[above] {$ \vec{f}_1 $} (-2.3,.25);
		\node at (1.75,-.2) {$ F_2 $};
		\draw[->] (1.5,.2) -- node[above] {$ \vec{f}_2 $} (2.3,.2);
	\end{tikzpicture}
\end{minipage}%
\begin{minipage}{.65\linewidth}
	\begin{equation*}
	\prd{}{t} \int_V \dd^3 r \rho(\vec{r},t) = - \int_{\partial V} \dd \vec{f} \cdot \vec{j}(\vec{r},t)
	\end{equation*}
	\begin{equation*}
	\Rightarrow \quad 0 = \ub{\prd{}{t} \int_V \dd ^3 r \rho(\vec{r},t)}_{=\int_V \dd^3 r \prt{\rho}{t}} + \ub{\int_{\partial V} \dd \vec{f} \cdot \vec{j}(\vec{r},t)}_{=\int_V \dd^3 r \vabla \cdot \vec{j}}
	\end{equation*}
	\begin{equation*}
	\Rightarrow \quad \int_V \dd^3 r \left(\prt{\rho}{t} + \vabla \cdot \vec{j}\right) = 0
	\end{equation*}
\end{minipage}%

\noindent
für beliebige $ V $
\frbox{Kontinuitätsgleichung}{
\begin{equation*}
\prt{\rho(\vec{r},t)}{t} + \vabla \cdot \vec{j}(\vec{r},t) = 0
\end{equation*}
}

\subsection{Magnetostatik}

Stationärer (zeitunabhängigen) Fall
$$\underline{\rho = \rho(\vec{r}), \quad \vec{j} = \vec{j}(\vec{r})}$$
$$\prt{}{t}\rho = 0 \quad \Rightarrow \quad \vabla \cdot \underbrace{\vec{j}(\vec{r})}_{\mathclap{\textrm{Stationäre Ströme}}} = 0$$
Konsequenz: Durch jeden Querschnitt eines Leiters fließt der selbe Strom.
$$0 = \int_V \difi{^3 r} \vabla\cdot\vec{j} = \int_{\partial V} \difi{\vec{f}} \cdot \vec{j} = \int_{F_1} \difi{\vec{f}} \cdot \vec{j} + \int_{F_2}\difi{\vec{f}} \cdot \vec{j} = -I_1 + I_2$$
$$\Rightarrow I_1 = I_2$$

% Vorlesung 03.12.18 (21 days until christmas)

\section{Gesetz von Biot-Savart}

stationärer Strom in Leiter $ \rightarrow $ Magnetfeld\\
\begin{minipage}{.7\linewidth}
	Das Magnetfeld $ \dd\vec{B} $ am Ort $ \vec{r} $ verursacht durch Strom $ I $ im Linienelement $ \dd \vec{l} $ in $ \vec{r}' $.
	\begin{equation*}
	\dd \vec{B}(\vec{r}) = k' I \dd \vec{l} \times \frac{\vec{r} - \vec{r}'}{|\vec{r} - \vec{r}'|^3}
	\end{equation*}
	\begin{equation*}
	|\dd \vec{B}| \propto I ,\  |\dd \vec{l}| ,\ \frac{1}{|\vec{r} - \vec{r}'|^2}
	\end{equation*}
	\begin{equation*}
	\tx{Richtung von:} \quad \dd \vec{B} \propto \dd\vec{l} \times (\vec{r},\vec{r}')
	\end{equation*}
\end{minipage}%
\begin{minipage}{.3\linewidth}
	%t1:
	\centering
	\begin{tikzpicture}
		\draw[->] (0,0) -- (0,1.5) node[anchor=south east] {$ z $};
		\draw[->] (0,0) -- (2,0) node[anchor=north west] {$ y $};
		\draw[->] (0,0) -- (-1,-1) node[anchor=north east] {$ x $};
		\draw[thick, ->] (0,0) -- node[above] {$ \vec{r} $} (1,.7);
		\draw[thick, ->] (1,.7) -- node[anchor=south west] {$ \dd\vec{B} $} ++(-.7,.7);
		\draw[thick, ->] (0,0) -- node[anchor=north east] {$ \vec{r}' $} (.5,-1);
		\draw[very thick, ->] (.5,-1) -- node[anchor=north west] {$ \dd \vec{l}' $} ++(.5,.5);
		\draw[thick] (-1,-2) to[out=20,in=-135] (.5,-1) to[out=45,in=-110] (2,1);
		\draw[->] (-1,-2) to[out=20,in=-145] (-.3,-1.67) node[below=5pt] {$ I $};
		\node[right] at (2,1) {$ \gamma $};
	\end{tikzpicture}
\end{minipage}%
\\
Die Konstante $ k' $ im SI-Einheiten-System ist:
\begin{equation*}
k' = \frac{\mu_0}{4 \pi}
\end{equation*}
$ \mu_0 $ ist die magnetische Feldkonstante, die Permeabilität des Vakuums
\begin{equation*}
\mu_0 = 4 \pi \cdot 10^{-7} \frac{\tx{V s}}{\tx{A m}}
\end{equation*}
Sie ist definiert über:
\begin{equation*}
\epsilon_0 \mu_0 = \frac{1}{c^2} \qquad c: \tx{ Lichtgeschw. in Vakuum}
\end{equation*}
Einheit:
\begin{equation*}
[\vec{B}] = \frac{\tx{V s}}{\tx{m}^2} = 1 \tx{ Tesla}
\end{equation*}\\[5pt]
\frbox{Biot-Savart-Gesetz}{
\begin{equation*}
\vec{B}(\vec{r}) = \frac{\mu_0}{4 \pi} I \int_{\gamma} \dd \vec{l}' \times \frac{\vec{r} - \vec{r}'}{|\vec{r} - \vec{r}'|^3}
\end{equation*}
}
\noindent
Diese Formel gibt das Magnetfeld für einen Stromdurchflossenen dünnen Leiter an.\\[5pt]
Für eine ausgedehnte Stromdichte $ \vec{j}(\vec{r}) $ gilt:
\begin{equation*}
\tx{,,} \dd^3 r \vec{j}(\vec{r}) = I \dd \vec{l}`` \quad \vec{B}(\vec{r}) = \frac{\mu_0}{4 \pi} \int \dd^3 r \vec{j}(\vec{r}') \times \frac{\vec{r}- \vec{r}'}{|\vec{r} - \vec{r}'|^3}
\end{equation*}
Ähnlich in der Elektrostatik:
\begin{equation*}
\vec{E}(\vec{r}) = \frac{1}{4 \pi \epsilon_0} \int \dd^3 r' \rho(\vec{r}') \frac{\vec{r} - \vec{r}'}{|\vec{r} - \vec{r}'|^3}
\end{equation*}
Hier ist $ \rho(\vec{r}) $ aber ein Skalarfeld. $ \vec{j}(\vec{r}) $ ist ein Vektorfeld! Deshalb ist die Berechnung von Magnetfeldern komplizierter.\\[5pt]
\begin{minipage}{.7\linewidth}
	\emph{Beispiel:} Magnetfeld eines langen Drahtes:
	\begin{equation*}
	\vec{j}(\vec{r}) = I \delta(x) \delta(y) \vec{e}_z
	\end{equation*}
	Setzen wir dies nun ins Biot-Savart-Gesetz ein erhalten wir das $ \vec{B} $-Feld dieses Leiters:
\end{minipage}%
\begin{minipage}{.3\linewidth}
	%t2:
	\centering
	\begin{tikzpicture}
		\draw[very thick, ->] (0,-1.5) -- (0,1.5) node[left] {$ I $};
		\draw[->] (0,-2) -- (0,2) node[anchor=south east] {$ z $};
		\draw[->] (0,0) -- (.7,.7) node[anchor=south east] {$ y $};
		\draw[->] (0,0) -- (1,0) node[anchor=north west] {$ x $};
	\end{tikzpicture}
\end{minipage}%
\\
\begin{align*}
\vec{B}(\vec{r}) \ &= \frac{\mu_0}{4 \pi} \int \dd^3 r' I \delta (x') \delta (y') \vec{e}_z \times \frac{\vec{r} - \vec{r}'}{|\vec{r} - \vec{r}'|^3}\\
&= \frac{\mu_0}{4 \pi} \int_{-\infty}^{\infty} \dd z' \vec{e}_z \times \frac{\vec{r} - \equaltoup{\vec{r}'}{\mathclap{\vec{r}' = (0,0,z')}}}{|\vec{r} - \vec{r}'|^3}\\
\end{align*}
Nebenrechnung:\\
$ \vec{B}(\vec{r}) $ hängt nicht von $ z $ ab $ \rightarrow \vec{r} = (x,y,0) \quad \rightarrow \vec{r} - \vec{r}' = (x,y,-z') $\\
$ \vec{e}_z \times (\vec{r} - \vec{r}') = \begin{pmatrix}
-y \\ x \\ 0
\end{pmatrix} $
\begin{align*}
\qquad \qquad &= \frac{\mu_0 I}{4 \pi} \begin{pmatrix}
-y \\ x \\ 0
\end{pmatrix} \int_{-\infty}^{\infty} \dd z' \frac{1}{[x^2 + y^2 + z'^2]^{3/2}}\\
&= \frac{\mu_0 I}{4 \pi} \frac{2}{x^2 + y^2} \begin{pmatrix}
-y \\ x \\ 0
\end{pmatrix}\\
\vec{B} (\vec{r}) &= \frac{\mu_0 I}{2 \pi} \frac{1}{x^2 + y^2} \begin{pmatrix}
-y \\ x \\ 0
\end{pmatrix}
\end{align*}
In Zylinderkoordinaten:
\begin{equation*}
\vec{r} = \begin{pmatrix}
\rho \cos \varphi \\ \rho \sin \varphi \\ z
\end{pmatrix} \qquad \rho^2 = x^2 + y^2
\end{equation*}
\begin{equation*}
\vec{e}_{\varphi} = \frac{\prt{\vec{r}}{\varphi}}{|\prt{\vec{r}}{\varphi}|} = \frac{1}{\rho} \begin{pmatrix}
- \rho \sin \varphi \\ \rho \cos \varphi \\ 0
\end{pmatrix} = \frac{1}{\sqrt{x^2 + y^2}} \begin{pmatrix}
-y \\ x \\ 0
\end{pmatrix}
\end{equation*}
\begin{minipage}{.5\linewidth}
	\begin{equation*}
	\Rightarrow \quad \vec{B} = \frac{\mu_0}{2 \pi} I \frac{1}{\rho} \vec{e}_{\varphi}
	\end{equation*}
\end{minipage}%
\begin{minipage}{.5\linewidth}
	\centering
	%t3:
	\begin{tikzpicture}
		\draw[->] (0,-1.5) -- (0,1.5) node[anchor=south east] {$ z $};
		\draw[very thick, ->] (0,-1) -- (0,1) node[left] {$ I $};
		\draw[->] (0,0) -- (2,0) node[anchor=north west] {$ x $};
		\draw[->] (0,0) -- (1,1) node[anchor=south west] {$ y $};
		\draw[->] ({.5 * sin(45)},-{.25 * cos(45)}) arc[start angle=-45,end angle=315, x radius=.5, y radius=.25];
		\draw[->] ({1 * sin(45)},-{.5 * cos(45)}) arc[start angle=-45, end angle=315, x radius=1, y radius=.5];
		\draw[->] ({1.5 * sin(45)}, -{.75 * cos(45)}) arc[start angle=-45, end angle=315, x radius=1.5, y radius=.75];
		\node at (1,-1) {$ \vec{B} $};
	\end{tikzpicture}
\end{minipage}%

\section{Kraft eines äußeren Magnetfeldes auf einen Stromdurchflossenen Leiter}

\begin{minipage}{.5\linewidth}
	%t4:
	\centering
	\begin{tikzpicture}
		\draw[->] (0,0) -- (0,1.5) node[anchor=south east] {$ z $};
		\draw[->] (0,0) -- (2.5,0) node[anchor=north west] {$ x $};
		\draw[->] (0,0) -- (-1,-1) node[anchor=north east] {$ y $};
		\foreach \y in {.4,1.2,-.4,-1.2,-2}
		\draw[->] (.5,\y) -- ++(1.5,0);
		\node[right] at (2,1.2) {$ \vec{B} $};
		\coordinate (o) at (1,-1);
		\draw[thick, ->] (-1,-.5) to[out=-30,in=180] (o) to[out=0,in=-150] (3,-.5) to[out=30,in=-120] (3.5,0) node[anchor=north west] {$ I $};
		\draw[blue,thick, ->] ($ (o) + (.5,0.02) $) -- node[left] {$ \dd \vec{F} $} ++(0,.5);
		\draw[red,thick, ->] ($ (o) + (.5,0.02) $) -- node[below,xshift=10pt] {$ \dd \vec{l} $} ++(1,.1);
	\end{tikzpicture}
\end{minipage}%
\begin{minipage}{.5\linewidth}
	\begin{equation*}
	\dd \vec{F} = I \dd \vec{l} \times \vec{B}(\vec{r})
	\end{equation*}
	\begin{equation*}
	|\dd\vec{F}| \propto I ,\ |\dd \vec{l}| ,\ |\vec{B}|
	\end{equation*}
	\begin{equation*}
	\tx{Richtung von:} \quad \dd \vec{F} \propto \dd \vec{l} \times \vec{B}(\vec{r})
	\end{equation*}
\end{minipage}%
\\
\begin{minipage}{.6\linewidth}
	Damit ist die Kraft auf eine beliebige Leiterschleife:
	\begin{equation*} % \gamma_2 ?
	\vec{F} = I \int_{\gamma} \dd \vec{l} \times \vec{B}(\vec{r})
	\end{equation*}
\end{minipage}%
\begin{minipage}{.4\linewidth}
	\centering
	%t5:
	\begin{tikzpicture}
		\draw plot [smooth cycle, tension=1] coordinates { (0,0) (1,.5) (.5,1) (-.5,.5) (-1,-1) (0,-1) };
		\draw[->] (0,0) -- ++(50:.1cm) node[anchor=north west] {$ I $};
		\node[right] at (1,.5) {$ \gamma $};
	\end{tikzpicture}
\end{minipage}%
\\
Für eine ausgedehnte Stromverteilung gilt dann:
\begin{equation*}
\vec{F} = \int \dd ^3 r \vec{j}(\vec{r}) \times \vec{B}(\vec{r})
\end{equation*}

\subsection{Kraft zwischen zwei Stromdurchflossenen Leitern}

$ I_2 $ erzeugt am Ort $ \vec{r}_1 $ das Magnetfeld:
\begin{equation*}
\vec{B}(\vec{r}_1)  = \frac{\mu_0}{4 \pi} I_2 \int_{\gamma} \dd \vec{l}_2 \times \frac{\vec{r}_1 - \vec{r}_2}{|\vec{r}_1 - \vec{r}_2|^3}
\end{equation*}
\begin{minipage}{.6\linewidth}
	$ \rightarrow $ Kraft auf Linienelement $ \dd \vec{l}_1 $ in $  \vec{r}_1 $:
	\begin{align*}
	\dd \vec{F}_{12} &= I_1 \dd \vec{l}_1 \times \vec{B}(\vec{r}_1)\\
	&= \frac{\mu_0}{4 \pi} I_1 I_2 \dd \vec{l}_1 \times \int_{\gamma_{2}} \times \frac{\vec{r}_1 - \vec{r}_2}{|\vec{r}_1 - \vec{r}_2|^3}
	\end{align*}
	%
	%
	% ich glaube hier fehlt was dl_2 im integral
	%
	%
\end{minipage}%
\begin{minipage}{.4\linewidth}
	\flushright
	%t6:
	\begin{tikzpicture}
		\draw plot [smooth cycle, tension=.9] coordinates { (0,0) (-1,0) (-1,-1) (-1.3,-1.7) (-1,-2) (-.3,-1.5) (0,-.8)  };
		\draw[shift={(2 cm, -2 cm)},rotate=200] plot [smooth cycle, tension=.9] coordinates { (0,0) (-1,0) (-1,-1) (-1,-2) (-.3,-1.5) (0,-.8)  };
		\coordinate (o) at (1,-2.5);
		\coordinate (l) at (-.3,-1.5);
		\coordinate (r) at ($ (o) + (.75,1)$);
		\draw[thick,->] (o) -- node[above] {$ \vec{r}_1 $} (l);
		\draw[thick,->] (o) -- node[above,xshift=-5pt] {$ \vec{r}_2 $} (r);
		\node[anchor=north west] at (o) {$ O $};
		\draw[red,thick,->] (l) -- node[left,yshift=5pt] {$ \dd \vec{l}_1 $} ++(50:.5cm);
		\draw[red,thick,->] (r) -- node[right] {$ \dd \vec{l}_2 $} ++(95:.5cm);
		\draw[->] (-.2,.11) -- ++(160:.1cm) node[anchor=south west] {$ I_1 $};
		\node at (-1.2,.2) {$ \gamma_1 $};
		\draw[->] (2.415,0) -- ++(105:.1cm) node[anchor=south west] {$ I_2 $};
		\node at (1.5,-.3) {$ \gamma_2 $};
	\end{tikzpicture}
\end{minipage}%
\\
Die Kraft auf Leiterschleife 1 ist dann:
\begin{equation*}
\vec{F}_{12} = \frac{\mu_0}{4 \pi} I_1 I_2 \int_{\gamma_1} \int_{\gamma_2} \dd \vec{l}_1 \times \left(\dd \vec{l}_2 \times \frac{\vec{r}_1 - \vec{r}_2}{|\vec{r}_1 - \vec{r}_2|^3}\right)
\end{equation*}
\emph{Beispiel:} \textbf{Kraft zwischen zwei parallelen Drähten}\\
\begin{minipage}{.6\linewidth}
	\begin{equation*}
	\dd \vec{F}_{12} = \frac{\mu_0}{4 \pi} I_1 I2 \equalto{\dd \vec{l}_1}{\vec{e}_z \dd l_1} \times \int_{\gamma} \dd \vec{l}_2 \times \frac{\vec{r}_1 - \vec{r}_2}{|\vec{r}_1 - \vec{r}_2|^3}
	\end{equation*}
	Aus der Skizze gilt:\\
	$ \dd \vec{l}_2 = \dd z_2 \vec{e}_z \qquad \vec{r}_1  = (0,0,0) \qquad \vec{r}_2 = (a,0,z_2) $
\end{minipage}%
\begin{minipage}{.4\linewidth}
	\flushright
	%t7:
	\begin{tikzpicture}[scale=1.2]
		\draw[->] (0,-1) -- (0,2) node[anchor=south east] {$ z $};
		\draw[->] (-.5,0) -- (3,0) node[anchor=north west] {$ x $};
		\draw (2,-1) -- (2,2);
		\draw[thick,->] (0,-.5) -- (0,1.5) node[left] {$ I_1 $};
		\draw[thick,->] (2,-.5) -- (2,1.5) node[right] {$ I_2 $};
		\coordinate (o) at (0,0);
		\coordinate (p) at (2,.5);
		\draw[decorate, decoration={brace,amplitude=10pt,mirror,raise=2pt}]
		(o) -- node[below=12pt] {$ a $} (2,0);
		\draw[thick,->] (o) -- node[above] {$ \vec{r}_{2} $} (p);
		\draw[red,thick,->] (o) -- node[left] {$ \dd \vec{l}_1 $} ($ (o) + (0,.5) $);
		\draw[red,thick,->] (p) -- node[right] {$ \dd \vec{l}_2 $} ($ (p) + (0,.5) $);
	\end{tikzpicture}
	\vspace{5pt}
\end{minipage}%
\\
Nebenrechnung:
\begin{equation*}
\dd \vec{l}_2 \times (\vec{r}_1 - \vec{r}_2) = \dd z_2 \begin{pmatrix}
0 \\ 0 \\ 1
\end{pmatrix} \times \begin{pmatrix}
-a \\ 0 \\ -z_2
\end{pmatrix} = \dd z_2 \begin{pmatrix}
0 \\ -a \\ 0
\end{pmatrix}
\end{equation*}
\vspace{5pt}

\noindent
\begin{minipage}{.75\linewidth}
	\begin{align*}
	\rightarrow \quad \dd \vec{F}_{12} = \frac{\mu_0}{4 \pi} I_1 I_2 \dd l_1 \ub{\vec{e}_z \times \begin{pmatrix}
		0 \\ -a \\ 0
		\end{pmatrix}}_{= a \vec{e}_x} \ub{\int_{-\infty}^{\infty} \dd z_2 \frac{1}{\left(a^2 + z_2^2\right)^{3/2}}}_{= \frac{2}{a^2}}
	\end{align*}
\end{minipage}%
\begin{minipage}{.25\linewidth}
	Richtung:\\
	\flushright
	%t8:
	\begin{tikzpicture}
	\draw[->] (0,0) -- (0,1);
	\draw[->] (1,0) -- (1,1);
	\draw[->] (0,.5) -- (.3,.5);
	\draw[->] (1,.5) -- (.7,.5);
	%2nd
	\coordinate (o) at (2.5,0);
	\draw[->] (o) -- ++(0,1);
	\draw[->] ($ (o) + (1,1)$) -- ++(0,-1);
	\draw[->] ($ (o) + (0,.5)$) -- ++(-.3,0);
	\draw[->] ($ (o) + (1,.5)$) -- ++(.3,0);
	\end{tikzpicture}
\end{minipage}%
\begin{center}
	\begin{minipage}{.5\linewidth}
		\frbox{Kraft pro Länge:}{
			\begin{equation*}
			\frac{\dd \vec{F}_{12}}{\dd l_1} = \frac{\mu_0}{2 \pi} \frac{I_1 I_2}{a} \vec{e}_x
			\end{equation*}}
	\end{minipage}
\end{center}

\section{Feldgleichungen der Magnetostatik und Vektorpotential}

\begin{equation*}
\vec{B}(\vec{r}) = \frac{\mu_0}{4 \pi} \int \dd^3 r' \vec{j}(\vec{r}) \times \frac{\vec{r} - \vec{r}'}{|\vec{r} - \vec{r}'|^3}
\end{equation*}

\subsection{Vektorpotential}

\begin{equation*}
\vec{B}(\vec{r}) = \frac{\mu_0}{4 \pi} \int \dd^3 r' \ub{\vec{j}(\vec{r'}) \times \ub{\frac{\vec{r} - \vec{r}'}{|\vec{r} - \vec{r}'|^3}}_{- \vabla_{\vec{r}} \frac{1}{|\vec{r} - \vec{r}'|}}}_{\vabla_{\vec{r}} \times \left(\vec{j}(\vec{r'}) \frac{1}{|\vec{r} - \vec{r}'|}\right)}
\end{equation*}
Mit einer Identität des ersten Übungsblattes:
\begin{equation*}
\vabla \times (f \vec{G}) = f \vabla \times \vec{G} - \vec{G} \times \vabla f
\end{equation*}
(Die Rotation von $ \vec{G} $ fällt weg, da $ \vec{j} $ nur von $ \vec{r}' $ abhängt.)
\begin{equation*}
\Rightarrow \quad \vec{B}(\vec{r}) = \vabla_{\vec{r}} \times \frac{\mu_0}{4 \pi} \int \dd^3 r'\frac{\vec{j}(\vec{r}')}{|\vec{r} - \vec{r}'|} = \vabla \times \custo{\rightarrow}{\vec{A}(\vec{r})}{\mathclap{\tx{Vektorpotential}}}
\end{equation*}
\begin{equation*}
\vec{B}(\vec{r}) = \vabla \times \vec{A}(\vec{r}) \quad \leftarrow \vec{A} \tx{ nicht eindeutig festgelegt}
\end{equation*}
\begin{align*}
& \vec{A}' = \vec{A} + \vec{G} \ \tx{ mit } \ \vabla \times \vec{G} = 0 \quad \rightarrow \quad \vec{G}(\vec{r}) = \vabla \Lambda (\vec{r})\\
& \rightarrow \quad \vabla \times \vec{A}' = \vabla \times \vec{A} + \vabla \times \vec{G} = \vec{B}
\end{align*}
\begin{equation*}
\Rightarrow \quad \vec{A}(\vec{r}) = \frac{\mu_0}{4 \pi} \int \dd^3 r' \frac{\vec{j}(\vec{r}')}{|\vec{r} - \vec{r}'|} + \vabla \Lambda (\vec{r}) \quad \Rightarrow \quad \vabla \times \vec{A}(\vec{r}) = \vec{B}(\vec{r})
\end{equation*}
\textbf{Transformation: Eichtransformation}
\begin{equation*}
\vec{A}(\vec{r}) \rightarrow \vec{A}' = \vec{A} + \vabla \Lambda
\end{equation*}
\textbf{Magnetostatik: übliche Wahl:} $ \Lambda \equiv 0 $
ß\begin{equation*}
\Rightarrow \quad \vec{A} = \frac{\mu_0}{4 \pi} \int \dd ^3 r' \frac{\vec{j}(\vec{r}')}{|\vec{r} - \vec{r}'|}
\end{equation*}
Eine andere Eichung ist die \textbf{Coulomb-Eichung:}
\begin{equation*}
\Rightarrow \quad \vabla \cdot \vec{A} = 0
\end{equation*}
\begin{align*}
\vabla \cdot \vec{A} &= \frac{\mu_0}{4 \pi} \int \dd^3 r' \ub{\vabla_{\vec{r}} \cdot \left(\vec{j}(\vec{r}') \frac{1}{|\vec{r} - \vec{r}'|}\right)}\\
& \qquad \qquad \qquad = \vec{j}(\vec{r}') \cdot \ub{\vabla_{\vec{r}} \frac{1}{|\vec{r} - \vec{r}'|}}_{-\vabla_{\vec{r}'} \frac{1}{|\vec{r} - \vec{r}'|}}\\
& \qquad \qquad \qquad = - \vabla_{\vec{r}'} \cdot \left(\vec{j}(\vec{r}') \frac{1}{|\vec{r} - \vec{r}'|}\right) + \ob{\left(\vabla_{\vec{r}'} \cdot \vec{j}(\vec{r}')\right)}^{=0} \frac{1}{|\vec{r} - \vec{r}'|}\\
&= - \frac{\mu_0}{4 \pi} \int_{\mathbb{R}^3} \dd^3 r' \vabla_{\vec{r}'} \cdot \left(\frac{\vec{j}(\vec{r}')}{|\vec{r} - \vec{r}'|}\right)\\
&= - \frac{\mu_0}{4 \pi} \lim\limits_{R \to \infty} \int\limits_{K_R(0)} \dd^3 r' \vabla_{\vec{r}'} \cdot \left(\frac{\vec{j}(\vec{r}')}{|\vec{r} - \vec{r}'|}\right)\\
&= \frac{\mu_0}{4 \pi} \ub{\lim\limits_{R\to \infty} \int\limits_{\partial K_R(0)} \dd \vec{f}' \cdot \frac{\vec{j} (\vec{r}')}{|\vec{r} - \vec{r}'|}}_{=0} = 0
\end{align*}
\emph{Beispiel:} homogenes Magnetfeld
\begin{equation*}
\vec{B} = \vec{B}_0 \qquad \vec{B} = \vabla \times \vec{A}
\end{equation*}
\begin{equation*}
\vec{A} = \frac{1}{2} \vec{B} \times \vec{r}
\end{equation*}
Mit der Identität:
\begin{equation*}
\vabla \times (\vec{a} \times \vec{b}) = \vec{a} (\vabla \cdot \vec{b}) - \vec{b} (\vabla \cdot \vec{a}) + (\vec{b} \cdot \vabla) \vec{a} - (\vec{a} \cdot \vabla) \vec{b}
\end{equation*}
\begin{equation*}
\rightarrow \quad \vabla \times \vec{A} = \frac{1}{2} \vabla \times (\vec{B} \times \vec{r}) = \frac{1}{2} \vec{B} \ub{\vabla \cdot \vec{r}}_{= 3} - \frac{1}{2} \ub{(\vec{B} \cdot \vabla) \vec{r}}_{= \vec{N}} = \vec{B}
\end{equation*}
Mit der Identität:
\begin{equation*}
\vabla \cdot (\vec{a} \times \vec{b}) = \vec{b} \cdot (\vabla \times \vec{a}) - \vec{a} \cdot (\vabla \times \vec{b})
\end{equation*}
\begin{equation*}
\Rightarrow \quad \vabla \cdot \vec{A} = \frac{1}{2} \vabla \cdot \left(\vec{B} \times \vec{r}\right) = - \vec{B} \cdot \ub{\left(\vabla \times \vec{r}\right)}_{= 0} = 0
\end{equation*}
andere mögliche Wahl:
\begin{equation*}
\vec{A}' = \frac{1}{2} \vec{B} \times \vec{r} + \ub{\vabla \frac{r^2}{2}}_{=\vec{r}} = \vec{A} + \vec{r}
\end{equation*}
\begin{equation*}
\vabla \times \vec{A}' = \ub{\vabla \times \vec{A}}_{= \vec{B}} + \ub{\vabla \times \vec{r}}_{= 0} = \vec{B}
\end{equation*}

% Vorlesung 6.12.18 (18 days until christmas)

\begin{equation*}
\rmbox{\vec{B}(\vec{r}) = \frac{\mu_0}{4 \pi} \int \dd ^3 r' \vec{j}(\vec{r}') \times \frac{\vec{r} - \vec{r}'}{|\vec{r} - \vec{r}'|^3} = \vabla \times \vec{A}(\vec{r})}
\end{equation*}

\subsection{Feldgleichungen der Magnetostatik}

\subsubsection{Divergenz (Quellen)}

\begin{equation*}
\vabla \cdot \vec{B}(\vec{r}) = \vabla \cdot (\vabla \times \vec{A}(\vec{r})) = 0
\end{equation*}
\begin{equation*}
\rmbox{\Rightarrow \quad \vabla \cdot \vec{B}(\vec{r}) = 0}
\end{equation*}
In der Elektrostatik gilt:
\begin{equation*}
\rmbox{\vabla \cdot \vec{E}(\vec{r}) = \frac{1}{\epsilon_0} \rho(\vec{r})}
\end{equation*}
Es gibt also keine ,,magnetischen Ladungen`` wie beim elektrischen Feld.\\[5pt]
integrale Formulierung:
\begin{equation*}
0 = \int_V \dd^3 r \vabla \vec{B} = \int_{\partial V} \dd \vec{f} \cdot \vec{B}
\end{equation*}

\subsubsection{Rotation (Wirbel)}

$ \vec{B} = \vec{\nabla} \times \vec{A} $
\begin{equation*}
\vec{A}(\vec{r}) = \frac{\mu_0}{4 \pi} \int \dd^3 r' \frac{\vec{r} - \vec{r}'}{|\vec{r} - \vec{r}'|} + \vabla \Lambda
\end{equation*}
\begin{align*}
\Rightarrow \quad \vabla \times \vec{B} &= \vabla \times (\vabla \times \vec{A}\\
&= \vabla (\vabla \cdot \vec{A}) - \Delta \vec{A}
\end{align*}
\begin{equation*}
\vabla \cdot \vec{A} = \ub{\vabla \cdot \left(\frac{\mu_0}{4 \pi} \int \dd ^3 r' \frac{\vec{j}(\vec{r}')}{|\vec{r} - \vec{r}'|}\right)}_{=0} + \vabla \cdot (\vabla \Lambda) = \Delta \Lambda 
\end{equation*}
\begin{equation*}
\Delta \vec{A} = \frac{\mu_0}{4 \pi} \int \dd^3 r' \vec{j}(\vec{r}) \ub{\Delta \frac{\Lambda}{|\vec{r} - \vec{r}'|}}_{= - 4 \pi \delta(\vec{r} - \vec{r}')} + \Delta \vabla \Lambda = - \mu_0 \vec{j}(\vec{r}) + \Delta (\vabla \Lambda)
\end{equation*}
mit: $ \Delta \vec{A} = \begin{pmatrix}
\Delta A_x \\ \Delta A_y \\ \Delta A_z
\end{pmatrix} $
\begin{equation*}
\Rightarrow \quad \vabla \times \vec{B} = \cancel{\vabla(\Delta \Lambda)} + \mu_0 \vec{j} - \cancel{\Delta (\vabla \Lambda)} = \mu_0 \vec{j}
\end{equation*}
Kürzbar, da partielle Ableitungen vertauschbar sind.
\begin{equation*}
\rmbox{\vabla \times \vec{B} = \mu_0 \vec{j}}
\end{equation*}
\begin{minipage}{.8\linewidth}
	integrale Formulierung:
	\begin{equation*}
	\int_{\partial F} \dd \vec{r} \cdot \vec{B} (\vec{r}) = \int_F \dd \vec{f} \cdot (\vec{\nabla} \times \vec{B}) = \mu_0 \ub{\int_F \dd \vec{f} \cdot \vec{j}(\vec{r})}_{I_F} = \mu_0 I_F
	\end{equation*}
\end{minipage}%
\begin{minipage}{.2\linewidth}
	%t1:
	\flushright
	\begin{tikzpicture}
		\draw[->,pattern=north east lines, pattern color=black!40!white] (1,0) arc[start angle=0,end angle=360,y radius=.5cm,x radius=1cm] node[right] {$ \partial F $};
		\node at (-.6,0) {$ F $};
		\draw[fill=white] (-.25,1.5) -- (-.25,0) arc[start angle=180,end angle=360,x radius=.25,y radius=.125] -- (.25,1.5);
		\draw (0,1.5) ellipse (.25cm and .125cm);
		\draw (-.25,-1)  arc[start angle=180,end angle=360,x radius=.25,y radius=.125];
		\draw (-.25,-1) -- (-.25,-.48);
		\draw (.25,-1) -- (.25,-.48);
		\node at (0,.75) {$ \vec{j} $};
	\end{tikzpicture}
\end{minipage}%
\\[5pt]
\frbox{Amper\`esches Durchflutungsgesetz}{
\begin{equation*}
\Rightarrow \quad \oint_{\partial F} \dd \vec{r}\cdot \vec{B}(\vec{r}) = \mu_0 I_F
\end{equation*}
}

\subsubsection{Magnetfeld eines stromdurchflossenen Leiters mit homogener Stromdichte}

\begin{minipage}{.7\linewidth}
	Aufgrund der Symmetrie verwenden wir Zylinderkoordinaten:
	\begin{equation*}
	\vec{r} = \begin{pmatrix}
	\rho \cos \varphi \\ \rho \sin \varphi \\ z
	\end{pmatrix}
	\end{equation*}
	Die Stromdichte ist dann:
	\begin{align*}
	\vec{j} &= \vec{e}_z \casess{\frac{I}{\pi R^2}}{\rho \le R}{0}{\tx{sonst.}}\\
	&= \vec{e}_z \frac{I}{\pi R^2} \theta(R - \rho)
	\end{align*}
	Symmetrie:
	\begin{equation*}
	\vec{B}(\vec{r}) = B_\varphi(\rho) \vec{e}_{\varphi} \qquad \vec{e}_{\varphi} = \begin{pmatrix}
	- \sin \varphi \\ \cos \varphi \\ 0
	\end{pmatrix}
	\end{equation*}
\end{minipage}%
\begin{minipage}{.3\linewidth}
	%t2:
	\flushright
	\begin{tikzpicture}[scale=1.4]
		\draw[dashed,fill=gray!20!white] (0,0) ellipse (1cm and .5cm);
		\draw[dashed] (0,-1.5) ellipse (1cm and .5cm);
		\draw[dashed] (0,1.5) ellipse (1cm and .5cm);
		\draw[->] (0,0) -- (0,2.5) node[anchor=south east] {$ z $};
		\draw (0,-.5) -- (0,-2.5);
		\draw[->] (0,0) -- (2,0) node[anchor=north west] {$ x $};
		\draw (-1,-1.5) -- (-1,1.5) (1,-1.5) -- (1,1.5);
		\draw[thick,->] (-.5,0) -- node[left=5pt] {$ \vec{j} $} (-.5,1);
		\draw[very thick] (0,0) -- (1,0) node[anchor=north west,xshift=5pt,yshift=-5pt] {R};
		\draw[thick,->] (0,0) -- node[above] {$ \vec{r} $} ({cos(45) * 1cm},{sin(45) * .5cm});
		\draw[thick,->] ({cos(45) * 1cm},{sin(45) * .5cm}) -- node[above=5pt] {$ \vec{e}_\varphi $} ++({-cos(45) * .7cm},{sin(45) * .35cm});
		\draw (.8,0) to[out=-60,in=180] (1.2,-.4);
	\end{tikzpicture}
\end{minipage}%

\noindent
\begin{minipage}{.6\linewidth}
	$ F $ ist ein Kreis mit Radius $ \rho $ (kleiner oder größer als $ R $)\\
	Hierauf wenden wir das Amp\`eresche Durchflutungsgesetz an:\\
	Unter Verwendung von:
	\begin{equation*}
	\dd \vec{f}' = \vec{e}_z \dd f = \vec{e}_z = \rho' \dd \rho' \dd \varphi'
	\end{equation*}
\end{minipage}%
\begin{minipage}{.4\linewidth}
	%t3:
	\flushright
	\begin{tikzpicture}[scale=.8]
		\draw[->] (-3,0) -- (3,0) node[anchor=north west] {$ x $};
		\draw[->] (0,-3) -- (0,3) node[anchor=south east] {$ z $};
		\draw[dashed] (0,0) circle (2cm);
		\draw[draw=none,pattern=north east lines, pattern color=black!30!white] (0,0) circle (2cm);
		\draw (0,0) circle (1cm);
		\node at (1,-1) {$ R $};
		\draw[thick,->] (0,0) -- node[above=5pt] {$ \vec{r} $} ({sin(60) * 2cm},{cos(60) * 2cm});
		\draw (-1.2,1.2) to[out=90,in=0] (-2,2) node[left] {$ F $};
	\end{tikzpicture}
\end{minipage}%
\\
\begin{align*}
\mu_0 \int_F \dd \vec{f} ' \cdot \vec{j}(\vec{r}') &= \int_{\partial F} \dd \vec{r}' \cdot \vec{B}(\vec{r}')\\
&= \mu_0 \int_{0}^{\rho} \dd\rho' \int_{0}^{2 \pi} \dd \varpi'\rho' \frac{I}{\pi R^2} \theta(R - \rho)\\
&= \mu_0 \frac{I}{\pi R^2} 2 \pi \ub{\int_{0}^{\rho} \dd \rho' \rho' \theta(R - \rho)}\\
& \qquad = \left\{\begin{array}{cc}
\int_{0}^{R} \dots = \frac{1}{2} R^2 & \rho > R \\[5pt]
\int_{0}^{\rho}\: \dots = \frac{1}{2} \rho^2 & \rho \le R
\end{array}\right.
\end{align*}
\begin{equation*}
\Rightarrow \quad \mu_0 \int_{F} \dd \vec{f}' \cdot \vec{j}(\vec{r}') = \mu_0 \casess{I}{\rho > R}{\frac{\rho^2}{R^2} I}{\rho \le R}
\end{equation*}
\begin{equation*}
\int_{\partial F}\dd r' \vec{B}(\vec{r}') = \int_{0}^{2 \pi} \dd \varphi \frac{\dd r'}{\dd \varphi} \cdot \vec{B} = \int_{0}^{2 \pi} \dd \varphi \rho B_{\varphi}(\rho) = 2 \pi \rho B_{\varphi}(\rho)
\end{equation*}
mit:
\begin{equation*}
\vec{r}(\rho) = \begin{pmatrix}
\rho \cos \varphi \\ \rho \sin \varphi \\ 0
\end{pmatrix} \qquad \frac{\dd r'}{\dd \varphi} = \begin{pmatrix}
- \rho \sin \varphi \\ \rho \cos \varphi \\ 0
\end{pmatrix} = \rho \vec{e}_{\varphi}
\end{equation*}
Damit erhalten wir:
\begin{equation*}
 \mu_0 \int_{F} \dd \vec{f}' \cdot \vec{j}(\vec{r}')  = 2 \pi \rho B_{\varphi}(\rho) 
\end{equation*}
Daraus folgt:
\begin{equation*}
\Rightarrow \quad \mu_0 I_F = \rho B_{\varphi}(\rho) 2 \pi
\end{equation*}
Dies können wir umstellen in:
\begin{equation*}
\Rightarrow \quad B_{\varphi} (\rho) = \frac{\mu_0}{2 \pi} \frac{I_F}{\rho}
\end{equation*}
\begin{equation*}
\Rightarrow \quad \vec{B}(\vec{r}) = \frac{\mu_0}{2 \pi} I \vec{e}_{\varphi} \casess{\frac{1}{\rho}}{\rho > R}{\frac{\rho}{R^2}}{\rho \le R}
\end{equation*}
\begin{minipage}{.5\linewidth}
	%t4:
	\centering
	\begin{tikzpicture}
		\draw[->] ({sin(45) * .6cm},{- cos(45) * .4cm}) arc[start angle=-45,end angle=315,x radius = .6cm,y radius=.4cm];
		\draw[->] ({sin(45) * 1.2cm},{- cos(45) * .8cm}) arc[start angle=-45,end angle=315,x radius = 1.2cm,y radius=.8cm];
		\draw[draw=none,fill=white] (-.25,0) rectangle (.25,1.5);
		\draw (-.25,-1.5) -- (-.25,1.5) (.25,-1.5) -- (.25,1.5);
		\draw[thick] node[above] {R}  (0,0) -- (.25,0);
		\node at (1.2,-1.2) {$ \vec{B} $};
		\draw[->,thick] (0,.6) -- (0,1.2) node[left=5pt,yshift=-5pt] {$ I $};
		\draw (0,1.5) ellipse (.25cm and .17cm);
		\draw (-.25,-1.5) arc[start angle=180,end angle=360, x radius=.25, y radius=.17];
	\end{tikzpicture}
\end{minipage}%
\begin{minipage}{.5\linewidth}
	\vspace{15pt}
	%t5:
	\centering
	\begin{tikzpicture}
		\draw[thick,->] (0,0) -- (0,3) node[anchor=south east] {$ B_\varphi (\rho) $};
		\draw[thick,->] (0,0) -- (4,0) node[anchor=north west] {$ \rho $};
		\draw (2,.1) -- (2,-.1) node[below] {R};
		\draw (0,0) -- (2,2) to[out=-75,in=170] (4,.25);
		\draw (.1,2) -- (-.1,2) node[left] {$ \frac{\mu_0}{2 \pi} I \frac{1}{R} $};
		\draw[dashed] (0,2) -| (2,0);
	\end{tikzpicture}
\end{minipage}%


\subsubsection{Differentialgleichung für das Vektorpotential}

\begin{equation*}
\vec{B} = \vabla \times \vec{A} \qquad \vabla \times \vec{B} = \mu_0 \vec{j}
\end{equation*}
\begin{align*}
\mu_0 \vec{j} &= \vabla \times \vec{B} = \vabla \times (\vabla \times \vec{A})\\
&= \vabla(\vabla \cdot \vec{A}) - \Delta \vec{A}
\end{align*}
falls $ \vabla\cdot \vec{A} = 0 $ (Coulomb-Gleichung)
\begin{equation*}
\Rightarrow \quad \rmbox{\Delta \vec{A} = \mu_0 \vec{j}}
\end{equation*}
Wichtig: Die Komponenten sind nicht unabhängig voneinander aufgrund unserer Annahme $ \vabla \cdot \vec{A} = 0 $
analog:$ \Delta \varphi = - \frac{1}{\epsilon_0} \rho $

\subsection{Feldgleichungen der Magnetostatik}

\begin{minipage}{.5\linewidth}
	\begin{equation*}
	\vabla \cdot \vec{B} = 0
	\end{equation*}
	\vspace{-5pt}
	\begin{equation*}
	\phantom{\quad (\tx{Amp\`ere})} \vabla \times \vec{B} = \mu_0 \vec{j} \quad (\tx{Amp\`ere})
	\end{equation*}
\end{minipage}%
\begin{minipage}{.5\linewidth}
	\begin{equation*}
	\int_{\partial V} \dd \vec{f} \cdot \vec{B} = 0
	\end{equation*}
	\begin{equation*}
	\int_{\partial F} \dd \vec{r} \cdot \vec{B} = \mu_0 I_F
	\end{equation*}
\end{minipage}%
\\
Und für das Vektorpotential:
\begin{equation*}
\vec{B} = \vabla \times \vec{A} \qquad \Rightarrow \Delta  \vec{A} = - \mu_0 \vec{j}
\end{equation*}

\section{Multipolentwicklung - Magnetisches Moment}

\begin{equation*}
\vec{A}(\vec{r}) = \frac{\mu_0}{4 \pi} \int \dd ^3 r' \frac{\vec{j}(\vec{r}')}{|\vec{r} - \vec{r}'|}
\end{equation*}
mit $ \vabla \cdot \vec{A} = 0 $\\
\begin{minipage}{.5\linewidth}
	Wir betrachten eine lokalisierte Ladungsverteilung:
	\begin{equation*}
	\vec{j}(\vec{r}) = \casess{\tx{beliebig}}{r < R}{0}{r > R}
	\end{equation*}
\end{minipage}%
\begin{minipage}{.5\linewidth}
	%t6:
	\flushright
	\begin{tikzpicture}
		\draw[fill=gray!15!white]  plot [smooth cycle, tension=0.8] coordinates { (-1,0) (-1,1) (0,1) (.5,.5) (1,0) (.7,-.7) (0,-1) (-.4,-.7) };
		\draw[->] (0,0) -- (0,3);
		\draw[->] (0,0) -- (3,0);
		\draw[->] (0,0) -- (-1.8,-1.8);
		\draw[dashed] (0,0) circle (1.75cm);
		\draw (0,0) -- ({- sin(45) * 1.75},{cos(45) * 1.75})node[left=5pt] {R};
		\draw[->] (0,0) -- (1.2,.5) node[above] {$ \vec{r}' $};
		\draw[->] (0,0) -- (2.25,.5) node[right] {$ \vec{r} $};
	\end{tikzpicture}
\end{minipage}%
\\
Für $ r > R > r' $ machen wir eine Taylorentwicklung:
\begin{equation*}
\frac{1}{|\vec{r} - \vec{r}'|} = \frac{1}{r} + \frac{\vec{r}}{r^3} \cdot \vec{r}' + \dots
\end{equation*}
\begin{equation*}
\Rightarrow \quad \vec{A}(\vec{r}) = \frac{\mu_0}{4 \pi} \left\{\frac{1}{r} \int \dd^3r' \vec{j} (\vec{r}') + \frac{1}{r^3} \int \dd ^3 r' (\vec{r} \cdot \vec{r}') \vec{j}(\vec{r}') + \dots \right\}
\end{equation*}
\textbf{Es gilt: in der Magnetostatik} $ \vabla \cdot \vec{j} = 0 $
\begin{enumerate}[i)]
	\item Das Integral über die Stromdichte verschwindet:\\
	\begin{minipage}{.5\linewidth}
		$$ \int \dd ^3 r' \vec{j} (\vec{r}') = 0 $$
	\end{minipage}%
	\begin{minipage}{.5\linewidth}
		%t7:
		\centering
		\begin{tikzpicture}[scale=.5]
			\draw[->] (0,0) arc[start angle=135, end angle=495, radius = 1];
			\node at ({sin(45)},{-cos(45)}) {$ \vec{j} $};
		\end{tikzpicture}
	\end{minipage}%
	\\
	\item $$ \int \dd^3 r' (\vec{r} \cdot \vec{r}') \vec{j}(\vec{r}) = - \vec{r} \times \ub{\frac{1}{2} \int \dd^3 r' (\vec{r}' \times \vec{j}(\vec{r}'))}_{\mathclap{\substack{\defeq \vec{m}\\ \tx{magnetisches} \\ \tx{Dipolmoment} }}} $$
\end{enumerate}
Die Entwicklung des Vektorpotentials wird dann zu:
\begin{equation*}
\Rightarrow \quad \vec{A}(\vec{r}) = \frac{\mu_0}{4 \pi} \ub{\frac{\vec{m} \times \vec{r}}{r^3}}_{\propto \frac{1}{r^2}}
\end{equation*}
Für das Magnetfeld gilt:
\begin{equation*}
\Rightarrow \quad \vec{B} = \vabla \times \vec{A} = \frac{\mu_0}{4 \pi} \vabla \times \left(\vec{m} \times \frac{\vec{r}}{r^3}\right)
\end{equation*}
Mit der Identität: $ \vabla \times (f \vec{F}) = f \vabla \times \vec{G} - \vec{G} \times \vabla f $
\begin{align*}
\vabla \times \left(\frac{1}{r^3} (\vec{m} \times \vec{r})\right) &= \frac{1}{r^3} \ub{\vabla \times (\vec{m} \times \vec{r})}_{= 2 \vec{m}} - (\vec{m} \times \vec{r}) \times \ub{\vabla \frac{1}{r^3}}_{= - \frac{3 \vec{r}}{r^5}}\\
&= \frac{1}{r^3} 2 \vec{m} + \frac{3}{r^5} \ub{(\vec{m} \times \vec{r}) \times \vec{r}}_{(\vec{m} \cdot \vec{r}) \vec{r} - (\vec{r} \cdot \vec{r}) \vec{m}}\\
&= \frac{3 \vec{r} (\vec{m} \vec{r})}{r^5} - \frac{\vec{m}}{r^3}
\end{align*}
Beim $ \vec{B} $-Feld erhalten wir für den ersten nicht verschwindenden Term den Dipolterm:\\
\frbox{Multipolentwicklung des Magnetfeldes (1. Term)}{
	\begin{equation*}
	\Rightarrow \quad \vec{B}(\vec{r}) = \frac{\mu_0}{4 \pi} \left[\frac{3 \vec{r}(\vec{m} \cdot \vec{r})}{r^5} - \frac{\vec{m}}{r^3}\right] \qquad r > R
	\end{equation*}
}

\noindent
Der große Unterschied zum $ \vec{E} $-Feld ist, dass der führende Term ein Dipol ist. Das $ \vec{B} $-Feld hat also keinen Monopol.\\[5pt]
\emph{Beispiel:} \textbf{Magnetisches Dipolmoment einer Drahtschleife}\\
\begin{minipage}{.5\linewidth}
	$$ \rho = \sqrt{x^2 + y^2} \qquad \vec{e}_{\varphi} = \begin{pmatrix}
	- \sin \varphi \\ \cos \varphi \\ 0
	\end{pmatrix}$$
\end{minipage}%
\begin{minipage}{.5\linewidth}
	%t8:
	\centering
	\begin{tikzpicture}
		\draw[->,fill=gray!10!white] ({-sin(45) * 1.25},{cos(45) * .8}) arc[start angle=135, end angle=495, x radius = 1.25, y radius = .8] node[left=5pt] {$ I $};
		\draw[->] (0,0) -- (0,2) node[anchor=north west] {$ x $};
		\draw[->] (0,0) -- (2,0) node[anchor=south east] {$ z $};
		\node[circle,fill=black,inner sep=1pt,minimum size=1pt] at (0,0) {};
		\draw (0,0) -- ({sin(45) * 1.25},{cos(45) * .8}) node[anchor=south west] {$ R $};
	\end{tikzpicture}
\end{minipage}%
\\
\begin{equation*}
\vec{j} = I \delta(\rho - R) \delta(z) \vec{e}_{\varphi}
\end{equation*}
\begin{equation*}
\vec{m} = \frac{1}{2} \int \dd^3 r \ \vec{r} \times \vec{j}(\vec{r})
\end{equation*}
Nebenrechnung:
\begin{align*}
\vec{r} \times \vec{e}_{\varphi} &= \begin{pmatrix}
\rho \cos \varphi \\ \rho \sin \varphi \\ z
\end{pmatrix} \times \begin{pmatrix}
- \sin \varphi \\ \cos \varphi \\ ß
\end{pmatrix}\\
&= \begin{pmatrix}
- z \cos \varphi \\ - z \sin \varphi \\ \rho
\end{pmatrix}\\
&= \rho \begin{pmatrix}
0 \\ 0 \\ 1
\end{pmatrix} - z \begin{pmatrix}
\cos \varphi \\ \sin \varphi \\ 0
\end{pmatrix}\\
&= \rho \vec{e}_z - z \vec{e}_{\varphi}
\end{align*}
\begin{align*}
\Rightarrow \quad \vec{m} &= \frac{1}{2} I \int_{0}^{\infty} \rho \dd \rho \int_{0}^{2 \pi} \dd \varphi \int_{-\infty}^{\infty} \dd z \delta(\rho - R) \delta(z) (\rho \vec{e}_z - \custo{\rightarrow}{z}{0} \vec{e}_{\varphi})\\
&= \frac{I}{2} R^2 2 \pi \vec{e}_z
\end{align*}
\begin{minipage}{.5\linewidth}
	\begin{equation*}
	\Rightarrow \quad \vec{m} = \ub{\pi R^2}_{F} I \vec{e}_z = F I \vec{e}_z
	\end{equation*}
\end{minipage}%
\begin{minipage}{.5\linewidth}
	%t9:
	\centering
	\begin{tikzpicture}[scale=.75]
		\draw[->,fill=gray!10!white] ({-sin(45) * 1.25},{cos(45) * .8}) arc[start angle=135, end angle=495, x radius = 1.25, y radius = .8] node[left=5pt] {$ I $};
		\draw[->] (0,0) -- (0,2);
		\draw[->] (0,0) -- (2,0);
		\node[circle,fill=black,inner sep=1pt,minimum size=1pt] at (0,0) {};
		\draw[very thick,->] (0,0) -- (0,1.5) node[anchor=north west] {$ \vec{m} $};
	\end{tikzpicture}
	\vspace{5pt}
\end{minipage}%
\\
\begin{minipage}{.5\linewidth}
	Das Magnetische Moment einer Spule:
	\begin{equation*}
	\vec{m} = N I \cdot F \vec{e}_z
	\end{equation*}
\end{minipage}%
\begin{minipage}{.5\linewidth}
	%t10:
	\centering
	\begin{tikzpicture}
		\draw[decorate,decoration={aspect=.5, segment length=4mm, amplitude=.5cm,coil},thick,<-] (0,2) node[anchor=south west] {$ I $} -- (0,0);
	\end{tikzpicture}
	\vspace{10pt}
\end{minipage}%
\\
\folie{Vergleich idealer Dipol und Leiterschleife}\\
\folie{Vergleich $ \vec{E} $-Feld einer elektrischer Dipol und Magnetfeld um Leiterschleife}

% Vorlesung 10.12.18 (14 Days until christmas)

\subsection{Kraft auf eine lokalisierte Stromverteilung in einem äußeren Magnetfeld \texorpdfstring{$ \vec{B} $}{B}}

\begin{minipage}{.6\linewidth}
	$ \vec{j}(\vec{r}) = 0 \qquad |\vec{r}| > 0 $
	Taylorentwicklung von $ \vec{B}(\vec{r}) $ um $ \vec{r} = 0 $:
	\begin{equation*}
	\vec{B}(\vec{r}) = \vec{B}(0) + (\vec{r} \cdot \vabla) \vec{B}(\vec{r}) \bigg|_{\vec{r} = 0} + \dots
	\end{equation*}
	\begin{equation*}
	\longrightarrow \quad \vec{F} = \int \dd^3 r \left(\vec{j}(\vec{r}) \times \vec{B} (\vec{r})\right) \quad \ \
	\end{equation*}
\end{minipage}%
\begin{minipage}{.4\linewidth}
	%t1:
	\flushright
	\begin{tikzpicture}
		\draw[fill=gray!10] plot [smooth cycle, tension=.8] coordinates { (.5,-1.5) (0,-1) (-.7,-.5) (-.7,.7) (.7,1) (.7,0) (1.2,-.7) };
		\draw[->] (0,0) -- (0,2) node[anchor=south east] {$ z $};
		\draw[->] (0,0) -- (2,0) node[anchor=north west] {$ x $};
		\draw[->] (0,0) -- (-1.4,-1.4) node[anchor=north east] {$ y $};
		\node at (-.3,.45) {$ \vec{j} $};
		\coordinate (a) at (1.5,1.7);
		\coordinate (b) at ($ (a) + (-44:.5) $);
		\coordinate (c) at ($ (a) + (-45:1) $);
		\coordinate (d) at ($ (a) + (-50:1.5) $);
		\draw[<-] (a) to[out=-150,in=10] ++(-160:4);
		\draw[<-] (b) to[out=-145,in=15] ++(-155:4.5);
		\draw[<-] (c) to[out=-145,in=35] ++(-145:4.5);
		\draw[<-] (d) to[out=-145,in=55] ++(-135:4);
		\node at (2.5,1.5) {$ \vec{B} $};
		\draw[dashed] (0,0) circle (1.7cm);
	\end{tikzpicture}
\end{minipage}%
\begin{equation*}
\vec{F} = \ub{\int\dd^3 r \left(\vec{j}(\vec{r}) \times \vec{B}(0)\right)}_{= \bigg(\ub{\int \dd^3 r \vec{j} (\vec{r})}_{= 0} \bigg) \times \vec{B}(0)} + \int\dd ^3 r \left[ \vec{j}(\vec{r}) \times (\vec{r} \cdot \vabla) \vec{B}(0) \right] + \dots
\end{equation*}
\begin{minipage}{.7\linewidth}
	Der verschwindende Teil ist ein homogenes $ \vec{B} $-Feld ($ \vec{B} = \const $), und übt daher keine Kraft auf Stromverteilung aus.
\end{minipage}%
\begin{minipage}{.3\linewidth}
	%t2:
	\flushright
	\begin{tikzpicture}[scale=.75]
		\draw[->] (-{sin(45) * 1},{cos(45) * 1}) arc[start angle=135,end angle=-225, radius=1cm] node[anchor=south east] {$ I $};
		\draw[thick,->] (-1,0) -- node[left] {$ \vec{j} $} ++(0,-1);
		\draw[thick,->] (1,0) -- node[right] {$ \vec{j} $} ++(0,1);
	\end{tikzpicture}
\end{minipage}%
\\
Die Komponenten des Kraftvektors sind:
\begin{equation*}
\vec{F}_i = \int \dd^3 r \left[\vec{j} \times (\vec{r} \cdot \vabla) \vec{B}\right]_i
\end{equation*}
Nun nutzen wie die folgende Identität:
\begin{equation*}
(\vec{a} \times \vec{b})_i = \sum_{k,l} \epsilon_{ikl} \ a_k b_l
\end{equation*}
$ \epsilon_{ikl} $ ist das Levi-Civita Symbol.\\
Damit ergibt sich:
\begin{align*}
\left(\vec{j} \times (\vec{r} \cdot \vabla) \vec{B}\right)_i &= \sum_{k,l} \epsilon_{ikl} \  j_k \ub{\left[(\vec{r} \cdot \vabla) \vec{B}\right]_l}_{(\vec{r} \cdot \vabla) B_l = (\vabla B_l) \cdot \vec{r}}\\
&= \sum_{k,l} \epsilon_{ikl} \ j_k (\vabla B_l \cdot \vec{r})
\end{align*}
\begin{align*}
\rightarrow \quad F_i &= \sum_{k,l} \epsilon_{ikl} \ \ub{\int \dd ^3 r \left[(\vabla B_l) \cdot \vec{r}\right] j_k}\\
&\qquad \qquad = \int \dd^3 r \left( \left[(\vabla B_l) \cdot \vec{r}\right] \vec{j}\right)_k \overset{\tx{Identität}}{=} - \frac{1}{2} \left[\vabla B_l \times \int \dd^3 r \vec{r} \times \vec{j} \right]
\end{align*}
Hier die benutzte Identität (aus den Hausaufgaben):
\begin{equation*}
\frac{1}{2} \vec{a} \times \int \dd ^3 r (\vec{r} \times \vec{j} (\vec{r})) = - \int \dd^3 r (\vec{a} \cdot \vec{r}) \vec{j}(\vec{r}) \qquad (\vabla \cdot \vec{j} = 0)
\end{equation*}
Damit ergibt sich für die Kraft:
\begin{align*}
F_i &= - \frac{1}{2} \sum_{k,l} \epsilon_{ikl} \ \left[\vabla B_l \times \ub{\int \dd^3 r \vec{j} \times \vec{j}(\vec{r})}_{=2 \vec{m}}\right]_k \\
&= - \sum_{k,l} \epsilon_{ikl} \ \ub{\left(\vabla B_l \times \vec{m}\right)}_{- \left[\vec{m} \times \vabla B_l\right]_k = - (\vec{m} \times \vabla)_k B_l}\\
&= \sum_{k,l} \epsilon_{ikl} \ (\vec{m} \times \vabla)_k B_l\\
&= \left[(\vec{m} \times \vabla) \times \vec{B}\right]_i
\end{align*}
\begin{equation*}
\rmbox{F_i = \left[(\vec{m} \times \vabla) \times \vec{B}\right]_i}
\end{equation*}
Wir können nun mit der Identität umschreiben:
\begin{equation*}
(\vec{a} \times \vec{b}) \times \vec{c} = \vec{b} (\vec{c} \cdot \vec{a}) - \vec{a} (\vec{b} \cdot \vec{c})
\end{equation*}
\begin{align*}
\vec{F} &= (\vec{m} \times \vabla) \times \vec{B}(0)\\
&= \vabla (\vec{m} \cdot \vec{B}) - \vec{m} (\ub{\vabla \cdot \vec{B}}_{=0})
\end{align*}
\begin{equation*}
\rmbox{\Rightarrow \quad \vec{F} = \vabla (\vec{m} \cdot \vec{B}(0))}
\end{equation*}
Also: $ \vec{m} \perp \vec{B} \Rightarrow \vec{F} = 0 $\\
\begin{minipage}{.5\linewidth}
	\textbf{$ \rightarrow $ potentielle Energie:}
	\begin{equation*}
	W = - \vec{m} \cdot \vec{B}(0)
	\end{equation*}
\end{minipage}%
\begin{minipage}{.5\linewidth}
	%t3:
	\centering
	\begin{tikzpicture}[scale=.75]
		\draw[->,fill=gray!10!white] ({-sin(45) * 1.25},{cos(45) * .8}) arc[start angle=135, end angle=495, x radius = 1.25, y radius = .8] node[left=5pt] {$ I $};
		\draw[->] (0,0) -- (0,2);
		\draw[->] (0,0) -- (2,0);
		\node[circle,fill=black,inner sep=1pt,minimum size=1pt] at (0,0) {};
		\draw[very thick,->] (0,0) -- (0,1.5) node[anchor=north west] {$ \vec{m} $};
		\draw[very thick,->] (2.5,0) -- node[right] {$ \vec{B} $} ++(0,1.5);
	\end{tikzpicture}
\end{minipage}%
\\
\textbf{Drehmoment:}
\begin{equation*}
\vec{N} = \vec{m} \times \vec{B}(0)
\end{equation*}
\begin{equation*}
\vec{N} = \int\dd^3 r \ \vec{r} \times (\vec{j} \times \vec{B})
\end{equation*}

\section{Magnetostatik in Materie}

mikroskopische Feldgleichungen:
\begin{equation*}
\vabla \cdot \vec{B} = 0 \qquad \vabla \times \vec{B} = \mu_0 \vec{j}
\end{equation*}

\subsection{Makroskopische Feldgleichungen}

Definition Mittlung:
\begin{equation*}
\langle \vec{B} \rangle (\vec{r}) = \int\dd^3r' f(\vec{r}') \vec{B}(\vec{r} - \vec{r}')
\end{equation*}
Es muss weiterhin gelten:
\begin{equation*}
\vabla \cdot \langle \vec{B} \rangle (\vec{r}) = 0 \qquad \vabla \times \langle \vec{B} \rangle (\vec{r}) = \mu_0 \langle \vec{j} \rangle (\vec{r})
\end{equation*}
\lcom{Aus Zeitgründen werden wir hier nichts weiter genau herleiten wie in der Elektrostatik, sondern im wesentlichen die Ergebnisse Besprechen.}\\[5pt]
Aufteilung der Stromdichte:
\begin{equation*}
\vec{j} = \vec{j}_g + \vec{j}_f
\end{equation*}
$ \langle j_g \rangle $\\
\begin{minipage}{.5\linewidth}
	\centering
	%t4:
	\begin{tikzpicture}
		\draw[fill=gray!10!white] (-1,1.8) -- (-1,-1.8) arc[start angle=180,end angle=360,x radius=1cm,y radius=.4cm] (1,-1.8) -- (1,1.8) arc[start angle=0,end angle=180,x radius=1cm,y radius=.4cm];
		\draw[fill=gray!10!white] (0,1.8) ellipse (1cm and .4cm);
		\draw[decorate,decoration={coil,aspect=.3,amplitude=12mm,segment length=12mm},brown,thick] (0,2.5) -- (0,-2.5);
		\node at (-.5,.7) {$ e^- $};
		\draw[->] (.65,.5) arc[start angle=-45,end angle=315,x radius=.5,y radius=.2];
		\draw (.9,.6) to[out=0,in=125] (2,0) node[right] {$ \vec{j}_g $};
	\end{tikzpicture}
\end{minipage}%
\begin{minipage}{.5\linewidth}
	\centering
	%t5:
	\begin{tikzpicture}
	\draw[draw=none,fill=gray!15!white] (1.5,.5) rectangle ++(1,1);
	\foreach \x in {.5,1.5,2.5}
	\draw (\x,0) -- (\x,3);
	\foreach \y in {.5,1.5,2.5}
	\draw (0,\y) -- (3,\y);
	\coordinate (o) at (3.5,-.5);
	\draw (o) -- ++(0,.5);
	\draw (o) -- ++(.5,0);
	\draw[thick,->] (o) -- node[below=10pt] {$ \vec{r}_n $} (2,1);
	\node[anchor=south west] at (1.5,.5) {$ n $};
	\draw (2,1) to[out=30,in=150] ++(1,0) to[out=-30,in=-150] ++(1,0) node[right] {$ \vec{j}_n $};
	\node[anchor=north east] at (o) {$ 0 $};
	\end{tikzpicture}
\end{minipage}%
\\
Mittlung:
\begin{align*}
\rightarrow \quad \langle \vec{j}_g \rangle &= \vabla \times \ub{\langle \sum_n \vec{m} \delta(\vec{r}' - \vec{r}_n) \rangle}_{\defeq \vec{M}(\vec{r})} (\vec{r}) + \dots \\
&= \frac{\tx{magnetisches Dipolmoment}}{\tx{Volumen}} = \vabla \times \vec{M}(\vec{r})
\end{align*}
Mit
$$ \vec{m}_n = \frac{1}{2} \int \dd^3 r' \vec{r}' \times \vec{j}_n(\vec{r}') $$
Und $ \vec{M} (\vec{r}) : $ makroskopische Magnetisierung
\begin{equation*}
\left[\vec{M}\right] = \frac{\tx{A}}{\tx{m}^2} \tx{m} = \frac{\tx{A}}{\tx{m}} = \frac{\tx{magnetisches Dipolmoment}}{\tx{Volumen}}
\end{equation*}
\begin{equation*}
\Rightarrow \quad \vabla \times \langle \vec{B} \rangle (\vec{r}) = \mu_0 \langle \vec{j} \rangle (\vec{r}) + \mu_0 \vabla \times \vec{M}(\vec{r}) + \dots
\end{equation*}
\begin{equation*}
\Rightarrow \quad \vabla \times \ub{\left(\frac{1}{\mu_0} \langle \vec{B} \rangle (\vec{r})  - \vec{M} (\vec{r}) - \dots \right)}_{\defeq \vec{H}(\vec{r})} = \langle \vec{j}_f \rangle (\vec{r})
\end{equation*}
\begin{equation*}
\left[\vec{H}\right] = \frac{\tx{A}}{\tx{m}}
\end{equation*}

\subsection{Makroskopische Feldgleichunge der Magnetostatik}

\begin{equation*}
\vabla \cdot \vec{B} = 0 \qquad \vabla \times \vec{H} = \vec{j}_f \qquad 
\end{equation*}
\begin{equation*}
\vec{H} = \frac{1}{\mu_0} \vec{B} - \vec{M} - \dots
\end{equation*}
Integrale Form:
\begin{equation*}
\oint_{F} \dd \vec{f} \cdot \vec{B} = 0
\end{equation*}
\begin{equation*}
\int_{\partial F} \dd \vec{r} \cdot \vec{H} = \int_{F} \dd \vec{f} \cdot \vec{j}_f = I_F
\end{equation*}
\begin{center}
	%t6:
	\begin{tikzpicture}
		\draw[fill=gray!10!white] (-1,1.8) -- (-1,-1.8) arc[start angle=180,end angle=360,x radius=1cm,y radius=.4cm] (1,-1.8) -- (1,1.8) arc[start angle=0,end angle=180,x radius=1cm,y radius=.4cm];
		\draw[fill=gray!10!white] (0,1.8) ellipse (1cm and .4cm);
		\draw[decorate,decoration={coil,aspect=.3,amplitude=13mm,segment length=12mm},<-,brown,thick] (0,2.5) node[above] {\color{black} $ \vec{j}_f $ \color{brown}} -- (0,-2.5);
		
		\foreach \x\y\c in {.65/.5/1,0/-.75/2,.65/-1.9/3}
		\coordinate (\c) at ({\x - sin(45) * .5}, {\y + cos(45) * .2});
		
		\foreach \x\y in {.65/.5,0/-.75,.65/-1.9}
		\draw[->] (\x,\y) arc[start angle=-45,end angle=315,x radius=.5,y radius=.2];
		
		\foreach \c in {1,2,3}
		\draw[very thick,->] (\c) -- ++(0,.5);
		
		\draw (.9,.8) to[out=0,in=125] (2,0) node[right] {Magnetisierung $ \vec{m} $};
		\draw[thick,->] (-1.1,-1) -- node[left=5pt] {$ \vec{B}_0 $} ++(0,2);
		
		% ugly alignment line
		\draw[white] (-2,0) -- (-5.3,0);
	\end{tikzpicture}
\end{center}
Magnetisierung $ \rightarrow $ Zusatzfeld $ \vec{B}_M $
\begin{equation*}
\vec{B} = \vec{B}_0 + \vec{B}_M
\end{equation*}

\subsection{Vektorpotential}

\begin{equation*}
\vec{B} = \vabla \times \vec{A} \quad \rightarrow \quad \langle \vec{B} \rangle = \vabla \times \langle \vec{A} \rangle
\end{equation*}
\begin{equation*}
\vec{A} (\vec{r}) = \frac{\mu_0}{4 \pi} \int \dd ^3 r' \frac{\vec{j}(\vec{r}')}{|\vec{r} - \vec{r}'|} \qquad (\Lambda = 0 \quad \vabla \cdot \vec{A} = 0)
\end{equation*}
\begin{align*}
\Rightarrow \quad \langle \vec{A} \rangle (\vec{r}) &= \frac{\mu_0}{4 \pi} \int \dd^3r' \frac{\langle \vec{j} \rangle (\vec{r}')}{|\vec{r} - \vec{r}'|}\\
&= \frac{\mu_0}{4 \pi} \int \dd^3 r' \frac{\langle \vec{j} \rangle (\vec{r}')}{|\vec{r} - \vec{r}'|} + \ub{\frac{\mu_0}{4 \pi} \int\dd^3 r ' \frac{\vabla_{\vec{r}'} \times \vec{M} (\vec{r}')}{|\vec{r} - \vec{r}'|}}\\
& \qquad \qquad \qquad \qquad \quad = \frac{\mu_0}{4 \pi} \int \dd^3 r' \vec{M}(\vec{r}') \times \frac{(\vec{r} - \vec{r}')}{|\vec{r} - \vec{r}'|^3}
\end{align*}
Erklärung der letzten Umformung:
\begin{align*}
\int \dd^3 r' \frac{1}{|\vec{r} - \vec{r}'|} \vabla_{\vec{r}'} \times \vec{M} (\vec{r}') &= \int \dd^3 r' \vabla_{\vec{r}'} \times \left(\frac{1}{|\vec{r} - \vec{r}'|} \vec{M}\right) - \ub{\left(\vabla_{\vec{r}'} \frac{1}{|\vec{r} - \vec{r}'|}\right)}_{= \frac{\vec{r} - \vec{r}'}{|\vec{r} - \vec{r}'|^3}} \times \vec{M}\\
&= \ub{\int\dd^3 r' \vabla_{\vec{r}'} \times \left(\frac{\vec{M}}{|\vec{r} - \vec{r}'|}\right)} + \int\dd^3r' \vec{M}(\vec{r}')  \times \frac{\vec{r} - \vec{r}'}{|\vec{r} - \vec{r}'|}\\
&= \lim\limits_{R \to \infty} \int_{K_R(0)} \dd^3 r' \vabla_{\vec{r}'} \times \left(\frac{\vec{M}}{|\vec{r} - \vec{r}'|}\right)
\end{align*}
%
%
%
% kim
%
%
%

\subsection{Magnetisierung und Suszeptibilität}

\begin{equation*}
\vec{M} = \vec{M}(\vec{H})
\end{equation*}
lineare Näherung, isotrope Medien:
\begin{equation*}
\vec{M} = \chi_m \vec{H}
\end{equation*}
$ \chi_m : $ magnetische Suszeptibilität
\begin{align*}
\rightarrow \quad \vec{B} &= \mu_0 (\vec{H} + \vec{M}) = \mu_0 (\vec{H} + \chi_m \vec{H})\\
&= \ub{(1 + \chi_m)}_{= \mu_r} \mu_0 \vec{H}\\
&= \mu \vec{H}
\end{align*}
$ \mu_r : $ relative Permeabilität\\
$ \mu = \mu_0 \mu_r : $ Permeabilität

% Vorlesung 13.12.18 (11 Days until christmas)


% ab Hier auch richtige Phi und keine varPhis

\chapter{Zeitabhängige elektromagnetische Felder - Elektrodynamik}

bisher Zeitunabhängige Felder:
\begin{equation*}
\, \vabla \cdot \vec{D} = \rho \qquad \, \vabla \times \vec{E} = 0 \qquad \tx{Elektrostatik}
\end{equation*}
\begin{equation*}
\ \ \: \vabla \cdot \vec{B} = 0 \qquad \vabla \times \vec{H} = \vec{j} \qquad \tx{Magnetostatik}
\end{equation*}
Zeitabhängige Felder:
\begin{equation*}
\vec{B}(\vec{r}, t) \leftrightarrow \vec{E}(\vec{r}, t)
\end{equation*}

\section{Maxwell-Gleichungen}

\begin{align*}
\vabla \cdot \vec{D} &= \rho \qquad \, \vabla \times \vec{E} + \prt{\vec{B}}{t} = 0 \\
\vabla \cdot \vec{B} &= 0 \qquad \vabla \times \vec{H} - \prt{\vec{D}}{t} = \vec{j}
\end{align*}
$ \prt{\vec{B}}{t} \ $ ist die Induktion\\[3pt]
$ \prt{\vec{D}}{t} \ $ ist der Verschiebungsstrom\\

\subsection{Faraday'sches Induktionsgesetz}

\begin{equation*}
\vabla \times \vec{E} = - \prt{\vec{B}}{t}
\end{equation*}
\begin{center}
	%t1: %t2:
	\begin{tikzpicture}
	% \draw (0,0) ellipse (1cm and .7cm);
	\coordinate (r) at (-2.2,0);
	\draw[->] ($ ({cos(45) * -.9},{sin(45) * -.6}) + (r) $) arc[start angle=-135, end angle=225, x radius = .9cm, y radius = .6cm] node[anchor=north east] {$ \partial F $};
	\draw[->] ($ ({cos(45) * .9},{sin(45) * .6}) + (r) $) arc[start angle=-315, end angle=45, x radius = .9cm, y radius = .6cm];
	\node at (r) {$ F $};
	\node[left] at ($ (r) + (-1,0) $) {$ U_{\tx{ind}} $};
	\draw[fill=gray!30!white] (-.3,-.7) rectangle node[] {$ \substack{\tx{\Large{S}} \\[10pt] \tx{\Large{N}}} $} (.3,.7);
	\draw[->] (0,0) -- (.7,0) node[right] {$ \vec{v} $};
	
	\draw[looseness=2] (.15,.7) to[out=80,in=90] (1.2,0) to[out=-90,in=-80] (.15,-.7);
	\draw[looseness=2] (.09,.7) to[out=84,in=90] (1.8,0) to[out=-90,in=-84] (.09,-.7);
	\draw[looseness=2.5] (.03,.7) to[out=88,in=90] (2.6,0) to[out=-90,in=-88] (.03,-.7);
	
	\draw[looseness=2] (-.15,.7) to[out=100,in=90] (-1.2,0) to[out=-90,in=-100] (-.15,-.7);
	\draw[looseness=2] (-.09,.7) to[out=96,in=90] (-1.8,0) to[out=-90,in=-96] (-.09,-.7);
	\draw[looseness=2.5] (-.03,.7) to[out=92,in=90] (-2.6,0) to[out=-90,in=-92] (-.03,-.7);
	\end{tikzpicture}
\end{center}
\begin{align*}
\int_F \dd \vec{f} \cdot (\vec{\nabla} \times \vec{E}) &= - \int \dd \vec{f} \cdot \prt{\vec{B}}{t} = \ub{\int_{\partial F} \dd \vec{r} \cdot \vec{E}}_{U_{\tx{ind}}} = - \prd{}{t} \ub{\int_F \dd \vec{f} \cdot \vec{B}}_{\Phi_m (t)}
\end{align*}
$ \Phi_m(t) $ ist der Fluss des $ \vec{B} $-Feldes durch F
\begin{equation*}
U_{\tx{ind}} = - \prd{}{t} \int_F \dd \vec{f} \cdot \vec{B}
\end{equation*}
\emph{Bemerkung:}
\begin{enumerate}[i)]
	\item Vorzeichen-Strom
	\begin{center}
			%t3:
			\begin{tikzpicture}[scale=1.25]
				\coordinate (a) at (0,0);
				\draw[->] ($ ({cos(45) * -.9},{sin(45) * -.6}) + (a) $) arc[start angle=-135, end angle=225, x radius = .9cm, y radius = .6cm] node[anchor=north east] {$ \partial F $};
				\draw[->] ($ ({cos(45) * .9},{sin(45) * -.6}) + (a) $) arc[start angle=-45, end angle=315, x radius = .9cm, y radius = .6cm] node[anchor=north west] {$ \vec{j} $};
				\draw[thick,->] ($ (a) + (.3,-.3) $) -- ++(0,1.2);
				\draw ($ (a) + (.4,0) $) -- ++(1,.5) node[right] {$ \vec{B}(t_1) = \vec{B}_1 $};
				\draw[thick,->] ($ (a) + (-.3,.3) $) -- node[left=10pt] {$ \dd \vec{f} $} ++(0,.6);
				
				\coordinate (b) at (6,0);
				\draw[->] ($ ({cos(45) * -.9},{sin(45) * -.6}) + (b) $) arc[start angle=-135, end angle=225, x radius = .9cm, y radius = .6cm] node[anchor=north east] {$ \partial F $};
				\draw[->] ($ ({cos(45) * .9},{sin(45) * -.6}) + (b) $) arc[start angle=-45, end angle=315, x radius = .9cm, y radius = .6cm] node[anchor=north west] {$ \vec{j} $};
				\draw[thick,->] ($ (b) + (.3,-.3) $) -- ++(0,.6);
				\draw ($ (b) + (.4,0) $) -- ++(1,.5) node[right] {$ \vec{B}(t_2) = \vec{B}_2 $};
				\draw[thick,->] ($ (b) + (-.3,.3) $) -- node[left=10pt] {$ \dd \vec{f} $} ++(0,.6);
			\end{tikzpicture}
	\end{center}
	\begin{equation*}
	t_2 > t_1 \qquad \vec{B}(t_1) > \vec{B}(t_2)
	\end{equation*}
	\begin{equation*}
	\Phi_m(t_1) = \int_F \dd \vec{f} \vec{B}_1 > \Phi_m(t_2) = \int_F \dd \vec{f} \vec{B}_2
	\end{equation*}
	\begin{equation*}
	\prd{}{t} \Phi_m(t) \le 0
	\end{equation*}
	\begin{equation*}
	\rightarrow \quad U_{\tx{ind}} = \int_{\partial F} \dd \vec{r} \cdot \vec{E} = - \prd{}{t} \Phi_m > 0
	\end{equation*}
	Mit $ \vec{j} = \sigma \vec{E} $, wobei $ \sigma $ die Leitfähigkeit des Leiters ist erhalten wir:
	\begin{equation*}
	\rightarrow \quad \int_{\partial F} \dd \vec{r} \cdot \vec{j} > 0 \qquad \rightarrow \quad \vec{j} \parallel \partial F
	\end{equation*}
	\begin{minipage}{.5\linewidth}
		$ \Rightarrow $ \textbf{Lenz'sche Regel:}\\
		Der induzierte Strom und das damit verbundene Magnetfeld sind so gerichtet, dass sie der Ursache ihrer Entstehung entgegenwirken.
	\end{minipage}%
	\begin{minipage}{.5\linewidth}
		\flushright
		%t4:
		\begin{tikzpicture}
			\draw[->,very thick] ({cos(70) * 2cm}, {sin(70) * -1cm}) arc[start angle=-70, end angle=290,x radius = 2cm, y radius = 1cm] node[anchor=north west] {$ \vec{j} $};
			\draw[->, thick] (1,0) arc[start angle=180,end angle=-180,x radius=1cm, y radius=1.25cm];
			\node[right] at (3,0) {$ \vec{B}_{\tx{ind}} $};
			\foreach \x\n\l in {-1.2/1/1.8,0/2/1.4}
			\draw[thick,->] (\x,0) node[below] {$ \vec{B}(t_{\n}) $} -- ++(0,\l);
		\end{tikzpicture}
	\end{minipage}%
	\\
	Hier fällt das äußere $ \vec{B} $-Feld ab, wodurch ein Strom entsteht, der wiederum ein $ \vec{B} $-Feld $ \vec{B}_{\tx{ind}} $ induziert welches gegen den Abfall wirkt, also das äußere $ \vec{B} $-Feld verstärkt.
	\item $ \Phi_m $ hängt nicht von der speziellen Form der Fläche $ F $ ab, sondern nur vom Rand $ \partial F $.\\
	\begin{center}
		%t5:
		\begin{tikzpicture}
			\coordinate (c1) at (0,0);
			\draw[fill=blue!30!white,->] ($ (c1) + ({sin(45) * 1cm},{cos(45) * -.5cm}) $) arc[start angle=-45, end angle=315, x radius=1cm, y radius=.5cm] node[anchor=north west] {$ \partial F $};
			\draw[->] ($ (c1) + (.25,.25) $)  -- node[right] {$ \dd\vec{f}_1 $} ++(0,1);
			\draw ($ (c1) + (-.5,-.25) $) to[out=-130,in=-20] ++(-1,-.2) node[left] {$ F_1 $};
			
			\coordinate (c2) at (4,0);
			\draw[fill=red!30!white,looseness=2] ($ (c2) + (-1,0) $) to[out=-90,in=-90] ($ (c2) + (1,0) $);
			\draw[fill=red!13!white,->] ($ (c2) + ({sin(45) * 1cm},{cos(45) * -.5cm}) $) arc[start angle=-45, end angle=315, x radius=1cm, y radius=.5cm] node[xshift=.7cm] {$ \partial F $};
			\draw[->] ($ (c2) + (.25,-.3) $)  -- node[right,yshift=10pt] {$ \dd\vec{f}_2 $} ++(0,1.55);
			\draw ($ (c2) + (.6,-.8) $) to[out=-50,in=200] ++(1,-.2) node[right] {$ F_2 $};
			\node at ($ (c2) + (3.5,0) $) {$ \Phi_{m_1} = \Phi_{m_2} $};
			
			\coordinate (c3) at (2,-3);
			\draw[fill=red!30!white,looseness=2] ($ (c3) + (-1,0) $) to[out=-90,in=-90] ($ (c3) + (1,0) $);
			\draw[fill=blue!30!white] ($ (c3) + ({sin(45) * 1cm},{cos(45) * -.5cm}) $) arc[start angle=-45, end angle=315, x radius=1cm, y radius=.5cm];
			\draw[->] ($ (c3) + (-.25,.25) $)  -- node[left=5pt] {$ \dd\vec{f}_1 $} ++(0,1);
			\draw[->] ($ (c3) + (.25,-.75) $) -- node[right=5pt] {$ - \dd \vec{f_2} $} ++(0,-1);
		\end{tikzpicture}
	\end{center}
	
	\noindent
	Aus der Skizze und mit Satz von Gauß. (Der Fluss durch eine geschlossene Fläche ist gleich 0):
	\begin{equation*}
	\Phi = \int_F \dd \vec{f} \cdot \vec{B} = 0
	\end{equation*}
	\begin{equation*}
	\Phi_{\tx{m}_1} = \Phi_{\tx{m}_2}
	\end{equation*}
	\begin{equation*}
	0 = \int_F \dd \vec{f} \cdot \vec{B} = \ub{\int_{F_1} \dd \vec{f}_1 \cdot \vec{B}}_{\Phi_{\tx{m}_1}} - \ub{\int_{F_2} \dd \vec{f}_2 \cdot \vec{B}}_{\Phi_{\tx{m}_2}}
	\end{equation*}
	\item Eine Flussänderung kann auf verschiedene Weise zustandekommen:\\
	\begin{minipage}{.5\linewidth}
		\begin{itemize}
			\item Veränderung des Magnetfeldes $ \vec{B}(t) $
			\item Bewegung der Leiterschleife im äußeren Feld
		\end{itemize}
		\begin{equation*}
		\vabla \times \vec{E} = - \prt{\vec{B}}{t}
		\end{equation*}
		\vspace{20pt}
	\end{minipage}%
	\begin{minipage}{.5\linewidth}
		\flushright
		%t6:
		\begin{tikzpicture}[scale=1]
			\coordinate (r) at (-2.2,0);
			\draw[thick,->] ($ ({cos(45) * -.9},{sin(45) * -.6}) + (r) $) arc[start angle=-135, end angle=225, x radius = .9cm, y radius = .6cm];
			\draw[fill=gray!30!white] (-.3,-.7) rectangle node[] {$ \substack{\tx{\Large{S}} \\[8pt] \tx{\Large{N}}} $} (.3,.7);
			\draw[thick,->] (0,0) -- (1.1,0) node[right] {$ \vec{v} $};
			\draw[thick,->] ($ (r) + (-1,0) $) -- ++(-1.1,0);
			
			\draw[looseness=2] (.15,.7) to[out=80,in=90] (1.2,0) to[out=-90,in=-80] (.15,-.7);
			\draw[looseness=2] (.09,.7) to[out=84,in=90] (1.8,0) to[out=-90,in=-84] (.09,-.7);
			\draw[looseness=2.5] (.03,.7) to[out=88,in=90] (2.6,0) to[out=-90,in=-88] (.03,-.7);
			
			\draw[looseness=2] (-.15,.7) to[out=100,in=90] (-1.2,0) to[out=-90,in=-100] (-.15,-.7);
			\draw[looseness=2] (-.09,.7) to[out=96,in=90] (-1.8,0) to[out=-90,in=-96] (-.09,-.7);
			\draw[looseness=2.5] (-.03,.7) to[out=92,in=90] (-2.6,0) to[out=-90,in=-92] (-.03,-.7);
		\end{tikzpicture}
	\end{minipage}%
	\\
	Ein zeitlich verändertes $ \vec{B} $-Feld erzeugt ein elektrisches Wirbelfeld analog zu:
	\begin{equation*}
	\vabla \times \vec{B} = \mu_0 \vec{j}
	\end{equation*}
	\begin{center}
		%t7:
		\begin{tikzpicture}
		\draw[very thick, ->] (0,-1) -- (0,1) node[left] {$ \vec{j} $};
		\draw[->] ({.5 * sin(45)},-{.25 * cos(45)}) arc[start angle=-45,end angle=315, x radius=.5, y radius=.25];
		\draw[->] ({1 * sin(45)},-{.5 * cos(45)}) arc[start angle=-45, end angle=315, x radius=1, y radius=.5];
		\node at (.7,-.7) {$ \vec{B} $};
		
		\coordinate (a) at (3,0);
		\draw[very thick, ->] ($ (a) + (-.175,-1) $) -- ++(0,2);
		\draw[very thick, ->] ($ (a) + (.175,-1) $) -- ++(0,2);
		\draw[->] ($ (a) + ({.5 * sin(45)},-{.25 * cos(45)}) $) arc[start angle=-45,end angle=315, x radius=.5, y radius=.25];
		\draw[->] ($ (a) + ({1 * sin(45)},-{.5 * cos(45)}) $) arc[start angle=-45, end angle=315, x radius=1, y radius=.5];
		\node at ($ (a) + (1,-1) $) {$ \prt{\vec{B}}{t} $};
		\end{tikzpicture}
	\end{center}
\end{enumerate}

\subsection{Maxwellscher Verschiebungsstrom}

Magnetostatik:
\begin{equation*}
\vabla \times \vec{H} = \vec{j} \qquad \tx{Amp\`ere}
\end{equation*}
Dies kann in der Elektrodynamik nicht gelten, dies zeigen wir indem wir auf beiden Seiten die Divergenz bilden.
\begin{equation*}
\ub{\vabla \cdot (\vabla \times \vec{H})}_{=0} = \vabla \cdot \vec{j} \quad \Rightarrow \quad \vabla \cdot \vec{j} = 0 \quad \tx{bei stationären Strömen}
\end{equation*}
im Allgemeinen haben wir zeitlich abhängige Ströme:
\begin{equation*}
\vabla \cdot \vec{j}(\vec{r},t) \overset{\tx{i.A.}}{\neq} 0
\end{equation*}
Kontinuitätsgleichung:
\begin{equation*}
\prt{\rho}{t} + \vabla \cdot \vec{j} = 0
\end{equation*}
Ergänzung:
\begin{equation*}
\vabla \times \vec{H} = \vec{j} + \vec{j}_0
\end{equation*}
\begin{equation*}
\Rightarrow \quad \ub{\vabla \cdot (\vabla \times \vec{H})}_{= 0} = \ub{\vabla \cdot \vec{j}}_{= - \prt{\rho}{t}} + \vabla \cdot \vec{j}_0
\end{equation*}
\begin{equation*}
\Rightarrow \quad \vabla \cdot \vec{j}_0 = \prt{\rho}{t} = \prt{}{t} \vabla \cdot \vec{D} = \vabla \cdot \left(\prt{\vec{D}}{t}\right)
\end{equation*}
\begin{equation*}
\rightarrow \quad \vec{j}_0 = \prt{\vec{D}}{t}
\end{equation*}
Damit erhalten wir für die die Stromdichte:
\begin{equation*}
\vabla \times \vec{H} - \prt{\vec{D}}{t} = \vec{j}
\end{equation*}
\emph{Beispiel:} \textbf{Plattenkondensator}\\
\begin{minipage}{.6\linewidth}
	\begin{equation*}
	\vec{D} = \casess{\sigma \vec{e}_x}{\tx{innen}}{0}{\tx{außen}}
	\end{equation*}
	$ \sigma = \frac{q}{F} $
	\begin{equation*}
	\prt{\vec{D}}{t} = \frac{\vec{e}_x}{F} \equalto{\prd{q}{t}}{I} = \frac{I}{F} \vec{e}_x
	\end{equation*}
	\begin{equation*}
	\vabla \times \vec{H} - \prt{\vec{D}}{t} = \vec{j}
	\end{equation*}
	innen: $ \vec{j} = 0 $
	\begin{equation*}
	\vabla \times \vec{H} = \prt{\vec{D}}{t} =  \frac{I}{F} \vec{e}_x
	\end{equation*}
\end{minipage}%
\begin{minipage}{.4\linewidth}
	\flushright
	%t8:
	\begin{tikzpicture}[scale=1]
		\draw[very thick] (-1.5,-2) -- (-1.5,2) node[above] {$ q $};
		\draw[very thick] (1.5,-2) -- (1.5,2) node[above] {$ - q $};
		
		\foreach \y in {-1.8,-.6,.6,1.8}
		\draw[->] (-1.3,\y) -- (1.3,\y);
		
		\foreach \y in {-1.8,-.6,.6,1.8}
		\draw[fill = white!20!red] (-1.8,\y) circle (.2cm) node[] {$ \color{white} + \color{white} $};
		
		\foreach \y in {-1.8,-.6,.6,1.8}
		\draw[fill = white!20!blue] (1.8,\y) circle (.2cm) node[] {$ \color{white} - \color{white} $};
		
		\node at (0,1.2) {$ \vec{D} $};
		\node at (0,-1.2) {$ \vec{H} $};
		
		\draw[->] (0,-3) -- ++(1.5,0) node[right] {$ x $};
		\draw (0,-3.1) -- (0,-2.9);
		
		\draw[very thick] (-1.5,-2) -- (-1.5,-2.5);
		\draw[very thick] (1.5,-2) -- (1.5,-2.5);
		\draw[->] (-1.5,-2.5) -| (-3,-4) -- (0,-4) node[below] {\Large{$ I $}};
		\draw (1.5,-2.5) -| (3,-4) -- (0,-4);
		\draw[thick,->] ($ (-2.5,-2.5) + ({sin(45) * .3},{cos(45) * -.8}) $) arc[start angle=-45, end angle=315, x radius=.3, y radius=.8];
		\draw[thick,->] ($ (2.5,-2.5) + ({sin(45) * .3},{cos(45) * -.8}) $) arc[start angle=-45, end angle=315, x radius=.3, y radius=.8];
	\end{tikzpicture}
\end{minipage}%

\subsection{Lösung der Differentialgleichungen} %check this

\begin{align*}
\vabla \cdot \vec{D} &= \rho \qquad \, \vabla \times \vec{E} + \prt{\vec{B}}{t} = 0 \\
\vabla \cdot \vec{B} &= 0 \qquad \vabla \times \vec{H} - \prt{\vec{D}}{t} = \vec{j}
\end{align*}
Zwei homogene und zwei inhomogene Differentialgleichungen. Zur Lösung benötigt man also zusätzliche Materialgleichungen:
\begin{equation*}
\vec{B} = \mu_0 (\vec{H} + \vec{M}) \qquad \vec{D} = \epsilon_0\vec{E} + \vec{P}
\end{equation*}
bei linearen Medien:
\begin{equation*}
\vec{B} = \mu \vec{H} \qquad \vec{D} = \epsilon \vec{E}
\end{equation*}
Die Kraft ist:
\begin{equation*}
\vec{F} = q (\vec{E} + \vec{v} \times \vec{B})
\end{equation*}
\textbf{Intagrale Form der Maxwell-Gleichungen}
\begin{align*}
\oint_{\partial V} \dd \vec{f} \cdot \vec{B} &= 0  \qquad \qquad \qquad \qquad \: \oint_{\partial F} \dd \vec{r} \cdot \vec{E} + \prd{}{t} \int_F \dd \vec{f} \cdot \vec{B} = 0 \\
\oint_{\partial V} \dd \vec{f} \cdot \vec{D} &= \int_V \dd^3 r \rho(\vec{r}) = Q \quad \ \ \oint_{\partial F} \dd \vec{r} \cdot \vec{H} - \prd{}{t} \int_F \dd \vec{f} \cdot \vec{D} = \int_F \dd \vec{f} \cdot \vec{j} = I
\end{align*}
\textbf{Mikroskopische Maxwell-Gleichungen:}\\
formal:
\begin{equation*}
\vec{D} = \epsilon_0 \vec{E} \qquad \vec{H} = \frac{1}{\mu_0} \vec{B}
\end{equation*}
\begin{align*}
\vabla \cdot \vec{B} &= 0 \qquad \quad \vabla \times \vec{E} + \prt{\vec{B}}{t} = 0\\
\vabla \cdot \vec{E} &= \frac{1}{\epsilon_0} \rho \qquad \vabla \times \vec{B} - \epsilon_0 \mu_0 \prt{\vec{E}}{t} = \mu_0 \vec{j}
\end{align*}

\section{Potentiale der Elektrodynamik - Eichtransformation}

\begin{equation*}
\vabla \cdot \vec{B} = 0 \qquad \rmbox{\vec{B}(\vec{r},t) = \vabla \times \vec{A}(\vec{r},t)}
\end{equation*}
\begin{align*}
\vabla \times \vec{E} + \prt{\vec{B}}{t} = 0 \qquad 0 &= \vabla \times \vec{E} + \prt{}{t} \vabla \times \vec{A} \\
&= \vabla \times \left(\vec{E} + \prt{\vec{A}}{t}\right)
\end{align*}
\begin{equation*}
\Rightarrow \quad \vec{E} + \prt{\vec{A}}{t} = - \vabla \Phi
\end{equation*}
$ \Phi $ ist ein skalares Potential
\begin{equation*}
\Rightarrow \quad \rmbox{\vec{E}(\vec{r},t) = - \vabla \Phi(\vec{r},t) - \prt{\vec{A}(\vec{r},t)}{t}}
\end{equation*}
Damit haben wir die beiden homogenen Gleichungen gelöst. Wir können $ \vec{E} $- und $ \vec{B} $-Felder nun durch die Potentiale $ \Phi $ und $ \vec{A} $ ausdrücken.

\subsection{Bestimmungsgleichung für \texorpdfstring{$ \Phi , \vec{A} $}{phi A}}

\begin{equation*}
\vec{D} = \epsilon \vec{E} \qquad \vec{H} = \frac{1}{\mu} \vec{B}
\end{equation*}
\begin{align*}
\rightarrow \quad \vabla \cdot \vec{E} = \frac{1}{\epsilon} \rho
\end{align*}
\begin{equation*}
\frac{1}{\epsilon} \rho = \vabla \cdot \left(- \vabla \Phi - \prt{\vec{A}}{t}\right) = - \Delta \Phi - \prt{}{t} \vabla \cdot \vec{A}
\end{equation*}
\begin{equation*}
\rmbox{\Rightarrow \quad \Delta \Phi + \prt{}{t} \vabla \cdot \vec{A} = - \frac{1}{\epsilon} \rho}
\end{equation*}
\begin{align*}
& \vabla \times \vec{B} - \epsilon \mu \prt{\vec{E}}{t} = \mu \vec{j}
\end{align*}
\begin{equation*}
\mu \vec{j} = \ub{\vabla \times (\vabla \times \vec{A})}_{=\vabla \cdot (\vabla \cdot \vec{A}) - \Delta \vec{A}} - \epsilon \mu \prt{}{t} \left(- \vabla \Phi - \prt{\vec{A}}{t}\right)
\end{equation*}
\begin{equation*}
\rmbox{\Rightarrow \quad \Delta \vec{A} - \epsilon \mu \prt{^2 \vec{A}}{t^2} - \vabla \cdot \left(\vabla \cdot \vec{A} + \epsilon \mu \prt{\Phi}{t}\right) = - \mu \vec{j}}
\end{equation*}

\subsection{Eichtransformation}

$ \Phi, \vec{A} $ sind durch:
\begin{equation*}
\vec{B} = \vabla \times \vec{A} \qquad \vec{E} = - \vabla \Phi - \prt{\vec{A}}{t}
\end{equation*}
nicht eindeutig festgelegt.\\[5pt]
\textbf{Eichtransformation:}
\begin{align*}
\vec{A}'(\vec{r},t) &= \vec{A}(\vec{r},t) + \vabla \Lambda(\vec{r},t)\\
\Phi'(\vec{r},t) &= \Phi(\vec{r},t) - \prt{\Lambda(\vec{r},t)}{t}
\end{align*}
\begin{equation*}
\vabla \times \vec{A}' = \vabla \times \vec{A} + \ub{ \vabla \times (\vabla \Lambda)}_{=0} = \vabla \times \vec{A} = \vec{B} \quad \checkmark
\end{equation*}
\begin{equation*}
- \vabla \Phi' - \prt{\vec{A}'}{t} = - \vabla \Phi + \cancel{\vabla \prt{\Lambda}{t}} - \prt{\vec{A}}{t} - \cancel{\prt{}{t} \vabla \Lambda} = \vec{E} \quad \checkmark 
\end{equation*}
$ \rightarrow $ verschiedene Eichungen.
\begin{enumerate}[1)]
	\item \textbf{Coulomb-Eichung}
	\begin{equation*}
	\vabla \cdot \vec{A}(\vec{r},t) = 0
	\end{equation*}
	\begin{equation*}
	\rmbox{\rightarrow \quad \Delta \Phi(\vec{r},t) = - \frac{1}{\epsilon} \rho(\vec{r},t)}
	\end{equation*}
	\begin{equation*}
	\Phi(\vec{r},t) = \frac{1}{4 \pi \epsilon} \int \dd ^3 r' \frac{\rho(\vec{r}',t)}{|\vec{r} - \vec{r}'|}
	\end{equation*}
	\begin{equation*}
	\rmbox{ \rightarrow \quad \Delta \vec{A} - \epsilon \mu \prt{^2 \vec{A}}{t^2} = \mu \vec{j} + \epsilon \mu \vabla \prt{\Phi}{t}}
	\end{equation*}
	Typische Anwendung:\\
	Bei statischen Problemen und bei nicht relativistischen Geschwindigkeiten.\\
	Die Coulomb-Eichung ist nicht Lorenz-invariant.
	\item \textbf{Lorenz-Eichung}
	\begin{equation*}
	\vabla \cdot \vec{A} + \epsilon \mu \prt{\Phi}{t} = 0
	\end{equation*}
	Im Vakuum: $ \epsilon \mu = \epsilon_0 \mu_0 = \frac{1}{c^2} $
	\begin{equation*}
	\rightarrow \quad \Delta \Phi + \prt{}{t} \left(- \epsilon \mu \prt{\Phi}{t}\right) = 0
	\end{equation*}
	\begin{equation*}
	\rmbox{\rightarrow \quad \Delta \Phi - \epsilon \mu \prt{^2 \Phi}{t^2} = 0 \ / \ - \frac{1}{\epsilon} \rho}
	\end{equation*}
	\begin{equation*}
	\rmbox{\rightarrow \quad \Delta \vec{A} - \epsilon \mu \prt{^2 \vec{A}}{t^2} = 0 \ / \ \mu_0 \vec{j} \ \ }
	\end{equation*}
\end{enumerate}

% Vorlesung 17.12.18 (7 Days until Christmas)

\section{Energie und Impuls elektromagnetischer Felder}

\subsection{Energie des EM-Feldes}

\begin{minipage}{.7\linewidth}
	System von Punktladungen $ q_i, m_i, \vec{r}_i, \vec{v}_i $ im Volumen $ V $:\\[5pt]
	Kraft auf Ladungen im elektromagnetischen Feld:
	\begin{equation*}
	\vec{F}_i = q_i \left(\vec{E}(\vec{r}_i, t) + \vec{v}_i \times \vec{B}(\vec{r}_i, t) \right)
	\end{equation*}
\end{minipage}%
\begin{minipage}{.3\linewidth}
	%t1:
	\centering
	\begin{tikzpicture}
		\draw[fill=red!12!white,rounded corners=2pt] (0,0) rectangle (3,3);
		\node[fill=black,circle,inner sep=1pt, minimum size=1pt] (a) at (1.5,2) {};
		\draw[->] (0.02,0.02) -- node[right] {$ \vec{r}_i $} (a);
		\draw[->] (a) -- node[below=3pt] {$ \vec{v}_i $} ++(-45:1cm);
		\node[anchor=south east] at (a) {$q_i, m_i$};
		\node[anchor=north east] at (3,3) {$ V $};
	\end{tikzpicture}
	\vspace{5pt}
\end{minipage}%
\\
Zeitliche Änderung der Energie der Punktladungen durch die elektromagnetischen Felder:
\begin{align*}
\prd{}{t} W_{\tx{mat}} &= \prd{}{t} \sum_i \frac{m_i}{2} \vec{v}_i^2 = \sum_i \vec{v}_i \cdot m_i \cdot \prd{\vec{v}_i}{t} = \sum_i \vec{v}_i \cdot \vec{F}_i\\
&= \sum_i q_i \vec{v}_i \cdot \vec{E}(\vec{r}_i, t) + \sum_i q_i \ub{ \vec{v}_i \cdot (\vec{v}_i) \times \vec{B}(\vec{r}_i, t) }_{=0}\\
&= \sum_i q_i \vec{v}_i \cdot \vec{E}(\vec{r}_i, t)
\end{align*}
formale Kontinuierliche Beschreibung: $ \vec{j}(\vec{r},t) = \sum_i q_i \dot{\vec{r}}_i \delta(\vec{r} - \vec{r}_i(t)) $
\begin{equation*}
\rightarrow \quad \prd{}{t} W_{\tx{mat}} = \int_V \dd^3 r \ \vec{j}(\vec{r},t) \cdot \vec{E}(\vec{r},t)
\end{equation*}
Energiedichte $ w_{\tx{mat}} $: $ W_{\tx{mat}} = \int_V \dd^3 r \ w_{\tx{mat}} (\vec{r}) $
\begin{equation*}
\Rightarrow \quad \int_V \dd^3 r \ \vec{j} \cdot \vec{E} = \prd{}{t} \int_V \dd^3 r \ w_{\tx{mat}} = \int_V \dd^3 r \ \prt{}{t} w_{\tx{mat}}
\end{equation*}
\begin{equation*}
\Rightarrow \quad \int_V \dd^3 r \ \left(\prt{w_{\tx{mat}}}{t} - \vec{j} \cdot \vec{E} \right) = 0 \qquad \forall \ V
\end{equation*}
\begin{equation*}
\Rightarrow \quad \prt{w_{\tx{mat}}}{t} = \vec{j} \cdot \vec{E} \quad \widehat{=} \quad \tx{Leistungsdichte}
\end{equation*}
\begin{equation*}
[\vec{j}] [\vec{E}] = \frac{\tx{A}}{\tx{m}^2} \frac{\tx{V}}{\tx{m}} = \frac{\tx{W}}{\tx{m}^2}
\end{equation*}
Die gesamte Änderung der Energie der Materie im Volumen $ V $ ist dann:
\begin{equation*}
\prd{}{t} W_{\tx{mat}_V} = \int_V \dd^3 r \ \vec{j} \cdot \vec{E}
\end{equation*}
Außerdem gilt auch:
\begin{equation*}
\prd{}{t} W_{\tx{mat}} = - \prd{}{t} W_{\tx{feld}}
\end{equation*}
Wir wollen nun $ \vec{j} $ ersetzen und nutzen:
\begin{equation*}
\vec{j} = \vabla \times \vec{H} - \prt{\vec{D}}{t} \quad \Rightarrow \quad \vec{j} \cdot \vec{E} = \vec{E} \cdot (\vabla \times \vec{H}) - \vec{E} \cdot \prt{\vec{D}}{t}
\end{equation*}
$ \vec{E} $ mal der Rotation von $ \vec{H} $ können wir umformen als:
\begin{equation*}
\vabla \cdot (\vec{E} \times \vec{H}) = \vec{H} \cdot \ub{(\vabla \times \vec{E})}_{- \prt{\vec{B}}{t}} - \vec{E} \cdot (\vabla \times \vec{H})
\end{equation*}
\begin{equation*}
\rightarrow \quad  \vec{j} \cdot \vec{E} = - \vec{H} \cdot \prt{\vec{B}}{t} - \vec{E} \cdot \prt{\vec{D}}{t} - \vabla \cdot (\vec{E} \times \vec{H}) = \prt{}{t} w_{\tx{mat}}
\end{equation*}
\begin{equation*}
\Rightarrow \quad \prd{}{t} W_{\tx{mat}_V} = - \int_V \dd^3 r \ \left(\vec{H} \cdot \prt{\vec{B}}{t} + \vec{E} \cdot \prt{\vec{D}}{t} \right) - \int_V \dd^3 r \ \vabla \cdot (\vec{E} \times \vec{H}) = - \prd{}{t} W_{\tx{feld}}
\end{equation*}
Zur Vereinfachung und Interpretation: $ \vec{H} = \frac{1}{\mu} \vec{B} \quad \vec{D} = \epsilon \vec{E} \quad \epsilon, \mu \const $
\begin{equation*}
\vec{H} \cdot \prt{\vec{B}}{t} = \frac{1}{\mu} \vec{B} \cdot \prt{\vec{B}}{t} = \frac{1}{2} \frac{1}{\mu} \prt{}{t} (\vec{B} \cdot \vec{B}) = \frac{1}{2} \prt{}{t} (\vec{H} \cdot \vec{B})
\end{equation*}
\begin{equation*}
\vec{E} \cdot \prt{\vec{D}}{t} = \epsilon \vec{E} \cdot \prt{\vec{E}}{t} = \frac{\epsilon}{2} \prt{}{t} (\vec{E} \cdot \vec{E}) = \frac{1}{2} \prt{}{t} (\vec{E} \cdot \vec{D})
\end{equation*}
\begin{equation*}
\prd{}{t} W_{\tx{mat}_V} = - \prd{}{t} \frac{1}{2} \int_V \dd^3r \ (\vec{H} \cdot \vec{B} + \vec{E} \cdot \vec{D}) - \int_V \dd^3r \ \vabla \cdot (\vec{E} \times \vec{H})
\end{equation*}
\textbf{Elektrostatik:}
\begin{equation*}
w_{\tx{el}} = \frac{\epsilon_0}{2} \vec{E}^2 \quad (\tx{Vakuum}) \qquad w_{\tx{el}} = \frac{1}{2} \vec{E} \cdot \vec{D} \quad (\tx{Medium})
\end{equation*}
\textbf{Magnetostatik:}
\begin{equation*}
w_{\tx{mag}} = \frac{1}{2 \mu_0} \vec{B}^2 \quad (\tx{Vakuum}) \qquad w_{\tx{mag}} = \frac{1}{2} \vec{H} \cdot \vec{B} \quad (\tx{Medium})
\end{equation*}
Energie des elektromagnetischen Feldes:
\begin{equation*}
\frac{1}{2} \int_V \dd^3 r \ (\vec{H} \cdot \vec{B} + \vec{E} \cdot \vec{D}) = W_{\tx{em}_V}
\end{equation*}
Energiedichte:
\begin{equation*}
w_{\tx{em}} = \frac{1}{2} (\vec{H} \cdot \vec{B} + \vec{E} \cdot \vec{D})
\end{equation*}
\begin{equation*}
\prd{}{t} W_{\tx{mat}_V} = - \prd{}{t} W_{\tx{em}_V} - \ub{\int_V \dd^3 r \ \vabla \cdot (\vec{E} \times \vec{H})}_{\mathclap{\substack{\tx{Energie, die aus } V \\ \tx{abfließt (Strahlung)}}}}
\end{equation*}
\begin{equation*}
\vec{S}(\vec{r},t) = \vec{E} \times \vec{H} \qquad \tx{\textbf{Pointing-Vektor}}
\end{equation*}
\begin{equation*}
\int_V \dd^3 r \ \vabla \vec{S} = \int_{\partial V} \dd\vec{f}  \cdot \vec{S}
\end{equation*}
\begin{equation*}
[\vec{S}] = [\vec{E}] [\vec{H}] = \frac{\tx{V}}{\tx{m}} \frac{\tx{A}}{\tx{m}} = \frac{\tx{W}}{\tx{m}^2} = \frac{\tx{J}}{\tx{m}^2 \tx{s}} = \frac{\tx{Energie}}{\tx{Fläche } \cdot \tx{ Zeit}}
\end{equation*}
$ \vec{S}: $ \textbf{Energiestromdichte}\\
\begin{minipage}{.8\linewidth}
	% Hier fehlt ein Text von Kim:
	$$ \int_{\partial V} \dd \vec{f} \cdot \vec{S} $$
	Energiestrom des EM-Feldes durch die Fläche $ \partial V $.
\end{minipage}%
\begin{minipage}{.2\linewidth}
	\flushright
	%t2:
	\begin{tikzpicture}
	\draw[shade, ball color=gray!20!white] (0,0) circle (1cm);
	\node at (.3,.3) {$ \partial V $};
	\end{tikzpicture}
\end{minipage}%

\subsubsection{Energiebilanz}

\begin{equation*}
\ub{\prd{}{t} W_{\tx{mat} _V}}_{\mathclap{\substack{\tx{an Ladungen} \\ \tx{in } V \tx{ verrichtete} \\ \tx{Arbeit}}}} \ \ + \ \ \ub{\prd{}{t} W_{\tx{em}_V}}_{\mathclap{\substack{\tx{Änderung der} \\ \tx{EM Feldenergie} \\ \tx{in } V}}} \ = \ \ub{- \int_{\partial V} \dd \vec{f} \cdot \vec{S}}_{\mathclap{\substack{\tx{Fluss der EM} \\ \tx{Feldenergie aus } V}}}
\end{equation*}
%
%
%

\pagebreak

%
%
%
\noindent
\emph{Bemerkungen:}
\begin{enumerate}[i)]
	\item abgeschlossenes System (ohne Energiestromdichte ach außen) z.B. $ \mathbb{R}^3 $
	\begin{equation*}
	\int_{\partial V} \dd \vec{f} \cdot \vec{S} \rightarrow 0
	\end{equation*}
	\item Gebiet ohne Ladungen und Ströme $ \rho = \vec{j} = 0 \ \ \tx{in} V $. $ w_{\tx{mat}} = \vec{j} \cdot \vec{E} = 0 $
	\begin{equation*}
	\prd{}{t} W_{\tx{mat}_V} = - \int_{\partial V} \dd \vec{f} \cdot \vec{S}
	\end{equation*}
	\item differentielle Form:
	\begin{equation*}
	\int_V \dd^3 r \ \left[\prt{w_{\tx{em}}}{t} + \vabla \cdot \vec{S} + \vec{j} \cdot \vec{E}\right] = 0
	\end{equation*}
	\frbox{Poyntigsches Theorem}{
	\begin{equation*}
	\rightarrow \quad \prt{w_{\tx{em}}}{t} + \prt{w_{\tx{mat}}}{t} + \vabla \cdot \vec{S} = 0
	\end{equation*}}
	oder auch: \textbf{Kontinuitätsgleichung für Energie}
	\begin{equation*}
	\prt{}{t} w + \vabla \cdot \vec{S} = 0
	\end{equation*}
	Analogie:
	\begin{equation*}
	\prt{\rho}{t} + \vabla \cdot \vec{j} = 0
	\end{equation*}
	\item \textbf{Poynting-Vektor}
	\begin{equation*}
	\vec{S} = \vec{E} \times \vec{H} \quad \tx{Energiestromdichte}
	\end{equation*}
	\textbf{Beachte:} Nur $ \vabla \cdot \vec{S} $ oder $ \int_{\partial V} \dd \vec{f} \cdot \vec{S} $ haben physikalische Bedeutung.\\[5pt]
	$ \vec{S} \to \vec{S} + \vabla \times \vec{G} $\\
	$ \vec{S} \neq 0 $, aber kein Energiestrom:
	\begin{equation*}
	\vec{E} = E \vec{e}_x \quad \vec{H} = H \vec{e}_y \qquad E,H \const
	\end{equation*}
	\begin{equation*}
	\vec{S} = \vec{E} \times \vec{H} = E H \vec{e}_z \neq 0
	\end{equation*}
	aber: $ \vabla \cdot \vec{S} = 0  \quad \rightarrow $ \textbf{kein Beitrag !}
\end{enumerate}
\emph{Beispiel:} \textbf{Energietransport in el./mag. Wellen im Vakuum}
\begin{center}
	%t3:
	\begin{tikzpicture}%[>=stealth]
		% E
		\draw[red,thick,looseness=5,fill=red!20!white,opacity=.8] (0,0) to[out=80,in=100] (1.5,0) to[out=-80,in=-100] (3,0);
		\foreach \z in {.5,1}
		\draw[thick,red,->,>=stealth] (\z,0) -- ++(0,1.9);
		\foreach \z in {2,2.5}
		\draw[thick,red,->,>=stealth] (\z,0) -- ++(0,-1.9);
		% B
		\draw[blue,thick,looseness=4.5,fill=blue!20!white,opacity=.8] (0,0) to[out=-125,in=-145] (1.5,0) to[out=35,in=55] (3,0) to[out=-125,in=-145] (4.5,0) to[out=35,in=55] (6,0);
		\foreach \z in {.5,1,3.5,4}
		\draw[thick,blue,->,>=stealth] (\z,0) -- ++(-136:1.7cm);
		\foreach \z in {2,2.5,5,5.5}
		\draw[thick,blue,->,>=stealth] (\z,0) -- ++(43:1.7cm);
		% rest E
		\draw[red,thick,looseness=5,fill=red!20!white,opacity=.8] (3,0) to[out=80,in=100] (4.5,0) to[out=-80,in=-100] (6,0);
		\foreach \z in {3.5,4}
		\draw[thick,red,->,>=stealth] (\z,0) -- ++(0,1.9);
		\foreach \z in {5,5.5}
		\draw[thick,red,->,>=stealth] (\z,0) -- ++(0,-1.9);
		% koordinatensystem
		\draw[->] (0,0) -- (0,2) node[anchor=south east] {$ \vec{y} $};
		\draw[->] (0,0) -- (-1.2,-1.2) node[anchor=north east] {$ \vec{x} $};
		\draw[->] (0,0) -- (7,0) node[anchor=north west] {$ \vec{z} $};
		\node (e) at (1.5,2) {\color{red} $ \vec{E} $ \color{black}};
		\node (b) at (.5,-1.25) {\color{blue} $ \vec{B} $ \color{black}};
	\end{tikzpicture}
\end{center}
ebene Welle:
\begin{align*}
\vec{E} &= E_0 \cos(kz - \omega t) \vec{e}_x \\
\vec{B} &= \frac{E_0}{c} \cos(kz - \omega t) \vec{e}_y
\end{align*}
\begin{equation*}
\frac{\omega}{k} = c \qquad \lambda = \frac{2 \pi}{k}
\end{equation*}
mit $ c^2 = \frac{1}{\epsilon_0 \mu_0} $\\
\begin{minipage}{.7\linewidth}
	\begin{align*}
	\vec{S} &= \vec{E} \times \vec{H} = \frac{1}{\mu_0} \vec{E} \times \vec{B} = \frac{1}{\mu_0} \frac{E_0^2}{c} \cos^2(kz - \omega t) \vec{e}_z\\
	&= c \epsilon_0 E_0^2 \cos^2(kz - \omega t) \vec{e}_z
	\end{align*}
\end{minipage}%
\begin{minipage}{.3\linewidth}
	\flushright
	%t4:
	\begin{tikzpicture}
		\draw[thick,->] (0,0) -- (0,1.5) node[anchor=south east] {$ \vec{E} $};
		\draw[thick,->] (0,0) -- (1.5,0) node[anchor=north west] {$ \vec{S} $};
		\draw[thick,->] (0,0) -- (-.75,-.75) node[anchor=north east] {$ \vec{B} $};
	\end{tikzpicture}
\end{minipage}%
\\
\textbf{explizite Energiebilanz:}
\begin{equation*}
\prt{}{t} (\equalto{w_{\tx{mat}}}{0} + w_{\tx{em}}) = - \vabla \cdot \vec{S}
\end{equation*}
\begin{align*}
w_{\tx{em}} &= \frac{1}{2} (\vec{E} \cdot \vec{D} + \vec{H} \cdot \vec{B}) = \frac{1}{2} (\epsilon_0 \vec{E}^2 + \frac{1}{\mu_0} \vec{B}^2) \\
&= \frac{\epsilon_0}{2} (\vec{E}^2 + c^2 \vec{B}^2) = \epsilon_0 E_0^2 \cos^2(kz - \omega t)
\end{align*}
\begin{equation*}
\prt{}{t} w_{\tx{em}} = 2 \omega \epsilon_0 E_0^2 \cos(\dots) \sin(\dots)
\end{equation*}
\begin{equation*}
\vabla \cdot \vec{S} = \prt{S}{z} = - 2 \ub{kc}_{= \omega} \epsilon_0 E_0^2 \cos(\dots) \sin(\dots)
\end{equation*}

\subsection{Impuls des EM-Feldes - Maxwellscher Spannungstensor}

Wir betrachten ein System von Punktladungen: $ q_i, m_i, \vec{r}_i, \vec{v}_i $
Impuls: $ \vec{p_i} = m_i \vec{v}_i $
\begin{equation*}
\vec{D} = \epsilon \vec{E} \qquad \vec{B} = \mu \vec{H}
\end{equation*}
Die zeitliche Änderung des gesamten Impulses der Teilchen durch EM-Felder:
\begin{align*}
\prd{}{t} \vec{P}_{\tx{mat}} &= \prd{}{t} \sum_i \vec{p_i} = \sum_i \prd{}{t} \vec{p}_i\\
&= \sum_i \vec{F}_i = \sum_i q_i \left(\vec{E}(\vec{r}_i, t) + \vec{v}_i \times \vec{B}(\vec{r}_i, t)\right)
\end{align*}
$ \rightarrow $ Kontinuierliche Beschreibung:
\begin{align*}
\rho(\vec{r},t) &= \sum_i q_i \delta(\vec{r} - \vec{r}_i(t))\\
\vec{j}(\vec{r},t) &= \sum_i q_i \vec{v}_i \delta(\vec{r} - \vec{r}_i(t))
\end{align*}
\begin{equation*}
\rightarrow \quad \prd{}{t} \vec{P}_{\tx{mat}} = \int_V \dd^3 r \ \ub{\left(\rho \vec{E}(\vec{r},t) + \vec{j} \times \vec{B}(\vec{r},t)\right)}_{\tx{Kraftdichte}}
\end{equation*}
$ \rho = \vabla \cdot \vec{D} \quad \vec{j} = \vabla \times \vec{H} - \prt{\vec{D}}{t} $
\begin{equation*}
\Rightarrow \quad \rho \vec{E} + \vec{j} \times \vec{B} = - \prt{}{t} (\vec{D} \times \vec{B}) + \sum_{i,k=1}^{3} \prt{T_{ik}}{x_k} \vec{e}_i
\end{equation*}
\textbf{Maxwellscher Spannungstensor:}\\[5pt]
Symmetrischer Tensor 2. Stufe:
\begin{equation*}
T_{ik} = \epsilon E_i E_k - \frac{1}{\mu} B_i B_k - \frac{1}{2} \delta_{ik}(\epsilon \vec{E}^2 - \frac{1}{\mu} \vec{B}^2)
\end{equation*}
physikalisch: \textbf{Impulsstromdichte}

\subsubsection{Impulsbilanz}

\begin{equation*}
\prd{}{t} \vec{P}_{\tx{mat}_V} + \prd{}{t} \int_V \dd^3 r \ (\vec{D} \times \vec{B}) = \sum_{i=1}^{3} \ub{\Bigg(\int_{V} \dd^3 r\ \ob{\sum_{k=1}^{3} \prt{T_{ik}}{x_k}}^{= \vabla \cdot \vec{T}_i} \vec{e}_i \Bigg)}_{= \int_{\partial V} \dd \vec{f} \cdot \vec{T}_i}
\end{equation*}
Die $ i $-te Zeile dieser $ 3 \times 3 $-Matrix wäre als Vektor: $ \vec{T}_i = (T_{i1}, T_{i2}, T_{i3}) $
\begin{equation*}
\ub{\prd{}{t} \vec{P}_{\tx{mat}_V}}_{\mathclap{\substack{\tx{Änderung des} \\ \tx{Impulses} \\ \tx{der Teilchen}}}} + \ub{\prd{}{t} \int_V \dd^3 r (\vec{D} \times \vec{B})}_{\mathclap{\substack{\tx{Änderug des} \\ \tx{Impulses des} \\ \tx{EM-Feldes} \\ \tx{in } V}}} = \ub{\sum_{i = 1}^{3} \int_{\partial V} \dd \vec{f} \cdot \vec{T}_i}_{\mathclap{\substack{\tx{Impulsfluss} \\ \tx{des EM-Feldes} \\ \tx{aus } V}}}
\end{equation*}

%Vorlesung 20.12.18 (4 Days until christmas)

\begin{comment}
\begin{align*}
\prd{}{t} P_{\tx{mat}_V} &= \sum_i \vec{F}_i = \sum_i q_i (\vec{E}(\vec{r},t) + \vec{v}_i \times \vec{B}(\vec{r}_i,t)\\
&= \int_V \dd^3r (\rho \vec{E} + \vec{j} \times \vec{B})
\end{align*}
\begin{equation*}
\Rightarrow \prd{}{t} P_{\tx{mat}_V} + \prd{}{t} \int_V\dd^3 r(\vec{D} \times \vec{B}) = \sum_{i=1}^{3} \ub{\int_V \dd^2 r \sum_{k=1}^{3} \prt{T_{ik}}{x_k} \vec{e}_i}_{= \int_{\partial V} \dd\vec{f} \cdot \vec{T}_i \vec{e}_i}
\end{equation*}
\end{comment}

\noindent
differentiell:
\begin{equation*}
\rho \vec{E} + \vec{j} \times \vec{B} + \prt{}{t} (\vec{D} \times \vec{B}) = \sum_i (\vabla \cdot \vec{T}_i) \vec{e}_i
\end{equation*}
\begin{enumerate}[i)]
	\item \textbf{Impuls des EM-Feldes}
	\begin{equation*}
	P_{\tx{em}_V} = \int_V \dd^3 r \ub{(\vec{D} \times \vec{B})}_{=\vec{g}}
	\end{equation*}
	$ \vec{g} \defeq $ Impulsdichte\\
	$ \vec{D} = \epsilon \vec{E} \qquad \vec{B} = \mu \vec{H} $
	\begin{equation*}
	\rightarrow \quad \vec{g} = \epsilon \mu \ub{\vec{E} \times \vec{H}}_{=\vec{S}} = \epsilon_r \mu_r \ub{\epsilon_0 \mu_0}_{= \frac{1}{c^2}} \vec{S} = \epsilon_r \mu_r \frac{1}{c^2} \vec{S}
	\end{equation*}
	$ \left[\frac{1}{c^2} \vec{S}\right] = \frac{\tx{Impuls}}{\tx{Volumen}} $\\[10pt]
	Impulsdichte einer ebenen Wellen:
	\begin{align*}
	\vec{E} &= E_0 \cos\left(kz - \omega t\right) \vec{e}_x\\
	\vec{B} &= \frac{E_0}{c} \cos\left(kz - \omega t\right) \vec{e}_y\\
	\vec{S} &= \vec{E} \times \vec{H} = c \epsilon_0 E_0^2 \cos^2(kz - \omega t) \vec{e}_z\\
	\vec{g} &= \frac{\epsilon_0}{c} E_0^2 \cos^2(kz - \omega t) \vec{e}_z
	\end{align*}
	\item \textbf{Maxwellscher Spannungstensor - Impulsdichte}
	\begin{equation*}
	T_{ik} = \epsilon E_i E-k + \frac{1}{\mu} B_i B_k - \frac{1}{2} \delta_{ik} \left(\epsilon \vec{E}^2 + \frac{1}{\mu} \vec{B^2}\right)
	\end{equation*}
	\begin{equation*}
	\prd{}{t} \left(\vec{P}_{\tx{mat}_V} + \vec{P}_{\tx{em}_V}\right) = \sum_i \left(\int_V \dd\vec{f} \cdot \vec{T}_i \right)\vec{e}_i
	\end{equation*}
	Analogie: Energiebilanz
	\begin{equation*}
	\prd{}{t} \left(W_{\tx{mat}_V} + W_{\tx{em}_V}\right) = - \int_{\partial V} \dd \vec{f} \cdot \vec{S}
	\end{equation*}
	$ \Rightarrow - \vec{T}_i $ Impulsstromstärke zu $ P_i $\\
	$ \left[T_{ik}\right] = \frac{\tx{Impuls}}{\tx{Fläche } \cdot \tx{ Zeit}} $\\[10pt]
	$ T_{ik} : $ Berechnung von Kräften auf geladene materielle Körper im EM-Feld.\\
	\begin{minipage}{.6\linewidth}
		\begin{align*}
		\prd{}{t} \vec{P} &= \sum_i \int_{\partial V} \dd \vec{f} \cdot \vec{T}_i \vec{e}_i\\
		&= \sum_{i=1}^{3} \ub{\left(\prd{}{t} P_i\right)}_{\mathclap{\tx{Kraft } \vec{K} , K_i}} \vec{e}_i
		\end{align*}
		\begin{equation*}
		\Rightarrow \quad \dd K_i = (\vec{n} \cdot \vec{T}_i) \dd f
		\end{equation*}
		$ \dd \vec{f} = \vec{n} \dd f $
		\begin{equation*}
		\Rightarrow \quad \left[\frac{\dd K_i}{\dd f}\right] = \frac{\tx{Kraft}}{\tx{Fläche}} = \tx{ Druck}
		\end{equation*}
	\end{minipage}%
	\begin{minipage}{.4\linewidth}
		\flushright
		%t1:
		\begin{tikzpicture}
			\pic at (0,0) {annotated cuboid={width=7,height=20,depth=20}};
			\draw[thick,->] (.4,-.6) -- ++(1,0) node[right] {$ \dd \vec{f} $};
			% \draw (0,0) -- (.78,-1.23); % diagonale
			\draw (.45,-1) to[out=-45,in=100] (1,-2) node[below] {$ \vec{P} $};
			% waves
			\coordinate (a) at (-2.5,-.6);
			\draw[red!90!white,thick,->] (a) -- ++(60:2.5cm);
			\draw[red!90!white,thick,->] ($ (a) + (-60:.4cm) $) -- ++(60:2.5cm);
			\draw[blue!90!white,thick,->] (a) -- ++(-60:2.5cm);
			\draw[blue!90!white,thick,->] ($ (a) + (60:.4cm) $) -- ++(-60:2.5cm);
			\node[red!90!white] at ($ (a) + (55:2.8cm) $) {$ \vec{E} $};
			\node[blue!90!white] at ($ (a) + (-55:2.8cm) $) {$ \vec{B} $};
		\end{tikzpicture}
	\end{minipage}%
	\\[10pt]
\end{enumerate}
\emph{Beispiel:} \textbf{Plattenkondensator}\\
\begin{minipage}{.6\linewidth}
	\begin{equation*}
	\vec{E} = \casess{E \vec{e}_x}{\tx{innen}}{0}{\tx{außen}} \qquad \vec{B} = 0
	\end{equation*}
\end{minipage}%
\begin{minipage}{.4\linewidth}
	\flushright
	%t2:
	\begin{tikzpicture}
	\draw[very thick] (-1.3,-2) -- (-1.3,2) node[above] {$ \sigma $};
	\draw[very thick] (1.3,-2) -- (1.3,2) node[above] {$ - \sigma $};
	\draw[->] (0,-2.5) -- ++(2,0) node[right] {$ x $};
	\draw ($ (0,-2.5) + (0,.1) $) -- ($ (0,-2.5) + (0,-.1) $);
	\foreach \y in {-1.8,-.6,.6,1.8}
	\draw[->] (-1.1,\y) -- (1.1,\y);
	\foreach \y in {-1.8,-.6,.6,1.8}
	%\node at (-1.8,\y) {$ + $};
	\draw[fill = white!20!red] (-1.6,\y) circle (.2cm) node[] {$ \color{white} + \color{white} $};
	\foreach \y in {-1.8,-.6,.6,1.8}
	%\node at (1.8,\y) {$ - $};
	\draw[fill = white!20!blue] (1.6,\y) circle (.2cm) node[] {$ \color{white} - \color{white} $};
	\node[above] at (0,2) {$ \vec{E} $};
	\end{tikzpicture}
\end{minipage}%
\\
Krafttensor innen:
\begin{equation*}
T_{ik} = \ \ub{\epsilon E_i E_k}_{\mathclap{= \epsilon_0 E^2 \delta_{ix} \delta_{kx}}} \ + \ub{\frac{1}{\mu} B_i B_k}_{=0} - \frac{1}{2} \delta_{ik} \bigg(\ub{\epsilon \vec{E}^2}_{\epsilon_0 E^2} + \ub{\frac{1}{\mu} \vec{B^2}}_{= 0}\bigg)
\end{equation*}
Krafttensor außen:
\begin{equation*}
T_{ik} = 0
\end{equation*}
\vspace{5pt}
\begin{equation*}
T_{xx} = \frac{\epsilon_0}{2} E^2 \qquad T_{yy} = T_{zz} = - \frac{\epsilon_0}{2} E^2 \qquad \tx{sonst } = 0
\end{equation*}
\begin{equation*}
T_{ik} = \frac{\epsilon_0}{2} E^2 \begin{pmatrix}
1 & 0 & 0\\
0 & -1 & 0\\
0 & 0 & -1
\end{pmatrix}
\end{equation*}
Kraft in x-Richtung:
\begin{equation*}
K_x = \int_{\partial V} \dd\vec{f} \cdot \ub{\vec{T}_x}_{\mathclap{= (T_{xx},T_{xy},T_{xz}) = \frac{\epsilon_0}{2} E^2 \vec{e}_x}} = \int_{\partial V} \dd \vec{f} \cdot \frac{\epsilon_0}{2} E^2 \vec{e}_x
\end{equation*}
\begin{minipage}{.6\linewidth}
	$ \dd \vec{f} = \dd y \dd z \vec{e}_x $
	\begin{equation*}
	K_x = \int_{F_x} \dd f \frac{\epsilon_0}{2} E^2 = \frac{\epsilon_0}{2} E^2 F_x
	\end{equation*}
	$\rightarrow \quad \frac{\tx{Kraft}}{\tx{Fläche}} : \quad \frac{K_x}{F_x} = \frac{\epsilon_0}{2} E^2 = \frac{\sigma^2}{2 \epsilon_0}$
	\begin{equation*}
	\vec{K} = (K_x,0,0)
	\end{equation*}
\end{minipage}%
\begin{minipage}{.4\linewidth}
	\flushright
	%t3:
	\begin{tikzpicture}
	\foreach \x\y in {-.1/-.5, -.1/-1.35, -.1/-2.2}
	\draw[shade, ball color = red] (\x,\y) circle (.2cm) node[] {$ \color{white} + \color{white} $};
	\pic at (0,0) {annotated cuboid={width=5,height=30,depth=30}};
	\draw (0,.2) to[out=90,in=-40] (-.5,1) node[left] {$ V $};
	\foreach \x\y in {.7/.3, .7/-.55, .7/-1.4}
	\draw[shade, ball color = red] (\x,\y) circle (.2cm) node[] {$ \color{white} + \color{white} $};
	% diagonale besser aber nicht perfekt
	% \draw (0,0) -- ($ (0,-3) +({sin(45) * 1.65},{cos(45) * 1.65}) $);
	\draw[thick, ->] (.6,-1) -- ++(1,0) node[right] {$ \dd\vec{f}_x $};
	\draw (.5,-1.8) -- ++(-30:1cm) node[anchor=north west] {$ \substack{\tx{nur diese Fläche ($F_x$)} \\ \tx{trägt bei}} $};
	\end{tikzpicture}
\end{minipage}%

\section{Elektromagnetische Wellen}

\subsection{Maxwell-Gleichungen in einem Isolator - Homogene Wellengleichung}

\begin{equation*}
\rho_f = 0 = \vec{j}_f
\end{equation*}
$ \vec{D} = \epsilon \vec{E} \quad \vec{B} = \mu \vec{H} $
\begin{equation*}
\vabla \cdot \vec{E} = 0 \qquad \qquad \vabla \cdot \vec{B} = 0
\end{equation*}
\begin{equation*}
\quad \, \vabla \times \vec{E} + \prt{\vec{B}}{t} = 0 \qquad \qquad \vabla \times \vec{B} - \epsilon \mu \prt{\vec{E}}{t} = 0
\end{equation*}
\begin{equation*}
0 = \ub{\vabla \times \left(\vabla \times \vec{E}\right)}_{\vabla\ub{(\vabla \cdot \vec{E})}_{= 0} - \Delta \vec{E}} + \prt{}{t} \ub{\vabla \times \vec{B}}_{= \epsilon \mu \prt{\vec{E}}{t}} = - \Delta \vec{E} + \epsilon \mu \prt{^2 \vec{E}}{t^2}
\end{equation*}
\begin{equation*}
\Rightarrow \quad \Delta \vec{E} - \epsilon \mu \prt{^2 \vec{E}}{t^2} = 0
\end{equation*}
$ \epsilon \mu = \epsilon_r \mu_r \epsilon_0 \mu_0 = \frac{\epsilon_r \mu_r}{c^2} \qquad n = \sqrt{\epsilon_r \mu_r} $\\[5pt]
$ n $ ist der Brechungsindex des betrachteten Materials $ v = \frac{c}{n} $ die Geschwindigkeit der EM-Welle in diesem Material.
\begin{equation*}
\Rightarrow \quad \rmbox{\Delta \vec{E} - \frac{n^2}{c^2} \prt{^2 \vec{E}}{t^2} = 0}
\end{equation*}
außerdem gilt:
\begin{align*}
0 &= \ub{\vabla \times (\vabla \times \vec{B})}_{\vabla \ub{(\vabla \cdot \vec{B})}_{= 0} - \Delta \vec{B}} - \epsilon \mu \prt{}{t} \ub{\vabla \times \vec{E}}_{= - \prt{\vec{B}}{t}}
\end{align*}
\begin{equation*}
\Rightarrow \quad \rmbox{\Delta \vec{B} - \frac{n^2}{c^2} \prt{^2 \vec{B}}{t^2} = 0}
\end{equation*}
homogene Wellengleichung ($ \Psi(\vec{r},t) $)
\begin{equation*}
\Delta \Psi - \frac{n^2}{c^2} \prt{^2 \Psi}{t^2} = 0
\end{equation*}
\begin{equation*}
\Psi = E_x, E_y, E_z, B_x, B_y, B_z
\end{equation*}

% Neue Vorlesung 7.01.19

\bbb{Wiederholung}{
Maxwell-Gl für Isolator: $ \rho_f = 0 = \vec{j}_f \quad \vec{D} = \epsilon \vec{E} \quad \vec{H} = \frac{1}{\mu} \vec{B} $
\begin{equation*}
\vabla \cdot \vec{E} = 0 \qquad \qquad \vabla \cdot \vec{B} = 0
\end{equation*}
\begin{equation*}
\quad \, \vabla \times \vec{E} + \prt{\vec{B}}{t} = 0 \qquad \qquad \vabla \times \vec{B} - \epsilon \mu \prt{\vec{E}}{t} = 0
\end{equation*}
$ \epsilon \mu = \epsilon_r \mu_r \epsilon_0 \mu_0 = \frac{\epsilon_r \mu_r}{c^2} = \rmbox{ \frac{n^2}{c^2} = \frac{1}{v^2} } \qquad n = \sqrt{\epsilon_r \mu_r} $
homogene Wellengleichungen:
\begin{equation*}
\Rightarrow \quad \rmbox{\Delta \vec{E} - \frac{n^2}{c^2} \prt{^2 \vec{E}}{t^2} = 0} \qquad 
\Rightarrow \quad \rmbox{\Delta \vec{B} - \frac{n^2}{c^2} \prt{^2 \vec{B}}{t^2} = 0}
\end{equation*}
für jede Komponente von $ \vec{E} $ und $ \vec{B} $ gilt:
\begin{equation*}
\Delta \Psi - \frac{1}{v^2} \prt{^2 \Psi}{t^2} = 0
\end{equation*}
\lcom{Aus den Maxwell-Gleichungen ist aber zu sehen, dass $ \vec{E} $ und $ \vec{B} $-Felder \textbf{nicht} unabhängig voneinander sind.}
}
\subsection{Homogene Wellengleichung für skalare Funktion in einer Raumdimension}
\begin{equation*}
\Psi(x,t) \qquad \prt{^2 \Psi}{x^2} - \frac{1}{v^2} \prt{^2 \Psi}{t^2} = 0
\end{equation*}
Funktionen der Form $ \Psi_{\pm}(x,t) = f(x \pm vt) $ erfüllen die Wellengleichung.\\
\begin{equation*}
\prt{^2}{x^2} f(x \pm vt) = f''(x \pm vt)
\end{equation*}
\begin{equation*}
\prt{^2 f}{t^2} = \prt{}{t} f' (x \pm vt) \cdot (\pm v) = v^2 f'' (x \pm vt)
\end{equation*}
Wie sehen diese Lösungen nun aus?
\begin{equation*}
\Psi_-(x,t) = f(x - vt)
\end{equation*}
$ \Psi_-(x,0) = f(x) \ \leftarrow t = 0 $\\
\begin{center}
	%t1:
	%\begin{tikzpicture}
	%\draw [domain=-4:4, samples=100,scale=.3] plot (\x, {5*exp(-(\x)^(2)*0.25});
	%\end{tikzpicture}
	\begin{tikzpicture}
		%rechts
		%\draw[red,line width=1] (4,0) .. controls (3.2,0.2) and (2.8,1.8) .. (2,2);
		%\draw[red,line width=1] (0,0) .. controls (0.8,0.2) and (1.2,1.8) .. (2,2);
		\draw[domain=-5:2, samples=80, blue, thick] plot (\x, {2*exp(-(\x + 2)^(2)*0.8});
		%mitte
		%\draw[line width=1] (-2,0) .. controls (-1.2,0.2) and (-0.8,1.8) .. (0,2);
		%\draw[line width=1] (2,0) .. controls (1.2,0.2) and (0.8,1.8) .. (0,2);
		\draw[domain=-5:5, samples=80, thick] plot (\x, {2*exp(-(\x + 0)^(2)*0.8});
		%links
		%\draw[blue,line width=1] (-4,0) .. controls (-3.2,0.2) and (-2.8,1.8) .. (-2,2);
		%\draw[blue,line width=1] (0,0) .. controls (-0.8,0.2) and (-1.2,1.8) .. (-2,2);
		\draw[domain=-2:5, samples=80, red, thick] plot (\x, {2*exp(-(\x - 2)^(2)*0.8});
		%achsen
		\draw[thick, <-] (5.5,0) node[anchor=north west] {$x$}-- (-5.5,0);
		\draw[thick, ->] (0,0) node[below] {$0$} -- (0,3); % node[anchor=south east] {$y$};
		%
		\draw[line width=1,->] (0.2,2) -- (1,2);
		\draw[dotted] (0.5,2) node[above] {$v$};
		\draw[thick, red] (2,0.1) -- (2,-0.1) node[below] {$vt$};
		\draw[thick, blue] (-2,0.1) -- (-2,-0.1) node[below] {$-vt$};
		\draw[dotted] (2,2) node[red,above] {$\psi (x,t)$};
		\draw[dotted] (-2,2) node[blue,above] {$\psi (x,-t)$};
		\draw (-.7,3.5) node[above] {$\psi (x,0)$} -- (-0.15,2.1);
	\end{tikzpicture}
\end{center}
$ t > 0: \quad \Psi_-(x,t) = f(x - vt) $\\
$ \Psi_+(x,t) = f(x + vt) $\\[5pt]
Wellengleichungen sind linear $ \Rightarrow $ Linearkombinationen von Lösungen sind wieder Lösungen.
\begin{align*}
\Psi(x,t) &= a_1 \Psi_1(x,t) + a_2 \Psi_2(x,t)\\
&= a_1 f_1(x+vt) + a_2 f_2(x-vt)
\end{align*}
\lcom{Im allgemeinen muss es aber nicht sein, dass eine Welle bei Zeittransformationen ihre Form beibehält!}\\
\begin{center}
	%t2:
	\begin{tikzpicture}
		%\draw[line width=1,blue] (-4.5,0) .. controls (-3.3,0.1) and (-3.8,1.8) .. (-3,2);
		%\draw[line width=1,blue] (-1.5,0) .. controls (-2.7,0.1) and (-2.2,1.8) .. (-3,2);
		\draw[domain=-5:-1, samples=80, blue, thick] plot (\x, {2*exp(-(\x + 3)^(2)*2});
		%
		%\draw[line width=1] (-1.5,0) .. controls (-0.3,0.1) and (-0.8,1.8) .. (0,2);
		%\draw[line width=1] (1.5,0) .. controls (0.3,0.1) and (0.8,1.8) .. (0,2);
		\draw[domain=-2:2, samples=80, thick] plot (\x, {2*exp(-(\x + 0)^(2)*2});
		%
		%\draw[line width=1] (-3.5,0) .. controls (-1.2,0) and (-1.1,0.9) .. (0,1);
		%\draw[line width=1] (3.5,0) .. controls (1.2,0) and (1.1,0.9) .. (0,1);
		\draw[domain=-4:4, samples=80, thick] plot (\x, {1*exp(-(\x + 0)^(2)*.4});
		%
		%\draw[line width=1,blue] (-1.5,0) .. controls (0.8,0) and (0.9,0.9) .. (2,1);
		%\draw[line width=1,blue] (5.5,0) .. controls (3.2,0) and (3.1,0.9) .. (2,1);
		\draw[domain=-3:5, samples=80, blue, thick] plot (\x, {1*exp(-(\x - 2)^(2)*.4});
		%achsen
		\draw[thick, <-] (5.5,0) node[anchor=north west] {$x$}-- (-5.5,0);
		\draw[thick, ->] (0,0) node[below] {$0$} -- (0,3); % node[anchor=south east] {$y$};
		%
		\draw (2,.3) -- (3,1.2) node[anchor=south west] {$f_2$};
		\draw (0.3,2) -- (0.8,2.4) node[right] {$f_1$};
	\end{tikzpicture}
\end{center}
Man kann zeigen, dass die allgemeine Lösung der Wellengleichung in einer Dimension geschrieben werden kann als:
\begin{equation*}
\Psi(x,t) = f(x + vt) + g(x - vt)
\end{equation*}
Besonders Wichtig:\\
\textbf{Ebene Wellen (1D)}
\begin{align*}
\Psi_{\pm}(x,t) &= a \cos(kx \pm \omega t)\\
&= a \cos(k (x \pm \frac{\omega}{k} t))
\end{align*}
$ a \in \mathbb{R} \quad \omega, k \in \mathbb{R} \quad \omega, k > 0 $\\
$ \Psi_\pm $ löst die Wellengleichung: (siehe später explizit bei §D ebener Welle)
\begin{equation*}
\prt{^2}{x^2} \Psi - \frac{1}{v^2} \prt{^2}{t^2} \Psi = 0
\end{equation*}
falls $ \frac{\omega}{k} = v $\\
\begin{minipage}{.5\linewidth}
	\textbf{Dispersionsrelation:}\\
	$ \omega = k v = \omega(k) $\\
	Wellenzahl $ k $, Wellenlänge $ \lambda = \frac{2 \pi}{k} $
\end{minipage}%
\begin{minipage}{.5\linewidth}
	%t3:
	\begin{tikzpicture}[scale=.7]
		\draw[<-] (4.5,0) node[anchor=north west] {$x$}-- (-4.5,0);
		\draw[->] (-pi/2,-2) -- (-pi/2,2); % node[anchor=south east] {$y$};
		\draw[thick,domain=-4.3:4.3, samples=80] plot (\x, {sin(2*\x r)});
		\draw[dotted] (-2,-0.3) node[left] {$0$};
		\draw[thick] (pi/2,0.15) -- (pi/2,-0.15) node[below] {$\lambda$};
	\end{tikzpicture}
\end{minipage}%
\begin{align*}
\Psi_{\pm} (x + n \lambda, t) &= a \cos(k (x + n \lambda) \pm \omega t)\\
&= a \cos( k x \pm \omega t + k n \lambda)\\
&= a \cos(kx \pm \omega t) \\
&= \Psi_{\pm} (x,t)
\end{align*}
Die Kreisfrequenz $ \omega $, Frequenz $ \nu = \frac{\omega}{2 \pi} $, Schwingungsdauer $ \tau = \frac{1}{\nu} = \frac{2 \pi}{\omega} $
\begin{align*}
\Psi_{\pm} (x,t + n \tau) &= a \cos( kx \pm \omega(t + n \tau)) = a \cos(kx \pm \omega t \pm n \ub{\omega \tau}_{= 2 \pi}) \quad n \in \mathbb{Z} \\
&= a \cos(kx \pm \omega t) = \Psi_{\pm}(x,t)
\end{align*}
\begin{center}
	%t4:
	\begin{tikzpicture}
		\draw[<-] (4.5,0) node[right] {$x$}-- (-4.5,0);
		\draw[->] (-pi/4,-2) -- (-pi/4,2); % node[right] {$y$};
		\draw[red,thick,domain=-4.5:4.3, samples=100] plot (\x, {sin(2*\x r)});
		\draw[dotted] (-.8,-0.3) node[left] {$0$};
		\draw[thick] (3*pi/4,0.15) -- ++(0,-0.3) node[below] {$\lambda$};
		\draw[thick,domain=-4.5:4.3, samples=100] plot (\x, {-sin(2*\x r)});
		\draw[dotted] (-0.4,1) node[above] {$v$};
		\draw[dotted] (-2,1.5) node[above] {$\psi(x,0)$};
		\draw (-1.2,1) -- (-2,1.5);
		\draw[line width=1,->] (-0.7,1) -- (0,1);
		\draw[dotted] (2,1.5) node[above] {$\psi(x,t=\frac{\pi}{2})$};
		\draw (1,1) -- (1.8,1.5);
	\end{tikzpicture}
\end{center}
Die Phasengeschwindigkeit $ v $:
\begin{equation*}
v = \frac{\omega}{k} = \frac{\lambda}{2 \pi} \omega = \lambda \nu = \frac{\lambda}{\tau}
\end{equation*}
Ebenso wird die Wellengleichung mit einem Sinus gelöst:
\begin{equation*}
\Psi_{\pm} (x,t) = a \sin(kx \pm \omega t)
\end{equation*}
Beide Lösungen sind also enthalten in:
\begin{equation*}
\Psi(x,t) = a e^{i(\ob{kx - \omega t}^{\varphi \tx{ Phase}})}
\end{equation*}
$ a \in \mathbb{C} \quad k \in \mathbb{R} $\\
$ \omega = v |k| $\\
Für Physikalische Felder gilt:
\begin{equation*}
\Psi(x,t) = \Re\left(a e^{i(kx - \omega t)}\right)
\end{equation*}

\subsection{Ebene Wellen in 3 Raumdimensionen}
\begin{equation*}
\Delta \Psi(\vec{r} ,t) - \frac{1}{v^2} \prt{^2}{t^2} \Psi(\vec{r},t) = 0
\end{equation*}
eben Welle:
\begin{equation*}
\Psi(\vec{r} ,t) = a e^{i (\vec{k} \cdot \vec{r} - \omega t)}
\end{equation*}
Mit dem Wellenvektor:\\
$ \vec{k} = (k_x, k_y, k_z)^\top $\\
und der Frequenz\\
$ \omega \ge 0 $
\begin{align*}
\Delta \Psi &= \left(\prt{^2}{x^2} + \prt{^2}{y^2} + \prt{^2}{z^2}\right) a e^{i(\vec{k} \cdot \vec{r} - \omega t)} \\
&= i^2 (k_x^2 + k_y^2 + k_z^2) a e^{i(\vec{k} \cdot \vec{r} - \omega t)}\\
&= - k^2 \Psi
\end{align*}
\begin{equation*}
\prt{^2 \Psi}{t^2} = - \omega^2 \Psi
\end{equation*}
\begin{equation*}
\Delta \Psi - \frac{1}{v^2} \prt{^2 \Psi}{t^2} = \ob{\left(- k^2 + \frac{\omega^2}{v^2}\right)}^{=0} \Psi = 0
\end{equation*}
$ k^2 = \frac{\omega^2}{v^2} \qquad \omega = v |\vec{k}| $\\[5pt]
\emph{Bemerkung:}\\
\begin{minipage}{.5\linewidth}
	\begin{enumerate}[i)]
		\item ebene Welle:\\
		$ \Psi(\vec{r,t}) $ konstant falls Phase konstant.\\
		$ \varphi (\vec{r} ,t) = \vec{k} \cdot \vec{r} - \omega t = \const $\\
		Für festes $ t $: $ \vec{k} \cdot \vec{r} = \const $
		\item Phasengeschwindigkeit
		\begin{equation*}
		v = \frac{\omega}{|\vec{k}|}
		\end{equation*}
	\end{enumerate}
\end{minipage}%
\begin{minipage}{.5\linewidth}
	\flushright
	%t5:
	\begin{tikzpicture}[scale=.8]
		\draw[->] (0,-1) -- (0,3) node[anchor=south east] {$ y $};
		\draw[->] (-1,0) -- (4.5,0) node[anchor=north west] {$ x $};
		\coordinate (o) at (.5,-1);
		\draw[thick, ->] (o) -- ++(45:4) node[anchor=south west] {$ \vec{x} \  \substack{\tx{Ausbreitungs-} \\ \tx{Richtung}} $ };
		\draw ($ (o) + (135:3) $) -- ++(-45:6);
		\draw ($ (o) + (135:3) + (45:1.5) $) -- ++(-45:6);
		\draw ($ (o) + (135:3) + (45:3) $) -- ++(-45:6);
		\draw[decorate, decoration={brace,amplitude=10pt,raise=18pt}, xshift=2pt, yshift=-2pt]  ($ (o) + (135:2.5) $) -- node[anchor=south east, xshift=-15pt, yshift=15pt] {$\lambda = \frac{2 \pi}{k}$}  ($ (o) + (135:2.5) + (45:1.5) $);
		\draw[thick, <->] ($ (o) + (135:2.5) $) -- ($ (o) + (135:2.5) + (45:1.5) $);
	\end{tikzpicture}
\end{minipage}%
\begin{enumerate}
	\item[iii)] Wellengleichung linear\\
	$ \Rightarrow $ Linearkombinationen von Lösungen sind wieder Lösungen.
	\begin{equation*}
	\Psi(\vec{r},t) = a_1 e^{i(\vec{k}_1 \cdot \vec{r} - \omega_1 t)} + a_2 e^{i(\vec{k}_2 \cdot \vec{r} - \omega_2 t)}
	\end{equation*}
\end{enumerate}
\subsubsection{Allgemeine Lösung der Wellengleichung}
\begin{equation*}
\Psi(\vec{r},t) = \int \dd^3 k \ a(\vec{k}) e^{i(\vec{k} \cdot \vec{r} - \omega t)}
\end{equation*}
\begin{equation*}
\omega(\vec{k}) = v |\vec{k}|
\end{equation*}

\subsection{Ebene elektromagnetische Wellen}

\begin{equation*}
\Rightarrow \quad \rmbox{\Delta \vec{E} - \frac{1}{v^2} \prt{^2 \vec{E}}{t^2} = 0} \qquad 
\Rightarrow \quad \rmbox{\Delta \vec{B} - \frac{1}{v^2} \prt{^2 \vec{B}}{t^2} = 0}
\end{equation*}
Lösungen:
\begin{equation*}
\vec{E} = \vec{E}_0 e^{i(\vec{k} \cdot \vec{r} - \omega t)} \qquad \vec{B} = \vec{B}_0 e^{i(\tilde{\vec{k}} \cdot \vec{r} - \tilde{\omega} t)}
\end{equation*}
$ \omega = v |\vec{k}| \quad \tilde{\omega} = v |\tilde{\vec{k}}| \quad \vec{k}, \tilde{\vec{k}} \in \mathbb{R}^3 $
\begin{enumerate}[i)]
	\item $ \vabla \times \vec{E} = - \prt{\vec{B}}{t} $
	\begin{equation*}
	i (\vec{k} \times \vec{E}_0) e^{i(\vec{k} \cdot \vec{r} - \omega t)} = i \tilde{\omega} \vec{B}_0 e^{i(\tilde{\vec{k}} \cdot \vec{r} - \tilde\omega t)} \qquad \forall \vec{r},t
	\end{equation*}
	$$ \Rightarrow \tilde{\vec{k}} = \vec{k} \quad \tilde{\omega} = \omega $$
	$ \Rightarrow \vec{k} \times \vec{E}_0 = \omega \vec{B}_0 \ \rightarrow \ \vec{B}_0 = \frac{1}{\omega} \vec{k} \times \vec{E}_0 $
	$$\vec{B} \perp \vec{k}, \vec{E}$$
	$$ \Rightarrow \vec{B} = \frac{1}{\omega} \vec{k} \times \vec{E} $$
	\item $ \vabla \cdot \vec{E} = 0 $
	\begin{align*}
	0 &= \vabla \cdot \vec{E}_0 e^{i(\vec{k} \cdot \vec{r} - \omega t)}\\
	&= i(\vec{k} \cdot \vec{E}_0) e^{i(\vec{k} \cdot \vec{r} - \omega t)}
	\end{align*}
	$ \Rightarrow \vec{k} \cdot \vec{E}_0 \ \rightarrow \ \vec{k} \cdot \vec{E}_0 = 0 $
	$$ \vec{k} \perp \vec{E} $$
	\item $ \vabla \cdot \vec{B} = 0 $\\
	$ \Rightarrow \vec{k} \cdot \vec{B}_0 = 0 \Rightarrow \vec{k} \cdot \vec{B} = 0 $\\
	$ \vec{B} \perp \vec{k} $ nichts neues, wissen wir schon aus i)
	\item $ \vabla \times \vec{B} = \frac{1}{v^2} \prt{\vec{E}}{t} $\\
	$ \Rightarrow i \vec{k} \times \vec{B}_0 = - i \frac{1}{v^2} \omega \vec{E}_0 $\\
	$ \Rightarrow \vec{k} \times \vec{B}_0 = - \frac{\omega}{v^2} \vec{E}_0 $\\
	$$ \Rightarrow \vec{E} = -\frac{v^2}{\omega} \vec{k} \times \vec{B} $$
\end{enumerate}
\begin{minipage}{.5\linewidth}
	$ \vec{E}, \vec{B}, \vec{k} $ orthogonal zueinander\\
	$ \Rightarrow $ transversale Welle\\
	Beziehung zwischen der Amplitude von $ \vec{B}, \vec{E} $:
	\begin{equation*}
	|\vec{B}| = \frac{1}{\omega} |\vec{k} \times \vec{E}| = \frac{|\vec{k}|}{\omega} |\vec{E}| = \frac{1}{v} |\vec{E}|
	\end{equation*}
\end{minipage}%
\begin{minipage}{.5\linewidth}
	\flushright
	%t6:
	\begin{tikzpicture}
		\draw[thick,->] (0,0) -- (0,1.5) node[anchor=south east] {$ \vec{E} $};
		\draw[thick,->] (0,0) -- (1.5,0) node[anchor=north west] {$ \vec{S} $};
		\draw[thick,->] (0,0) -- (-.75,-.75) node[anchor=north east] {$ \vec{B} $};
	\end{tikzpicture}
\end{minipage}%
\\
o.B.d.A. $ \quad \vec{k} = k \vec{e}_z $
\begin{align*}
\vec{E} &= (E_{0_x} \vec{e}_x + E_{0_y} \vec{e}_y) e^{i(k \cdot z - \omega t)} \\
\vec{B} &= \frac{1}{\omega} \vec{k} \times \vec{E} = \frac{k}{\omega} \vec{e}_z \times (E_{0_x} \vec{e}_x + E_{0_y} \vec{e}_y) e^{i(k \cdot z - \omega t)} \\
&= \frac{1}{v} (- E_{0_y} \vec{e}_x + E_{0_x} \vec{e}_y) e^{i(k \cdot z - \omega t)} 
\end{align*}
i.A. $ E_{0_x}, E_{0_y} \in \mathbb{C} $
\begin{equation*}
\vec{E} = \Re\left(\vec{E}_0 e^{i(k \cdot z - \omega t)} \right)
\end{equation*}
\emph{Beispiel:} $ E_{0_y} = 0 \quad E_{0_x} \in \mathbb{R} $
\begin{align*}
\vec{E} &= E_{0_x} \cos (kz - \omega t) \vec{e}_x \\
\vec{B} &= \frac{E_{0_x}}{v} \cos(kz - \omega t) \vec{e}_y
\end{align*}

%t7: (EM-Welle Tikz): (v in ausbreitungs richtung einzeichnen)
\begin{center}
	%t3:
	\begin{tikzpicture}%[>=stealth]
	% E
	\draw[red,thick,looseness=5,fill=red!20!white,opacity=.8] (0,0) to[out=80,in=100] (1.5,0) to[out=-80,in=-100] (3,0);
	\foreach \z in {.5,1}
	\draw[thick,red,->,>=stealth] (\z,0) -- ++(0,1.9);
	\foreach \z in {2,2.5}
	\draw[thick,red,->,>=stealth] (\z,0) -- ++(0,-1.9);
	% B
	\draw[blue,thick,looseness=4.5,fill=blue!20!white,opacity=.8] (0,0) to[out=-125,in=-145] (1.5,0) to[out=35,in=55] (3,0) to[out=-125,in=-145] (4.5,0) to[out=35,in=55] (6,0);
	\foreach \z in {.5,1,3.5,4}
	\draw[thick,blue,->,>=stealth] (\z,0) -- ++(-136:1.7cm);
	\foreach \z in {2,2.5,5,5.5}
	\draw[thick,blue,->,>=stealth] (\z,0) -- ++(43:1.7cm);
	% rest E
	\draw[red,thick,looseness=5,fill=red!20!white,opacity=.8] (3,0) to[out=80,in=100] (4.5,0) to[out=-80,in=-100] (6,0);
	\foreach \z in {3.5,4}
	\draw[thick,red,->,>=stealth] (\z,0) -- ++(0,1.9);
	\foreach \z in {5,5.5}
	\draw[thick,red,->,>=stealth] (\z,0) -- ++(0,-1.9);
	% koordinatensystem
	\draw[->] (0,0) -- (0,2) node[anchor=south east] {$ \vec{y} $};
	\draw[->] (0,0) -- (-1.2,-1.2) node[anchor=north east] {$ \vec{x} $};
	\draw[->] (0,0) -- (7,0) node[anchor=north west] {$ \vec{z} $};
	\node (e) at (1.5,2) {\color{red} $ \vec{E} $ \color{black}};
	\node (b) at (.5,-1.25) {\color{blue} $ \vec{B} $ \color{black}};
	\draw[thick,->] (5.5,2.5) -- node[above] {$ \vec{v} $} ++(1.5,0);
	\end{tikzpicture}
\end{center}

\subsection{Polarisation ebener EM-Wellen}

Charakterisierung der Schwingungsrichtung $ \vec{k} = k \vec{e}_z $
\begin{equation*}
\vec{E} = (E_{0_x} \vec{e}_x + E_{0_y} \vec{e}_y) e^{i(kx - \omega t)}
\end{equation*}
\begin{equation*}
E_{0_x} = |E_{0_x}| e^{i \varphi} \qquad E_{0_y} = |E_{0_y}| e^{i (\varphi + \delta)}
\end{equation*}
Physikalisches Feld:
\begin{align*}
\vec{E} &= \Re \left[ (E_{0_x} \vec{e}_x + E_{0_y} \vec{e}_y) e^{i(k \cdot r - \omega t)} \right]\\
&= |E_{0_x}| \cos(kz - \omega t + \varphi) \vec{e}_x + |E_{0_y}| \cos(kz - \omega t + \varphi + \delta) \vec{e}_y
\end{align*}
\textbf{Fälle:}\\
\begin{enumerate}[i)]
	\item
	$ \delta = 0 $ oder $ \delta = \pm \pi $
	\begin{equation*}
	\Rightarrow \quad \vec{E} = \ub{(|E_{0_x}| \vec{e}_x \pm |E_{0_y}| \vec{e}_y)}_{\mathclap{\substack{\tx{orts- und} \\ \tx{Zeitunabh.}}}} \cos(kz - \omega t)
	\end{equation*}
	$ \rightarrow \vec{E} $ schwingt in fester Richtung! Die \textbf{Polarisationsrichtung}
	$ \rightarrow $ \textbf{linear Polarisiert}
	\item
	$ \delta = \pm \frac{\pi}{2} \quad |E_{0_x}| = |E_{0_y}| = E $
	\textbf{Zirkulär Polarisiert}
	\begin{align*}
	\vec{E} = E\ub{(\vec{e}_x \cos(kz - \omega t + \varphi)) \mp \vec{e}_y \sin(kz - \omega t + \varphi))}_{ = \vec{r} (t)}
	\end{align*}
	\lcom{$ |\vec{r}| = 1 $ und $ \vec{r} $ läuft mit $ \omega $ um die $ z $-Achse}
	\item
	$ \delta , |E_{0_x}|, |E_{0_y}| $ beliebig:
	\textbf{elliptisch Polarisiert}
\end{enumerate}
\begin{center}
	%t8:
	\begin{tikzpicture}[scale=.95]
		\draw[->] (0,-2) -- (0,2) node[anchor=south east] {$ y $};
		\draw[->] (-2,0) -- (2,0) node[anchor=north west] {$ x $};
		\draw[thick,->] (0,0) -- (1.2,1.2) node[anchor=south west] {$ \vec{E} $};
		\draw[->] (0,0) -- (-1.2,-1.2);
		\draw[thick] ($ (1.2,0) + (0,.15) $) -- ++(0,-.3) node[below] {$ |E_{0_x}| $};
		\draw[thick] ($ (0,1.2) + (.15,0) $) -- ++(-.3,0) node[left] {$ |E_{0_y}| $};
		%
		\draw[white] (2,-2) -- ++(0,-.225);
	\end{tikzpicture}%
	%t9:
	\begin{tikzpicture}[scale=.76]
		\draw[->] (0,-2.5) -- (0,2.5) node[anchor=south east] {$ y $};
		\draw[->] (-2.5,0) -- (2.5,0) node[anchor=north west] {$ x $};
		\draw[thick,->] (0,0) -- ({cos(45) * 2}, {sin(45) * 2}) node[anchor=south west] {$ \vec{r}(t) $};
		\draw (0,0) circle (2cm);
		\centerarc[thick, <->](0,0)(120:-30:1.2);
		\node (a) at (-30:1.4) {};
		\draw[<-] (a) to[out=0,in=90] ++(1,-.7) node[anchor=north west] {$ - \frac{\pi}{2} $};
		\node at (-3,2) {rechtszirkulär};
		\node (b) at (120:1.4) {};
		\draw[<-] (b) to[out=90,in=0] ++(-.7,1) node[anchor=south east] {$ + \frac{\pi}{2} $};
		\node at (3,-2.5) {linkszirkulär};
	\end{tikzpicture}%
	%t10:
	\begin{tikzpicture}[scale=.76]
		\draw[->] (0,-2.5) -- (0,2.5) node[anchor=south east] {$ y $};
		\draw[->] (-2.5,0) -- (2.5,0) node[anchor=north west] {$ x $};
		\draw[rotate=25] (0,0) ellipse (2cm and 1cm);
		\draw (-1.86,-1.25) rectangle (1.86,1.25);
		\draw[thick, ->] (0,0) -- node[above] {$ \vec{E} $} (1.7,1); %({cos(25) * 2},{sin(25) * 2});
		%
		\draw[white] (2,-2.5) -- ++(0,-0.315);
	\end{tikzpicture}
\end{center}

% Vorlesung 10.01.18

\noindent
\bbb{Wiederholung}{
\begin{align*}
\vec{E} &= \vec{E}_0 e^{i(\vec{k} \cdot \vec{r} - \omega t)} \qquad \omega = |\vec{k}| v \\
\vec{B} &= \vec{B}_0 e^{i(\tilde{\vec{k}} \cdot \vec{r} - \tilde{\omega} t)} \ \ \buildrel ! \over = \ \ \frac{1}{\omega} \vec{k} \times \vec{E}
\end{align*}
}

\subsubsection{andere Formen der EM-Wellen}

\begin{minipage}{.45\linewidth}
	\textbf{Wellenpakete:}
	\begin{equation*}
	\vec{E}(\vec{r},t) = \int \dd^3 k \vec{a}(\vec{k}) e^{i(\vec{k} \cdot \vec{r} - \omega(k) t)}
	\end{equation*}
\end{minipage}%
\begin{minipage}{.55\linewidth}
	\flushright
	%t1:
	\begin{tikzpicture}[scale=.8]
		\draw[thick,fill=black!10!red!12!white, domain=-3.75:3.75, samples=100] plot (\x, {2 * cos(1 / 4 * 15 * \x r) * exp(- (\x)^(2) * 0.25});
		\draw[->] (-5,0) -- (5,0) node[anchor=north west] {$ x $};
		\draw[->] (0,-2.5) -- (0,2.5) node[anchor=south east] {$ y $};
	\end{tikzpicture}
\end{minipage}%
\\
\begin{minipage}{.5\linewidth}
	\textbf{Kugelwellen}
	\begin{equation*}
	\vec{E}(\vec{r},t) = \vec{E}_0 \frac{e^{i(\vec{k} \cdot \vec{r} - \omega t)}}{r}
	\end{equation*}
	eine Kugelwelle hat Amplitude $ \rho \propto \frac{1}{r} $\\
	Flächen gleicher Phase (wie im Bild zu sehen).
	\begin{center}
		%t2:
		\begin{tikzpicture}[scale=.7]
			\draw (0,0) circle (.4cm);
			\draw (0,0) circle (1cm);
			\draw (0,0) circle (1.6cm);
		\end{tikzpicture}
	\end{center}
\end{minipage}%
\begin{minipage}{.5\linewidth}
	\flushright
	%t3:
	\begin{tikzpicture}[scale=.7]
		\draw[->] (-1,0) -- (9.5,0) node[anchor=north west] {$ r $};
		\draw[->] (0,-2.5) -- (0,2.5) node[anchor=south east] {$ |\vec{E}| $};
		\draw[thick,domain=0.001:9.3, samples=100] plot (\x, {2 * cos(1 / 4 * 15 * (\x * .5) r) * exp(- (\x * .5)^(2) * 0.25 * (1) / (\x * 0.2) });
		\draw[thick, dashed] (0,2) to[out=-28,in=178] (9.3,.2);
		\draw (3,1) -- ++(30:1cm) node[anchor=south west] {$ \rho \propto \frac{1}{r} $};
	\end{tikzpicture}
\end{minipage}%

\section{Reflexion und Brechung von EM-Wellen an Grenzflächen}

Wir betrachten eine ebene Grenzfläche ($ x $-$ y $-Ebene) zwischen zwei ungeladenen, nicht leitenden Medien.\\
\begin{minipage}{.55\linewidth}
	Aus der Skizze:
	\begin{equation*}
	v_1 = \frac{1}{\sqrt{\epsilon_1 \mu_1}} , \ \omega_e = v_1 k_e
	\end{equation*}
	und $ \vec{k}_e $ ohne $ y $-Komponente
	\begin{equation*}
	\vec{k}_e = \begin{pmatrix}
	k_{e_x} \\ 0 \\ k_{e_z}
	\end{pmatrix}
	\end{equation*}
\end{minipage}%
\begin{minipage}{.45\linewidth}
	%t4:
	\begin{tikzpicture}
		\draw[->] (-2,0) -- (4,0) node[right] {$x$};
		\draw[->] (0,-2) -- (0,2) node[right] {$z$};
		\draw[thick,->] (-1.1,-1.5) node[anchor=north east] {$\vec{k}_e$} -- (-0.01,-0.01); 
		\draw[thick,->] (0,0) -- (1.5,1.25) node[anchor=south west] {$\vec{k}_2$}; 
		\draw[thick,->] (0,0) -- (1.1,-1.5) node[anchor=north west] {$\vec{k}_r$};
		\centerarc[](0,0)(40:90:1);
		\node at (0.3,0.6) {$\theta_2$};
		\centerarc[](0,0)(234:270:1);
		\node at (-0.2,-0.7) {$\theta_e$};
		\centerarc[](0,0)(270:306:1);
		\node at (0.3,-0.7) {$\theta_r$};
		\node at (-1.5,.7) {$\epsilon_2 \mu_2$};
		\node at (-1.5,-.7) {$\epsilon_1 \mu_1$};
		\draw[->] (3.25,0) -- (3.25,.85) node[anchor=north west] {$\vec{n}=\vec{e}_z$};
	\end{tikzpicture}
	\vspace{5pt}
\end{minipage}%
\\
Die \textbf{einfallende Welle} sieht folgendermaßen aus:
\begin{align*}
\vec{E}_e &= \vec{E}_{e_0} e^{i(\vec{k}_e \cdot \vec{r} - \omega_e t)}\\
\vec{B}_e &= \frac{1}{\omega_e} \vec{k}_e \times \vec{E}_e = \frac{1}{v_1} \frac{\vec{k}_e}{k_e} \times \vec{E}_e
\end{align*}
\textbf{reflektierte Welle}
\begin{align*}
\vec{E}_r &= \vec{E}_{r_0} e^{i(\vec{k}_r \cdot \vec{r} - \omega_r t)} \\
\vec{B}_r &= \frac{1}{v_1} \frac{\vec{k}_r}{k_r} \times \vec{E}_r
\end{align*}
gesamtes Feld in Medium 1:
\begin{equation*}
\vec{E}_1 = \vec{E}_e + \vec{E}_r
\end{equation*}
\textbf{transmittierte Welle}
\begin{align*}
\vec{E}_2 &= \vec{E}_{2_0} e^{i(\vec{k}_2 \cdot \vec{r} - \omega_2 t)} \\
\vec{B}_2 &= \frac{1}{v_2} \frac{\vec{k}_2}{k_2} \times \vec{E}_2
\end{align*}

\subsection{Stetigkeitsbedingunen an Grenzflächen}

Hier: ungeladene, nicht leitende Medien: $ \rho_f = 0 = \vec{j}_f $\\[5pt]
Maxwell-Gl. $ \Rightarrow $ Stetigkeitsbedingungen für $ \vec{r} = (x,y,0) $
\begin{enumerate}[i)]
	\item mit $ \vabla \cdot \vec{D} = 0 $
	\begin{equation*}
	\vec{n} \cdot (\vec{D}_1 - \vec{D}_2) = 0 \quad \Rightarrow \quad \epsilon_1 \vec{n} \cdot \vec{E}_1 - \epsilon_2 \vec{n} \cdot \vec{E}_2 = 0
	\end{equation*}
	\item mit $ \vabla \times \vec{E} = - \prt{\vec{B}}{t} $ und $ \vec{t} \sim \vec{e}_x, \vec{e}_y $
	\begin{equation*}
	\vec{t} \cdot (\vec{E}_1 - \vec{E}_2) = 0
	\end{equation*}
	\item mit $ \vabla \cdot \vec{B} = 0 $
	\begin{equation*}
	\vec{n} \cdot (\vec{B}_1 - \vec{B}_2) = 0
	\end{equation*}
	\item mit $ \vabla \times \vec{H} = \cancel{\vec{j}} + \prt{\vec{D}}{t} $
	\begin{equation*}
	\vec{t} \cdot (\vec{H}_1 - \vec{H}_2) = 0 \quad \Rightarrow \quad \frac{1}{\mu_1} \vec{t} \cdot \vec{B}_1 - \frac{1}{\mu_2} \vec{t} \cdot \vec{B}_2 = 0
	\end{equation*}
\end{enumerate}
Erläuterung zu Punkt ii) und iv):
\begin{equation*}
\vabla \times \vec{E} = - \prt{\vec{B}}{t} \quad \rightarrow \quad \vec{t} \cdot (\vec{E}_2 - \vec{E}_1) = 0
\end{equation*}
\begin{minipage}{.6\linewidth}
	\begin{align*}
	&\int \limits _F \dd \vec{f} \cdot (\vabla \times \vec{E}) = - \int \limits_F \dd \vec{f} \cdot \prt{\vec{B}}{t} = \ub{- \prd{}{t} \int \limits_F \dd \vec{f} \vec{B}}_{\overset{\Delta x \to 0}{\rightarrow} 0} \\
	= & \int \limits_{\partial F} \dd \vec{r} \cdot \vec{E} \qquad \rightarrow \qquad \int \dd s \ \vec{t} \cdot (\vec{E}_2 - \vec{E}_1) = 0
	\end{align*}
	\vspace{5pt}
\end{minipage}%
\begin{minipage}{.4\linewidth}
	\flushright
	%t5:
	\begin{tikzpicture}[scale=.7]
		\draw[->] (-2.2,0) -- (3,0) node[anchor=north west] {$ y $};
		\draw[->] (0,-2) -- (0,2) node[anchor=south east] {$ x $};
		\draw[thick] (-2,-1) rectangle (2,1);
		\draw[ultra thick, ->] (0,0) -- (1,0);
		\draw (.5,-.2) -- ++(-60:1.5cm) node[below=2pt] {$ \vec{E} $};
		\draw[decorate, decoration={brace,amplitude=10pt,raise=2pt}, thick]  ($ (-2,-1) $) -- node[left, xshift=-15pt] {$\Delta x$}  ($ (-2,1) $);
	\end{tikzpicture}
\end{minipage}%
\\
Aus den vier Stetigkeitsbedingungen ergeben sich folgende Schlussfolgerungen:\\[5pt]
ii)
\begin{equation*}
\vec{t} \cdot \vec{E}_2 = \vec{t} \cdot \vec{E}_1 = \vec{t} \cdot (\vec{E}_e + \vec{E}_r)
\end{equation*}
\begin{equation*}
\Rightarrow \vec{t} \cdot \vec{E}_{2_0}  e^{i(\vec{k}_2 \cdot \vec{r} - \omega_2 t)} = \vec{t} \cdot \vec{E}_{e_0}  e^{i(\vec{k}_e \cdot \vec{r} - \omega_e t)} + \vec{t} \cdot \vec{E}_{r_0}  e^{i(\vec{k}_r \cdot \vec{r} - \omega_r t)}
\end{equation*}
$ \forall \vec{r} = (x,y,0) , \forall t $
$ \rightarrow $ Die Orts- und Zeitabhängigkeit im Exponenten muss gleich sein!
\begin{equation*}
\vec{k}_2 \cdot \vec{r} - \omega_2 t =  \vec{k}_e \cdot \vec{r} - \omega_e t = \vec{k}_r \cdot \vec{r} - \omega_r t
\end{equation*}
$ \vec{r} = 0 : $ $ \Rightarrow \ \omega_2 = \omega_e = \omega_r \defeq \omega $\\
\lcom{Die gleiche Schlussfolgerung geht für die Wellenvektoren nicht, da wir hier die $ z $-Kompo-\\
nente gar nicht beachten und damit keine Aussagen über die Gleichheit der Vektoren machen können.}\\[5pt]
Für die einfallende und reflektierte Welle im Medium 1 gilt:\\
$ v_1 = \frac{\omega_e}{|\vec{k}_e|} = \frac{\omega_r}{|\vec{k}_r|} $
\begin{equation*}
\rightarrow \quad |\vec{k}_e| = |\vec{k}_r| \defeq k_1
\end{equation*}
Somit sind auch beide Wellenlängen $ \lambda_1 $ gleich!\\[5pt]
Für die Welle im Medium 2 gilt:
\begin{align*}
k_2 &= \frac{\omega_2}{v_2} = \omega \sqrt{\epsilon_2 \mu_2} \\
&= \omega \sqrt{\frac{\epsilon_2 \mu_2}{\epsilon_1 \mu_1}} \sqrt{\epsilon_1 \mu_1} = \ub{\frac{\omega}{v_1}}_{k_1} \sqrt{\frac{\epsilon_2 \mu_2}{\epsilon_1 \mu_1}} = 	k_1 \sqrt{\frac{\epsilon_2 \mu_2}{\epsilon_1 \mu_1}} = k_1 \cdot \frac{n_2}{n_1}
\end{align*}
weiterhin:
$ t=0 : $ $ \Rightarrow \ \vec{k}_2 \cdot \vec{r} = \vec{k}_e \cdot \vec{r} = \vec{k}_r \cdot \vec{r} $\\[5pt]
$ k_{2_x} x + k_{2_y} y = k_{e_x} x + \ub{k_{e_y} y}_{=0} = k_{r_x} x + k_{r_y} y =   $\\[5pt]
$ k_{e_y} = 0 \quad \Rightarrow k_{2_y} = 0 = k_{r_y}$
$ \Rightarrow \vec{k}_e, \vec{k}_r, \vec{k}_2 $ liegen in einer Ebene (hier: $ x $-$ z $-Ebene), der Einfallsebene.\\
\begin{minipage}{.5\linewidth}
	\begin{align*}
	\vec{k}_e &= k_e (\sin \theta_e \vec{e}_x + \cos \theta _e \vec{e}_z)\\
	\vec{k}_r &= k_r (\sin \theta_r \vec{e}_x - \cos \theta _r \vec{e}_z)\\
	\vec{k}_2 &= k_2 (\sin \theta_2 \vec{e}_x + \cos \theta _2 \vec{e}_z)
	\end{align*}
	$ k_e = k_r \defeq k_1 $
\end{minipage}%
\begin{minipage}{.5\linewidth}
	\flushright
	%t6:
	\begin{tikzpicture}
		\draw[->] (-2,0) -- (2,0) node[right] {$x$};
		\draw[->] (0,-2) -- (0,2) node[right] {$z$}; 
		\draw[thick,->] (0,0) -- (0.7,1.25) node[anchor=south west] {$\vec{k}_2$}; 
		\centerarc[](0,0)(60:90:1.2);
		\node at (0.2,.85) {$\theta_2$};
		\draw[thick,->] (-1.1,-1.5) node[anchor=north east] {$\vec{k}_e$} -- (-0.01,-0.01); 
		\draw[thick,->] (0,0) -- (1.1,-1.5) node[anchor=north west] {$\vec{k}_r$};
		\centerarc[](0,0)(234:270:1);
		\node at (-0.2,-0.7) {$\theta_e$};
		\centerarc[](0,0)(270:306:1);
		\node at (0.3,-0.7) {$\theta_r$};
	\end{tikzpicture}
\end{minipage}%

\subsubsection{Reflexionsgesetz}

$ \vec{r} = (x,0,0): \quad \vec{k}_e \cdot \vec{r} = \vec{k}_r \cdot \vec{r} $
\begin{align*}
\vec{k}_e \cdot \vec{r} &= x k_1 \sin \theta_e\\
\vec{k}_r \cdot \vec{r} &= x k_1 \sin \theta_r\\
\end{align*}
\begin{center}
	\begin{minipage}{.5\linewidth}
		\frbox{Reflexionsgesetz}{
		\begin{equation*}
		\theta_e = \theta_r
		\end{equation*}
		}
	\end{minipage}
\end{center}


\subsubsection{Brechungsgesetz}

$ \vec{r} = (x,0,0): \quad \vec{k}_2 \cdot \vec{r} = \vec{k}_e \cdot \vec{r} $
\begin{align*}
&k_2 \sin \theta_2 = k_1 \sin \theta_1 \\
k_2 = &k_1 \sqrt{\frac{\epsilon_2 \mu_2}{\epsilon_1 \mu_1}} = k_1 \frac{n_2}{n_1}
\end{align*}
$ n : $ Brechungsindex
\begin{center}
	\begin{minipage}{.5\linewidth}
		\frbox{Snellius'sches Brechungsgesetz}{
		\begin{equation*}
		\frac{n_2}{n_1} = \frac{\sin \theta_1}{\sin \theta_2}
		\end{equation*}
		}
	\end{minipage}
\end{center}
\begin{center}
	%t7:
	\begin{tikzpicture}
		\draw[->] (-2,0) -- (2,0) node[right] {$x$};
		\draw[->] (0,-2) -- (0,2) node[right] {$z$};
		\draw[thick,->] (0,0) -- (0.7,1.25) node[anchor=south west] {$\vec{k}_2$}; 
		\centerarc[](0,0)(60:90:1.2);
		\node at (0.2,.85) {$\theta_2$};
		\draw[thick,->] (-1.1,-1.5) node[anchor=north east] {$\vec{k}_e$} -- (-0.01,-0.01);
		\draw[thick,->] (0,0) -- (1.1,-1.5) node[anchor=north west] {$\vec{k}_r$};
		\centerarc[](0,0)(234:270:1);
		\node at (-0.2,-0.7) {$\theta_e$};
		\centerarc[](0,0)(270:306:1);
		\node at (0.3,-0.7) {$\theta_r$};
		\node at (-1,1.75) {$n_2>n_1$};
	\end{tikzpicture}
	\hspace{1.5cm}
	%t8:
	\begin{tikzpicture}
		\draw[->] (-2,0) -- (2,0) node[right] {$x$};
		\draw[->] (0,-2) -- (0,2) node[right] {$z$}; 
		\draw[thick,->] (0,0) -- (1.5,1.25) node[anchor=south west] {$\vec{k}_2$}; 
		\centerarc[](0,0)(40:90:1);
		\node at (0.3,0.6) {$\theta_2$};
		\draw[thick,->] (-1.1,-1.5) node[anchor=north east] {$\vec{k}_e$} -- (-0.01,-0.01); 
		\draw[thick,->] (0,0) -- (1.1,-1.5) node[anchor=north west] {$\vec{k}_r$};
		\centerarc[](0,0)(234:270:1);
		\node at (-0.2,-0.7) {$\theta_e$};
		\centerarc[](0,0)(270:306:1);
		\node at (0.3,-0.7) {$\theta_r$};
		\node at (-1,1.75) {$n_2<n_1$};
	\end{tikzpicture}
\end{center}
Für $ n_2 > n_1 : $ Grenzfall $ \theta_1 \defeq \theta_g $ bei dem \textbf{Totalreflexion} auftritt (d.h. $ \theta_2 = \frac{\pi}{2} $)
\begin{equation*}
\sin \theta_g = \frac{n_2}{n_1} \sin \frac{\pi}{2} = \frac{n_2}{n_1}
\end{equation*}
z.B. Wasser $ \ n_1 \approx 1{,}33 $, Luft $ \ n_2 \approx 1 $ $ \Rightarrow \theta_g \approx 49 ^\circ $
\begin{comment}
\begin{tikzpicture}
\draw[->] (-5,0) -- (5,0);
\draw[->] (0,-5) -- (0,5);
\draw [domain=-3.75:3.75, samples=100] plot (\x, {2 * cos(1 / 4 * 15 * \x r) * exp(- (\x)^(2) * 0.25});
\end{tikzpicture}
\end{comment}
% teyt of tikz plotting:
%\begin{tikzpicture}
%\draw [domain=-4:4, samples=100,scale=.3] plot (\x, {5*exp(-(\x)^(2)*0.25 * cos(\x * 40)});
%\end{tikzpicture}

%Vorlesung 14.01.18

\subsection{Intensitätsbeziehungen bei Reflexion und Brechung}

\bbb{Wiederholung}{
\begin{minipage}{.6\linewidth}
	\begin{equation*}
	\vec{E}_l = \vec{E}_{l_0} e^{i(\vec{k}_l \vec{r} - \omega t)} \qquad l = e, r, 2
	\end{equation*}
	\begin{align*}
	\vec{B}_l &= \frac{1}{\omega} \vec{k}_e \times \vec{E}_l \\
	\vec{E}_1 &= \vec{E}_e + \vec{E}_r
	\end{align*}
\end{minipage}%
\begin{minipage}{.4\linewidth}
	%t1:
	\begin{tikzpicture}
		\draw[->] (-2,0) -- (2,0) node[right] {$x$};
		\draw[->] (0,-2) -- (0,2) node[right] {$z$};
		\draw[thick,->] (0,0) -- (0.7,1.25) node[anchor=south west] {$\vec{k}_2$}; 
		\centerarc[](0,0)(60:90:1.2);
		\node at (0.2,.85) {$\theta_2$};
		\draw[thick,->] (-1.1,-1.5) node[anchor=north east] {$\vec{k}_e$} -- (-0.01,-0.01);
		\draw[thick,->] (0,0) -- (1.1,-1.5) node[anchor=north west] {$\vec{k}_r$};
		\centerarc[](0,0)(234:270:1);
		\node at (-0.2,-0.7) {$\theta_e$};
		\centerarc[](0,0)(270:306:1);
		\node at (0.3,-0.7) {$\theta_r$};
		\node at (-1.75,.6) {$ \epsilon_2, \mu_2, n_2 $};
		\node at (-1.75,-.6) {$ \epsilon_1, \mu_1, n_1 $};
	\end{tikzpicture}
	\vspace{5pt}
\end{minipage}%
\\
\textbf{Stetigkeitsbedingungen an der Grenzfläche} $ \vec{r} = (x,y,0) $ hier: $ \vec{r} = 0  , \ t = 0 \quad \vec{t} \sim \vec{e}_x , \vec{e}_y $
\begin{enumerate}[i)]
	\item
	\begin{equation*}
	\vec{n} \cdot (\vec{D}_2 - \vec{D}_1) = 0
	\end{equation*}
	\begin{equation*}
	\Rightarrow \quad \vec{e}_z \cdot (\epsilon_2 \vec{E}_{2_0} - \epsilon_1 (\vec{E}_{e_0} + \vec{E}_{r_0})) = 0
	\end{equation*}
	\item
	\begin{equation*}
	\vec{t} \cdot (\vec{E}_2 - \vec{E}_1) = 0
	\end{equation*}
	\begin{equation*}
	\Rightarrow \quad \vec{t} \cdot (\vec{E}_{2_0} - \vec{E}_{e_0} - \vec{E}_{r_0}) = 0
	\end{equation*}
	\item
	\begin{equation*}
	\vec{n} \cdot (\vec{B}_2 - \vec{B}_1) = 0
	\end{equation*}
	\begin{equation*}
	\Rightarrow \quad \vec{e}_z \cdot \bigg[\vec{k}_2 \times \vec{E}_{2_0} - \vec{k}_e \times \vec{E}_{e_0} - \vec{k}_r \times \vec{E}_{r_0}\bigg] = 0
	\end{equation*}
	\item
	\begin{equation*}
	\vec{t} \cdot (\vec{H}_2 - \vec{H}_1) = 0 \qquad \vec{H}_i = \frac{1}{\mu} \vec{B}_i
	\end{equation*}
	\begin{equation*}
	\Rightarrow \quad \vec{t} \cdot \bigg[\frac{1}{\mu_2} \vec{k}_2 \times \vec{E}_{2_0} - \frac{1}{\mu_1} \vec{k}_r \times \vec{E}_{r_0} - \frac{1}{\mu_1} \vec{k}_e\ \times \vec{E}_{e_0} \bigg] = 0
	\end{equation*}
\end{enumerate}
}

\noindent
Wir zerlegen die Amplitude der einfallenden Welle in Anteile parallel und senkrecht zur Einfallsebene ($ x $-$ z $-Ebene).
\begin{align*}
\vec{E}_{e_0} &= \vec{E}_{e_{0_\parallel}} + \vec{E}_{e_{0_\perp}} \\
\vec{E}_{e_{0_\perp}} &= (\vec{e}_y \cdot \vec{E}_{e_0}) \cdot \vec{e}_y
\end{align*}
$ \Rightarrow \quad \vec{E}_e = \vec{E}_{e_\parallel} + \vec{E}_{e_\perp} $\\
ebenso für $ \vec{E}_r $ und $ \vec{E}_2 $\\[10pt]
\lcom{Den einen Fall betrachten wir genauer die anderen sind ähnlich und werden deswegen nicht genauer besprochen.}
\begin{enumerate}[I)]
	\item $ \vec{E}_e $ senkrecht zur Einfallsebene polarisiert:
	$$ \vec{E}_{e_0}  = E_{e_0} \vec{e}_y $$
	\begin{minipage}{.6\linewidth}
		Es folgt aus der Skizze:
		\begin{align*}
		\vec{k}_2 &= k_2 (\sin \theta_2 \vec{e}_x + \cos \theta_2 \vec{e}_z)\\
		\vec{k}_e &= k_e (\sin \theta_1 \vec{e}_x + \cos \theta_1 \vec{e}_z)\\
		\vec{k}_r &= k_r (\sin \theta_1 \vec{e}_x + \cos \theta_1 \vec{e}_z)
		\end{align*}
	\end{minipage}%
	\begin{minipage}{.4\linewidth}
		\flushright
		%t2:
		\begin{tikzpicture}
			\draw[->] (-2.5,0) -- (2.5,0) node[right] {$x$};
			\draw[->] (0,-2.5) -- (0,2.5) node[right] {$z$};
			\draw[thick,->] (0,0) -- (30:2.5cm) node[anchor=south west] {$\vec{k}_2$}; 
			\centerarc[](0,0)(30:90:.9);
			\node at (0.3,.5) {$\theta_2$};
			\draw[thick,->] (-135:2.5cm) node[anchor=north east] {$\vec{k}_e$} -- (-0.01,-0.01);
			\node at (-0.2,-0.5) {$\theta_e$};
			\centerarc[](0,0)(-90:-45:.9);
			\draw[thick,->] (0,0) -- (-45:2.5cm) node[anchor=north west] {$\vec{k}_r$};
			\centerarc[](0,0)(-135:-90:.9);
			\node at (0.25,-0.5) {$\theta_r$};
			%E und B Felder
			\coordinate (2) at (30:1.5);
			\draw[black!20!blue,thick,->] (2) -- node[anchor=south west] {$ \vec{B}_2 $} ++(120:1cm);
			\draw[black!15!red,thick] (2) circle (.15cm);
			\draw[black!15!red,thick] ($ (2) + (-135+30:.15) $) -- ++(45+30:.3);
			\draw[black!15!red,thick] ($ (2) + (135+30:.15) $) -- ++(-45+30:.3);
			\node[black!15!red,anchor=north west] at (2) {$ \vec{E}_2 $};
			%
			\coordinate (e) at (-135:1.5);
			\draw[black!20!blue,thick,->] (e) -- node[anchor=north east] {$ \vec{B}_e $} ++(135:1cm);
			\draw[black!15!red,thick] (e) circle (.15cm);
			\draw[black!15!red,thick] ($ (e) + (-135+45:.15) $) -- ++(45+45:.3);
			\draw[black!15!red,thick] ($ (e) + (135+45:.15) $) -- ++(-45+45:.3);
			\node[black!15!red,anchor=north west] at (e) {$ \vec{E}_e $};
			%
			\coordinate (r) at (-45:1.5);
			\draw[black!20!blue,thick,->] (r) -- node[anchor=north west] {$ \vec{B}_r $} ++(45:1cm);
			\draw[black!15!red,thick] (r) circle (.15cm);
			\draw[black!15!red,thick] ($ (r) + (-135+45:.15) $) -- ++(45+45:.3);
			\draw[black!15!red,thick] ($ (r) + (135+45:.15) $) -- ++(-45+45:.3);
			\node[black!15!red,anchor=north east] at (r) {$ \vec{E}_r $};
		\end{tikzpicture}
	\end{minipage}%
	\\
	Aus den Stetigkeitsbedingunen folgt:  % vielleicht muss eins auch i) sein
	\begin{equation*}
	\vec{E}_{r_0} = E_{r_0} \vec{e}_y \qquad \vec{E}_{2_0} = E_{2_0} \vec{e}_y
	\end{equation*}
	\begin{enumerate}
		\item[ii)] $ \vec{t} = \vec{e}_y $
		\begin{equation*}
		\Rightarrow \quad \rmbox{0 = E_{2_0} - E_{e_0} - E_{r_0}}
		\end{equation*}
		\item[iii)] Diese Bedingung liefert im wesentlichen das Brechungsgesetz
		\item[iv)] $ \vec{t} = \vec{e}_x: $
		%
		%
		%
		% make me nicer:
		%
		%
		%
		\begin{equation*}
		0 = \frac{1}{\mu_2} \ub{\vec{e}_x \cdot (\vec{k}_2 \times \vec{E}_{2_0})}_{\substack{= - \vec{k}_2 \cdot (\vec{e}_x \times \vec{E}_{2_0}) \\ = - \vec{k}_2 \cdot \vec{e}_z E_{2_0} \\ = - E_{2_0} k_2 \cos \theta_2}} - \frac{1}{\mu_1} \ub{\vec{e}_x \cdot (\vec{k}_e \times \vec{E}_{e_0})}_{= - E_{e_0} k_e \cos \theta_1} - \frac{1}{\mu_1} \ub{\vec{e}_x \cdot (\vec{k}_r \times \vec{E}_{r_0})}_{= E_{r_0} k_r \cos \theta_1}
		\end{equation*}
		\begin{equation*}
		\Rightarrow \quad \rmbox{0 = - \frac{k_2}{\mu_2} E_{2_0} \cos \theta_2 + \frac{k_e}{\mu_1} E_{e_0} \cos \theta_1 - \frac{k_r}{\mu_1} E_{r_0} \cos \theta_1}
		\end{equation*}
	\end{enumerate}
	$ E_{r_0} = E_{2_0} - E_{e_0} $\\[5pt]
	$ k_e = k_r = k_1 = \frac{\omega}{v_1} = \omega \sqrt{\epsilon_1 \mu_1} $\\
	$ k_2 = \omega \sqrt{\epsilon_2 \mu_2} $
	\begin{equation*}
	\Rightarrow \quad 0 = - \omega \sqrt{\frac{\epsilon_2}{\mu_2}} E_{2_0} \cos \theta_2 + \omega \sqrt{\frac{\epsilon_1}{\mu_1}} E_{e_0} \cos \theta_1 - \omega \sqrt{\frac{\epsilon_1}{\mu_1}} (E_{2_0} - E_{e_0}) \cos \theta_1
	\end{equation*}
	Umformen mit Brechungsindex $ n_i = \sqrt{\epsilon_{r_i} \mu_{e_i}} $
	\begin{equation*}
	\Rightarrow \quad \left(\frac{E_{2_0}}{E_{e_0}}\right)_{\perp} = \frac{2\sqrt{\frac{\epsilon_1}{\mu_1}} \cos \theta_1}{\sqrt{\frac{\epsilon_1}{\mu_1}} \cos \theta_1 + \sqrt{\frac{\epsilon_2}{\mu_2}} \cos \theta_2} = \frac{2 n_1 \cos \theta_1}{n_1 \cos \theta_1 + \frac{\mu_1}{\mu_2} n_2 \cos \theta_2}
	\end{equation*}
	Einfalls- und Ausfallswinkel können wir mit dem Brechungsgesetz ineinander umformen: $ \cos \theta_2 = \sqrt{1 - \sin^2 \theta_2} = \sqrt{1 - \frac{n_1^2}{n_2^2} \sin^2 \theta_1} $ und
	$ E_{r_0} = E_{2_0} - E_{e_0} $\\[5pt]
	\frbox{Senkrechter Einfall}{
	\begin{equation*}
	\Rightarrow \quad \left(\frac{E_{2_0}}{E_{e_0}}\right)_{\perp} = \frac{n_1 \cos \theta_1 - \frac{\mu_1}{\mu_2} n_2 \cos \theta_2}{n_1 \cos \theta_1 + \frac{\mu_1}{\mu_2} n_2 \cos \theta_2}
	\end{equation*}
	}
	%
	%
	%
	\pagebreak
	%
	%
	%
	\item  $ \vec{E}_e $ parallel zur Einfallsebene\\
	\frbox{Paralleler Einfall}{
	\begin{equation*}
	\left(\frac{E_{2_0}}{E_{e_0}}\right)_{\parallel} = \frac{2 n_1 \cos \theta_1}{\frac{\mu_1}{\mu_2}n_2 \cos \theta_1 + n_1 \cos \theta_2}
	\end{equation*}
	\begin{equation*}
	\left(\frac{E_{r_0}}{E_{e_0}}\right)_{\parallel} = \frac{\frac{\mu_1}{\mu_2} n_2 \cos \theta_1 - n_1 \cos \theta_2}{\frac{\mu_1}{\mu_2} n_2 \cos \theta_1 + n_1 \cos \theta_2}
	\end{equation*}
	}
	\begin{center}
		%t3:
		\begin{tikzpicture}[scale=.9]
		\draw[->] (-2.5,0) -- (2.5,0) node[right] {$x$};
		\draw[->] (0,-2.5) -- (0,2.5) node[right] {$z$};
		\draw[thick,->] (0,0) -- (30:2.5cm) node[anchor=south west] {$\vec{k}_2$}; 
		\centerarc[](0,0)(30:90:.9);
		\node at (0.3,.5) {$\theta_2$};
		\draw[thick,->] (-135:2.5cm) node[anchor=north east] {$\vec{k}_e$} -- (-0.01,-0.01);
		\node at (-0.2,-0.5) {$\theta_e$};
		\centerarc[](0,0)(-90:-45:.9);
		\draw[thick,->] (0,0) -- (-45:2.5cm) node[anchor=north west] {$\vec{k}_r$};
		\centerarc[](0,0)(-135:-90:.9);
		\node at (0.25,-0.5) {$\theta_r$};
		%E und B Felder
		\coordinate (2) at (30:1.5);
		\draw[black!20!red,thick,->] (2) -- node[anchor=south west] {$ \vec{E}_2 $} ++(120:1cm);
		%
		\coordinate (e) at (-135:1.5);
		\draw[black!20!red,thick,->] (e) -- node[anchor=north east] {$ \vec{E}_e $} ++(135:1cm);
		%
		\coordinate (r) at (-45:1.5);
		\draw[black!20!red,thick,->] (r) -- node[anchor=north west] {$ \vec{E}_r $} ++(45:1cm);
		\end{tikzpicture}
	\end{center}
\end{enumerate}
\emph{Bemerkungen:}
\begin{enumerate}[i)]
	\item \textbf{Fresnelsche Formeln}: da meist gilt $ \frac{\mu_1}{\mu_2} \approx 1 $
	\item Senkrechter Einfall:\\
	\begin{minipage}{.6\linewidth}
		\begin{equation*}
		\frac{E_{2_0}}{E_{e_0}} = \frac{2n_1}{n_1 + n_2} \qquad \frac{E_{r_0}}{E_{e_0}} = \frac{n_1 - n_2}{n_1 + n_2}
		\end{equation*}
		für $ n_1 = n_2 $ erfüllt $ \checkmark $\\[5pt]
		für $ n_2 \gg n_1 : \qquad \frac{E_{2_0}}{E_{e_0}} = \frac{2}{1 + \frac{n_2}{n_1}} $ erfüllt $ \checkmark $
	\end{minipage}%
	\begin{minipage}{.4\linewidth}
		\flushright
		%t4:
		\begin{tikzpicture}
			\draw[->] (-2.5,0) -- (2.5,0) node[anchor=north west] {$ x $};
			\draw[thick,->] (0,0) -- node[right] {$ \vec{k}_2 $} (0,1.5);
			\draw[thick,->] (-0.1,-1.5) -- node[left] {$ \vec{k}_e $} (-0.1,0);
			\draw[thick,->] (0.1,0) -- node[right] {$ \vec{k}_r $} (0.1,-1.5);
		\end{tikzpicture}
	\end{minipage}%
	\\
	\item \textbf{Brewster-Winkel:}
	\begin{equation*}
	\left(\frac{E_{r_0}}{E_{e_0}}\right)_{\parallel} = \frac{n_2 \cos \theta_1 - n_1 \cos \theta_2}{n_2 \cos \theta_1 + n_1 \cos \theta_2}
	\end{equation*}
	Wir wollen für $ \theta_B:  \quad \left(\frac{E_{r_0}}{E_{e_0}}\right)_\parallel = 0 $ also Zähler = 0:
	\begin{align*}
	n_2 \cos \theta_1 &= n_1 \cos \theta_2
	&= n_1 \sqrt{1 - \sin^2 \theta_2}
	&= n_1 \sqrt{1 - \frac{n_1^2}{n_2^2} \sin^2 \theta_1}
	\end{align*}
	\begin{equation*}
	\frac{n_2}{n_1} = \frac{\sqrt{1 - \frac{n_1^2}{n_2^2} \sin^2 \theta_B}}{\cos \theta_B}
	\end{equation*}
	$ \Rightarrow \quad \tan \theta_B = \frac{n_2}{n_1} $\\[5pt]
	\begin{minipage}{.5\linewidth}
		\emph{Beispiel:}\\
		Luft $ \rightarrow $ Glas $ \frac{n_2}{n_1} \approx 1{,}5 \quad \rightarrow \quad \theta_B \approx 56^\circ $\\[5pt]
		Es gilt: $ \theta_B + \theta_w = \frac{\pi}{2} $
	\end{minipage}%
	\begin{minipage}{.5\linewidth}
		\centering
		%t5:
		\begin{tikzpicture}
			\draw[->] (-2,0) -- (2,0) node[right] {$x$};
			\draw[->] (0,-2) -- (0,2) node[right] {$z$};
			\draw[thick,->] (0,0) -- (60:1.7cm) node[anchor=south west] {$\vec{k}_2$}; 
			\centerarc[](0,0)(60:90:1.2);
			\node at (0.2,.85) {$\theta_2$};
			\draw[thick,->] (-150:1.7cm) node[anchor=north east] {$\vec{k}_e$} -- (-0.01,-0.01);
			\draw[thick,->] (0,0) -- (-30:1.7cm) node[anchor=north west] {$\vec{k}_r$};
			\centerarc[](0,0)(210:270:1);
			\node at (-0.3,-0.6) {$\theta_e$};
			\centerarc[](0,0)(270:330:1);
			\node at (0.4,-0.6) {$\theta_r$};
			\centerarc[thick,black!20!red](0,0)(-30:60:.7cm);
			\node[circle,fill=black!20!red,inner sep=1pt,minimum size=1pt] at (.4,.1) {};
		\end{tikzpicture}
	\end{minipage}%
	\\
\end{enumerate}

\subsubsection{Einschub}

Nun noch eine Erklärung die etwas umständlicher ist: (Zu senkrechtem Einfall)\\
\begin{minipage}{.5\linewidth}
	$ \vec{E}_{e_0} = E_{e_0} \vec{e}_y $\\
	$ \vec{E}_{r_0} = E_{r_0} \vec{e}_y \quad \vec{E}_{2_0} = E_{2_0} \vec{e}_y $\\[5pt]
\end{minipage}%
\begin{minipage}{.5\linewidth}
	\flushright
	%t6:
	\begin{tikzpicture}[scale=.9]
		\draw[->] (-2.5,0) -- (2.5,0) node[right] {$x$};
		\draw[->] (0,-2.5) -- (0,2.5) node[right] {$z$};
		\draw[thick,->] (0,0) -- (30:2.5cm) node[anchor=south west] {$\vec{k}_2$}; 
		\centerarc[](0,0)(30:90:.9);
		\node at (0.3,.5) {$\theta_2$};
		\draw[thick,->] (-135:2.5cm) node[anchor=north east] {$\vec{k}_e$} -- (-0.01,-0.01);
		\node at (-0.2,-0.5) {$\theta_e$};
		\centerarc[](0,0)(-90:-45:.9);
		\draw[thick,->] (0,0) -- (-45:2.5cm) node[anchor=north west] {$\vec{k}_r$};
		\centerarc[](0,0)(-135:-90:.9);
		\node at (0.25,-0.5) {$\theta_r$};
		%E und B Felder
		\coordinate (2) at (30:1.5);
		\draw[black!20!red,thick] (2) circle (.15cm);
		\draw[black!20!red,thick] ($ (2) + (-135+30:.15) $) -- ++(45+30:.3);
		\draw[black!20!red,thick] ($ (2) + (135+30:.15) $) -- ++(-45+30:.3);
		\node[black!20!red,anchor=south east] at (2) {$ \vec{E}_2 $};
		%
		\coordinate (e) at (-135:1.5);
		\draw[black!20!red,thick] (e) circle (.15cm);
		\draw[black!20!red,thick] ($ (e) + (-135+45:.15) $) -- ++(45+45:.3);
		\draw[black!20!red,thick] ($ (e) + (135+45:.15) $) -- ++(-45+45:.3);
		\node[black!20!red,anchor=south east] at (e) {$ \vec{E}_e $};
		%
		\coordinate (r) at (-45:1.5);
		\draw[black!20!red,thick] (r) circle (.15cm);
		\draw[black!20!red,thick] ($ (r) + (-135+45:.15) $) -- ++(45+45:.3);
		\draw[black!20!red,thick] ($ (r) + (135+45:.15) $) -- ++(-45+45:.3);
		\node[black!20!red,anchor=south west] at (r) {$ \vec{E}_r $};
	\end{tikzpicture}
\end{minipage}%
\\
\begin{enumerate}
	\item[ii)] $ \vec{t} = \vec{e}_x: $
	\begin{equation*}
	0 = \vec{e}_x \cdot (\vec{E}_{2_0} - \vec{E}_{r_0} - \vec{E}_{e_0})
	\end{equation*}
	\begin{equation*}
	\Rightarrow \quad E_{2_{0_x}} = E_{r_{0_x}}
	\end{equation*}
	\item[i)]
	\begin{equation*}
	0 = \vec{e}_z \cdot (\epsilon_2 \vec{E}_{2_0} - \epsilon_1 \vec{E}_{r_0} - \epsilon_1 \vec{E}_{e_0})
	\end{equation*}
	\begin{equation*}
	\Rightarrow \quad \epsilon_2 E_{2_{0_z}} = \epsilon_1 E_{r_{0_z}}
	\end{equation*}
	$ \vec{k}_e, \vec{k}_r, \vec{k}_2 $ in $ x $-$ z $-Ebene
	\begin{equation*}
	\vec{k}_r = \begin{pmatrix}
	k_{r_x} \\ 0 \\ k_{r_z}
	\end{pmatrix} \qquad \vec{k}_2 = \begin{pmatrix}
	k_{2_x} \\ 0 \\ k_{2_z}
	\end{pmatrix}
	\end{equation*}
	\begin{align*}
	0 &= \vec{k}_r \cdot \vec{E}_{r_0} = k_{r_x} E_{r_{0_x}} + k_{r_z} E_{r_{0_z}} \quad \leftarrow \quad 0 = k_{r_x} E_{2_{0_x}} + k_{r_z} \frac{\epsilon_2}{\epsilon_1} E_{2_{0_z}} \\
	0 &= \vec{k}_2 \cdot \vec{E}_{2_0} = k_{2_x} E_{e_{0_x}} + k_{2_z} E_{2_{0_z}} \quad \rightarrow \quad E_{2_{0_x}} = - \frac{k_{r_z}}{k_{r_x}} \frac{\epsilon_2}{\epsilon_1} E_{2_{0_z}}
	\end{align*}
	\begin{equation*}
	\Rightarrow \quad 0 = \custo{\neq}{\left[- \cancel{k_{z_x}} \frac{k_{r_z}}{\cancel{k_{r_x}}} \frac{\epsilon_2}{\epsilon_1} + k_{2_z}\right]}{0} \equalto{E_{2_{0_z}}}{0}
	\end{equation*}
	$ k_{2_x} = k_{r_x} $
\end{enumerate}

% Vorlesung 17.01.18

\section{Aspekte der Absorption und Dispersion elektromagnetischer Felder}

bisher galt: freie Ausbreitung der EM-Wellen\\
in Medien gilt:\\[5pt]
\begin{minipage}{.5\linewidth}
	\textbf{Absorption:} 
\end{minipage}%
\begin{minipage}{.5\linewidth}
	$ \ \, $ und \textbf{Dispersion:}
\end{minipage}%
\\
\begin{minipage}{.5\linewidth}
	\flushleft
	%t1:
	\begin{tikzpicture}[xscale=.75,scale=.8]
		\fill[gray!10!] (0,-1.5) rectangle (6.3,1.5);
		\fill[pattern=north east lines,pattern color = gray!50!] (0,-1.5) rectangle (6.3,1.5);
		\draw[thick] (0,-1.5) -- (0,1.5);
		\draw[domain=-4.725:0, samples=100, thick] plot (\x, {sin((\x * 4) r)});
		\draw[domain=0:6.3, samples=100, thick] plot (\x, {sin((\x * 4) r) * exp(-\x * 0.4)});
	\end{tikzpicture}
\end{minipage}%
\begin{minipage}{.5\linewidth}
	\flushright
	%t2:
	\begin{tikzpicture}[scale=1.05]
		\draw[fill=gray!10!] (0,0) -- ++(-60:2cm) -- ++(-180:2cm) -- cycle;
		\draw[<-, thick] (-120:1cm) -- ++(-160:2cm) node[left] {$ h(\omega) , \epsilon(\omega) $};
		\foreach \x\a\l in {.4/5/1, .5/-7.5/2, .6/-20/3}
		\draw[->, thick] ($ (0,0)!\x!(-60:2cm) $) -- ++(\a:2cm) node[right] {$ \omega_{\l} $};
	\end{tikzpicture}
\end{minipage}%

\subsection{Elektromagnetische Wellen in elektrischen Leitern}

bisher galt: \textbf{Isolator/Vakuum} freie Ladungen und freie Ströme = 0 $\qquad \rho_f = 0 = \vec{j}_f $\\
im Leiter gilt: \textbf{Ohmsche Leiter} $ \vec{j} = \sigma \vec{E}  \qquad \rho = 0 \qquad \sigma : $ Leitfähigkeit
\begin{equation*}
\vec{D} = \epsilon \vec{E} \qquad \vec{B} = \mu \vec{H}
\end{equation*}
Maxwell-Gleichungen:
\begin{align*}
\vabla \cdot \vec{E} &= 0 \qquad \qquad \qquad \ \, \vabla \cdot \vec{B} = 0\\
\vabla \times \vec{E} + \prt{\vec{B}}{t} &= 0 \qquad \vabla \times \vec{B} - \mu \epsilon \prt{\vec{E}}{t} = \mu \vec{j} = \mu \sigma \vec{E}
\end{align*}
\emph{Bemerkung:}\\
$ \vabla \cdot \vec{E} = \frac{1}{\epsilon_0} \rho $\\
$ \prt{\rho}{t} = - \vabla \vec{j} = - \sigma \vabla \vec{E} = - \frac{\sigma}{\epsilon} \rho $\\
$ \rho(\vec{r},t) = e^{-\frac{\sigma}{\epsilon} t} \rho(\vec{r},t=0) $\\
somit ist die Annahme $ \rho = 0 $ gerechtfertigt.\\[15pt]
\begin{align*}
0 &= \vabla \times (\vabla \times \vec{E}) + \prt{}{t} (\vabla \times \vec{B}) \\
&= \vabla (\cancel{\vabla \vec{E}}) - \Delta \vec{E} + \prt{}{t} (\mu \epsilon \prt{\vec{E}}{t} + \mu \sigma \vec{E})
\end{align*}
\begin{equation*}
\Rightarrow \quad \rmbox{\Delta \vec{E} - \mu \epsilon \prt{^2 \vec{E}}{t^2} - \mu \sigma \prt{\vec{E}}{t} = 0}
\end{equation*}
Analog kann man auch nach $ \Delta \vec{B} $ auflösen:
\begin{equation*}
\Rightarrow \quad \rmbox{\Delta \vec{B} - \mu \epsilon \prt{^2 \vec{B}}{t^2} - \mu \sigma \prt{\vec{B}}{t} = 0}
\end{equation*}
Diese Beiden Gleichungen nennt man auch \textbf{Telegraphen-Gleichungen}.\\[5pt]
Ansatz zur Lösung: monochromatische, ebene Welle $ \qquad \vec{k} = k \vec{e}_z $
\begin{align*}
\vec{E}(\vec{r},t) &= \vec{E}_0 e^{i(\vec{k} \cdot \vec{r} - \omega t)} = \vec{E}_0 e^{i(k r - \omega t)}\\
\vec{B}(\vec{r},t) &= \vec{B}_0 e^{i(k r - \omega t)}
\end{align*}
\begin{equation*}
\Rightarrow \quad 0 = - k^2 + \mu \epsilon \omega^2 + i \omega \mu \sigma = - k^2 + \mu \omega^2 \ub{( \epsilon + i \frac{\sigma}{\omega})}_{\eqdef \eta}
\end{equation*}
$ \eta : $ \textbf{verallgemeinerte Dielektrizitätskonstante}
\frbox{Dispersionsrelation im leitenden Medium ($ k(\omega) $)}{
\begin{equation*}
\Rightarrow \quad k^2 = \mu \eta \omega^2 = \mu \omega^2 \left(\epsilon + i \frac{\sigma}{\omega}\right)
\end{equation*}
}
\noindent
Hieraus folgt, dass $ \vec{k} $ komplex ist und einen Imaginärteil haben muss: $ k = k_{R} + i k_{I} $
\begin{align*}
\Rightarrow \quad k^2 &= k_R^2 - k_I^2 + 2 i k_R k_I\\
&\overset{!}{=} \mu \epsilon \omega^2 + i \mu \sigma \omega
\end{align*}
\begin{equation*}
\Rightarrow \quad k_{R/I} = \omega \sqrt{\frac{\epsilon \mu}{2}} \left[\sqrt{1 + \left(\frac{\sigma}{\epsilon \omega}\right)^2} \pm 1\right]^{\frac{1}{2}}
\end{equation*}
\begin{align*}
\vec{E} &= \vec{E}_0 e^{i(k z - \omega t)} \\
&= \vec{E}_0 e^{i((k_R + i k_I) z - \omega t)} \\
&= \vec{E}_0 e^{-k_I z} e^{i(k_R z - \omega t)}
\end{align*}
\begin{minipage}{.4\linewidth}
	\textbf{Eindringtiefe:} $ d = \frac{1}{k_I} $\\[10pt]
	typische Werte:\\
	Kupfer: $ \sigma = 5{,}81 \cdot 10^{7} \ \frac{1}{\Omega \tx{m}} $\\
	\begin{tabular}{c|c}
		$ \nu $ & $ d $ \\
		\hline
		1 MHz & 66 $ \mu $m \\
		$ 6 \cdot 10^{14} $ Hz & 2,7 nm
	\end{tabular}\\[15pt]
\end{minipage}%
\begin{minipage}{.6\linewidth}
	\flushright
	%t3:
	\begin{tikzpicture}[xscale=.8]
		\fill[gray!10!] (0,-1.5) rectangle (6.3,1.5);
		\fill[pattern=north east lines,pattern color = gray!50!] (0,-1.5) rectangle (6.3,1.5);
		\draw[thick,->] (0,-1.7) -- (0,1.7);
		\draw[thick,->] (-5,0) -- (6.7,0);
		\draw[domain=-4.725:0, samples=100, thick] plot (\x, {sin((\x * 4) r)});
		\draw[domain=0:6.3, samples=100, thick] plot (\x, {sin((\x * 4) r) * exp(-\x * 0.5)});
		\draw[domain=0:6.3, samples=100, thick, dashed] plot (\x, {exp(- \x * 0.5)});
		\draw (2.5,.5) to[out=40,in=-180] (3.5,1) node[right, fill=white, opacity=.8, rounded corners=5pt] {$ \propto e^{- i k z} $};
	\end{tikzpicture}
\end{minipage}%
\\
\emph{Bemerkung:}
\lcom{Man kann Zeigen, dass, anders als im Vakuum, im Medium das $ \vec{B} $-Feld Phasenverschoben zum $ \vec{E} $-Feld ist. Wir werden hierauf aus Zeitgründen nicht genauer eingehen.}

\subsection{Frequenzabhängigkeit der Dielektrizitätskonstanten - Dispersion}

\subsubsection{Modell für $ \epsilon(\omega) $}

Wir betrachten hier nicht magnetisierbare Medien. Es soll gehen $ \mu_r \approx 1 $.
\begin{equation*}
\epsilon = \epsilon_0 (1 + \chi_e) \qquad \vec{P} = \epsilon_0 \chi_e \vec{E} = \frac{\tx{Dipole}}{\tx{Volumen}} = N \langle \vec{p} \rangle
\end{equation*}
\lcom{Die Polarisation ist die mittlere Dipol-dichte pro Volumen.}\\[5pt]
Die Verschiebungsolarisation ist:
\begin{equation*}
\langle \vec{p} \rangle = \alpha \vec{E}
\end{equation*}
$ \alpha : $ atomare Polarisierbarkeit
\begin{equation*}
\chi_e = \frac{N \alpha}{\epsilon_0}
\end{equation*}
$ \chi_e : $ elektrische Suszeptibilität\\
All diese Gleichungen würde man für langsam schwingende Felder erhalten also für $ \omega \to 0 $.\\
\lcom{Bei Tieffrequenten Wellen können die Ladungen den Feldern folgen und schwächen diese dadurch ab. Bei Hochfrequenten Wellen können die Ladungen den Feldern nicht mehr folgen und der Effekt ist ein anderer.}

\subsubsection{Klassisches Oszillatormodell für die Bindung von Elektronen in Atomen}

\begin{center}
	%t4:
	\begin{tikzpicture}
		\draw[domain=-3.15:3.15, samples=250, thick, draw=gray!70!] plot(\x, {1.5 * sin(\x * 8 r)});
		\node at (2,-.7) {$ \vec{E}(\vec{r},t) $};
		\coordinate (p) at (-1.5,-.8);
		\coordinate (m) at (1.5,.8);
		\draw[decorate,decoration={aspect=.5, segment length=4mm, amplitude=.5cm,coil},thick] ($ (p) + (30:0.15) $) -- ($ (m) + (30:0.15) $);
		\node[circle,draw=black,fill=black!10!red] at (p) {$ + $};
		\node[circle,draw=black,fill=black!10!blue] at (m) {$ - $};
		\draw[thick,->] ($ (p) + (120:.8) $) -- node[above,xshift=2pt] {$ \vec{r} $} ($ (m) + (120:.8) $);
	\end{tikzpicture}
\end{center}
$ \omega_0 : $ Frequenz\\
$ \gamma : $ Dämpfung\\
Dies können wir durch einen getriebenen, gedämpften, harmonischen Oszillator darstellen:
\begin{equation*}
m(\ddot{\vec{r}} + \gamma \dot{\vec{r}} + \omega_0^2 \vec{r}) = q \vec{E}(\vec{r},t)
\end{equation*}
$ \rightarrow $ Dipolmoment: $ \vec{p} = q \vec{r} \qquad q = -e $
\begin{equation*}
\Rightarrow \quad m(\ddot{\vec{p}} + \gamma \dot{\vec{p}} + \omega_0^2 \vec{p}) = e^2 \vec{E}
\end{equation*}
\textbf{statischer Grenzfall:} $ \vec{E}(\vec{r},t) = \vec{E}(\vec{r}) $\\
$ \Rightarrow \quad \dot{\vec{p}} = 0 = \ddot{\vec{p}} $\\
\begin{equation*}
\Rightarrow \quad \vec{p} = \frac{e^2 \vec{E}}{m \omega_0^2} = \alpha \vec{E} \quad \rightarrow \quad \alpha = \frac{e^2}{m \omega_0^2}
\end{equation*}
\begin{equation*}
\Rightarrow \quad \chi_e = \frac{N e^2}{\epsilon_0 m \omega_0^2}
\end{equation*}
\textbf{zeitabhängiger Fall:} $ \vec{E} = \vec{E}_0 e^{-i \omega t} \qquad e^{ikz} = e^{i \frac{2 \pi}{\lambda} z} $\\[5pt]
Mechanik:\\
$ \Rightarrow \quad \vec{p}(t) : $ erzwungene Schwingung mit Frequenz $ \omega $
\begin{equation*}
\vec{p}(t) = \vec{p}_0 e^{- i \omega t}
\end{equation*}
\begin{equation*}
m ( - \omega^2 - i \gamma \omega + \omega_0^2) \vec{p}_0 e^{-i \omega t} = e^2 \vec{E}_0 e^{-i \omega t}
\end{equation*}
\begin{equation*}
\Rightarrow \quad \vec{p}_0 = \ub{\frac{e^2}{m} \frac{1}{\omega_0^2 - \omega^2 - i \gamma \omega}}_{\alpha (\omega)} \vec{E}_0
\end{equation*}
\begin{align*}
\Rightarrow \quad \chi_e(\omega) &= \frac{N \alpha(\omega)}{\epsilon_0} \\
&= \frac{N e^2}{\epsilon_0 m} \frac{1}{\omega_0^2 - \omega^2 - i \gamma \omega}
\end{align*}
\begin{align*}
\Rightarrow \quad \epsilon(\omega) &= \epsilon_0 \epsilon_r(\omega) \\
&= \epsilon_0 (1 + \chi_e(\omega))
\end{align*}
\frbox{Dispersionsrelation im leitenden Medium $ \epsilon_r(\omega) $}{
\begin{equation*}
\epsilon_r(\omega) = 1 + \frac{N e^2}{\epsilon_0 m} \frac{1}{\omega_0^2 - \omega^2 - i \gamma \omega}
\end{equation*}
}
\noindent
Realteil:
$$ \Re(\epsilon_r(\omega)) = 1 + \frac{Ne^2}{\epsilon_0 m} \frac{\omega_0^2 - \omega^2}{(\omega_0^2 - \omega^2)^2 + \gamma^2 \omega^2} $$
Imaginärteil:
$$ \Im(\epsilon_r(\omega)) = \frac{N e^2}{\epsilon_0 m} \frac{\gamma \omega}{(\omega_0^2 - \omega^2)^2 + \gamma^2 \omega^2} \quad \ \  $$
\begin{minipage}{.5\linewidth}
	%t5:
	\begin{tikzpicture}[scale=.7]
		\draw[->] (-.1,0) -- (8,0) node[anchor=north west] {$ \omega $};
		\draw[->] (0,-.1) node[below] {$ 0 $} -- (0,5) node[left] {$ \Re(\epsilon_r(\omega)) $};
		\coordinate (R) at (0,4);
		\coordinate (w) at (3,0);
		\node[left] at (R) {$ \Re(\epsilon_r(0)) $};
		\node[below] at ($ (w) + (-.1,0) $) {$ \omega_0 $};
		\draw ($ (R) + (-.1,0) $) -- ++(.2,0);
		\draw ($ (w) + (0,-.1) $) -- ++(0,.2);
		\draw (-.1,1.5) node[left] {1} -- ++(.2,0);
		%\draw[thick,looseness=.8] (R) to[out=0,in=-135] (1.5,2.8) to[out=45,in=-180] (2.5,3.8) to[out=0,in=100] (3,1.75);
		%\draw[thick,looseness=1] (R) to[out=0,in=-135] (2.2,5) to[out=45,in=100] (2.8,5.8) to[out=-80,in=100] (3.2,0.2);
		%\draw[domain=0:5, samples=100, thick] plot (\x, {(10 * (\x)) / ((4 - (\x)^2)^2 + 2 * (\x)^2))});
		\draw[domain=0:8, samples=100, thick] plot (\x, {0.5 + 1 +  10 * (  4 - (\x * 0.668)^2) / ((4 - (\x * 0.668)^2)^2+ 2 * (\x * 0.668)^2)});
		\draw[thick,dashed] (0,1.5) -- (8,1.5);
		\draw[thick,dashed] (w) -- ++(0,2);
		\draw[dashed] (1.7,0) -- ++(0,5);
		\draw[dashed] (3.7,0) -- ++(0,5);
		\node at (.9,5.5) {\small normal};
		\node at (2.7,6) {\small anormal};
		\node at (4.5,5.5) {\small normal};
		\node at (7,5.8) {\small Dispersion};
	\end{tikzpicture}
	\begin{comment} % alternative:
	\begin{tikzpicture}
		\draw[scale=1.5,red,line width=1,domain=-2:2, samples=500] plot (\x, {((2 * \x) / (\x - 10  * ((\x)^2) - 0.25))});
		\coordinate (a) at (-3,-1.5);
		\coordinate (b) at (0,0);
		% Koordianten Achsen
		\draw[->] (a) -- ++(6,0) node[right] {$\omega$};
		\draw[->] (a) -- ++(0,3) node[left] {$Re(\epsilon(\omega)$};
		\node[left] at ($(a) + (0,1.8)$) {$Re(\epsilon(0)$};
		%Unterteilung
		\draw[dashed] ($ (a) + (0,1.36) $) node[left] {$1$} -- ++(6,0);
		\draw[dashed] (0,-1.6) node[below] {$\omega_0$} -- (0,0);
		\draw[dashed] ($ (b) + (0.25,-1.8) $)  -- ++(0,3);
		\draw[dashed] ($ (b) + (-0.25,-1.8) $) -- ++(0,3);
		%Beschriftung
		\node at (-2,1) {normal};
		\node at (2.5,1) {normale Dispersion};
		\node at (0,1.5) {anormal};
	\end{tikzpicture}\\
	\end{comment}
	\vspace{5pt}
\end{minipage}%
\begin{minipage}{.5\linewidth}
	\flushright
	\vspace{5pt}
	%t6:
	\begin{tikzpicture}[scale=.7]
		\draw[->] (-.1,0) -- (8,0) node[anchor=north west] {$ \omega $};
		\draw[->] (0,-.1) node[below] {$ 0 $} -- (0,5) node[anchor=south east] {$ \Im(\epsilon_r(\omega)) $};
		\coordinate (w) at (3,0);
		\node[below,yshift=-3pt] at ($ (w) + (-.1,0) $) {$ \omega_0 $};
		\draw ($ (w) + (0,-.1) $) -- ++(0,.2);
		%\draw[thick,looseness=.8] (R) to[out=0,in=-135] (1.5,2.8) to[out=45,in=-180] (2.5,3.8) to[out=0,in=100] (3,1.75);
		%\draw[thick,looseness=1] (R) to[out=0,in=-135] (2.2,5) to[out=45,in=100] (2.8,5.8) to[out=-80,in=100] (3.2,0.2);
		\draw[domain=0:8, samples=100, thick] plot (\x, {(13 * (\x * 0.63)) / ((4 - (\x * 0.63)^2)^2 + 2 * (\x * 0.63)^2))});
		%\draw[domain=0:8, samples=100, thick] plot (\x, {0.5 + 1 +  10 * (  4 - (\x * 0.668)^2) / ((4 - (\x * 0.668)^2)^2+ 2 * (\x * 0.668)^2)});
		\draw[dashed, thick] (w) -- ++(0,4);
	\end{tikzpicture}
\end{minipage}%
\\
\emph{Bemerkung:} \textbf{Dispersion}\\
Frequenzabhängigkeit von $ \epsilon = \epsilon (\omega) = \epsilon_R + i \epsilon_I $\\
Brechungsindex $ n = \sqrt{\epsilon_r \mu_r} = n (\omega) $\\
normale Dispersion: $ \prd{\epsilon_R}{\omega} > 0 $\\
anormale Dispersion: $ \prd{\epsilon_R}{\omega} < 0 $\\[5pt]
\begin{minipage}{.5\linewidth}
	\textbf{Verallgemeinert auf $ j $ Elektronen}\\
	\textbf{im Atom:}\\
	$ f_j : $ Elektronen mit Frequenz $ \omega_j , \gamma_j $
	\begin{equation*}
	\epsilon_r(\omega) = 1 + \frac{N e^2}{\epsilon_0 m} \sum_j \frac{f_j}{\omega_j^2 - \omega^2 - i \gamma_j \omega_j}
	\end{equation*}
\end{minipage}%
\begin{minipage}{.5\linewidth}
	\flushright
	%t7:
	\begin{tikzpicture}[yscale=.7,scale=.7]
		\draw[->] (-.1,0) -- (10,0) node[anchor=north west] {$ \omega $};
		\draw[->] (0,-.1) node[below] {$ 0 $} -- (0,8);
		
		\draw[domain=0:10, samples=200, thick] plot (\x, {3 + 1 +  10 * (4 - (\x)^2) / ((4 - (\x)^2)^2 + 2 * (\x)^2) +  10 * (25  - (\x)^2) / ((25 - (\x)^2)^2 + 2 * (\x)^2) + 10 * (64 - (\x)^2) / ((64 - (\x)^2)^2 + 2 * (\x)^2});
		%1 +  10 * (4 - (\x)^2) / ((4 - (\x)^2)^2 + 2 * (\x)^2)
		%1 +  10 * (4 - (\x)^2) / ((4 - (\x)^2)^2 + 2 * (\x)^2) +  15 * ( 49  - (\x)^2) / ((49 - (\x)^2)^2 + 2 * (\x)^2) + 10 * (144 - (\x)^2) / ((144 - (\x)^2)^2 + 2 * (\x)^2)
		
		\draw[domain=0:10, samples=200, thick] plot (\x, {(10 * (\x)) / ((4 - (\x)^2)^2 + 2 * (\x)^2) +  (15 * (\x)) / ((25 - (\x)^2)^2 + 2 * (\x)^2) +  (15 * (\x)) / ((64 - (\x)^2)^2 + 2 * (\x)^2)});
		%(10 * (\x)) / ((4 - (\x)^2)^2 + 2 * (\x)^2) +  (15 * (\x)) / ((49 - (\x)^2)^2 + 2 * (\x)^2) + (10 * (\x)) / ((144 − (\x)^2)^2 + 2 * (x)^2)
		%\draw[domain=0:8, samples=500, thick] plot (\x, {0.5 + 1 +  10 * (  4 - (\x * 0.668)^2) / ((4 - (\x * 0.668)^2)^2+ 2 * (\x * 0.668)^2)});
		
		\foreach \x in {1.9,5,8}
		\draw[thick,dashed] (\x,0) -- ++(0,8);
		\node at (10,1.5) {$ \epsilon_{I} $};
		\node at (10,4.5) {$ \epsilon_{R} $};
	\end{tikzpicture}
	\vspace{10pt}
\end{minipage}%
\\
\begin{minipage}{.5\linewidth}
	Die Quantenmechanik erklärt diese\\
	Sprünge durch Energieniveaus:\\
\end{minipage}%
\begin{minipage}{.5\linewidth}
	%t8:
	\begin{tikzpicture}[scale=.8]
		\foreach \y\l in {0/0, 2/1, 3/2}
		\draw[thick] (0,\y) node[left] {$ E_{\l} $} -- ++(2,0);
		\draw[->] (1,.1) -- ++(0,1.8);
		\draw[decorate, decoration={brace,amplitude=5pt,mirror,raise=2pt}, xshift=4pt,black] (2,0) -- node[right,xshift=4pt] {$ \hbar \omega = h f = E_1 - E_0 $} (2,2);
	\end{tikzpicture}
\end{minipage}

\subsection{Bedeutung der komplexen/frequenzabhängigen Dielektrizitäts- \texorpdfstring{\\}{zeilenumbruch} konstante}

Wellengleichung im Medium $ \quad \epsilon = \epsilon(\omega) \quad \mu \approx \mu_0 \quad \sigma = 0 $
\begin{equation*}
\Delta \vec{E} - \mu_0 \epsilon \prt{^2 \vec{E}}{t^2} = 0
\end{equation*}
\begin{equation*}
\qquad \qquad \qquad \qquad \qquad \qquad \vec{E} = \vec{E}_0 e^{i(kz - \omega t)} \begin{tikzpicture}[scale=.7]
\draw[thick, ->] (0,0) -- (1,0) -- (1,1);
\end{tikzpicture}\qquad \quad \Big| \quad - k^2 + \mu_0 \epsilon \omega^2 = 0 \ \
\end{equation*}
\begin{equation*}
\Rightarrow \quad k^2 = \mu_0 \epsilon \omega^2 = \mu_0 (\epsilon_R + i \epsilon_I) \omega^2
\end{equation*}
$ k = k_R + i k_I $
\begin{equation*}
k_R^2 - k_I^2 + 2 i k_R k_I = \mu_0 \epsilon_R \omega^2 + i \mu_0 \epsilon_I \omega^2
\end{equation*}
\begin{equation*}
k_{R/I} = \omega \sqrt{\frac{\mu_0 \epsilon_R}{2}} \left[\sqrt{1 + \left(\frac{\epsilon_I}{\epsilon_R}\right)^2} \pm 1\right]^{\frac{1}{2}}
\end{equation*}
\begin{equation*}
\epsilon_I \approx 0 \quad k_I \approx 0 \quad k_R \approx \omega \sqrt{\mu_0 \epsilon_R(\omega)}
\end{equation*}
\begin{minipage}{.5\linewidth}
	\begin{equation*}
	\Rightarrow \quad \vec{E}(\vec{r},t) = \vec{E}_0 e^{- k_I z} e^{i(k_R z - \omega t)}
	\end{equation*}
\end{minipage}%
\begin{minipage}{.5\linewidth}
	\flushright
	%t9:
	\begin{tikzpicture}[xscale=.75, scale=.8]
		\fill[gray!10!] (0,-1.5) rectangle (6.3,1.5);
		\fill[pattern=north east lines,pattern color = gray!50!] (0,-1.5) rectangle (6.3,1.5);
		\draw[thick] (0,-1.5) -- (0,1.5);
		\draw[domain=-4.725:0, samples=100, thick] plot (\x, {sin((\x * 4) r)});
		\draw[domain=0:6.3, samples=100, thick] plot (\x, {sin((\x * 4) r) * exp(-\x * 0.4)});
	\end{tikzpicture}
\end{minipage}%
\\
Ausbreitungsgeschwindigkeit:
$$ v = \frac{\omega}{k_R} = v(\omega) = \frac{c}{n(\omega)} $$

%Vorlesung 21.01. Vertretung von Prof Thoss

\section{Allgemeine Lösung der Maxwell-Gl. - Retardierte Potentiale}

Wir suchen Lösungen der Gleichungen:
\begin{align*}
\vabla \cdot \vec{B} &= 0 \qquad \qquad \quad \vabla \times \vec{E} + \prt{\vec{B}}{t} = 0\\
\vabla \cdot \vec{E} &= \frac{1}{\epsilon_0} \rho \qquad \qquad \vabla \times \vec{B} - \ub{\epsilon_0 \mu_0}_{\frac{1}{c^2}} \prt{\vec{E}}{t} = \mu_0 \vec{j}
\end{align*}
gegeben: $ f(\vec{r},t) , \vec{j}(\vec{r},t) $\\
gesucht: $ \vec{E}(\vec{r},t), \vec{B}(\vec{r},t) $\\[5pt]
zur formalen Lösung der Gleichungen verwenden wir die Potentiale $ \Phi , \vec{A} $ der Elektrodynamik
\begin{equation*}
\vec{B} = \vabla \times \vec{A} \qquad \qquad \vec{E} = - \vabla \Phi - \prt{\vec{A}}{t}
\end{equation*}
\textbf{Lorenzeichung:}
\begin{equation*}
\vabla \cdot \vec{A} + \frac{1}{c^2} \prt{\Phi}{t} = 0
\end{equation*}
Aus den inhomogenen Gleichungen erhalten wir die entkoppelten Wellengleichungen der Potentiale:
\begin{equation*}
\Delta \Phi - \frac{1}{c^2} \prt{^2 \Phi}{t^2} = - \frac{1}{\epsilon_0} \rho
\end{equation*}
\begin{equation*}
\Delta \vec{A} - \frac{1}{c^2} \prt{^2 \vec{A}}{t^2} = - \mu_0 \vec{j}
\end{equation*}
\emph{Nebenbemerkung:}
\begin{equation*}
\Delta \Psi (\vec{r},t) - \frac{1}{c^2} \prt{^2}{t^2} \Psi (\vec{r},t) \eqdef \square \Psi (\vec{r},t)
\end{equation*}
mit $ \square : $ \textbf{d'Alembert Operator}
\begin{equation*}
\square = \Delta - \frac{1}{c^2} \prt{^2}{t^2}
\end{equation*}
Aus der Elektro-/Magnetostatik ist bekannt:
\begin{equation*}
\prt{}{t} \Phi = 0 = \prt{}{t} \vec{A}
\end{equation*}
\begin{align*}
\Delta \Phi(\vec{r}) &= - \frac{1}{\epsilon_0} \rho \qquad \tx{ Poisson-Gleichung}\\
\Delta \vec{A}(\vec{r}) &= - \mu_0 \vec{j}
\end{align*}
Die Lösungen waren:
\begin{equation*}
\Phi(\vec{r}) = \Phi_{\tx{hom}}(\vec{r}) + \Phi_{\tx{inh}}(\vec{r})
\qquad \qquad
\tx{mit}
\qquad 
\Phi_{\tx{inh}}(\vec{r}) = \kq \int \dd^3 \vec{r}' \frac{\rho(\vec{r}')}{| \vec{r} - \vec{r}' |}
\end{equation*}
sowie für das Vektorpotential:
\begin{equation*}
\vec{A}(\vec{r}) = \vec{A}_{\tx{hom}} (\vec{r}) + \vec{A}_{\tx{inh}} (\vec{r})
\qquad \qquad 
\tx{mit}
\qquad
\vec{A}_{\tx{inh}} (\vec{r}) = \frac{\mu_0}{4 \pi} \int \dd ^3 \vec{r}' \frac{\vec{j}(\vec{r})}{| \vec{r} - \vec{r}' |}
\end{equation*}
und $ \Delta \vec{A}_{\tx{hom}}(\vec{r}) = 0 $\\[10pt]
Wir können die Diskussion auf das skalare Potential $ \Phi $ beschränken:
\begin{equation*}
\Delta \Phi - \frac{1}{c^2} \prt{^2}{t^2} \Phi = - \frac{1}{\epsilon_0} \rho
\end{equation*}
$ \Phi(\vec{r},t) = \Phi_{\tx{hom}} + \Phi_{\tx{inh}} $\\
wobei:\\
$ \Phi_{\tx{hom}} : $ allgemeine Lösung der homogenen DGL $ \Delta \Phi_{\tx{hom}} - \frac{1}{c^2} \prt{^2}{t^2} \Phi_{\tx{hom}} = 0 $ \\
$ \Phi_{\tx{inh}} : $ \ \, spezielle Lösung der inhomogenen DGL $ \Delta \Phi_{\tx{inh}} - \frac{1}{c^2} \prt{^2}{t^2} \Phi_{\tx{inh}} = - \frac{1}{\epsilon_0} \rho $\\[5pt]
Lösung der homogenen Wellengleichung liefert z.B. Kugelwellen oder ebene Wellen:
\begin{equation*}
\Phi_{\tx{hom}}(\vec{r},t) = \Re \left\{ a e^{i(\vec{k} \cdot \vec{r} - \omega t)} \right\}
\end{equation*}
mit $ \vec{k} \in \mathbb{R}^3 $, $ |\vec{k}| = \frac{\omega}{c} $ bzw.: $ \omega(\vec{k}) = |\vec{k}| c $\\[5pt]
Die allgemeine Lösung erhalten wir nun durch Linearkombination:
\begin{equation*}
\Phi_{\tx{hom}}(\vec{r},t) = \Re \left\{ \int \dd ^3 \vec{k} \ a (\vec{k}) e^{i (\vec{k} \cdot \vec{r} - \omega(\vec{k}) t)} \right\}
\end{equation*}
Bestimmung einer speziellen Lösung der inhomogenen Wellengleichung:
\begin{equation*}
\Delta \Phi - \frac{1}{c^2} \prt{^2}{t^2} \Phi = - \frac{1}{\epsilon_0} \rho
\end{equation*}
zur Lösung der DGL betrachten wir eine Fouriertransformation von $ \Phi (\vec{r},t) $ bezüglich der Variablen $ t $.
\begin{equation*}
\tilde{\Phi}(\vec{r},\omega) \defeq \frac{1}{\sqrt{2 \pi}} \int_{- \infty}^{\infty} \dd t \Phi (\vec{r},t) e^{i \omega t}
\end{equation*}
Man kann danach zurücktransformieren:
\begin{equation*}
\Phi (\vec{r},t) = \frac{1}{\sqrt{2 \pi}} \int_{-\infty}^{\infty} \dd \omega \tilde{\Phi} (\vec{r},t) e^{- i \omega t}
\end{equation*}
wir können dies nun ineinander einsetzen um zu sehen ob dasselbe wieder herauskommt oder ob wir bei der Berechnung etwas verloren haben.
\begin{align*}
\Phi(\vec{r},t) &= \frac{1}{\sqrt{2 \pi}} \int_{- \infty}^{\infty} \dd \omega \left\{ \frac{1}{\sqrt{2 \pi}} \int_{- \infty}^{\infty} \dd t' \Phi(\vec{r},t) e^{i \omega t}\right\} e^{- i \omega t} \\
&= \int_{- \infty}^{\infty} \dd t' \Phi(\vec{r},t) \ub{\frac{1}{2 \pi} \int_{-\infty}^{\infty} \dd \omega e^{i \omega (t - t')}}_{= \delta(t - t')}\\
&= \Phi (\vec{r},t)
\end{align*}
Somit haben wir keine Information gewonnen oder erzeugt. Wir haben den Ausdruck lediglich in eine andere Darstellung umgeformt.\\[5pt]
Wir setzen nun die Fourierdarstellung in die DGL ein:
\begin{align*}
\prt{^2}{t^2} \Phi(\vec{r},t) &= \frac{1}{\sqrt{2 \pi}} \int \dd \omega \ \tilde{\Phi} (\vec{r},t) \ub{\prt{^2}{t^2} e^{- i \omega t}}_{= - \omega^2 e^{- i \omega t}}\\
&= \frac{1}{\sqrt{2 \pi}} \int \dd \omega (- \omega ^2) \tilde{\Phi} (\vec{r},t) e^{- i \omega t}
\end{align*}
Damit gilt:
\begin{align*}
& \frac{1}{\sqrt{2 \pi}} \int \dd \omega e^{- i \omega t} \left(\Delta + \frac{\omega^2}{c^2}\right) \tilde{\Phi} (\vec{r},t) = - \frac{1}{\epsilon_0} \rho (\vec{r},t)\\
\overset{\substack{\tx{Fouriertrafo.} \\ \tx{von } \rho}}{=} & \frac{1}{\sqrt{2 \pi}} \int \dd \omega e^{- i \omega t} \left(- \frac{1}{\epsilon_0}\right) \tilde{\rho} (\vec{r},\omega)
\end{align*}
\frbox{Helmholtz-Gleichung}{
\begin{equation*}
\Rightarrow \quad \left(\Delta + \frac{\omega^2}{c^2}\right) \tilde{\Phi} (\vec{r},t) = - \frac{1}{\epsilon_0} \tilde{\rho} (\vec{r},\omega)
\tag{*}
\label{Helmholtz}
\end{equation*}
}

\noindent
Zur Bestimmung der Lösung von \eqref{Helmholtz} betrachten wir zunächst eine Analogie zur Elektrostatik: $ \omega \to 0 \quad (\widehat{=} \ \prt{}{t} = 0) $
\begin{equation*}
\Rightarrow \quad \Delta \tilde{\Phi} = - \frac{1}{\epsilon_0} \tilde{\rho} \qquad \tx{Poisson-Gleichung}
\end{equation*}
spezielle Lösung (ohne Randbedingungen)
\begin{equation*}
\tilde{\Phi} (\vec{r},\omega = 0) = \kq \int \dd^3 r' \frac{\rho(\vec{r}')}{| \vec{r} - \vec{r}' |} = \int \dd^3 r' \grr \rho(\vec{r}')
\end{equation*}
mit der Green'schen Funktion:
\begin{equation*}
\grr = \kq \frac{1}{| \vec{r} - \vec{r}' |}
\end{equation*}
Diese erfüllt die Laplace Gleichung:
\begin{equation*}
\Delta_{\vec{r}} \grr = - \frac{1}{\epsilon_0} \delta(\vec{r} - \vec{r}')
\end{equation*}
Wir benötigen eine Verallgemeinerung für $ \omega \neq 0 $:
\begin{equation*}
\left(\Delta_{\vec{r}} + \frac{\omega^2}{c^2}\right) \gre_{\omega} (\vec{r}, \vec{r}') = - \frac{1}{\epsilon_0} \delta(\vec{r} - \vec{r}')
\end{equation*}
Es zeigt sich, dass gilt:
\begin{equation*}
\left(\Delta_{\vec{r}} + \frac{\omega^2}{c^2}\right) \kq \frac{e^{\pm i k |\vec{r} - \vec{r}'|}}{|\vec{r} - \vec{r}'|} = - 4 \pi \delta(\vec{r} - \vec{r}') = \grr
\end{equation*}
%
%
%
% oben stand Grr = - 4 \pi \delta (shit) aber es sollte glaub ich sein Grr = - 1/\epsilon_0 \delta(shit)
%
%
%
Mit $ k = \frac{\omega}{c} $ folgt daraus:
\begin{equation*}
\left(\Delta_{\vec{r}} + \frac{\omega^2}{c^2}\right) \kq \frac{e^{\pm i \frac{\omega}{c} |\vec{r} - \vec{r}'|}}{|\vec{r} - \vec{r}'|} = - \frac{1}{\epsilon_0} \delta(\vec{r} - \vec{r}')
\end{equation*}
und somit:
\begin{equation*}
\tilde{\Phi} (\vec{r},\omega) = \kq \int \dd^3 r' \ub{\frac{e^{\pm i \frac{\omega}{c} |\vec{r} - \vec{r}'|}}{|\vec{r} - \vec{r}'|}}_{\gre_{\omega} (\vec{r},\vec{r}')} \tilde{\rho}(\vec{r}',\omega)
\end{equation*}
\lcom{Da die Lösung gleich der Green'schen Funktion ($ \gre_{\omega} $) mal der inhomogenität ($ \tilde{\rho} $) ist.}\\[5pt]
Für das Potential $ \Phi $ erhalten wir nun durch Rücktransformation:
\begin{align*}
\Phi (\vec{r},t) &= \frac{1}{\sqrt{2 \pi}} \int_{- \infty}^{\infty} \dd \omega \tilde{\Phi} (\vec{r},\omega) e^{- i \omega t}\\
&= \frac{1}{\sqrt{2 \pi}} \int \dd \omega \kq \int \dd^3 r' \frac{e^{\pm i \frac{\omega}{c} |\vec{r} - \vec{r}'|}}{|\vec{r} - \vec{r}'|} e^{- i \omega t} \tilde{\rho} (\vec{r}' , \omega)\\
&= \kq \int \dd^3 r' \frac{1}{|\vec{r} - \vec{r}'|} \ub{\frac{1}{\sqrt{2 \pi}} \int \dd \omega \ \tilde{\rho} (\vec{r}', \omega) e^{- i \omega (t - \frac{|\vec{r} - \vec{r}'|}{c} )}}_{= \rho(\vec{r}', t - \frac{|\vec{r} - \vec{r}'|}{c})}
\end{align*}
\begin{equation*}
\Rightarrow \quad \Phi(\vec{r},t) = \kq \int\dd^3 r' \frac{\rho(\vec{r}', t - \frac{|\vec{r} - \vec{r}'|}{c})}{|\vec{r} - \vec{r}'|}
\end{equation*}
wie in der Elektrostatik, aber:
\begin{equation*}
t \to \tilde{t} = t - \frac{|\vec{r} - \vec{r}'|}{c}
\label{tikzrho} \tag{*}
\end{equation*}
\lcom{Die neue Zeit ist also die alte Zeit minus der räumliche Abstand durch die Ausbreitungsgeschwindigkeit also die alte Zeit minus die Zeit in der die Änderung voranschreiten konnte.}\\[5pt]
\begin{minipage}{.5\linewidth}
	physikalische Bedeutung:\\[5pt]
	Betrachte Änderung der Ladungsverteilung $ \rho $ in $ \vec{r}' $ zur Zeit $ \tilde{t} $.
\end{minipage}%
\begin{minipage}{.5\linewidth}
	\flushright
	%t1:
	\begin{tikzpicture}
		\draw[fill=gray!10, scale=.7] plot [smooth cycle, tension=.8] coordinates { (.5,-1.5) (0,-1) (-.7,-.5) (-.7,.7) (.7,1) (.7,0) (1.2,-.7) };
		\coordinate (o) at (1,-1.5);
		\draw[thick,->] (o) -- node[anchor=south west] {$ \vec{r}' $} ($ (0,0) + (-45:.5) $);
		\coordinate (p) at (5,-.5);
		%\node[fill=black,circle,minimum size=1pt, inner sep=1pt] (p2) at (5,-.5) {};
		\draw[thick] (o) -- node[anchor=north west] {$ \vec{r} $} (p);
		\draw[thick] ($ (0,0) + (-45:.5) $) -- node[above] {$ | \vec{r} - \vec{r}' | $} (p) node[right] {$ \Phi(\vec{r},t) $};
		\draw[thick,->] ($ (p) + (-0.05,0) $) -- ++(10:.1);
		\draw (45:.5) to[out=70,in=-180] (30:1) node[right] {$ \rho(\vec{r},\tilde{t}) $};
		%\node[circle,minimum size=1cm, draw=black] (a) at (4,4){};
		%\draw[->] (0,0) -- (a);
	\end{tikzpicture}
	\vspace{10pt}
\end{minipage}%
\\
$ \Rightarrow $ Die Änderung des elektro-magnetischen Feldes (oder $ \Phi $) verbreitet sich mit Lichtgeschwindigkeit $ c $. In der Entfernung $ |\vec{r} - \vec{r}'| $ ändert sich das Feld erst zur späteren Zeit $ t = \tilde{t} + \frac{|\vec{r} - \vec{r}'|}{c} $. Wegen der verzögerten Änderung heißt die Potential-Lösung auch \textbf{retardierendes Potential}.
\vspace{5pt}
\frbox{Retardierendes Potential}{
\begin{equation*}
\Phi_{\tx{ret}} (\vec{r},t) = \kq \int \dd^3 r' \frac{\rho (\vec{r}', t - \frac{|\vec{r} - \vec{r}'|}{c})}{|\vec{r} - \vec{r}'|}
\end{equation*}
}
\noindent
Dies ist eine spezielle Lösung der inhomogenen Wellengleichung:
\begin{equation*}
\left(\Delta - \frac{1}{c^2} \prt{^2}{t^2}\right) \Phi = - \frac{1}{\epsilon_0} \rho(\vec{r},t)
\end{equation*}
Die entsprechende Lösung der Gleichung für das Vektorpotential $ \vec{A} $:
\begin{equation*}
\left(\Delta - \frac{1}{c^2} \prt{^2}{t^2}\right) \vec{A} = - \mu_0 \vec{j}(\vec{r},t)
\end{equation*}
ist:
\vspace{5pt}
\frbox{Retardierendes Vektorpotential}{
\begin{equation*}
\vec{A}_{\tx{ret}} (\vec{r},t) = \frac{\mu_0}{4 \pi} \int \dd^3 r' \frac{\vec{j}(\vec{r}', t - \frac{|\vec{r} - \vec{r}'|}{c})}{|\vec{r} - \vec{r}'|}
\end{equation*}
}
\noindent
Damit haben wir die spezielle oder auch partikuläre Lösung der inhomogene Wellengleichung gefunden.\\[10pt]
Die Felder erhalten wir gemäß:
\begin{equation*}
\vec{B} = \vabla \times \vec{A} \qquad \vec{E} = - \vabla \Phi - \prt{A}{t}
\end{equation*}
\vspace{10pt}

\noindent
\emph{Bemerkung:} Wir haben vorhin nur die Lösung mit $ e^{+ i \frac{\omega}{c} |\vec{r} - \vec{r}'|} $ betrachtet, die uns auf die retardierenden Potentiale geführt hat. Die andere Lösung mit $ e^{- i \frac{\omega}{c} |\vec{r} - \vec{r}'|} $ würde uns auf andere Potentiale führen: \textbf{avancierte Potentiale}
\begin{equation*}
\Phi_{\tx{av}} (\vec{r},t) = \kq \int \dd^3 r' \frac{\rho(\vec{r}', t + \frac{|\vec{r} - \vec{r}'|}{c})}{|\vec{r} - \vec{r}'|}
\end{equation*}
Diese Lösung der avancierten Potentiale führt uns darauf, dass die Änderung von $ \rho $ für $ \tilde{t} = t + \frac{|\vec{r} - \vec{r}'|}{c} > t $ die Felder zu einem früheren Zeitpunkt $ t $ beeinflusst. Dies \textbf{widerspricht dem Kausalitätsprinzip} und wurde deshalb vorhin von uns ignoriert. Die Lösung ist dennoch hilfreich falls man ein Problem nur mit der allgemeinen Lösung ($ \pm $) Lösen kann (Siehe: Feldtheorie, S-Matrix-Formalismus).\\

% Vorlesung 24.01

\bbb{Wiederholung}{ % make me nicer !!!
Maxwellgleichungen:
\begin{align*}
\vabla \cdot \vec{B} &= 0 \qquad \qquad \quad \vabla \times \vec{E} + \prt{\vec{B}}{t} = 0\\
\vabla \cdot \vec{E} &= \frac{1}{\epsilon_0} \rho \qquad \qquad \vabla \times \vec{B} - \ub{\epsilon_0 \mu_0}_{\frac{1}{c^2}} \prt{\vec{E}}{t} = \mu_0 \vec{j}
\end{align*}
Potentiale:
\begin{equation*}
\vec{B} = \vabla \times \vec{A} \qquad \qquad \vec{E} = - \vabla \Phi - \prt{\vec{A}}{t}
\end{equation*}
\textbf{Lorenzeichung:}
\begin{equation*}
\vabla \cdot \vec{A} + \frac{1}{c^2} \prt{\Phi}{t} = 0
\end{equation*}
Entkoppelte Wellengleichungen der Potentiale:
\begin{equation*}
\Delta \Phi - \frac{1}{c^2} \prt{^2 \Phi}{t^2} = - \frac{1}{\epsilon_0} \rho \qquad \qquad 
\Delta \vec{A} - \frac{1}{c^2} \prt{^2 \vec{A}}{t^2} = - \mu_0 \vec{j}
\end{equation*}
Retardierende Potentiale:
\begin{equation*}
\Phi_{\tx{ret}} (\vec{r},t) = \kq \int \dd^3 r' \frac{\rho (\vec{r}', t - \frac{|\vec{r} - \vec{r}'|}{c})}{|\vec{r} - \vec{r}'|}
\qquad \qquad
\vec{A}_{\tx{ret}} (\vec{r},t) = \frac{\mu_0}{4 \pi} \int \dd^3 r' \frac{\vec{j}(\vec{r}', t - \frac{|\vec{r} - \vec{r}'|}{c})}{|\vec{r} - \vec{r}'|}
\end{equation*}
%t1:
Zum Verständnis des Retardierenden Potentials Skizze (\ref{tikzrho}).
}

\section{Elektromagnetische Potentiale und Felder bewegter Punktladungen}

%
%
%
% T2
%
%
%
\begin{equation*}
\rho(\vec{r},t) = q \delta (\vec{r} - \vec{R}(t)) \qquad \vec{j}(\vec{r},t) = q \vec{V}(t) \delta(\vec{r} - \vec{R}(t))
\end{equation*}
mit $ \vec{V}(t) = \dot{\vec{R}}(t) $\\
\begin{equation*}
\Phi(\vec{r},t) = \kq \int \dd^3 r' \frac{q \delta(\vec{r}' - \vec{R}(\tilde{t}))}{|\vec{r} - \vec{r}'|}
\end{equation*}
$ \tilde{t} = t - \frac{|\vec{r} - \vec{r}'|}{c} $\\
Wir wollen nun das folgende Integral lösen:
\begin{align*}
\int \dd^3 r' \frac{q \delta(\vec{r}' - \vec{R}(\tilde{t}))}{|\vec{r} - \vec{r}'|} &= \int \dd^3 r' \int_{-\infty}^{\infty} \dd t' \frac{q \delta(\vec{r}' - \vec{R}(\custoup{\leftrightarrow}{\tilde{t}}{t'}))}{|\vec{r} - \vec{r}'|} \delta(t' - \tilde{t})\\
&= \int \dd^3 r' \int \dd t' \frac{\delta(\vec{r}' - \vec{R}(t')) \delta(t' - t + \frac{|\vec{r} - \vec{r}'|}{c})}{|\vec{r} - \vec{r}'|}\\
&= \int \dd t' \int \dd^3 r' \frac{\delta(\vec{r}' - \vec{R}(t')) \delta(t' - t + \frac{|\vec{r} - \vec{r}'|}{c})}{|\vec{r} - \vec{r}'|}\\
&= \int \dd t' \frac{\delta(\ob{t' - t + \frac{|\vec{r} - \vec{R}(t')|}{c}}^{= f(t')})}{|\vec{r} - \vec{R}(t')|}\\
&= \int \dd t' \frac{\delta(f(t'))}{|\vec{r} - \vec{R}(t')|}
\end{align*}
Wir benutzen eine Eigenschaft der $ \delta $-Funktion:
\begin{equation*}
\delta (f(t')) = \sum_{j} \frac{\delta(t' - t_j)}{\left|\left(\frac{\dd f}{\dd t'}\right)_{t' = t_j}\right|}
\end{equation*}
$ t_j : $ Nullstellen von $ f(t') $\\
$ f(t') $ hat höchstens eine Nullstelle: $ t_j \eqdef t_{\tx{ret}} \qquad t_{\tx{ret}}: \ f(t_{\tx{ret}}) = 0$
\begin{equation*}
t_{\tx{ret}} = t - \frac{|\vec{r} - \vec{R}(t_{\tx{ret}}) |}{c}
\end{equation*}
gegeben: $ t , \vec{r} \rightarrow t_{\tx{ret}}(\vec{r},t) $\\[5pt]
Berechnen wir zuerst die Ableitung:
\begin{align*}
\prd{f}{t'} &= 1 + \frac{1}{c} \prd{}{t'} |\vec{r} - \vec{R}(t')|\\
&= 1 - \frac{1}{c} \frac{(\vec{r} - \vec{R}(t')) \cdot \ob{\dot{\vec{R}}(t')}^{=\vec{V}(t')}}{|\vec{r} - \vec{R}(t')|}\\
&= 1 - \ub{\vec{e}_{\vec{r} - \vec{R}(t')} \cdot \frac{\vec{V}(t')}{c}}_{\le\ |\vec{e} \cdot \frac{\vec{V}}{c}| \ \le \ |\frac{\vec{V}}{c}| \ < \ 1}
\end{align*}
%
%
%
% T3
%
%
%
\begin{align*}
\int \dd t' \frac{\delta(f(t'))}{|\vec{r} - \vec{R}(t')|} &= \int \dd t' \frac{\delta(t' - t_{\tx{ret}})}{\left|\left(\frac{\dd f}{\dd t'}\right)_{t' = t_{\tx{ret}}}\right|} \cdot \frac{1}{| \vec{r} - \vec{R}(t')|}\\
&= \int \dd t' \frac{\delta(t' - t_{\tx{ret}})}{|\vec{r} - \vec{R}(t_{\tx{ret}})| - \frac{1}{c} (\vec{r} - \vec{R}(t_{\tx{ret}})) \cdot \vec{V}(t_{\tx{ret}})}\\
&= \frac{1}{|\vec{r} - \vec{R}(t_{\tx{ret}})| - \frac{1}{c} (\vec{r} - \vec{R}(t_{\tx{ret}})) \cdot \vec{V}(t_{\tx{ret}})}
\end{align*}
Damit haben wir das integral gelöst und erhalten:
\frbox{Lienard-Wiechert-Potentiale}{
\begin{align*}
\Rightarrow \quad \Phi(\vec{r},t) &= \kq \frac{q}{|\vec{r} - \vec{R}(t_{\tx{ret}})| - \frac{1}{c} (\vec{r} - \vec{R}(t_{\tx{ret}})) \cdot \vec{V}(t_{\tx{ret}})}\\[5pt]
\Rightarrow \quad \vec{A}(\vec{r},t) &= \frac{\mu_0}{4 \pi} \ \ \ \frac{q \vec{V}(t_{\tx{ret}})}{|\vec{r} - \vec{R}(t_{\tx{ret}})| - \frac{1}{c} (\vec{r} - \vec{R}(t_{\tx{ret}})) \cdot \vec{V}(t_{\tx{ret}})}
\end{align*}
}
\noindent
\emph{Beispiele:}
\begin{enumerate}[i)]
	\item statischer Grenzfall: ruhende Punktladung:\\
	$ \vec{V} = 0 \quad \vec{R} = \vec{R}_0 \quad \vec{A}(\vec{r},t) = 0 $
	%
	%
	%
	% T4
	%
	%
	%
	\begin{equation*}
	\Phi(\vec{r},t) = \kq \frac{q}{|\vec{r} - \vec{R}_0|}
	\end{equation*}
	\item gleichförmig bewegte Punktladung:\\
	$ \vec{V} = \vec{V}_0 = \const \quad \vec{R}(t) = \vec{R}_0 + \vec{V}_0 \cdot t $\\[5pt]
	Bestimmung von $ t_{\tx{ret}} $:
	\begin{equation*}
	t_{\tx{ret}} = t - \frac{|\vec{r} - \vec{R}(t_{\tx{ret}})|}{c}
	\end{equation*}
	\begin{align*}
	c(t - t_{\tx{ret}}) &= |\vec{r} - \vec{R}(t_{\tx{ret}})|\\
	&= |\ub{\vec{r} - \vec{R}(t)}_{\eqdef \vec{d}(\vec{r},t)} + \ub{\vec{R}(t) - \vec{R}(t_{\tx{ret}})}_{= \vec{V}_0 \cdot (t - t_{\tx{ret}})}|
	\end{align*}
	%
	%
	%
	% T5
	%
	%
	%
	\begin{align*}
	\Rightarrow \quad c^2(t - t_{\tx{ret}})^2 &= | \vec{d} - \vec{V}_0 \cdot (t - t_{\tx{ret}})|^2\\
	&= d^2 + 2 \cdot \ub{\vec{d} \cdot \vec{V}_0}_{d V_0 \cos \alpha} \cdot (t - t_{\tx{ret}}) + V_0^2 \cdot (t-t_{\tx{ret}})^2
	\end{align*}
	\begin{align*}
	\Rightarrow \quad t - t_{\tx{ret}} &= \frac{d V_0 \cos \alpha}{c^2 - V_0^2} + \frac{d}{c^2 - V_0^2} \sqrt{V_0^2 \cos^2 \alpha + c^2 - V_0^2} \\
	&= \frac{d V_0 \cos \alpha}{c^2 - V_0^2} + \frac{d}{c^2 - V_0^2} \sqrt{- V_0^2 \sin^2 \alpha + c^2}\\
	&= \frac{|\vec{d}(\vec{r},t)|}{c^2 - V_0^2} \left(V_0 \cos \alpha + \sqrt{c^2 - V_0^2 \sin^2 \alpha}\right)
	\end{align*}
	Im Nenner des Retardierenden Potentials steht:
	\begin{align*}
	& \ub{|\vec{r} - \vec{R}(t_{\tx{ret}})|}_{c \cdot ( t - t_{\tx{ret}})} - \frac{1}{c} (\ub{\vec{r} - \vec{R}(t_{\tx{ret}})}) \cdot \equaltoup{\vec{V}(t_{\tx{ret}})}{\vec{V}_0}\\
	% }_{\mathclap{\substack{\vec{r} - \vec{R}(t) + \vec{R}(t) - \vec{R}(t_{\tx{ret}}) \\ = \vec{d} + \vec{V}_0 (t - t_{\tx{ret}})}}}
	& \quad \qquad \qquad \qquad = \vec{r} - \vec{R}(t) + \vec{R}(t) - \vec{R}(t_{\tx{ret}}) \\
	& \quad \qquad \qquad \qquad = \vec{d} + \vec{V}_0 \cdot (t - t_{\tx{ret}})\\
	&= c \cdot (t - t_{\tx{ret}}) - \frac{1}{c} \vec{V}_0 \cdot (\vec{d} + \vec{V}_0 (t - t_{\tx{ret}}))\\
	&= \frac{1}{c} (c^2 - V_0^2) (t - t_{\tx{ret}}) - \frac{1}{c} V_0 d \cos \alpha\\
	&= \frac{1}{c} |\vec{d}| \left(\cancel{ V_0 \cos \alpha} + \sqrt{c^2 - V_0^2 \sin^2 \alpha} - \cancel{ V_0 \cos \alpha}\right)\\
	&= |\custo{\rightarrow}{\vec{d}}{\mathclap{\vec{r} - \vec{R}(t)}}(\vec{r},t)| \sqrt{1 - \frac{V_0^2}{c^2} \sin^2 \alpha}
	\end{align*}
	\begin{align*}
	\Rightarrow \quad \Phi(\vec{r},t) &= \kq \frac{q}{|\vec{r} - \vec{R}(t)| \sqrt{1 - \frac{V_0^2}{c^2} \sin \alpha^2}}\\
	\Rightarrow \quad \vec{A}(\vec{r},t) &= \frac{\mu_0}{4 \pi} \ \ \ \frac{q \vec{V}_0}{|\vec{r} - \vec{R}(t)| \sqrt{1 - \frac{V_0^2}{c^2} \sin \alpha^2}}
	\end{align*}
	für langsame Teilchen gilt: $ \frac{V_0}{c} \ll 1 $ (nichtrelativistisch):
	\begin{equation*}
	\rightarrow \quad \Phi(\vec{r},t) = \kq \frac{q}{|\vec{r} - \vec{R}(t)|}
	\end{equation*}
\end{enumerate}

\subsection{Felder einer gleichförmig-bewegten Punktladung}

\begin{equation*}
\vec{E} = - \vabla \Phi - \prt{\vec{A}}{t} \qquad \vec{B} = \vabla \times \vec{A}
\end{equation*}
Komplikationen $ \alpha = \alpha(\vec{r},t) $
%
%
%
% T6
%
%
%
\lcom{Dies wird z.B. im Griffitz genau hergeleitet, ist aber obwohl es nicht zu kompliziert ist, zu langwierig für de Vorlesung.}
\begin{align*}
\vec{E} &= \frac{q}{4 \pi \epsilon_0} \frac{1 - \frac{V_0^2}{c^2}}{\left(1 - \frac{V_0^2}{c^2} \sin^2 \alpha^2\right)^{\frac{3}{2}}} \frac{\vec{r} - \vec{R}(t)}{|\vec{r} - \vec{R}(t)|^3}\\
\vec{B} &= \frac{1}{c^2} \left(\vec{V}_0 \times \vec{E}\right)
\end{align*}
%
%
%
% I33 (l.2743) oben steht sin^2 \alpha^2 aber ich glaube dass 2. ^2 ist zuviel
%
%
%
Bei kleinen Geschwindigkeiten gilt:\\
$ \frac{V_0}{c} \ll 1 : $
\begin{equation*}
\rightarrow \quad \vec{E} \approx \frac{q}{4 \pi \epsilon_0} \frac{\vec{r} - \vec{R}(t)}{|\vec{r} - \vec{R}(t)|^3} \qquad \vec{B} \approx 0
\end{equation*}
$ \frac{V_0}{c} : $ endlich
%
%
%
% T7
%
%
%
\begin{enumerate}[1)]
	\item $ \vec{d} \perp \vec{R} \quad \alpha = \frac{\pi}{2} $
	\begin{equation*}
	\Rightarrow \quad \vec{E} = \frac{q}{4 \pi \epsilon_0} \frac{1}{\sqrt{1 - \frac{V_0^2}{c^2}}} \frac{\vec{r} - \vec{R}(t)}{|\vec{r} - \vec{R}(t)|^3} \ > \ \frac{q}{4 \pi \epsilon_0} \frac{\vec{r} - \vec{R}(t)}{|\vec{r} - \vec{R}(t)|^3}
	\end{equation*}
	\item $ \vec{d} \parallel \vec{R} \quad \alpha = 0 $
	\begin{equation*}
	\Rightarrow \quad \vec{E} = \frac{q}{4 \pi \epsilon_0} \left(1 - \frac{V_0^2}{c^2}\right) \frac{\vec{r} - \vec{R}(t)}{|\vec{r} - \vec{R}(t)|^3} \ < \ \frac{q}{4 \pi \epsilon_0} \frac{\vec{r} - \vec{R}(t)}{|\vec{r} - \vec{R}(t)|^3}
	\end{equation*}
\end{enumerate}
\folie{$ \vec{E} $- und $ \vec{B} $-Felder von Ladungen die sich mit nahezu Lichtgeschwindigkeit bewegen.}

\subsection{Felder einer (allgemein) bewegten Punktladung}

\begin{equation*}
\vec{E} = - \vabla \Phi - \prt{\vec{A}}{t} \qquad \vec{B} = \vabla \times \vec{A}
\end{equation*}
\lcom{Wird ebenfalls nicht in der Vorlesung berechnet da es zu langwierig ist.}\\[5pt]
%
%
%
% T8
%
%
%
$ \vec{d}_{\tx{ret}} \defeq \vec{r} - \vec{R}(t_{\tx{ret}}) \qquad \vec{e}_{\tx{ret}} \defeq \frac{\vec{d}_{\tx{ret}}}{|\vec{d}_{\tx{ret}}|} \qquad \vec{\beta}(t) = \frac{\vec{V}(t)}{c} $
\begin{align*}
\vec{E}(\vec{r},t) &= \frac{q}{4 \pi \epsilon_0} \frac{(\vec{e}_{\tx{ret}} - \vec{\beta}(t_{\tx{ret}})) ( 1 - \beta^2 (t_{\tx{ret}}) )}{(1 - \vec{e}_{\tx{ret}} \cdot \vec{\beta}(t_{\tx{ret}}))^3 |\vec{d}_{\tx{ret}}(\vec{r},t)|^2}\\
&+ \frac{q}{4 \pi \epsilon_0} \frac{\vec{e}_{\tx{ret}} \times \left[\left(\vec{e}_{\tx{ret}} - \vec{\beta}(t_{\tx{ret}}) \times \dot{\vec{\beta}}(t_{\tx{ret}}\right)\right]}{c \ ( 1 - \vec{e}_{\tx{ret}} \cdot \vec{\beta}(t_{\tx{ret}}))^3 |\vec{d}_{\tx{ret}}(\vec{r},t)|}\\[5pt]
\vec{B}(\vec{r},t) &= \frac{1}{c} \vec{e}_{\tx{ret}} \times \vec{E}(\vec{r},t)
\end{align*}
\emph{Bemerkung:}
\begin{enumerate}[a)]
	\item Der erste Term des $ \vec{E} $-Feldes:\\
	$ \sim \frac{1}{d_{\tx{ret}}^2} \overset{r \to \infty}{\longrightarrow} \frac{1}{r^2} $ wie in der Elektrostatik\\
	unabhängig von der Beschleunigung $ \dot{\vec{\beta}} $
	\item Der zweite Term des $ \vec{E} $-Feldes:\\
	$ \sim \frac{1}{d_{\tx{ret}}} \overset{r \to \infty}{\longrightarrow} \frac{1}{r} $ dominiert für $ r \to \infty $\\[5pt]
	$ \sim \dot{\vec{\beta}} $ verschwindet für eine gleichförmig Bewegte Ladung
\end{enumerate}

%Vorlesung 28.01

\section{Elektromagnetische Felder oszillierender Quellen - \texorpdfstring{\\}{newline} EM-Strahlung}

oszillierende Ladungs-/Stromverteilung
\begin{align*}
\rho(\vec{r},t) &= \rho(\vec{r}) e^{- i \omega t}\\
\vec{j}(\vec{r},t) &= \vec{j}(\vec{r}) e^{- i \omega t}
\end{align*}
\lcom{Aus Einfachheitshalber in der Notation bekommen die Orts- und Zeitabhängigen Funktionen dieselben symbole wie die nur Ortsabhängigen.}\\
\emph{Beispiele:}
%
%
%
% T1
%
%
%

\noindent
Vektorpotential:
\begin{align*}
\vec{A}(\vec{r},t) &= \frac{\mu_0}{4 \pi} \int \dd^3 r' \frac{\vec{j}(\vec{r}', t - \frac{|\vec{r} - \vec{r}'|}{c})}{|\vec{r} - \vec{r}'|} = \frac{\mu_0}{4 \pi} \int \dd^3 r' \frac{\vec{j}(\vec{r}')}{|\vec{r} - \vec{r}'|} e^{-i\omega \left(t - \frac{|\vec{r} - \vec{r}'|}{c}\right)} \qquad \quad k = \frac{\omega}{c} \\
&= e^{-i\omega t} \ub{\frac{\mu_0}{4 \pi} \int \dd^3 r' \frac{\vec{j}(\vec{r}') e^{i k |\vec{r} - \vec{r}'|}}{|\vec{r} - \vec{r}'|}}_{\eqdef \vec{A}(\vec{r})} = e^{ - i \omega t} \vec{A}(\vec{r})
\end{align*}
Wir interessieren uns für zeitabhängige Felder in Raumbereichen außerhalb der Ladung-/Strom\-verteilung\\[5pt]
\textbf{Annahmen:}
\begin{enumerate}[i)]
	\item zeitabhängige Felder im großen Abstand zur Quelle
	%
	%
	%
	% T2
	%
	%
	%
	\begin{equation*}
	\vec{j} (\vec{r}') = \casess{0}{r' > R_0}{\tx{sonst}}{\tx{beliebig}}
	\end{equation*}
	$ \rightarrow $ Betrachtung von $ \vec{A} $ (von $ \vec{A} $ ist Ausreichend)
	\begin{equation*}
	\vec{B} = \vabla \times \vec{A} \quad \checkmark
	\end{equation*}
	$ \vec{E} = ? $
	\begin{align*}
	\vabla \times \vec{B}(\vec{r},t) - \frac{1}{c^2} \prt{\vec{E}(\vec{r},t)}{t} = \mu_0 \vec{j}(\vec{r},t) = 0 \quad \tx{für } |\vec{r}| > R_0
	\end{align*}
	\begin{align*}
	\prt{}{t} \vec{E}(\vec{r},t) &= c^2 \vabla \times \vec{B} = c^2 \vabla \times (\vabla \times \vec{A}(\vec{r},t))\\
	&= e^{-i\omega t} c^2 \vabla \times (\vabla \times \vec{A}(\vec{r}))
	\end{align*}
	\begin{equation*}
	\Rightarrow \quad \vec{E}(\vec{r},t) = \frac{i}{\omega} e^{-i\omega t} c^2 \vabla \times (\vabla \times \vec{A}(\vec{r})) + \vec{f}(\vec{r})
	\end{equation*}
	$ \omega \neq 0 \quad \vec{f} \equiv 0 $
	\item Wellenlänge $ \lambda \gg $ Ausdehnung $ R_0 $ (Langwellennäherung)
	\begin{equation*}
	\lambda = \frac{2 \pi}{k} = \frac{2 \pi c}{\omega} \gg R_0
	\end{equation*}
\end{enumerate}
i) \& ii) $ R_0 \ll \lambda , r $
Wir integrieren über
\begin{equation*}
\frac{e^{i k |\vec{r} - \vec{r}'|}}{|\vec{r} - \vec{r}'|}
\end{equation*}
\begin{align*}
|\vec{r} - \vec{r}'| &= \sqrt{r^2 - 2 \vec{r} \cdot \vec{r}' + r'^2}\\
&= r \sqrt{1 - e \vec{e}_{\vec{r}} \cdot \frac{\vec{r}'}{r} + \left(\frac{r'}{r}\right)^2}\\
&\approx r (1 - \vec{e}_{\vec{r}} \cdot \frac{\vec{r}'}{r}) \qquad \qquad \vec{r}' \ll r
\end{align*}
\begin{align*}
e^{i k |\vec{r} - \vec{r}'|} &\approx e^{i k r} \ub{ e ^{ - i k \vec{e}_{\vec{r}} \cdot \vec{r}'}}_{\approx 1 - i k \vec{e}_{\vec{r}} \cdot \vec{r}'}
\end{align*}
mit $ k r' = 2 \pi \frac{r'}{\lambda} \ll 1 $, da $ k R_0 \ll 2 \pi $
\begin{align*}
\frac{1}{|\vec{r} - \vec{r}'|} \approx \frac{1}{r} ( 1 + \vec{e}_{\vec{r} \cdot \frac{\vec{r}'}{r}})
\end{align*}
(aus der Multipolentwicklung)
\begin{align*}
\frac{e^{i k |\vec{r} - \vec{r}'|}}{|\vec{r} - \vec{r}'|} &\approx e^{i k r} (1 - i k \vec{e}_{\vec{r}} \cdot \frac{\vec{r}'}{r}) (1 + \vec{e}_{\vec{r}} \cdot \frac{\vec{r}'}{r})\\
&= e^{i k r} \Bigg[\frac{1}{r} - i k \vec{e}_{\vec{r}} \cdot \frac{\vec{r}'}{r} + \frac{1}{r} \vec{e}_{\vec{r}} \cdot \frac{\vec{r}'}{r} - i k \custo{\rightarrow}{\cancel{\left(\vec{e}_{\vec{r}} \cdot \frac{\vec{r}'}{r}\right)^2}}{\mathcal{O} \left(\left(\frac{\vec{r}'}{r}\right)^2\right)}\Bigg]\\
&\approx \frac{e^{i k r}}{r} \left[1 + \vec{e}_{\vec{r}} \cdot \vec{r}' \left(\frac{1}{r} - i k\right)\right]
\end{align*}
\begin{align*}
\vec{A}(\vec{r}) &= \frac{\mu_0}{4 \pi} \int \dd^3 r' \vec{j}(\vec{r}') \frac{e^{i k |\vec{r} - \vec{r}'|}}{|\vec{r} - \vec{r}'|}\\
&= \ub{\frac{\mu_0}{4 \pi} \frac{e^{i k r}}{r} \int \dd^3 r' \vec{j} (\vec{r}')}_{\mathclap{\substack{\tx{elektrischer Dipolanteil} \\ E1}}} + \ub{\frac{\mu_0}{4 \pi} \frac{e^{i k r}}{r} \left(\frac{1}{r} - ik\right) \int \dd^3 r' \vec{j}(\vec{r}') (\vec{e}_{\vec{r}} \cdot \vec{r}')}_{\mathclap{\substack{\tx{el. Quadrupolanteil und magnetischer} \\ \tx{Dipolanteil } (E2, M1)}}}
\end{align*}
\lcom{Wir betrachten nun den elektrischen Dipolterm, um zu erklären wie eine bewegte Ladung und die damit verbundene Stromdichte einen elektrischen Dipol Darstellen.}

\subsubsection{elektrischen Dipolterm:}

lokalisierte Stromverteilung $ \vec{j}_g $:
\begin{equation*}
\int \dd^3 r' \vec{j}(\vec{r}') = - \int \dd^3 r' \vec{r}' \vabla \cdot \vec{j}(\vec{r}')
\end{equation*}
\begin{equation*}
\vabla \cdot ( x \vec{j}(\vec{r}')) = x \vabla \vec{j} + \vec{j} \cdot \ub{\vabla x}_{\vec{e}_x} = x \vabla \vec{j} + j_x
\end{equation*}
\begin{equation*}
\Rightarrow \quad \int_{\mathbb{R}^3} \dd^3 r' \vec{j}_x(\vec{r}') = \ub{\int_{\mathbb{R}^3} \dd^3 r' \vabla \cdot \left(x' \vec{j}(\vec{r})\right)}_{= 0} - \int_{\mathbb{R}^3} \dd^3 r' x' \vabla \cdot \vec{j}
\end{equation*}
\textbf{Kontinuitätsgleichung:}
\begin{align*}
0 &= \prt{}{t} \rho(\vec{r},t) + \vabla \cdot \vec{j}(\vec{r},t)\\
&= \prt{}{t} e^{-i\omega t} \rho(\vec{r}) + \vabla e^{-i\omega t} \vec{j}(\vec{r})\\
&= - i \omega t \rho(\vec{r},t) + e^{- i \omega t} \vabla \cdot \vec{j}(\vec{r})\\
\vabla \vec{j}(\vec{r}) &= i \omega t e^{i \omega t} \rho(\vec{r},t) = i \omega \rho(\vec{r})
\end{align*}
\begin{align*}
\Rightarrow \quad \int \dd^3 r' \vec{j}(\vec{r}') &= - i \omega \ub{\int \dd^3 r' \vec{r}' \rho(\vec{r})}_{\mathclap{= \vec{p} \quad \tx{Dipolmoment}}} = - i \omega \vec{p}
\end{align*}
\begin{equation*}
\Rightarrow \quad \vec{A}_{E1} (\vec{r}) = - i \omega \frac{\mu_0}{4 \pi} \frac{e^{i k r}}{r} \vec{p}
\end{equation*}
\begin{equation*}
\vec{A}_{E1} (\vec{r},t) = e^{- i \omega t} \vec{A}_{E_1}(\vec{r}) = - i \omega \frac{\mu_0}{4 \pi} \frac{e^{i k r}}{r} \ub{e^{- i \omega t} \vec{p}}_{\int \dd^3 r' \vec{r}' \rho(\vec{r},t) = \vec{p}(t)}
\end{equation*}

\subsection{Elektrische Dipolstrahlung}

\begin{equation*}
\vec{A}(\vec{r},t) = e^{- i \omega t} \vec{A}(\vec{r})
\end{equation*}
\begin{equation*}
\vec{A}(\vec{r}) = - i \omega \frac{\mu_0}{4 \pi} \frac{e^{i k r}}{r} \vec{p}
\end{equation*}
\lcom{Wir wollen nun hieraus die elektrischen und magnetischen Felder berechnen.}
\begin{equation*}
\vec{B}(\vec{r},t) = \vabla \times \vec{A}(\vec{r},t) = e^{- i \omega t} \vabla \times \vec{A}(\vec{r})
\end{equation*}
\begin{align*}
\vabla \times \vec{A} &= - i \omega \frac{\mu_0}{4 \pi} \ub{\vabla \times \left(\frac{e^{ikr}}{r} \vec{p}\right)}_{ = - \vec{p} \times \ub{\vabla \frac{e^{ikr}}{r}}_{\mathclap{= \vec{e}_{r} \prt{}{t} \frac{e^{ikr}}{r} = i k \frac{e^{ikr}}{r} \left(1 - \frac{1}{ikr}\right)}}}\\
&= i \omega \frac{\mu_0}{4 \pi} i k \frac{e^{ikr}}{r} \left(1 - \frac{1}{ikr}\right) (\vec{p} \times \vec{e}_{r})\\
&= \frac{\mu_0}{4 \pi} ck^2 \frac{e^{ikr}}{r} \left(1 - \frac{1}{ikr}\right) (\vec{e}_{r} \times \vec{p})
\end{align*}
\begin{equation*}
\Rightarrow \quad \vec{B}(\vec{r},t) = \frac{\mu_0}{4 \pi} ck^2 \frac{e^{i(kr - \omega t)}}{r} \left(1 - \frac{1}{ikr}\right) \vec{e}_r \times \vec{p} \qquad \qquad \qquad \vec{B} \perp \vec{e}_r , \vec{r}
\end{equation*}
%
%
%
% T3
%
%
%
\begin{equation*}
\vec{E}(\vec{r},t) = \frac{i}{\omega} e^{- i \omega t} c^2 \vabla \times (\vabla \times \vec{A}(\vec{r}))
\end{equation*}
\begin{equation*}
\Rightarrow \quad \vec{E}(\vec{r},t) = \kq \frac{e^{i (kr - \omega t)}}{r} \Bigg\{ \, k^2 \, \ub{(\vec{e}_r \times \vec{p}) \times \vec{e}_r}_{\mathclap{\substack{\tx{transversal Komponenten} \\ (\perp \tx{ zu } \vec{e}_r)}}} + \left(\frac{1}{r^2} - \frac{ik}{r}\right) \ub{\left[3 \vec{e}_r (\vec{e}_r \cdot \vec{p}) - \vec{p}\right]}_{\mathclap{\substack{\tx{auch Komponenten} \\ \tx{entlang } \vec{e}_r}}} \Bigg\}
\end{equation*}

\subsubsection{Grenzfall:}

statischer Dipol $ \quad \omega \to 0 \quad k \to 0 $
\begin{align*}
\vec{E}(\vec{r},t) &\rightarrow \kq \frac{3 \vec{e}_r (\vec{e}_r \cdot \vec{p}) - \vec{p}}{r^3}\\
\vec{B}(\vec{r},t) &\rightarrow 0
\end{align*}
\subsubsection{Fernbereich (Strahlungszone)}
$ kr \gg 1 \quad r \gg \lambda $
%
%
%
% T4
%
%
%
$ \frac{1}{kr} \ll 1 \qquad \frac{k^2}{r} = \frac{k^2 r^2}{r^3} \gg \frac{1}{r^3} \qquad \frac{k}{r^2} = \frac{k^2}{kr r} \ll \frac{k^2}{r} $
\begin{equation*}
\Rightarrow \quad \vec{E}(\vec{r},t) = \kq e^{i (kr - \omega t)} \Bigg\{ \frac{k^2}{r} (\vec{e}_r \times \vec{p}) \times \vec{e}_r + \left(\cancel{\frac{1}{r^3}} - \cancel{\frac{ik}{r^2}}\right) \left[3 \vec{e}_r (\vec{e}_r \cdot \vec{p}) - \vec{p}\right]\Bigg\}
\end{equation*}
da, wie oben gezeigt, diese Terme $ \frac{1}{r^3}, \ \frac{ik}{r^2} $ vernachlässigbar klein im Vergleich zu $ \frac{k^2}{r} $ sind.
\begin{align*}
\vec{B}(\vec{r},t) &\approx \frac{\mu_0}{4 \pi} c k^2 \frac{e^{i (kr - \omega t)}}{r} \vec{e}_r \times \vec{p} \\
\vec{E}(\vec{r},t) &\approx \kq k^2 \frac{e^{i(kr - \omega t)}}{r} (\vec{e}_r \times \vec{p}) \times \vec{e}_r\\
&= c \vec{B}(\vec{r},t) \times \vec{e}_r \qquad \qquad &\vec{E} \perp \vec{B}, \vec{e}_r
\end{align*}
%
%
%
% glaube oben muss ein k^2 bei E hin aber da stand keins
%
%
%
\lcom{wir erhalten also sich transversal ausbreitende Kugelwellen.}\\[10pt]
\emph{Bemerkungen:}
\begin{enumerate}[i)]
	\item $ \vec{E},\vec{B},\vec{e}_r $ orthogonal
	%
	%
	%
	% T5
	%
	%
	%
	\item $ |\vec{E}| \sim \frac{1}{r} \quad |\vec{B}| \sim \frac{1}{r} $\\
	\textbf{Strahlungsfelder}\\
	\folie{Strahlungsfelder}
\end{enumerate}

% Vorlesung 31.01

\bbb{Widerholung}{
%
%
%
% T1: T2 der letzten Vorlesung mit <-> R_0 aus T4
%
%
%
$ R_0 \ll \lambda \ll r $
\begin{align*}
\vec{B}(\vec{r},t) &\approx \frac{\mu_0}{4 \pi} c k^2 \frac{e^{i (kr - \omega t)}}{r} \vec{e}_r \times \vec{p} \\
\vec{E}(\vec{r},t) &\approx \kq k^2 \frac{e^{i(kr - \omega t)}}{r} (\vec{e}_r \times \vec{p}) \times \vec{e}_r\\
&= c \vec{B}(\vec{r},t) \times \vec{e}_r \qquad \qquad &\vec{E} \perp \vec{B}, \vec{e}_r
\end{align*}
mit $ \vec{p} = \int \dd^3 r' \vec{r}' \rho(\vec{r}') $
}

\subsection{Energiestromdichte}

\begin{align*}
\vec{S} &= \vec{E} \times \vec{H} = \frac{1}{\mu_0} \vec{E} \times \vec{B}\\
&= \frac{c}{\mu_0} \ub{(\vec{B} \times \vec{e}_r) \times \vec{B}}_{\vec{e}_r \vec{B}^2 + \vec{B} (\equalto{\vec{e}\cdot \vec{B}}{0})}\\
\Rightarrow \quad \vec{S} &= \frac{c}{\mu_0} \vec{B}^2 \vec{e}_r
\end{align*}
\begin{align*}
\vec{B} &= \Re \left\{ \frac{\mu_0}{4 \pi} c k^2 \frac{e^{i(kr-\omega t)}}{r} \vec{e}_r \times \vec{p} \right\}\\
&= \frac{\mu_0}{4 \pi} ck^2 \vec{e}_r \times \vec{p} \frac{\cos(kr - \omega t)}{r}
\end{align*}
\begin{align*}
\Rightarrow \quad \vec{S}(\vec{r},t) &= \frac{c}{\mu_0} \left(\frac{\mu_0}{4 \pi} ck^2\right)^2 (\equalto{\vec{e}_r \times \vec{p}}{p^2 \sin^2 \theta} )^2 \vec{e}_r \frac{\cos^2(kr - \omega t)}{r}
\end{align*}
\begin{equation*}
\Rightarrow \quad \vec{S} = \frac{c}{16 \pi^2 \epsilon_0} \frac{k^4 p^2}{r^2} \sin^2 \theta \cos^2(k t - \omega t) \vec{e}_r
\end{equation*}
gemittelt über Periode $ T = \frac{2 \pi}{\omega} $
\begin{align*}
\langle \vec{S} \rangle &= \frac{1}{T} \int_{0}^{T} \dd t \vec{S}(\vec{r},t)\\
&= \frac{c}{32 \pi^2 \epsilon_0} \frac{k^4 p^2}{r^2} \sin^2 \theta \vec{e}_r & k = \frac{\omega}{c}
\end{align*}
\begin{equation*}
\rmbox{\langle \vec{S} \rangle = \frac{1}{32 \pi^2 \epsilon_0} \frac{\omega^4 p^2}{c^3 r^2} \sin^2 \theta \vec{e}_r}
\end{equation*}
\begin{enumerate}[i)]
	\item Abstrahlung $ \propto \vec{e}_r $
	\item Energieflussdichte $ \propto \frac{1}{r^2} $
	\item in Raumwinkelelementen $ \dd \Omega $ abgestrahlte Leistung
	\begin{equation*}
	\dd P = \langle \vec{S} \rangle \cdot \dd \vec{f}
	\end{equation*}
 	mit $ \dd \vec{f} = \vec{e}_r r^2 \dd \Omega $
	%
	%
	%
	% T3
	%
	%
	%
	\begin{equation*}
	\Rightarrow \quad \dd P = \frac{\omega^4 p^2}{32 \pi^2 \epsilon_0 c^3 r^2} r^2 \sin^2 \theta \dd \Omega
	\end{equation*}
	\begin{equation*}
	\prt{P}{\Omega} = \frac{\omega^4 p^2}{32 \pi^2 \epsilon_0 c^3} \sin^2 \theta
	\end{equation*}
	\item Winkelabhängigkeit: $ \propto \sin^2 \theta $
	%
	%
	%
	% T4
	%
	%
	%
	\item \textbf{gesamte abgestrahlte Leistung}
	\begin{align*}
	P = \int_{4\pi} \dd \Omega \frac{\dd P}{\dd \Omega} \quad \overset{\mathclap{\substack{ \int \dd \Omega \sin^2 \theta \\ = \frac{8 \pi}{3} \\ \phantom{0} }}}{=} \quad \frac{1}{12 \pi \epsilon_0 c^3} \omega	^4 p^2
	\end{align*}
	\begin{equation*}
	\rmbox{P = \int_{4\pi} \dd \Omega \frac{\dd P}{\dd \Omega} = \frac{1}{12 \pi \epsilon_0 c^3} \omega^4 p^2}
	\end{equation*}
\end{enumerate}
\vspace{20pt}
\lcom{Der rest der Vorlesung ist nicht mehr Klausurrelevant.}

%\bibliographystyle{plain}
%\bibliography{literature}
%\addcontentsline{toc}{section}{Literatur}

\end{document}